% Adjust these for the path of the theme and its graphics, relative to this file
%\usepackage{beamerthemeFalmouthGamesAcademy}
\usepackage{../../beamerthemeFalmouthGamesAcademy}
\usepackage{multimedia}
\graphicspath{ {../../} }

% Default language for code listings
\lstset{language=C++,
        morekeywords={each,in,nullptr}
}

% For strikethrough effect
\usepackage[normalem]{ulem}
\usepackage{wasysym}

\usepackage{pdfpages}

% http://www.texample.net/tikz/examples/state-machine/
\usetikzlibrary{arrows,automata}

\newcommand{\modulecode}{COMP140 GAM160}\newcommand{\moduletitle}{Hacking Hardware/Advanced Programming}\newcommand{\sessionnumber}{Session 6}

\begin{document}
\title{\sessionnumber: Computer Graphics Workshop 2}
\subtitle{\modulecode: \moduletitle}

\frame{\titlepage} 

\begin{frame}
	\frametitle{Learning outcomes}
	\begin{itemize}
		\item \textbf{Implement} a basic procedural mesh
		\item \textbf{Implement} some basic primitives
		\item \textbf{Manipulate} geometry in shaders
	\end{itemize}
\end{frame}

\begin{frame}{Exercise 1 - Geometry}
	\begin{enumerate}
		\item Download one of the following projects
		\begin{itemize}
			\item Unity - \url{https://github.com/Falmouth-Games-Academy/COMP280-Unity-Mesh-Example}
			\item UE4 - \url{https://github.com/Falmouth-Games-Academy/COMP280-UE4-Mesh-Example}
		\end{itemize}
		\item Instead of the triangle, implement the following primitives
		\begin{enumerate}
			\item Plane
			\item Pyramid 
			\item Sphere
			\item Cylinder
		\end{enumerate}
	\end{enumerate}
\end{frame}

\begin{frame}{Exercise 2 - Manipulation}
	\begin{enumerate}
		\item Manipulate the vertices over time, perhaps using a sine wave
		\item Use one of the other vertex element to dampen or expand the effect
		\item Use one of the other vertex elements to implement vertex animation
	\end{enumerate}
\end{frame}
\end{document}
