% Adjust these for the path of the theme and its graphics, relative to this file
%\usepackage{beamerthemeFalmouthGamesAcademy}
\usepackage{../../beamerthemeFalmouthGamesAcademy}
\usepackage{multimedia}
\graphicspath{ {../../} }

% Default language for code listings
\lstset{language=C++,
        morekeywords={each,in,nullptr}
}

% For strikethrough effect
\usepackage[normalem]{ulem}
\usepackage{wasysym}

\usepackage{pdfpages}

% http://www.texample.net/tikz/examples/state-machine/
\usetikzlibrary{arrows,automata}

\newcommand{\modulecode}{COMP140 GAM160}\newcommand{\moduletitle}{Hacking Hardware/Advanced Programming}\newcommand{\sessionnumber}{Session 6}

\begin{document}
\title{\sessionnumber: Geometry}
\subtitle{\modulecode: \moduletitle}

\frame{\titlepage} 

\begin{frame}{Learning outcomes}
	\begin{itemize}
		\item \textbf{Understand} how a mesh is represented in memory
		\item \textbf{Implement} custom meshes in UE4 or Unity
		\item \textbf{Manipulate} these meshes in a shader
	\end{itemize}
\end{frame}

\begin{frame}{Intro}
	\begin{itemize}
		\item One of the most important points in 3D Graphics is that we are manipulating data on the GPU
		\pause\item This means we need to understand how to package that data on the Application side
		\pause\item You need to understand how this data is represented in memory
		\pause\item Add how to operate on the data in shaders to achieve certain effects
	\end{itemize}
\end{frame}

\part{Complex meshes}
\frame{\partpage}

\begin{frame}{Mesh Formats}
	\begin{itemize}
		\item Typically we invent our own mesh format (see Doom's MD6, Valve's smd formats)
		\pause\item These formats are optimised for realtime rendering and are very efficient
		\pause\item Usually developers write exporters for Maya or 3DSMax to support their format
		\pause\item We are going to use FBX (Autodesk Filmbox) as our model format, this known as an 'interchange' format
	\end{itemize}
\end{frame}

\begin{frame}{Quick Tour of the FBX Format}
	\begin{center}
		\includegraphics[width=\textwidth,height=0.8\textheight]{scene_org}
	\end{center}
\end{frame}
\part{Vertices}
\frame{\partpage}

\begin{frame}{Interleaved Vertices}
	\begin{itemize}
		\item Up until this point we have been storing vertex positions as floats
		\item If we need a vertex to have colours, we can store these in a separate Vertex Buffer
		\item Or we can create a \textbf{C structure} which represents a Vertex
		\item This is known as Interleaved Vertices and in \textbf{MOST} cases is more efficient
	\end{itemize}
\end{frame}

\begin{frame}}[fragile]{Vertex Structure 1}
	\begin{lstlisting}
		struct Vertex
		{
			float x,y,z;
		};
	\end{lstlisting}
\end{frame}

\begin{frame}}[fragile]{Vertex Structure 2}
	\begin{lstlisting}
		struct Vertex
		{
			float x,y,z;
			float r,g,b,a;
		};
	\end{lstlisting}
\end{frame}

\begin{frame}{Changes to the Vertex Buffer}
	\begin{itemize}
		\item There will be a slight change 
	\end{itemize}
\end{frame}

\part{Indices}
\frame{\partpage}

\begin{frame}{Indices and Rendering}
	\begin{itemize}
		\item Indices are integers which specify the vertices that make up a mesh
		\pause\item In rendering this allows us to send less data to the pipeline
		\pause\item A cube would have 36 vertices, this will be at least 432 bytes (12 bytes per vertex)
		\pause\item With indices, we used 8 vertices and 36 indices, which is around 240 bytes in total.
	\end{itemize}
\end{frame}

\begin{frame}{Index Buffer}
\begin{itemize}
	\item These indices are stored in what is know as a Index Buffer(Directx) or Element Buffer(OpenGL)
	\pause \item In OpenGL and Directx, we fill these up and tell the pipeline what Buffer to use
	\pause \item In UE4 and Unity, these buffers are usually hidden away from us
	\pause \item We typically use higher level classes such as C\# Mesh and UE4 Procedural Mesh to add in our own indices 
\end{itemize}
\end{frame}
\input{primitives}
\part{Vertex Shader}
\frame{\partpage}

\begin{frame}{Reminder}
	\begin{itemize}
		\item As a reminder, the vertex shader takes in exactly one vertex
		\pause \item This vertex will contain data we 
		\pause \item We then carry out operations on that vertex
		\pause \item Then return that vertex back to the pipeline
	\end{itemize}
\end{frame}

\begin{frame}{Unity's Vertex}
	\begin{itemize}
		\item Unity has built in vertex types
		\begin{itemize}
			\pause \item \textbf{appdata\_base}: position, normal and one texture coordinate
			\pause \item \textbf{appdata\_tan}: position, tangent, normal and one texture coordinate
			\pause \item \textbf{appdata\_full}: position, tangent, normal, four texture coordinates and colour
		\end{itemize}
		\pause \item \url{https://docs.unity3d.com/Manual/SL-VertexProgramInputs.html}
	\end{itemize}
\end{frame}

\begin{frame}{UE4 Vertex - Expressions}
	\begin{itemize}
		\item UE4 has a bunch of Expression nodes which allow you to interact with Vertices
		\begin{itemize}
			\pause \item Vector Expressions: \url{https://docs.unrealengine.com/en-US/Engine/Rendering/Materials/ExpressionReference/Vector/index.html}
			\pause \item Coordinate Expressions: \url{https://docs.unrealengine.com/en-US/Engine/Rendering/Materials/ExpressionReference/Coordinates/index.html}
		\end{itemize}
	\end{itemize}
\end{frame}

\begin{frame}[fragile]{Vertex Shader - GLSL Example}
	\begin{lstlisting}
	#version 330 core
	
	layout(location = 0) in vec3 vertexPosition;
	layout(location = 1) in vec2 vertexTextureCoord;
	
	uniform mat4 modelMatrix;
	uniform mat4 viewMatrix;
	uniform mat4 projectionMatrix;
	
	out vec2 vertexTextureCoordOut;
	
	void main(){
	
		mat4 mvpMatrix=projectionMatrix*viewMatrix*modelMatrix;
		
		vec4 mvpPosition=mvpMatrix*vec4(vertexPosition,1.0f);
		
		vertexTextureCoordOut=vertexTextureCoord;
		
		gl_Position=mvpPosition;
	}
	\end{lstlisting}
\end{frame}
\part{Fragment Shader}
\frame{\partpage}
\part{Meshes Example}
\frame{\partpage}

\begin{frame}{Unity3D - Meshes}
	\begin{itemize}
		\item Mesh Class - \url{https://docs.unity3d.com/ScriptReference/Mesh.html}
	\end{itemize}
\end{frame}

\begin{frame}{UE4 - Meshes}
	\begin{itemize}
		\item Procedural Mesh Blueprints - \url{https://docs.unrealengine.com/en-US/BlueprintAPI/Components/ProceduralMesh/index.html}
		\url{https://www.youtube.com/watch?v=dKlMEmVgbvg}
		\item Procedural Mesh C++ -
		\url{http://wlosok.cz/procedural-mesh-in-ue4-1-triangle/}
	\end{itemize}
\end{frame}

\end{document}
