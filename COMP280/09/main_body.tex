% Adjust these for the path of the theme and its graphics, relative to this file
%\usepackage{beamerthemeFalmouthGamesAcademy}
\usepackage{../../beamerthemeFalmouthGamesAcademy}
\usepackage{multimedia}
\graphicspath{ {../../} }

% Default language for code listings
\lstset{language=C++,
        morekeywords={each,in,nullptr}
}

% For strikethrough effect
\usepackage[normalem]{ulem}
\usepackage{wasysym}

\usepackage{pdfpages}

% http://www.texample.net/tikz/examples/state-machine/
\usetikzlibrary{arrows,automata}

\newcommand{\modulecode}{COMP140 GAM160}\newcommand{\moduletitle}{Hacking Hardware/Advanced Programming}\newcommand{\sessionnumber}{Session 6}

\begin{document}
\title{\sessionnumber: The graphics pipeline}
\subtitle{\modulecode: \moduletitle}

\frame{\titlepage} 

\begin{frame}
	\frametitle{Learning outcomes}
	By the end of today's session, you will be able to:
	\begin{itemize}
		\item \textbf{Recall} the key stages of the graphics pipeline
		\item \textbf{Explain} the differences between a CPU and a GPU
		\item \textbf{Write} basic programs using SDL and OpenGL
	\end{itemize}
\end{frame}

\part{Module introduction}
\frame{\partpage}

\begin{frame}{Aim}
\begin{center}
To empower you to leverage mathematics and mathematical modelling in the design and implementation of real-time 3D worlds and simulations.
\end{center}
\end{frame}

\begin{frame}{Description}
On this module, you learn the fundamental mathematics involved in the design, development and maintenance of real-time 3D worlds and simulations. In doing so, you will leverage mathematics practically to generate and manipulate worlds and simulations relevant to a range of creative computing contexts. Indicatively, content spans topics such as linear algebra (vectors, matrices and quaternions), geometry, trigonometry, 3D transformation, collision detection, Newtonian mechanics, numerical control, calculus, and efficiency and optimisation of numerical methods.
\end{frame}

\begin{frame}{Learning Outcome}
	\begin{itemize}
		\item SOLVE
		\item Apply knowledge of algorithms, data structures, and mathematics to solve well-defined problems.
		\item Assessment criteria category: PROCESS
	\end{itemize}
\end{frame}

\begin{frame}{Topic schedule}
	\begin{center}
		On LearningSpace
	\end{center}
\end{frame}

\begin{frame}{Timetable}
	\begin{center}
		\url{http://mytimetable.falmouth.ac.uk}
	\end{center}
\end{frame}

\begin{frame}{Assignments}
	\begin{itemize}
		\pause\item Assignment 1: worksheet tasks
			\begin{itemize}
				\pause\item \textbf{Four} worksheets --- roughly 2 weeks each
			\end{itemize}
		\pause\item See LearningSpace for assignment briefs and worksheets
		\pause\item See MyFalmouth for deadlines
	\end{itemize}
\end{frame}

\begin{frame}{Worksheet A}
	\begin{itemize}
		\item B\'ezier curves
		\item Due \textbf{Monday week 4 (14th October)}
	\end{itemize}
\end{frame}


% http://tex.stackexchange.com/questions/30720/footnote-without-a-marker
\newcommand\blfootnote[1]{%
  \begingroup
  \renewcommand\thefootnote{}\footnote{#1}%
  \addtocounter{footnote}{-1}%
  \endgroup
}

\part{Graphics and simulation hardware}
\frame{\partpage}

\begin{frame}{How computers work?}
		\textit{``A computer is like a small cardboard box. Inside this box lives an elf.
		The elf obeys instructions from my program, in the order they are given.
		Some instructions tell the elf to draw pictures onto the screen.''}
	\blfootnote{\url{https://blogs.msdn.microsoft.com/shawnhar/2008/03/31/an-elf-in-a-box/}}
\end{frame}

\begin{frame}{How computers really work}
		\textit{``A computer is like a cardboard box inhabited by a pair of elves,
		Charles Pitchwife Underhill plus his younger sibling George Pekkala Underhill 
		(both more commonly known by their initials).''}
	\blfootnote{\url{https://blogs.msdn.microsoft.com/shawnhar/2008/03/31/an-elf-in-a-box/}}
\end{frame}

\begin{frame}{How computers really work (cont.)}
		\textit{``Charles is smart, well educated, and fluent in dozens of languages 
		(C, C++, C\#, and Python, to name a few).
		George, on the other hand [...]
		finds it difficult to communicate with anyone other than his brother Charles, 
		prefers to plan his day well in advance, and gets flustered if asked to change 
		activities with insufficient warning. He has an amazing ability to multiply 
		floating point numbers, especially enjoying computations that involve vectors 
		and matrices.''}
	\blfootnote{\url{https://blogs.msdn.microsoft.com/shawnhar/2008/03/31/an-elf-in-a-box/}}
\end{frame}

\begin{frame}{How computers really work (cont.)}
		\textit{``When you run a program on this computer, Charles reads it and does whatever it says. 
		Any time he encounters a graphics drawing instruction, he notes that down on a piece of paper. 
		At some later point (when the paper fills up, or if he sees a Present instruction) 
		he translates the entire paper from the original language into a secret code which 
		only he and George can understand, then hands these translated instructions to his brother, 
		who carries them out.''}
	\blfootnote{\url{https://blogs.msdn.microsoft.com/shawnhar/2008/03/31/an-elf-in-a-box/}}
\end{frame}

\begin{frame}{CPUs vs GPUs}
	\begin{itemize}
		\pause \item (CPU = central processing unit; GPU = graphics processing unit)
		\pause \item GPUs are \textbf{highly parallelised}
			\begin{itemize}
				\pause \item Intel i7 6900K: \textbf{8} cores
				\pause \item Nvidia GTX 1080: \textbf{2560} shader processors
			\end{itemize}
		\pause \item GPUs are \textbf{highly specialised}
			\begin{itemize}
				\pause \item Optimised for floating-point calculations rather than logic
				\pause \item Optimised for performing the same calculation on several thousand vertices or pixels at once
			\end{itemize}
	\end{itemize}
\end{frame}

\begin{frame}{Physics processing unit}
	\begin{columns}
		\begin{column}{0.35\textwidth}
			\includegraphics[width=\textwidth]{physx_ppu}
		\end{column}
		\begin{column}{0.63\textwidth}
			\begin{itemize}
				\pause \item Ageia PhysX PPU briefly appeared on the market in 2006-2008
				\pause \item Similar architecture to GPU (many cores, optimised for floating point calculations)
				\pause \item Ageia acquired in 2008 by Nvidia...
				\pause \item Now PhysX is Nvidia's middleware for performing physics simulation on the GPU
			\end{itemize}
		\end{column}
	\end{columns}
\end{frame}

\begin{frame}{General purpose GPU (GPGPU)}
	\begin{itemize}
		\pause \item Early GPUs used a \textbf{fixed pipeline} -- could only be used for rendering 3D graphics
		\pause \item Modern GPUs use a \textbf{programmable pipeline} -- can be programmed for other tasks
		\pause \item Physics simulation (e.g.\ PhysX)
		\pause \item Scientific computing (e.g.\ CUDA)
		\pause \item Deep learning
	\end{itemize}
\end{frame}

\begin{frame}{Graphics APIs}
	\begin{itemize}
		\pause\item Graphics APIs \textbf{abstract} away the differences between different manufacturers' GPUs
		\pause\item There are several APIs in use today:
		\begin{itemize}
			\pause\item \textbf{OpenGL}: Open standard, very mature, very widely supported
			\pause\item \textbf{Vulkan}: Open standard, very new, support still growing
			\pause\item \textbf{Direct3D}: Microsoft only
			\pause\item \textbf{Metal}: Apple only
			\pause\item Sony and Nintendo consoles have their own APIs; Microsoft consoles use Direct3D
		\end{itemize}
		\pause\item Most general-purpose game engines (e.g.\ Unity, Unreal) support several graphics APIs
		\pause\item On this module we will use \textbf{OpenGL} (but the principles are transferable)
	\end{itemize}
\end{frame}


\part{The 3D graphics pipeline}
\frame{\partpage}

\begin{frame}{The 3D graphics pipeline}
	\begin{center}
		\includegraphics[height=0.7\textheight]{pipeline}
	\end{center}
\end{frame}

\begin{frame}{Vertex processing}
	\begin{columns}
		\begin{column}{0.3\textwidth}
			\includegraphics[width=\textwidth]{pipeline_1}
		\end{column}
		\begin{column}{0.65\textwidth}
			\begin{itemize}
				\pause\item Geometry is provided to the GPU as a \textbf{mesh} of \textbf{triangles}
				\pause\item Each triangle has three \textbf{vertices} specified in 3D space $(x,y,z)$
				\pause\item Vertex processor \textbf{transforms} (rotates, moves, scales) vertices
					and \textbf{projects} them into 2D screen space $(x,y)$
				\pause\item May also apply particle simulations, skeletal animations or deformations, etc.
			\end{itemize}
		\end{column}
	\end{columns}
\end{frame}

\begin{frame}{Rasterisation}
	\begin{columns}
		\begin{column}{0.3\textwidth}
			\includegraphics[width=\textwidth]{pipeline_2}
		\end{column}
		\begin{column}{0.65\textwidth}
			\begin{itemize}
				\pause\item Determine \textbf{which fragments} are covered by the triangle
				\pause\item In practical terms, ``fragment'' = ``pixel''
				\pause\item Vertex processor can associate \textbf{data} with each vertex;
					this is \textbf{interpolated} across the fragments
			\end{itemize}
		\end{column}
	\end{columns}
\end{frame}

\begin{frame}{Fragment processing}
	\begin{columns}
		\begin{column}{0.3\textwidth}
			\includegraphics[width=\textwidth]{pipeline_3}
		\end{column}
		\begin{column}{0.65\textwidth}
			\begin{itemize}
				\pause\item Determine the \textbf{colour} of each fragment covered by the triangle
				\pause\item \textbf{Textures} are 2D images that can be \textbf{wrapped} onto a 3D object
				\pause\item Colour is calculated based on \textbf{texture}, \textbf{lighting} and other
					properties of the surface being rendered (e.g.\ shininess, roughness)
			\end{itemize}
		\end{column}
	\end{columns}
\end{frame}

\begin{frame}{Blending}
	\begin{columns}
		\begin{column}{0.3\textwidth}
			\includegraphics[width=\textwidth]{pipeline_4}
		\end{column}
		\begin{column}{0.65\textwidth}
			\begin{itemize}
				\pause\item Combine these fragments with the existing content of the image buffer
				\pause\item \textbf{Depth testing}: if the new fragment is ``in front'' of the old one, replace it;
					if it is ``behind'', discard it
				\pause\item \textbf{Alpha blending}: combine the old and new colours for a semi-transparent appearance
			\end{itemize}
		\end{column}
	\end{columns}
\end{frame}



\part{Your first OpenGL program}
\frame{\partpage}

\begin{frame}{SDL and OpenGL}
	\begin{itemize}
		\pause\item OpenGL only handles rendering of graphics
		\pause\item We need something else to handle windows, events, audio etc
		\pause\item We will use our old friend \textbf{SDL}
	\end{itemize}
\end{frame}

\begin{frame}{Live coding}
	\begin{center}
		\url{https://github.com/Falmouth-Games-Academy/bsc-live-coding}
	\end{center}
\end{frame}

\begin{frame}[fragile]{Live coding}
	\begin{itemize}
		\pause\item Start with the basic SDL application
		\pause\item Link with \texttt{opengl32.lib}
		\pause\item \lstinline{#include <gl/GL.h>}
		\pause\item Pass \lstinline{SDL_WINDOW_OPENGL} to \lstinline{SDL_CreateWindow}
		\pause\item Set some attributes:
	\end{itemize}
	\begin{lstlisting}
SDL_GL_SetAttribute(SDL_GL_CONTEXT_PROFILE_MASK, SDL_GL_CONTEXT_PROFILE_CORE);
SDL_GL_SetAttribute(SDL_GL_CONTEXT_MAJOR_VERSION, 3);
SDL_GL_SetAttribute(SDL_GL_CONTEXT_MINOR_VERSION, 2);
SDL_GL_SetAttribute(SDL_GL_DOUBLEBUFFER, 1);
	\end{lstlisting}
	\begin{itemize}
		\pause\item We don't need \lstinline{SDL_CreateRenderer} any more...
		\pause\item ... but we do need \lstinline{SDL_GLContext}
	\end{itemize}
\end{frame}

\begin{frame}[fragile]{Clearing the screen}
	\pause With SDL:
	\begin{lstlisting}
SDL_SetRenderDrawColor(renderer, 255, 128, 0, 255);
SDL_RenderClear(renderer);
SDL_RenderPresent(renderer);
	\end{lstlisting}
	
	\pause With OpenGL:
	\begin{lstlisting}
glClearColor(1.0f, 0.5f, 0.0f, 1.0f);
glClear(GL_COLOR_BUFFER_BIT);
SDL_GL_SwapWindow(window);
	\end{lstlisting}
\end{frame}

\begin{frame}{Our first triangle}
	\begin{center}
		\url{http://www.opengl-tutorial.org/beginners-tutorials/tutorial-2-the-first-triangle/}
	\end{center}
\end{frame}



\begin{frame}
	\frametitle{Debrief}
	It's the end of today's session. You are now able to:
	\begin{itemize}
		\item \textbf{Recall} the key stages of the graphics pipeline
		\item \textbf{Explain} the differences between a CPU and a GPU
		\item \textbf{Write} basic programs using SDL and OpenGL
	\end{itemize}
\end{frame}

\end{document}
