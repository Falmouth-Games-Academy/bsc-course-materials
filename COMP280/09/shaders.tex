\part{Shaders}
\frame{\partpage}

\begin{frame}{Shaders}
\begin{itemize}
	\item The Programmable units of the pipeline include
	\begin{itemize}
		\pause \item Vertex Processor
		\pause \item Tessellation Control
		\pause \item Tessellation Evaluation
		\pause \item Geometry Processor
		\pause \item Fragment Processor 
	\end{itemize}
	\pause\item Programs for these units are called \textbf{shaders}
\end{itemize}
\end{frame}

\begin{frame}{Vertex \& Fragment Shader}
\begin{itemize}
	\item These two units are a requirement for any Rendering to occur in D3D or OpenGL, the the other units are optional
	\pause\item The vertex processor and fragment processor are \textbf{programmable}
	\pause\item Programs for these units are called \textbf{shaders}
	\pause\item \textbf{Vertex shader}: responsible for geometric transformations, deformations, and projection
	\pause\item \textbf{Fragment shader}: responsible for the visual appearance of the surface
\end{itemize}
\end{frame}

\begin{frame}{Vertex Shader}
\begin{itemize}
	\item Takes in exactly one vertex as input
	\pause \item Outputs one vertex
	\pause \item Typical operations include Transformations and Animation
\end{itemize}
\end{frame}

\begin{frame}{Fragment Shader}
\begin{itemize}
	\item Takes in a pixel fragment (see rasterization)
	\pause \item Outputs colour and depth values
	\pause \item Typically used for shading calculations and texturing
\end{itemize}
\end{frame}
	
