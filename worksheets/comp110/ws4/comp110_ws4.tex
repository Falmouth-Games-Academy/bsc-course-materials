\documentclass{scrartcl}

\usepackage{amsmath}
\usepackage{varwidth}
\usepackage[dvipsnames]{xcolor}
\usepackage{algpseudocode}
\usepackage{listings}
\lstset{
	basicstyle=\footnotesize\ttfamily,
	tabsize=4,
	showstringspaces=false,
	breaklines=true,
	prebreak={\space\hbox{\textcolor{Gray}{$\hookleftarrow$}}},
	language=C++
}
\usepackage{enumitem}

\makeatletter
\@addtoreset{section}{part}
\makeatother  

\title{Worksheet 4}
\subtitle{COMP110: Principles of Computing}
\author{Ed Powley}
\date{January 2016}

\renewcommand\thepart{\Alph{part}}

\begin{document}

\maketitle

\section*{Introduction}

In this assignment, you will create three small C++ programs:
\begin{enumerate}[label=\Alph*.]
	\item A console application implementing the word guessing game Hangman;
	\item A console application implementing the 2-player strategy game Connect~4;
	\item A graphical application which generates and displays the Mandelbrot fractal.
\end{enumerate}

This worksheet tests your ability to translate various program notations (pseudocode, flowcharts,
mathematics, narrative descriptions) into C++ code.

\section*{Submission instructions}

todo

\section*{Marking}

todo

\clearpage
\part{Hangman}

\section{}
Do a thing

\section{}

The following algorithm takes the current partially revealed word, the secret word, and a guessed letter.
It returns a new partially revealed word, in which the guessed letter has been filled in where it appears in the string.

\noindent\begin{algorithmic}
	\Procedure{FillInLetter}{partialWord, secretWord, letter}
		\State $\text{result} \gets \text{empty string}$
		\For{$i = 0, 1, \dots, \text{secretWord.length} - 1$}
			\If{$\text{secretWord}[i] = \text{letter}$}
				\State append letter to result
			\Else
				\State append $\text{partialWord}[i]$ to result
			\EndIf
		\EndFor
		\State \textbf{return} result
	\EndProcedure
\end{algorithmic}

The following table gives some examples of possible input and output:

\begin{center}
\begin{tabular}{|ccc|c|}
\hline
partialWord & secretWord & letter & result \\ \hline
\lstinline{"B-----"} & \lstinline{"BANANA"} & \lstinline{'A'} & \lstinline{"BA-A-A"} \\ \hline
\lstinline{"B-----"} & \lstinline{"BANANA"} & \lstinline{'E'} & \lstinline{"B-----"} \\ \hline
\lstinline{"-----"} & \lstinline{"APPLE"} & \lstinline{'L'} & \lstinline{"---L-"} \\ \hline
\lstinline{"-----"} & \lstinline{"APPLE"} & \lstinline{'B'} & \lstinline{"-----"} \\ \hline
\end{tabular}
\end{center}

Implement the $\Call{FillInLetter}$ algorithm as a C++ function with the following signature:

\begin{lstlisting}
std::string fillInLetter(std::string partialWord,
                         std::string secretWord,
                         char letter)
\end{lstlisting}

\clearpage
\part{Connect 4}

\section{}
Do a thing

\section{}
Do another thing

\clearpage
\part{Mandelbrot}

\end{document}
