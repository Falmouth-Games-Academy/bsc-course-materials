\part{Classes Review}
\frame{\partpage}

\begin{frame}{Classes}
	\begin{itemize}
		\pause \item Let us look at Classes again
		\pause \item Classes allow us to create our own data types
		\pause \item They consist of a series of data(variables) and functions that operate on the data
		\pause \item Functions and variables inside the class can be marked with the following \textbf{access specifiers}
		\begin{itemize}
			\pause \item \textbf{Public}: Can be accessed directly
			\pause \item \textbf{Private}: Can only be accessed inside the class
			\pause \item \textbf{Protected}: Acts like private, but child classes can access
		\end{itemize}
	\end{itemize}
\end{frame}

\begin{frame}[fragile]{Class Examples}
	\begin{lstlisting}[language=C++,basicstyle=\tiny,]
	class Player
	{
	public:
		Player()
		{
			Health=100;
		};

		void TakeDamage(int health)
		{
			Health-=health;
		};

		void HealDamage(int health)
		{
			Health+=health;
		};

		~Player(){};
	private:
		int Health;
	};
\end{lstlisting}
\end{frame}


\begin{frame}{Classes vs Structs}
\begin{itemize}
	\pause \item A \textbf{Struct} is pretty much the same as a \textbf{Class}
	\pause \item The only difference in functionally, by default:
	\begin{itemize}
		\pause \item Everything in a \textbf{Class} is \textbf{private}
		\pause \item Everything in a \textbf{Struct} is \textbf{public}
	\end{itemize}
	\pause \item Difference by convention:
	\begin{itemize}
		\pause \item Structs are used for holding related data and tend not to have functions
		\pause \item Classes hold data and functions
	\end{itemize}
\end{itemize}
\end{frame}

\begin{frame}[fragile]{Creating an Instance}
	\begin{lstlisting}[language=C++,basicstyle=\tiny,]
		//Creating on the stack, this will be deleted when it drops out of scope
		Player player1=Player();
		
		//Call take damage function, notice we use . to access functions
		player.TakeDamage(20);
		
		//Creating on the Heap, please delete!!
		Player * player2=new Player();
		
		//Call take damage function, note we use -> to access functions
		player->TakeDamage(20);
		
		//Deleting player2 on the heap
		if (player2)
		{
			delete player2;
			player2=nullptr;
		}
	\end{lstlisting}
\end{frame}


\begin{frame}{Constructor \& Deconstructor}
	\begin{itemize}
		\pause \item \textbf{Constructors} are called when you create an instance
		\pause \item Constructors can take in zero or many parameters
		\pause \item You need to declare different version of the constructor
		\pause \item Deconstructors are called when the instance has been deleted (by the dropping out of scope, or explicitly deleted in C++)
		\pause \item Constructors have to be names the same as the class
		\pause \item Deconstructors have the same name as the class but prefixed with \textbf{$\sim$} (tilde symbol)
	\end{itemize}
\end{frame}

\begin{frame}[fragile]{Constructors}
	\begin{lstlisting}[language=C++,basicstyle=\tiny,]
	//Create a player
	Player * player1=new Player();
	
	//Create another player with the one parameter constructor
	Player player2=Player(10);
	
	//Create another player with the two parameter constructor
	Player * player3=new Player(100,20);
	
	delete player1;
	delete player2;
	\end{lstlisting}
\end{frame}


\begin{frame}[fragile]{Using Constructors}
	\begin{lstlisting}[language=C++,basicstyle=\tiny,]
		//Create a player with the default no parameter constructor
		Player player1=new Player();
		
		//Create a player with one parameter constructor
		Player player2=new Player(50);
		
		//Create a player with two parametes constructor
		Player player3=new Player(120,50);
	\end{lstlisting}
\end{frame}

\begin{frame}{Encapsulation}
	\begin{itemize}
		\pause \item In OOP, Encapsulation is a key principle 
		\pause \item This refers to the idea that all data in a class should be hidden by the caller
		\pause \item This means that \textbf{all} variables should be marked \textbf{private} or \textbf{protected}
		\pause \item And only functions inside the class can operate on the data
	\end{itemize}
\end{frame}

\begin{frame}[fragile]{Encapsulation Examples}
\begin{lstlisting}[language=C++,basicstyle=\tiny,]
class Player
{
	//see class above
	
	void TakeDamage(int health)
	{
		Health-=health;
		if (Health>0)
		{
			Kill()
		}
	};
	
	void Kill()
	{
		//Mark for deletion
		//remove from screen
	};
};
	\end{lstlisting}
\end{frame}

