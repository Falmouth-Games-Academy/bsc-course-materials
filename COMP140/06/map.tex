\part{Associative Array: Map \& Dictionary}
\frame{\partpage}

\begin{frame}{The Problem}
	\begin{itemize}
		\pause \item If you need to store one unique copy of an element
		\pause \item You want to access an element via a key
		\pause \item You are doing lots of searches for an element
	\end{itemize}
\end{frame}

\begin{frame}{The Solution}
	\begin{itemize}
		\pause \item You should use an Associative array
		\begin{itemize}
			\pause \item in C++ we have the \textbf{map} or \textbf{unordered\_map}  class
		\end{itemize} 
		\pause \item These data structures are structured as key-value pair
		\pause \item It allows you to retrieve the items via the key
		\pause \item This makes it a good choice for looking up large data sets
	\end{itemize}
\end{frame}

\begin{frame}{Use Case}
	\begin{itemize}
		\pause \item If you are creating a resource management system for handling textures, models or other assets
		\pause \item Localisation system, each language is stored in an Associative Array
		\pause \item Unit Manager, a class to manage units created in the game
		\pause \item Save Game System
	\end{itemize}
\end{frame}


\begin{frame}[fragile]{C++ Map
Example}
\begin{lstlisting}
Map<string,int> highScores;

highScores["Brian"]=200;
highScores["Sarah"]=2000;
highScores["Julia]=4000;

for(auto iter : highScores)
{
	std::cout<<"High Score "+iter.first<<" "<<iter.second<<std::endl;
}
\end{lstlisting}
\end{frame}

\begin{frame}[fragile]{C++ Map
	Example}
\begin{lstlisting}
auto iter=highScores.find("Brian");
if (iter!=highScores.end())
{
	int score=highScores["Brian];
}

highScores.earse("Sarah");
\end{lstlisting}
\end{frame}

\begin{frame}{Additional Notes}
\begin{itemize}
	\pause \item Iterating over a map has a slightly annoying syntax
	\pause \item Associative Arrays tend to have good performance for retrieval (O (log n))
	\pause \item If you add an item and its key already exists it may overwrite the value
\end{itemize}
\end{frame}