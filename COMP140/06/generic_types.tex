\part{Generic Types}
\frame{\partpage}

\begin{frame}{Quick Aside - Generic Programming}
	\begin{itemize}
		\pause \item Generic Programming is where you write one piece of code which operates on many different types
		\pause \item This uses a concept called Templates which act in proxy for the type
		\pause \item The Compiler then generates the code which uses the actual type
	\end{itemize}
\end{frame}

\begin{frame}{Look back at Vector/List}
	\begin{itemize}
		\pause \item In the previous section you would have noticed the following
		\begin{itemize}
			\pause \item vector\textbf{\textless}int\textbf{\textgreater}
		\end{itemize} 
		\pause \item These are know as generic parameters and you should insert the data type that the collection will handle (including your own data types aka classes and structs)
	\end{itemize}
\end{frame}

\begin{frame}{Generic Programming}
	\begin{itemize}
		\pause \item You can write your own generic classes and functions but this is beyond the scope of this class
		\pause \item C++ examples - \url{https://www.codeproject.com/Articles/257589/An-Idiots-Guide-to-Cplusplus-Templates-Part}
		\pause \item Word of warning, it is often difficult to write generic code
		\pause \item If you have errors they are often difficult to isolate as the compiler messages are so cryptic
	\end{itemize}
\end{frame}