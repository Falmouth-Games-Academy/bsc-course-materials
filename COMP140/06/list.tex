\part{Linked List}
\frame{\partpage}

\begin{frame}{The Problem}
	\begin{itemize}
		\pause \item You have started using a dynamic array and you have notice performance is poor on adding/removing
		 \pause \item You then realise that you are adding/removing elements from the middle of the collection
		 \pause \item You also realise that you don't require random access to elements in the collection
	\end{itemize}
\end{frame}

\begin{frame}{The Solution}
	\begin{itemize}
		\pause \item In this case a Linked List would be a better choice
		\begin{itemize}
			\pause \item In C++ we have the \textbf{list} class
		\end{itemize}
		\pause \item Linked Lists contain elements (called Nodes) which usually have a reference (or pointer) to the previous and next Node in the list
		\pause \item This means that there is a slight increase in memory needed when working with lists
	\end{itemize}
\end{frame}

\begin{frame}{Use Case}
	\begin{itemize}
		\pause \item If you AI character has to visit a series of waypoints, these could be stored in a list
		\pause \item Your Player has a number of quests they can try and complete
		\pause \item If the AI/Player carries an action and a number of systems need to be notified of the event 
	\end{itemize}
\end{frame}

\begin{frame}[fragile]{C++ List
Example}
\begin{lstlisting}
	list<vec2> waypoints;
	
	waypoints.push_back(vec2(10,10));
	waypoints.push_back(vec2(15,15));
	waypoints.push_back(vec2(20,20));
	
	for(vec2 position:waypoints)
	{
		std::cout<<"Waypoint Locations "<<position.x<<" "<<position.y<<std::endl;
	}
\end{lstlisting}
\end{frame}

\begin{frame}[fragile]{C++ List
	Example}
\begin{lstlisting}
	waypoints.push_front(vec2(0,0));
	
	auto iter=std::find(waypoints.begin(),waypoints.end(),vec2(15,15));
	waypoints.insert(iter, vec3(25,25));
\end{lstlisting}
\end{frame}

\begin{frame}{Additional Notes}
	\begin{itemize}
		\pause \item Linked Lists usually support constant time insertions and deletions in the collection (O(1))
		\pause \item Also perform better than dynamic arrays for moving elements around the collection
		\pause \item This feature means that Linked Lists are a good data structure if you need to sort your data
		\pause \item Main drawback of Linked Lists is that you can't have direct access to elements in the list, it takes linear time (O(n)) to access
	\end{itemize}
\end{frame}