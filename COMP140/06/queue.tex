\part{Queue}
\frame{\partpage}

\begin{frame}{The Problem}
	\begin{itemize}
		\pause \item If you need to visit items in a certain (e.g front to back)
		\pause \item Examples of this could be waypoints or commands to an AI character 
	\end{itemize}
\end{frame}

\begin{frame}{The Solution}
	\begin{itemize}
		\pause \item In this case a Queue would be a good choice
		\begin{itemize}
			\pause \item In C\# we have the \textbf{Queue} class
			\pause \item in C++ we have the \textbf{queue} class
		\end{itemize}
		\pause \item This is \textbf{F}irst-\textbf{I}n-\textbf{L}ast-\textbf{O}ut data structure
		\pause \item You add elements to the end of the queue and you remove elements from the start
	\end{itemize}
\end{frame}

\begin{frame}{Use Case}
	\begin{itemize}
		\pause \item An RTS game where you can add orders to a unit, these are then carried out sequence
		\pause \item An RTS where you have a base which produces units
		\pause \item A spawning system, where you have to defeat enemies in a specific order
	\end{itemize}
\end{frame}

\begin{frame}[fragile]{C\# Queue
Example}
\begin{lstlisting}
Queue<GameObject> unitsToBuild=new Queue<GameObject>();

unitsToBuild.Enqeue(soliderPrefab);
unitsToBuild.Enqeue(builderPrefab);
unitsToBuild.Enqeue(tankPrefab);

foreach(GameObject go in unitsToBuild)
{
	Debug.Log("Units to build "+go.name);
}

\end{lstlisting}
\end{frame}


\begin{frame}[fragile]{C\# Queue
	Example}
\begin{lstlisting}
	GameObject nextUnitToBuild=unitsToBuild.Peek();
	
	unitsToBuild.Dequeue();
\end{lstlisting}
\end{frame}

\begin{frame}[fragile]{C++ Queue
Example}
\begin{lstlisting}
queue<Command> aiCommands;

aiCommands.push(Command("Attack"));
aiCommands.push(Command("Recharge"));
aiCommands.push(Command("Run"));
\end{lstlisting}
\end{frame}

\begin{frame}[fragile]{C++ Queue
	Example}
\begin{lstlisting}
Command nextCommand=aiCommands.front();

aiCommands.pop();
\end{lstlisting}
\end{frame}