% Adjust these for the path of the theme and its graphics, relative to this file
%\usepackage{beamerthemeFalmouthGamesAcademy}
\usepackage{../../beamerthemeFalmouthGamesAcademy}
\usepackage{multimedia}
\graphicspath{ {../../} }

% Default language for code listings
\lstset{language=C++,
        morekeywords={each,in,nullptr,int32, TCHAR, uint8, int8, uint16, int16,
        uint32, int32, uint64, int64, PTRINT, UObject. AActor, SWidget, FName,
        FString, UClass, USoundCue, UTexture}
}

% For strikethrough effect
\usepackage[normalem]{ulem}
\usepackage{wasysym}
\usepackage{listings}
\usepackage{pdfpages}

% http://www.texample.net/tikz/examples/state-machine/
\usetikzlibrary{arrows,automata}

\newcommand{\modulecode}{COMP260}\newcommand{\moduletitle}{Distributed Systems}\newcommand{\sessionnumber}{5}

\begin{document}
\title{\sessionnumber: Epic Coding Standards}
\subtitle{\modulecode: \moduletitle}

\frame{\titlepage}

\begin{frame}
	\frametitle{Learning outcomes}
	\begin{itemize}
		\item \textbf{Understand} Epic coding standards
		\item \textbf{Compare} different coding standards
		\item \textbf{Apply} Epic coding standard to your own Unreal Projects
	\end{itemize}
\end{frame}

\begin{frame}
  \frametitle{Why Coding Standards?}
  \begin{itemize}
    \item Aids in software maintance
    \item Improves readability and understandability
    \item Adds to the documentation of the project
  \end{itemize}
\end{frame}

\begin{frame}
  \frametitle{Naming Conventions 1}
  \begin{itemize}
    \item First letter of each word in name is capitalised, and no underscores between words
    \begin{itemize}
      \item E.g. Health and UPrimitiveComponent, not lastMouseCoordinates or delta\_coordinates
    \end{itemize}
    \item Type and variable names are nouns
    \item Method names are verbs that describe the effects or describe return value
    \item Names should be clear unambigous, and discriptive. Avoid over-abbreviation.
  \end{itemize}
\end{frame}

\begin{frame}
  \frametitle{Naming Conventions 2}
  \begin{itemize}
    \item All variables should be declared one at a time to allow commments
    \item All functions that return bool should ask a true/false questions
    \item A procedure(a function with no return) should use strong verb followed by an Object
    \item Prefix function parameters passed by reference with \textbf{Out}
  \end{itemize}
\end{frame}

\begin{frame}
  \frametitle{Naming Conventions 3}
  \begin{itemize}
    \item Type names are prefixed with an additional upper-case letter, to distinguish them from variable names.
    E.g. \textbf{FSkin} for type, and \textbf{Skin} is an instance of \textbf{FSkin}
    \begin{itemize}
      \item Template classes are prefixed by T
      \item Classes that inherit from UObject are prefixed by U
      \item Classes that inheirt from AActor are prefixed by A
      \item Classes that inheirt from SWidget are prefixed by S
      \item Classes that are Interfaces are prexfixed by I
      \item Enums are prexfixed by E
      \item Boolean variables must be prefixed by b (e.g. bIsDead or bHasFallen)
      \item Most other classes are prefixed by F
    \end{itemize}
  \end{itemize}
\end{frame}

\begin{frame}[fragile]
  \frametitle{Naming Convention Examples}
  \begin{lstlisting}
    float TeaWeight;

    int32 TeaCount;

    bool bDoesTeaStink;

    FName TeaName;

    FString TeaFriendlyName;

    UClass* TeaClass;

    USoundCue* TeaSound;

    UTexture* TeaTexture;

    bool IsTeaFresh(UTea Tea) \{...\}
  \end{lstlisting}
\end{frame}

\begin{frame}
  \frametitle{Portable Aliases for C++ Types}
  \begin{table}
      \begin{tabular}{ | l | l | l | }
        \hline
        \textbf{Unreal Type} & \textbf{C++ Type} & \textbf{Size} \\
        \hline
        bool & bool or BOOL & Never assume size \\
        \hline
        TCHAR & TCHAR or char & Never assume size \\
        \hline
        uint8 & unsigned bytes & 1 byte \\
        \hline
        int8 & bytes & 1 byte \\
        \hline
        uint16 & unsigned short & 2 bytes \\
        \hline
        int16 & short & 2 bytes \\
        \hline
        uint32 & unsigned int & 4 bytes \\
        \hline
        int32 & int & 4 bytes \\
        \hline
        uint64 & unsigned long & 8 bytes \\
        \hline
        int64 & long & 8 bytes \\
        \hline
        float & float & 4 bytes \\
        \hline
        double & double & 8 bytes \\
        \hline
        PTRINT & void* & Never assume size \\
        \hline
      \end{tabular}
    \end{table}
\end{frame}
\end{document}
