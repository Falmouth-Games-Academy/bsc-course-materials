\part{Representing numbers}
\frame{\partpage}

\begin{frame}{Powers of 10}
	\pause
	$$ 10^6 = 1\underbrace{000000}_{\text{6 zeroes}} $$
	\pause
	$$ 10^1 = 10 $$
	\pause
	$$ 10^0 = 1 $$
	\pause
	$$ 10^{-1} = 0.1 $$
	\pause
	$$ 10^{-6} = 0.\underbrace{00000}_{\text{5 zeroes}}1 $$
\end{frame}

\begin{frame}{Scientific notation}
	\begin{itemize}
		\pause\item A way of writing \textbf{very large} and \textbf{very small} numbers
		\pause\item $a \times 10^b$, where
			\begin{itemize}
				\pause\item $a$ ($1 \leq |a| < 10$) is the \textbf{mantissa}
				\pause\item ($a$ is a positive or negative number
					with a single non-zero digit before the decimal point)
				\pause\item $b$ (an integer) is the \textbf{exponent}
			\end{itemize}
		\pause\item E.g.\ 1 light year = $9.461 \times 10^{15}$ metres
		\pause\item E.g.\ Planck's constant = $6.626 \times 10^{-34}$ joules
		\pause\item Socrative \texttt{FALCOMPED}
	\end{itemize}
\end{frame}

\begin{frame}[fragile]{Scientific notation in C++}
	\pause Instead of writing $\times 10$, write \lstinline{e}
	\pause
	\begin{lstlisting}
double lightYear = 9.461e15;
double plancksConstant = 6.626e-34;
	\end{lstlisting}
	\pause This also works in Python and many other programming languages
\end{frame}

\begin{frame}{IEEE floating point format}
\end{frame}
