\part{Modular arithmetic}
\frame{\partpage}

\begin{frame}{Remainders}
	\begin{itemize}
		\pause\item Primary school maths!
		\pause\item Dividing one whole number by another gives a \textbf{quotient} and a \textbf{remainder}
		\pause\item E.g.\ dividing 20 by 3 gives quotient 6, remainder 2
		\pause\item This is because $20 = 6 \times 3 + 2$
	\end{itemize}
\end{frame}

\begin{frame}{Congruence}
	\begin{itemize}
		\pause\item Two numbers $a$ and $b$ are \textbf{congruent modulo $n$} if dividing by $n$ leaves the same remainder
		\pause\item We write $$a \equiv b \mod n$$
		\pause\item E.g.\ $20 \div 3$ leaves a remainder of $2$, and so does $14 \div 3$, so
			$$20 \equiv 14 \mod 3$$
		\pause\item E.g.\ $20 \div 5$ leaves a remainder of $0$, but $14 \div 5$ leaves a remainder of $4$, so
			$$20 \not\equiv 14 \mod 5$$
	\end{itemize}
\end{frame}
