\part{MIPS machine code}
\frame{\partpage}

\begin{frame}{MIPS instructions}
	\begin{itemize}
		\pause\item Each line of MIPS assembly code can be translated into a \textbf{machine code instruction}
		\pause\item 1 line of assembly = 1 instruction
		\pause\item Each instruction is a \textbf{32 bit} value
		\pause\item First 6 bits specify the \textbf{opcode}; how the remaining 26 bits are interpreted depends on which opcode it is
	\end{itemize}
\end{frame}

\begin{frame}{Anatomy of an instruction}
	\pause \textbf{R-type} instruction:
	\setlength{\tabcolsep}{4.8pt}
	\begin{tabular}{|*{32}{c|}}
		\hline &&&&&&&&&&&&&&&&&&&&&&&&&&&&&&& \\\hline
		\multicolumn{6}{|c|}{opcode} &
		\multicolumn{5}{|c|}{\lstinline{$s}} &
		\multicolumn{5}{|c|}{\lstinline{$t}} &
		\multicolumn{5}{|c|}{\lstinline{$d}} &
		\multicolumn{5}{|c|}{shift} &
		\multicolumn{6}{|c|}{function} \\
		\multicolumn{6}{|c|}{6 bits} &
		\multicolumn{5}{|c|}{5 bits} &
		\multicolumn{5}{|c|}{5 bits} &
		\multicolumn{5}{|c|}{5 bits} &
		\multicolumn{5}{|c|}{5 bits} &
		\multicolumn{6}{|c|}{6 bits} \\\hline
	\end{tabular}
	\begin{itemize}
		\pause\item \textbf{opcode} and \textbf{function} together specify the operation to execute
			\begin{itemize}
				\pause\item E.g.\ \lstinline{ADD} has opcode \texttt{000000} and function \texttt{100000}
				\pause\item E.g.\ \lstinline{SUB} has opcode \texttt{000000} and function \texttt{100010}
			\end{itemize}
		\pause\item Some instructions specify a \textbf{shift} amount
			\begin{itemize}
				\pause\item For \lstinline{ADD} \lstinline{SUB} etc these 5 bits are ignored
			\end{itemize}
		\pause\item Registers are identified by a 5-bit number
			\begin{itemize}
				\pause\item E.g.\ \lstinline{$zero} $\to$ \texttt{00000}, \lstinline{$s0} $\to$ \texttt{01000}, \lstinline{$s1} $\to$ \texttt{01001}
				\pause\item There are 32 registers
			\end{itemize}
	\end{itemize}
\end{frame}

\begin{frame}[fragile]{Example}
	\begin{lstlisting}
ADD $s0, $s0, $s1
	\end{lstlisting}
	$$ \downarrow $$
	\begin{lstlisting}
opcode   s     t     d   shift function
000000 01000 01001 01001 00000 100000
	\end{lstlisting}
\end{frame}

\begin{frame}{Anatomy of an instruction}
	\pause \textbf{I-type} instruction:
	\setlength{\tabcolsep}{4.8pt}
	\begin{tabular}{|*{32}{c|}}
		\hline &&&&&&&&&&&&&&&&&&&&&&&&&&&&&&& \\\hline
		\multicolumn{6}{|c|}{opcode} &
		\multicolumn{5}{|c|}{\lstinline{$s}} &
		\multicolumn{5}{|c|}{\lstinline{$t}} &
		\multicolumn{16}{|c|}{\lstinline{C}} \\
		\multicolumn{6}{|c|}{6 bits} &
		\multicolumn{5}{|c|}{5 bits} &
		\multicolumn{5}{|c|}{5 bits} &
		\multicolumn{16}{|c|}{16 bits} \\\hline
	\end{tabular}
	\begin{itemize}
		\pause\item \textbf{opcode} specifies the operation to execute
			\begin{itemize}
				\pause\item E.g.\ \lstinline{ADDI} has opcode \texttt{001000}
			\end{itemize}
		\pause\item \lstinline{C} is specified as a 16-bit number
	\end{itemize}
\end{frame}

\begin{frame}[fragile]{Example}
	\begin{lstlisting}
ADDI $s0, $s1, 123
	\end{lstlisting}
	$$ \downarrow $$
	\begin{lstlisting}
opcode   s     t           C
001000 01001 01000 0000000001111011
	\end{lstlisting}
\end{frame}

\begin{frame}{Anatomy of an instruction}
	\pause \textbf{J-type} instruction:
	\setlength{\tabcolsep}{4.8pt}
	\begin{tabular}{|*{32}{c|}}
		\hline &&&&&&&&&&&&&&&&&&&&&&&&&&&&&&& \\\hline
		\multicolumn{6}{|c|}{opcode} &
		\multicolumn{26}{|c|}{address} \\
		\multicolumn{6}{|c|}{6 bits} &
		\multicolumn{26}{|c|}{26 bits} \\\hline
	\end{tabular}
	\begin{itemize}
		\pause\item \textbf{opcode} specifies the operation to execute
			\begin{itemize}
				\pause\item E.g.\ \lstinline{J} has opcode \texttt{000010}
			\end{itemize}
		\pause\item \textbf{address} is specified as a 26-bit number
	\end{itemize}
\end{frame}

