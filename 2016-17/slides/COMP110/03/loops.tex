\newcommand{\socrative}{
	\begin{center}
		Socrative room code: \texttt{FALCOMPED}
	\end{center}
}

\newcommand{\codeslide}[2]{
	\begin{columns}
		\begin{column}{0.48\textwidth}
			\lstinputlisting{#1}
		\end{column}
		\pause
		\begin{column}{0.48\textwidth}
			\begin{center}
				\colorbox{white}{
					\color{black}
					\begin{tabular}{|c|c|}
						\hline
						\textbf{Variable} & \textbf{Value} \\\hline
						#2
					\end{tabular}
				}
			\end{center}
		\end{column}
	\end{columns}
}

\newcommand{\trow}[1]{ & \\ \texttt{#1} & \\ & \\\hline}

\part{Loops (from last time)}
\frame{\partpage}

\begin{frame}{For loops and ranges}
	\lstinputlisting{for0.py}
	\begin{itemize}
		\pause\item \lstinline{xrange(n)} is the \textbf{sequence}
			$0, 1, 2, \dots, n-1$
		\pause\item So \lstinline{xrange(5)} is the \textbf{sequence}
			$0, 1, 2, 3, 4$
		\pause\item Note: \lstinline{xrange(n)} \textbf{does not include} $n$
		\pause\item The \lstinline{for} loop iterates through the items in a sequence \textbf{in order}
		\pause\item Can also use \lstinline{range} instead of \lstinline{xrange},
			but \lstinline{range} is less efficient
			\begin{itemize}
				\item Homework (advanced): what is the difference between \lstinline{range} and \lstinline{xrange}?
			\end{itemize}
	\end{itemize}
\end{frame}

\begin{frame}{For loops (1)}
	\socrative
	\codeslide{for1.py}{\trow{a}\trow{b}\trow{i}}
\end{frame}

\begin{frame}{For loops (2)}
	\socrative
	\codeslide{for2.py}{\trow{a}\trow{b}\trow{i}}
\end{frame}

\begin{frame}{More ranges}
	\begin{itemize}
		\pause\item Can optionally specify \textbf{start point}
		\pause\item \lstinline{xrange(3, 10)} $\to [3, 4, 5, 6, 7, 8, 9]$
		\pause\item If start point is specified, can optionally specify \textbf{step}
		\pause\item \lstinline{xrange(0, 20, 2)} $\to [0, 2, 4, 6, 8, 10, 12, 14, 16, 18]$
		\pause\item Step can be negative:
		\pause\item \lstinline{xrange(10, 0, -1)} $\to [10, 9, 8, 7, 6, 5, 4, 3, 2, 1]$
	\end{itemize}
\end{frame}

\begin{frame}{While loops}
	\socrative
	
	The \lstinline{while} loop keeps executing while the condition is \textbf{true}
	
	\codeslide{while1.py}{\trow{a}}
\end{frame}

\begin{frame}{Looping forever}
	\lstinputlisting{while2.py}
\end{frame}

\begin{frame}{Summary}
	\pause We have seen some basic code constructions in Python
	\begin{itemize}
		\pause\item \lstinline{print} and \lstinline{raw_input} for command-line input and output
		\pause\item Variable assignment using \lstinline{=}
		\pause\item \lstinline{if} statements for choosing whether or not to execute a block of code
		\pause\item \lstinline{for} loops to execute a block of code a specified number of times
		\pause\item \lstinline{while} loops to execute a block of code until a condition is no longer true
	\end{itemize}
	\pause These are enough to write some simple programs, but you will see several more in coming weeks...
\end{frame}

% Show PyCharm bouncing ball example? -- think about this between now and Michael's session next Wednesday
