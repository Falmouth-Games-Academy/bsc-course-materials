\part{Pseudocode}
\frame{\partpage}

\begin{frame}{Pseudocode}
	\pause Flowcharts are useful, but...
	\begin{itemize}
		\pause\item Can be time-consuming to draw
		\pause\item Do not reflect structured programming concepts well
	\end{itemize}
	\pause \textbf{Pseudocode} expresses an algorithm in a way that looks more like a structured program
\end{frame}

\begin{frame}{Pseudocode example}
	\begin{algorithmic}
		\State \textbf{print} ``How old are you?''
		\State \textbf{read} $age$
		\If{$age < 13$}
			\State \textbf{print} ``You are a child''
		\ElsIf{$age < 18$}
			\State \textbf{print} ``You are a teenager''
		\Else
			\State \textbf{print} ``You are an adult''
		\EndIf
	\end{algorithmic}
\end{frame}

\begin{frame}{Pseudocode example}
	\begin{columns}
		\begin{column}{0.6\textwidth}
			\begin{algorithmic}
				\State $sum \gets 0$ \Comment{initialisation}
				\For{$i$ in $1, \dots, 9$}
					\State $sum \gets sum + i$
				\EndFor
				\State \textbf{print} $sum$ \Comment{print the result}
			\end{algorithmic}
		\end{column}
	\end{columns}
\end{frame}

\begin{frame}{Pseudocode example}
	\begin{columns}
		\begin{column}{0.6\textwidth}
			\begin{algorithmic}
				\State $a \gets 1$ \Comment{initialisation}
				\While{$a < 100$}
					\State $a \gets a \times 2$
				\EndWhile
				\State \textbf{print} $a$ \Comment{print the result}
			\end{algorithmic}
		\end{column}
	\end{columns}
\end{frame}

\begin{frame}{Formatting pseudocode}
	\begin{itemize}
		\pause\item Pseudocode is a \textbf{communication tool}, not a \textbf{programming language}
		\pause\item Important: \textbf{clear}, \textbf{concise}, \textbf{unambiguous}, \textbf{consistent}
		\pause\item \textbf{Not} important: adhering to a strict set of style guidelines,
			ensuring direct translatability to your chosen programming language
	\end{itemize}
\end{frame}

\begin{frame}{Level of abstraction}
	\pause Whether working with flowcharts or pseudocode, choose your \textbf{level of abstraction} carefully
\end{frame}

\begin{frame}{Level of abstraction: Good}
	\begin{algorithmic}
		\State Fill kettle
		\State Turn kettle on
		\State Put instant coffee in mug
		\If{sugar wanted}
			\State Add sugar
		\EndIf
		\State Wait for kettle to boil
		\If{milk wanted}
			\State Pour water to $\frac45$ full
			\State Add milk
		\Else
			\State Fill mug with water
		\EndIf
		\State Stir
	\end{algorithmic}
\end{frame}

\begin{frame}{Level of abstraction: Not so good}
	\begin{algorithmic}
		\State Position kettle beneath tap
		\State Turn tap on
		\While{water is below halfway point}
			\State Wait
		\EndWhile
		\State Turn tap off
		\State Place kettle on base
		\State Press power button
		\State ...
	\end{algorithmic}
\end{frame}

\begin{frame}{Level of abstraction: Silly}
	\begin{algorithmic}
		\State Place right palm on kettle handle
		\State Bend fingers on right hand
		\State Lift arm upwards
		\While{tap spout is not directly above kettle}
			\State Move arm to the right
		\EndWhile
		\State Place left palm on tap handle
		\State Bend fingers on left hand
		\State Rotate left hand
		\State ...
	\end{algorithmic}
\end{frame}

\begin{frame}{Level of abstraction: also silly}
	\begin{algorithmic}
		\State Make a cup of coffee
	\end{algorithmic}
\end{frame}

\begin{frame}{Activity}
	A number guessing game: The computer chooses a number between 1 and 20 at random.
	The player guesses a number.
	The computer says whether the guessed number is ``too high'', ``too low'' or ``correct''.
	The game ends when the correct number is guessed, or after 5 incorrect guesses.

	\begin{itemize}
		\item In \textbf{groups of 2-3}
		\item \textbf{Write} pseudocode for the number guessing game
		\item Write your pseudocode with \textbf{pen and paper} or using your favourite \textbf{text editor or word processor}
	\end{itemize}
\end{frame}

