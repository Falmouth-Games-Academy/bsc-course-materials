% Adjust these for the path of the theme and its graphics, relative to this file
%\usepackage{beamerthemeFalmouthGamesAcademy}
\usepackage{../../beamerthemeFalmouthGamesAcademy}
\usepackage{multimedia}
\graphicspath{ {../../} }

% Default language for code listings
\lstset{language=C++,
        morekeywords={each,in,nullptr}
}

% For strikethrough effect
\usepackage[normalem]{ulem}
\usepackage{wasysym}

\usepackage{pdfpages}

% http://www.texample.net/tikz/examples/state-machine/
\usetikzlibrary{arrows,automata}

\newcommand{\modulecode}{COMP140 GAM160}\newcommand{\moduletitle}{Hacking Hardware/Advanced Programming}\newcommand{\sessionnumber}{Session 6}

\begin{document}
\title{\sessionnumber: Text-based Adventure}
\subtitle{\modulecode: \moduletitle}

\frame{\titlepage} 

\begin{frame}
	\frametitle{Workshop}
	In the morning session you will:
	
	\begin{itemize}
		\item \textbf{Self-organise} into \textbf{THREE} groups of approximately equal size.
		\item \textbf{Design} a short text-based adventure game with the following themes:
		\begin{itemize}
			\item Group wearing the \textit{most red}: ``Iron Bull''.
			\item Group wearing the \textit{most green}: ``Ghostly Spymaster''.
			\item Remaining group: ``Atoning Monk''.
			\item Note the checking order: red $\rightarrow$ green
		\end{itemize}
		\item \textbf{Implement} a paper prototype as appropriate.
	\end{itemize}
\end{frame}

\begin{frame}
	\frametitle{Workshop}
	After the mid-session break you will:
	
	\begin{itemize}
		\item \textbf{Develop} a working design document to flesh-out details not in the prototype. 
		\item For more information, review the material here:
		\begin{itemize}
			\item \url{http://www.gamasutra.com/view/feature/131791/the_anatomy_of_a_design_document_.php}
			\item \url{http://www.gamasutra.com/view/feature/130127/design_document_play_with_fire.php}
		\end{itemize}
	\end{itemize}
\end{frame}

\end{document}
