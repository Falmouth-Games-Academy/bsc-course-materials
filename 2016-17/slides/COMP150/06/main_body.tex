% Adjust these for the path of the theme and its graphics, relative to this file
%\usepackage{beamerthemeFalmouthGamesAcademy}
\usepackage{../../beamerthemeFalmouthGamesAcademy}
\usepackage{multimedia}
\graphicspath{ {../../} }

% Default language for code listings
\lstset{language=C++,
        morekeywords={each,in,nullptr}
}

% For strikethrough effect
\usepackage[normalem]{ulem}
\usepackage{wasysym}

\usepackage{pdfpages}

% http://www.texample.net/tikz/examples/state-machine/
\usetikzlibrary{arrows,automata}

\newcommand{\fullbleed}[1]{
\begin{frame}[plain]
	\begin{tikzpicture}[remember picture, overlay]
		\node[at=(current page.center)] {
			\includegraphics[width=\paperwidth]{#1}
		};
	\end{tikzpicture}
\end{frame}
}

\newcommand{\picturepage}[2]{
\begin{frame}[plain]
	\begin{tikzpicture}[remember picture, overlay]
		\node[at=(current page.center)] {
			\includegraphics[width=\paperwidth]{#1}
		};
		\draw<1>[draw=none, fill=black, opacity=0.9] (-1,-5.2) rectangle (current page.south east);
		\node[draw=none,text width=0.96\paperwidth, align=right] at (5.5,-5.5) {\tiny{#2}};
	\end{tikzpicture}
\end{frame}
}

\newcommand{\notepicx}[5]{
\begin{frame}[plain]
	\begin{tikzpicture}[remember picture, overlay]
		\node[at=(current page.center)] {
			\includegraphics[width=\paperwidth]{#1}
		};
		\node[draw=none, fill=black, text width=#5\paperwidth] at ([xshift=#3, yshift=#4] current page.center) {\small{#2}};
	\end{tikzpicture}
\end{frame}
}

\newcommand{\notepic}[4]{
	\notepicx{#1}{#2}{#3}{#4}{0.4}
}

\newcommand{\modulecode}{COMP260}\newcommand{\moduletitle}{Distributed Systems}\newcommand{\sessionnumber}{5}

\begin{document}
\title{\sessionnumber: Scholarly Writing}
\subtitle{\modulecode: \moduletitle}

\frame{\titlepage} 

\begin{frame}
	\frametitle{Learning Outcomes}
	\begin{itemize}
		\item \textbf{Compare} the key features of academic discussions \textbf{and} arguments
		\item \textbf{Explain how} to structure a scholarly paper
		\item \textbf{Use} LaTeX to \textbf{write} a scholarly paper
		\item \textbf{Reference} sources using BiBTeX
	\end{itemize}
\end{frame}

\begin{frame}
	\frametitle{Academic Discussion}
	
	In your pre-production development teams:
	
	\vspace{2em}
	
	\textbf{Discuss} what a `discussion' is, and what it does, in an academic context 
	
	\begin{itemize}
		\item Discuss
		\item Make Notes in Slack
		\item 10 minutes
	\end{itemize}
\end{frame}

\begin{frame}
	\frametitle{Academic Discussion}
	
	The core of academic discussion is `evidence-based argument of a non-obvious position'' with the following goals:
		
	\begin{itemize}	
		\item Present new ideas and insights
		\item Argue that ideas and insights are likely to be true (or, at least supported by credible evidence)
		\item Defend the ideas and insights from likely criticisms
		\item Propose useful applications of the idea and/or how the idea could be further developed
	\end{itemize}
\end{frame}

\begin{frame}
	\frametitle{Academic Discussion}
	
	All papers should \textit{argue} something.
	
	\vspace{2em}
	
	Construct some sort of judgement, and then to be able to construct an effective, convincing argument to defend it. This is the essence of academic writing.		
\end{frame}

\begin{frame}
	\frametitle{Academic Discussion}
	
	A common criticism of student papers is that they are \textit{descriptive} rather than \textit{argumentative}.
	
	\begin{itemize}	
		\item There must be evidence of original insight through analysis:
		\begin{itemize}	
			\item Compare and Contrast
			\item Synthesis and Inference
		\end{itemize}	
		\item There must be a purpose to the argument:
		\begin{itemize}	
			\item Argument is centred on a question
			\item The paper has a key take-away point that justifies its existence
			\item The question is actually answered, and the answer defended
		\end{itemize}	
		\item Avoid describing things the reader is likely to already know
	\end{itemize}		
\end{frame}

\begin{frame}
	\frametitle{Academic Discussion}
		
	\begin{itemize}	
		\item Relevant
		\item Manageable in terms of research and other practical considerations
		\item Specific, yet sustainable
		\item Original, novel, and useful
		\item Consistent with Requirements
		\item Clear and Simple
		\item Interesting
	\end{itemize}
\end{frame}

\begin{frame}
	\frametitle{Relevance}
		
	The question will be of academic and intellectual interest to people in the field you have chosen to study. 
	The question may arise from curiosities about, or issues raised in, the literature, or queries about practice.

	\vspace{2em}

	You should be able to establish a clear purpose for your research in relation to the chosen field. 
	For example, are you filling a gap in knowledge, analysing academic assumptions or professional practice, 
	monitoring a development in practice, comparing different approaches or testing theories within a specific population?

\end{frame}

\begin{frame}
	\frametitle{Management}
		
	You need to be realistic about the scope and scale of the project. The question you ask must be within your ability to tackle. 
	Can this data be accessed within the limited time and resources you have available to you?
	
	\vspace{2em}
	
	Sometimes a research question appears feasible, but when you start your fieldwork or library study, it proves otherwise. 
	In this situation, it is important to write up the problems honestly and to reflect on what has been learnt.

\end{frame}

\begin{frame}
	\frametitle{Original}
		
	The question should not simply copy others.
	It should show your own imagination and your ability to construct and develop research issues. 
	
	\vspace{2em}
	
	The best insights cause pause and are derived by the author, rather than merely copied from others.
\end{frame}

\begin{frame}
	\frametitle{Specific, yet Sustainable}
		
	The question should be very specific, but should be able to sustain discussion.
	It should show your insight into one key aspects of the field. However, it should also show your ability to construct and develop insights about that field. 
	There needs to be sufficient scope to develop into a research project in the future.

	\vspace{2em}

	\textit{Depth} over \textbf{breadth}.
\end{frame}

\begin{frame}
	\frametitle{Consistent with Requirements}
		
	The question must allow you the scope to satisfy the learning outcomes of the course.
	
	\vspace{2em}
	
	For example, in this module you should conduct a theoretical study and/or literature review, one that does not contain analysis of empirical data. 
	You should conduct an appropriate review of the academic literature and show how you have explored theory and reasoned analytically to produce new insights about the subject.
\end{frame}

\begin{frame}
	\frametitle{Clarity}
		
	If you create a clear and simple research question, you may find that it becomes more complex as you think about the situation you are studying and undertake the literature review. 
	
	\vspace{2em}
	
	Having one key question with several sub-components will guide your research here.
\end{frame}

\begin{frame}
	\frametitle{Interesting}
		
	The question needs to intrigue you and maintain your interest throughout the project. There are two traps to avoid:
	
	\begin{itemize}
		\item Some questions are\textit{convenient} --- the best you can come up with when asked to state a question or, perhaps, the question fits in with the assessment so you decide it will suffice.
		\item Some questions are \textit{fads} --- they arise out of a set of personal circumstances, for example a job application. Once the circumstances change you may lose interest for the topic and it becomes very tedious.
	\end{itemize}
\end{frame}

\begin{frame}
	\frametitle{Essay Writing}
	
	``Writing  an essay does not simply ‘happen’ on a particular day. Effectively, you start the writing process as soon as you begin to study the topic of your next essay.''

	--- Northedge 2005: 297
\end{frame}

\fullbleed{writing_stages}

\begin{frame}
	\frametitle{Essay Structure}
	
	\begin{itemize}
		\item A clear structure is evident from a clear question and a focused argument
		\item With a constrained word count, you must focus on a very specific question
		\item \textit{Depth} is needed over \textbf{breadth}:
		\begin{itemize}
			\item There must be enough evidence to convince
			\item The question must be answerable	
		\end{itemize}
	\end{itemize}
\end{frame}

\begin{frame}
	\frametitle{Essay Structure}
	
	There are many parallels between academic writing and storytelling:
	
	\begin{itemize}
		\item Similar rules of writing will apply (though formal)
		\item Beginning, middle, and end:
		\begin{itemize}
			\item Who is the story about? Who are the characters? What do you need to understand to follow the story? What is the larger challenge that is being addressed?
			\item What actions are taken to address the challenge? What do the characters do?
			\item What have the characters accomplished? How have the characters and their world changed as a result of the action?
		\end{itemize}
	\end{itemize}
\end{frame}

\begin{frame}
	\frametitle{Essay Structure}
	
	Schimel, J. (2012) presents an interesting overview of the structures that can be found in academic papers:
	
	\begin{itemize}
		\item OCAR Structure
		\item Opening, Challenge, Action, Resolution

		\item ABDCE Structure
		\item Action, Background, Development, Climax, Ending
		
		\item LDR Structure
		\item Lead, Development, Review
	\end{itemize}
\end{frame}

\begin{frame}
	\frametitle{Essay Structure}
	
	Focus on OCAR!

\end{frame}

\fullbleed{ocar}

\fullbleed{ocar_flow}

\begin{frame}
	\frametitle{Essay Structure}
	
	\begin{itemize}
		\item Nobody uses OCAR as explicit headings in essays
		\item It is a general framework --- use your own headings!
		\item It is broadly compatible with other styles of academic writing
		\item i.e., IMRAD as used in primary research
	\end{itemize}
\end{frame}

\begin{frame}
	\frametitle{LaTeX and BiBTeX}
	
	In pairs.
	
	\vspace{2em}
	
	Complete the tutorials here:
	\url{https://www.latex-tutorial.com/tutorials/}
	
	\vspace{2em}
	
	30 minutes.	
	
\end{frame}

\begin{frame}
	\frametitle{LaTeX and BiBTeX}
	
	In pairs.
	
	\vspace{2em}
	
	Complete the tutorials here:
	\url{https://www.latex-tutorial.com/tutorials/beginners/latex-bibtex/}
	
	\vspace{2em}
	
	30 minutes.	
	
\end{frame}

\begin{frame}
	\frametitle{Proposal}
	
	Synch your fork:
	\url{https://github.com/Falmouth-Games-Academy/comp150-agile}
	
	\vspace{2em}
	
	For detailed instructions, see:
	\url{https://help.github.com/articles/syncing-a-fork/}
	
	\vspace{2em}
	
	Add your proposal and references. Then, start your essay.
	
	\vspace{2em}
	
	30 minutes.	
	
\end{frame}

\end{document}
