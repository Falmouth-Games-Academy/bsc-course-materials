% Adjust these for the path of the theme and its graphics, relative to this file
%\usepackage{beamerthemeFalmouthGamesAcademy}
\usepackage{../../beamerthemeFalmouthGamesAcademy}
\usepackage{multimedia}
\graphicspath{ {../../} }

% Default language for code listings
\lstset{language=C++,
        morekeywords={each,in,nullptr,int32, TCHAR, uint8, int8, uint16, int16,
        uint32, int32, uint64, int64, PTRINT, UObject. AActor, SWidget, FName,
        FString, UClass, USoundCue, UTexture}
}

% For strikethrough effect
\usepackage[normalem]{ulem}
\usepackage{wasysym}
\usepackage{listings}
\usepackage{pdfpages}

% http://www.texample.net/tikz/examples/state-machine/
\usetikzlibrary{arrows,automata}

\newcommand{\modulecode}{COMP140 GAM160}\newcommand{\moduletitle}{Hacking Hardware/Advanced Programming}\newcommand{\sessionnumber}{Session 6}

\begin{document}
\title{\sessionnumber: Memory and Profiling}
\subtitle{\modulecode: \moduletitle}

\frame{\titlepage}

\begin{frame}
	\frametitle{Learning outcomes}
	\begin{itemize}
		\item \textbf{Understand} Memory in modern object orientated languages
		\item \textbf{Compare} memory models in managed and unmanaged languages
		\item \textbf{Understand} the role of the profiler in tunning performance in games
	\end{itemize}
\end{frame}

\begin{frame}
  \frametitle{Memory}
  \begin{itemize}
    \item Dynamic memory, allocated on the Heap and is growable
    \item Static memory, allocated on the Stack and is fixed size
  \end{itemize}
\end{frame}

\begin{frame}
  \frametitle{Heap Memory}
\end{frame}

\begin{frame}
  \frametitle{Stack Memory}
\end{frame}

\begin{frame}
  \frametitle{Types - Value, Reference & Pointers}
\end{frame}

\begin{frame}
  \frametitle{Strings}
\end{frame}

\begin{frame}
  \frametitle{Memory Management}
\end{frame}

\begin{frame}
  \frametitle{Profiling}
\end{frame}

\begin{frame}
  \frametitle{Calling Unmanaged code}
\end{frame}

\end{document}
