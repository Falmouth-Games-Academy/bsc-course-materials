% Adjust these for the path of the theme and its graphics, relative to this file
%\usepackage{beamerthemeFalmouthGamesAcademy}
\usepackage{../../beamerthemeFalmouthGamesAcademy}
\usepackage{multimedia}
\graphicspath{ {../../} }

% Default language for code listings
\lstset{language=C++,
        morekeywords={each,in,nullptr,int32, TCHAR, uint8, int8, uint16, int16,
        uint32, int32, uint64, int64, PTRINT, UObject. AActor, SWidget, FName,
        FString, UClass, USoundCue, UTexture},
        breaklines=true,
        basicstyle=\tiny
}

% For strikethrough effect
\usepackage[normalem]{ulem}
\usepackage{wasysym}
\usepackage{listings}
\usepackage{pdfpages}

% http://www.texample.net/tikz/examples/state-machine/
\usetikzlibrary{arrows,automata}

\newcommand{\modulecode}{COMP140 GAM160}\newcommand{\moduletitle}{Hacking Hardware/Advanced Programming}\newcommand{\sessionnumber}{Session 6}

\begin{document}
\title{\sessionnumber: Streams \& Serialization}
\subtitle{\modulecode: \moduletitle}

\frame{\titlepage}

\begin{frame}
	\frametitle{Learning outcomes}
	\begin{itemize}
		\item \textbf{Understand} the concept of serialization in Computer Science
		\item \textbf{Explain} how streams can be used to send data to a location
		\item \textbf{Implement} a save system for games
	\end{itemize}
\end{frame}

\begin{frame}
	\frametitle{Working with the Filesystem}
	\begin{itemize}
		\item Typically in a game we don't interact with the filesystem
		\item We work with some sort of virtual filesystem which is local to the game
		\item This is because
		\begin{itemize}
				\item We don't have access to the users root directory
				\item We want to limit cheating
				\item Game Engines usually access file resources in a crossplatform manner
		\end{itemize}
	\end{itemize}
\end{frame}

\begin{frame}
	\frametitle{The Filesystem \& Game Engines}
	\begin(itemize)
  \item Often Game Engines use a compressed format to store assets
  \item This aids in performance but also adds in a layer of security
  \item This means that if you want to load assets programmatically you can't use standard file reading functions
  \item See \textbff{Resource.Load} in Unity or \textbff{Referencing Assets} in Unreal
	\end{itemize)
\end{frame}

\begin{frame}
	\frametitle{Typical Filesytem operations}
  \begin{itemize}
    \item In most modern operating system you can't access the local filesystem
    \item You can only use certain directories such as the Users Document or AppData directory
    \item See \textbff{Application.*Path} in Unity \textbff{FPath} in Unreal
    \item These sort of functions/variables will give us access to safe directories to use
    \item We then use other functions to create directories and/or files
    \item See \textbff{File} \& \textbff{Directory} class in C# or \textbff{File Management} for Unreal
  \end{itemize}
\end{frame}

\begin{frame}
	\frametitle{Serialization}
  \begin{itemize}
    \item Is the process of converting a data structure to a series of bytes
    \item This can be used to transmit the data structure to a file, over the network or store in memory
    \item We can also take serialized data and reconstruct a data structure
    \item This makes serialization the perfect candidate for saving game data
  \end{itemize}
\end{frame}

\begin{frame}
	\frametitle{Streams}
  \begin{itemize}
    \item Are just generic series of bytes that are sent to a resource
    \item This resource could be memory, a file or network
    \item Often used in conjunction with Serialization to save data
    \item See \textbff{IO Streams} in C# or \textbff{FBufferArchive} in Unreal
  \end{itemize}
\end{frame}

\end{document}
