% Adjust these for the path of the theme and its graphics, relative to this file
%\usepackage{beamerthemeFalmouthGamesAcademy}
\usepackage{../../beamerthemeFalmouthGamesAcademy}
\usepackage{multimedia}
\graphicspath{ {../../} }

% Default language for code listings
\lstset{language=C++,
        morekeywords={each,in,nullptr,int32, TCHAR, uint8, int8, uint16, int16,
        uint32, int32, uint64, int64, PTRINT, UObject. AActor, SWidget, FName,
        FString, UClass, USoundCue, UTexture},
        breaklines=true,
        basicstyle=\tiny
}

% For strikethrough effect
\usepackage[normalem]{ulem}
\usepackage{wasysym}
\usepackage{listings}
\usepackage{pdfpages}
\usepackage{enumerate}

% http://www.texample.net/tikz/examples/state-machine/
\usetikzlibrary{arrows,automata}

\newcommand{\modulecode}{COMP140 GAM160}\newcommand{\moduletitle}{Hacking Hardware/Advanced Programming}\newcommand{\sessionnumber}{Session 6}

\begin{document}
\title{\sessionnumber: Streams \& Serialization}
\subtitle{\modulecode: \moduletitle}

\frame{\titlepage}

\begin{frame}
	\frametitle{Learning outcomes}
	\begin{itemize}
		\item \textbf{Understand} the concept of serialization in Computer Science
		\item \textbf{Explain} how streams can be used to send data to a location
		\item \textbf{Implement} a save system for games
	\end{itemize}
\end{frame}

\begin{frame}
	\frametitle{Working with the Filesystem}
	\begin{itemize}
		\item Typically in a game we don't interact with the filesystem
		\item We work with some sort of virtual filesystem which is local to the game
		\item This is because
		\begin{itemize}
				\item We don't have access to the users root directory
				\item We want to limit cheating
				\item Game Engines usually access file resources in a crossplatform manner
		\end{itemize}
	\end{itemize}
\end{frame}

\begin{frame}
	\frametitle{The Filesystem \& Game Engines}
	\begin{itemize}
  		\item Often Game Engines use a compressed format to store assets
  		\item This aids in performance but also adds in a layer of security
  		\item This means that if you want to load assets programmatically you can't use standard file reading functions
  		\item See\textbf{Resource.Load} in Unity or \textbf{Referencing Assets} in Unreal
	\end{itemize}
\end{frame}

\begin{frame}
	\frametitle{Typical Filesytem operations}
  \begin{itemize}
    \item In most modern operating system you can't access the local filesystem
    \item You can only use certain directories such as the Users Document or AppData directory
    \item See \textbf{Application.*Path} in Unity \textbf{FPath} in Unreal
    \item These sort of functions/variables will give us access to safe directories to use
    \item We then use other functions to create directories and/or files
    \item See \textbf{File} \& \textbf{Directory} class in C\# or \textbf{File Management} for Unreal
  \end{itemize}
\end{frame}

\begin{frame}
	\frametitle{Serialization}
  \begin{itemize}
    \item Is the process of converting a data structure to a series of bytes
    \item This can be used to transmit the data structure to a file, over the network or store in memory
    \item We can also take serialized data and reconstruct a data structure
    \item This makes serialization the perfect candidate for saving game data
  \end{itemize}
\end{frame}

\begin{frame}
	\frametitle{Streams}
  \begin{itemize}
    \item Are just generic series of bytes that are sent to a resource
    \item This resource could be memory, a file or network
    \item Often used in conjunction with Serialization to save data
    \item See \textbf{IO Streams} in C\# or \textbf{FBufferArchive} in Unreal
  \end{itemize}
\end{frame}

\begin{frame}
	\frametitle{Exercise 1 - Basic Saving}
	\begin{enumerate}
		\item Create a Player Info Screen where the player can enter a username
		\item This is then saved and when the game is restarted the username is displayed on the screen
		\begin{itemize}
		\item \textbf{Hint} Unity - use \textbf{PlayerPrefs} 
		\item \textbf{Hint} Unreal - search for \textbf{Saving Your Game}
		\end{itemize}
		\item Add in buttons to delete the saved data
	\end{enumerate}
\end{frame}

\begin{frame}
	\frametitle{Exercise 2 - Dynamic Loading}
	\begin{enumerate}
		\item Create a basic scene with three cubes
		\item Create three materials (red, green, blue)
		\item Assign the red material to the cubes
		\item On a key press load and assign the green material (you must do this dynamically) to the cubes
		\item Create a Prefab/Actor of a sphere which uses the blue material
		\item Dynamically Load and Spawn this sphere when a key is pressed 
	\end{enumerate}
\end{frame}

\begin{frame}
	\frametitle{Exercise 2 - Hints}
	\begin{itemize}
		\item Unity - Use \textbf{Resource.Load}, all assets loaded this way should be in the Resource Folder. See \textbf{Loading Resources at Runtime}
		\item Unreal - See \textbf{Referencing Assets} for loading assets such as materials and see \textbf{spawning actors} for spawning actors
	\end{itemize}
\end{frame}

\begin{frame}
	\frametitle{Exercise 3 - Serialization}
	\begin{enumerate}
		\item Create a directory in My Documents called \textbf{GameTitle}
		\item Inside \textbf{GameTitle} create a directory called \textbf{Saves}
		\item Create a scene which has a number of triggers dotted around the level
		\begin{itemize}
			\item When the FPS Player enters one of these triggers, save the players position to the\textbf{Saves} folder, the data should be saved in a file called \textbf{save.sav}
			\item If the player quits the game and restarts, you should check for the existence of this file and load the last position
		\end{itemize}
	\end{enumerate}
\end{frame}

\begin{frame}
	\frametitle{Exercise 3 - Hints}
	\begin{itemize}
		\item Unity - See \textbf{JSON Serialization in Unity}, this will get the data into a string. You will then need to use the \textbf{Path}, \textbf{StreamWriter} and \textbf{StreamReader} C\# classes to complete the task
		\item Unreal - See \textbf{FBufferArchive} and \textbf{Save System, Read \& Write Any Data to Compressed Binary Files}
	\end{itemize}
\end{frame}

\begin{frame}
	\frametitle{Exercise 4 - Extension}
	\begin{enumerate}
		\item Instead of saving to one file each time, create a filename with the date and time of the save. Save your game data to that
		\item Create a basic UI which manages the saves (delete \& load)
		\item Create some basic GameObjects/Actors, these should move up and down. Save the position of these to file
	\end{enumerate}
\end{frame}

\end{document}
