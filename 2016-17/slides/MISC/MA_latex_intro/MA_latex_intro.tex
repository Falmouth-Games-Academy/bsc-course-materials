\documentclass[xcolor={dvipsnames}]{beamer}

\usepackage{etoolbox}\newtoggle{printable}\togglefalse{printable}
\usepackage{../../beamerthemeFalmouthGamesAcademy}
\graphicspath{ {../../} }

\title{Introduction to LaTeX}
\author{Ed Powley}

\begin{document}

\frame{\maketitle}

\part{Introducing LaTeX}
\frame{\partpage}

\begin{frame}{What is LaTeX?}
\begin{itemize}
	\pause\item A \textbf{typesetting} system
	\pause\item A \textbf{markup language} (like HTML or Markdown)
	\pause\item \textbf{Not} a WYSIWYG system
\end{itemize}
\end{frame}

\begin{frame}{Why LaTeX?}
\begin{itemize}
	\pause\item Plain text format
	\begin{itemize}
		\pause\item Can use any text editor
		\pause\item Can use version control (e.g.\ Git)
	\end{itemize}
	\pause\item Separates content from formatting
	\begin{itemize}
		\pause\item Similar to HTML and CSS
		\pause\item Unlike most WYSIWYG systems
	\end{itemize}
	\pause\item Produces professional-looking papers, reports, theses, books, slideshows, ...
\end{itemize}
\end{frame}

\begin{frame}{Getting LaTeX}
\begin{itemize}
	\pause\item LaTeX is \textbf{free open source software}
	\pause\item Consists of:
	\begin{itemize}
		\pause\item Several \textbf{executables} (pdflatex, bibtex, makeindex, ...)
		\pause\item A large library of \textbf{packages}
		\pause\item An \textbf{integrated development environment (IDE)} (optional)
	\end{itemize}
\end{itemize}
\end{frame}

\part{Your first LaTeX document}
\frame{\partpage}

\end{document}
