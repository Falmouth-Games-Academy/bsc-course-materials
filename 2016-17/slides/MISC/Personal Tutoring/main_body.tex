% Adjust these for the path of the theme and its graphics, relative to this file
%\usepackage{beamerthemeFalmouthGamesAcademy}
\usepackage{../../beamerthemeFalmouthGamesAcademy}
\usepackage{multimedia}
\graphicspath{ {../../} }

% Default language for code listings
\lstset{language=C++,
        morekeywords={each,in,nullptr}
}

% For strikethrough effect
\usepackage[normalem]{ulem}
\usepackage{wasysym}

\usepackage{pdfpages}

% http://www.texample.net/tikz/examples/state-machine/
\usetikzlibrary{arrows,automata}

\newcommand{\modulecode}{COMP260}\newcommand{\moduletitle}{Distributed Systems}\newcommand{\sessionnumber}{5}

\begin{document}
\title{\sessionnumber: Personal Tutoring}
\subtitle{\modulecode: \moduletitle}

\frame{\titlepage} 

\begin{frame}
	\frametitle{Activity}
	
	Raise your hand when you see your first games console:
	
	\begin{itemize}
		\item Magnavox Odyssey \pause
		\item Atari 2600 \pause
		\item BBC Micro \pause
		\item Nintendo Entertainment System \pause
		\item Sega Master System \pause
		\item Sega Mega Drive \pause
		\item Nintendo GameBoy \pause
		\item Super Nintendo Entertainment System

	\end{itemize}
\end{frame}

\begin{frame}
	\frametitle{Activity}
	
	Raise your hand when you see your first games console:
	
	\begin{itemize}
		\item Sony PlayStation \pause
		\item Nintendo 64 \pause
		\item Nintendo GameBoy Color \pause
		\item Sega Dreamcast \pause
		\item Sony PlayStation 2 \pause
		\item Microsoft Xbox \pause
		\item Nintendo GameCube \pause
		\item Microsoft Xbox 360 
	\end{itemize}
\end{frame}

\begin{frame}
	\frametitle{Learning Outcomes}
	\begin{itemize}
		\item \textbf{Identify} your personal tutor
		\item \textbf{Explain} the role of the personal tutor
		\item \textbf{Recall where} to look for support
		\item \textbf{Explain how} to book meetings with your personal tutor
	\end{itemize}
\end{frame}

\begin{frame}
	\frametitle{Who Is Your Personal Tutor?}	
	\begin{itemize}
		\item 2015-16 Entry --- Dr. Edward Powley \pause
		\item \textbf{2016-17 Entry --- Dr. Michael Scott} \pause
		\item 2017-18 Entry --- Alcwyn Parker \pause 
		\\~\\
		\item We prefer to be called Ed, Michael, and Al
	\end{itemize}
\end{frame}

\begin{frame}
	\frametitle{Role of the Personal Tutor}	
	
	Your personal tutor is your:
	
	\begin{itemize}
		\item Your point of contact for any non-academic issue \textbf{or} any issue affecting your studies
		\item Your signpost to Falmouth University and FXU Services
		\item Your referee when you are looking for references
		\item Your champion for extra-curricular and enterprise opportunities 
	\end{itemize}
\end{frame}

\begin{frame}
	\frametitle{Role of the Personal Tutor}	
	
	\begin{center}
		Check-in with us regularly to let us know how things are going!
	\end{center}
\end{frame}

\begin{frame}
	\frametitle{Role of the Personal Tutor}	
	
	Please:
	
	\begin{itemize}
		\item There is no such thing as a minor or silly issue
		\item Do not wait for problems to get worse - we can help you to resolve them while they are small
		\item Trust your tutor - let them know about any challenges you encounter
		         \textbf{before} submitting a claim for extenuating circumstances
		\item Inform your tutor about any accessibility issues
	\end{itemize}
\end{frame}

\begin{frame}
	\frametitle{How Do You Contact Your Personal Tutor?}	
	\begin{itemize}
		\item Email --- michael.scott@falmouth.ac.uk \pause
		\item \textbf{NOT} mike.scott@falmouth.ac.uk --- this is a different person
	\end{itemize}
\end{frame}

\begin{frame}
	\frametitle{How Do You Contact Your Personal Tutor?}	
	\begin{itemize}
		\item Slack --- @adrir \pause
		\item Always use the @ symbol as this will send me an email and ping my phone
	\end{itemize}
\end{frame}

\begin{frame}
	\frametitle{How Do You Contact Your Personal Tutor?}	
	\begin{itemize}
		\item Please \textbf{do not} use Facebook
		\item I do not receive any notifications from Facebook
	\end{itemize}
\end{frame}

\begin{frame}
	\frametitle{How Do You Contact Your Personal Tutor?}	
	\begin{itemize}
		\item Please \textbf{do not} call tutors on your phones
		\item Academics are rarely at their desks
		\item Falmouth University does not recognise a telephone call as a point of contact because there is no record
	\end{itemize}
\end{frame}

\begin{frame}
	\frametitle{Personal Tutor Meetings}	
	\begin{itemize}
		\item You are expected to meet your tutor once every two weeks to check-in
		\item This fortnightly check-in is \textbf{mandatory}
		\item Book these meetings in advance on the LearningSpace
		
		\item Tutors have specified office hours, which differ based on their commitments
		\item We do not have enough time to see everyone every week! 
	\end{itemize}
\end{frame}

\begin{frame}
	\frametitle{Office Hours: Dr Michael Scott}	
	\begin{itemize}
		\item Wednesday
		\item 1pm --- 5pm
		\\~\\
		\item Please note that other activities share this timeslot in weeks: 1, 3, 4, 5, 7, 9, 10, and 12. Your CPD sessions do count as a check-in, but if there is an issue your tutor will still be able see you after such a session.
	\end{itemize}
\end{frame}

\begin{frame}
	\frametitle{How Do I Book, Exactly?}	
	\begin{center}
		Live Demonstration (5 minutes)
	\end{center}
\end{frame}

\begin{frame}
	\frametitle{Your First Tutor Meeting}	
	
	We will check that you have:
	
	\begin{itemize}
	\item Collected your Student ID card
	\item Enrolled on the BSc Computing for Games Course
	\item Completed All Enrolment Activities at myfalmouth.falmouth.ac.uk
	\item Know how to access to your Personal Timetable
	\end{itemize}
\end{frame}

\begin{frame}
	\frametitle{Your First Tutor Meeting}	
	
	We will check that you have:
	
	\begin{itemize}
	\item Know how to access to your Personal Email Account
	\item Have subscribed to the Course Page on LearningSpace
	\item Have subscribed to the Relevant Modules on LearningSpace
	\item Joined the Facebook Group
	\end{itemize}
\end{frame}

\begin{frame}
	\frametitle{Your First Tutor Meeting}	
	
	We will check that you have:
	
	\begin{itemize}
	\item Joined the Slack Group
	\item Joined the Falmouth Games Academy GitHUB Team
	\item Setup a Twitter Account
	\item Downloaded the Socrative App
	\end{itemize}
\end{frame}

\begin{frame}
	\frametitle{Your First Tutor Meeting}	
	
	We will check that you have:
	
	\begin{itemize}
	\item Successfully Logged Into a Lab Computer
	\item Registered on FXU and Nominated a Student Rep
	\item Familiarised yourself with the BSc Course Requirements
	\end{itemize}
\end{frame}

\begin{frame}
	\frametitle{What About Academic Support?}	
	\begin{enumerate}
		\item Review the resources on LearningSpace
		\item Leave a message in the relevant Slack channel
		\item Make a pull-request on GitHub
		\item Email the module leader
	\end{enumerate}
	
	We are a community, and I expect students to help each other.
\end{frame}

\begin{frame}
	\frametitle{And Everything Else...?}	
	\begin{itemize}
		\item Falmouth University --- \texttt{portal.falmouth.ac.uk}
		\item Student Union --- \texttt{https://www.fxu.org.uk/}
	\end{itemize}
\end{frame}

\begin{frame}
	\frametitle{Any Advice?}	
	
	Time Management:
	
	\begin{itemize}
		\item Purchase a diary \textbf{or} setup a calendar app on your phone
		\item Buy a large calender \textbf{and} mount it on your wall, then label it with all of the assignment deadlines
		\item Setup a kanban-style task board on another wall, then use post-it notes as a to-do list
		\item Specify in advance what nights you will go out and stick to it
	\end{itemize}
\end{frame}

\begin{frame}
	\frametitle{Any Advice?}	
	
	\begin{itemize}
		\item All \textbf{summative} assignment deadlines are on \texttt{myfalmouth.falmouth.ac.uk}
		\\~\\
		\item All \textbf{formative} assignment deadlines are in the assignment briefs on \texttt{learningspace.falmouth.ac.uk}, and should hopefully be clear in session titles
	\end{itemize}
\end{frame}


\begin{frame}
	\frametitle{Any Advice?}	
	
	Get A Head Start:
	
	\begin{itemize}
		\item Play SpaceChem
		\item It is available in the lab, but a copy on Steam is only about £5
		\\~\\
		\item Your first worksheet will be to complete several of the levels
	\end{itemize}
\end{frame}

\begin{frame}
	\frametitle{Any Advice?}	
	
	Update Your Email:
	
	\begin{itemize}
		\item Setup an auto-forward rule to redirect your university email to an account you actually use
		\\~\\
		\item It is really easy and it takes less than 5 minutes:
		\item \url{https://www.youtube.com/watch?v=otr6fmAXKxs}
	\end{itemize}
\end{frame}

\begin{frame}
	\frametitle{Any Advice?}	
	
	Get the Socrative App on Your Phone:
	
	\begin{itemize}
		\item It is a question response system we frequently use in lectures
		\\~\\
		\item It's free and takes 5-minutes to download and install:
		\item \url{http://www.socrative.com/apps.php}
	\end{itemize}
\end{frame}

\begin{frame}
	\frametitle{Any Advice?}	
	
	Familiarise Yourself with the LearningSpace:
	\\~\\
	\url{learningspace.falmouth.ac.uk}
\end{frame}

\begin{frame}
	\frametitle{Questions \& Answers}	
	\begin{center}
		Thank you for listening. 
		\\~\\
		Please feel welcome to ask questions or raise concerns.
	\end{center}
\end{frame}

\end{document}
