% Adjust these for the path of the theme and its graphics, relative to this file
%\usepackage{beamerthemeFalmouthGamesAcademy}
\usepackage{../../beamerthemeFalmouthGamesAcademy}
\usepackage{multimedia}
\graphicspath{ {../../} }

% Default language for code listings
\lstset{language=C++,
        morekeywords={each,in,nullptr}
}

% For strikethrough effect
\usepackage[normalem]{ulem}
\usepackage{wasysym}

\usepackage{pdfpages}

% http://www.texample.net/tikz/examples/state-machine/
\usetikzlibrary{arrows,automata}

\newcommand{\modulecode}{COMP260}\newcommand{\moduletitle}{Distributed Systems}\newcommand{\sessionnumber}{5}

\begin{document}
\title{Introduction to Unity}
\subtitle{\modulecode\ \moduletitle}

\frame{\titlepage} 

\begin{frame}{Games courses at Falmouth}
	\begin{itemize}
		\pause\item \textbf{Studio-based}: 100\% coursework, mainly team-based game development projects
		\pause\item BA Game Development
			\begin{itemize}
				\pause\item Six routes: animation, art, audio, design, programming, writing
			\end{itemize}
		\pause\item BSc Computing for Games
		\pause\item BA Game Art
	\end{itemize}
\end{frame}

\begin{frame}{What is Unity?}
	\begin{itemize}
		\pause\item A \textbf{game engine}
		\pause\item Two parts:
			\begin{itemize}
				\pause\item \textbf{Editor}: used by developers to create the game
				\pause\item \textbf{Player}: packages the game to run on the customer's machine
			\end{itemize}
	\end{itemize}
\end{frame}

\begin{frame}{Why Unity?}
	\begin{itemize}
		\pause\item It's \textbf{cross platform}
			\begin{itemize}
				\pause\item Editor runs on Windows and Mac
				\pause\item Games can run on Windows, Mac, Linux, Android, iOS, all major consoles, and web via HTML5
			\end{itemize}
		\pause\item It's relatively (!) \textbf{easy to use}
		\pause\item Very flexible, can be \textbf{programmed} using C\#
		\pause\item Has \textbf{asset pipelines} for artists, animators, musicians etc.
		\pause\item \textbf{Asset store} to download sprites, 3D models, animations, C\# scripts, ...
	\end{itemize}
\end{frame}

\begin{frame}{Made with Unity}
	\center\url{https://madewith.unity.com/}
\end{frame}

\begin{frame}{Key concepts}
	\begin{itemize}
		\pause\item A game is made up of one or more \textbf{scenes}
		\pause\item Each scene contains a number of \textbf{game objects}
		\pause\item Each game object has several \textbf{components} which affect how it looks, moves and behaves
		\pause\item Programmers can create custom components called \textbf{behaviours}
		\pause\item NB: behaviours can create or delete game objects, or change scenes...
		\pause\item No main loop, but behaviours have an \lstinline{Update} method which is called each frame
	\end{itemize}
\end{frame}

\begin{frame}{What to make?}
	\center\url{http://www.ludocraft.com/gigster/index.html}
\end{frame}

\begin{frame}{What to make?}
	\begin{itemize}
		\pause\item Think simple --- \textbf{don't overscope}!
		\pause\item Work in \textbf{teams}
		\pause\item Focus on \textbf{core mechanics} --- what will the player do in your game, and why is that fun?
		\pause\item Think about your \textbf{unique selling points (USPs)} --- what will make your game different from all the others?
		\pause\item Aim for a \textbf{minimum viable product} --- don't worry about polishing yet
	\end{itemize}
\end{frame}

\begin{frame}{Unity tutorial}
	\center\url{http://bit.ly/unity_roll_a_ball}
\end{frame}

\end{document}
