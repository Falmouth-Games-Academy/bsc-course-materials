% Adjust these for the path of the theme and its graphics, relative to this file
%\usepackage{beamerthemeFalmouthGamesAcademy}
\usepackage{../../beamerthemeFalmouthGamesAcademy}
\usepackage{multimedia}
\graphicspath{ {../../} }

% Default language for code listings
\lstset{language=C++,
        morekeywords={each,in,nullptr}
}

% For strikethrough effect
\usepackage[normalem]{ulem}
\usepackage{wasysym}

\usepackage{pdfpages}

% http://www.texample.net/tikz/examples/state-machine/
\usetikzlibrary{arrows,automata}

\newcommand{\modulecode}{COMP260}\newcommand{\moduletitle}{Distributed Systems}\newcommand{\sessionnumber}{5}

\setbeamertemplate{navigation symbols}{}


\newcommand{\fullbleed}[1]{
\begin{frame}[plain]
	\begin{tikzpicture}[remember picture, overlay]
		\node[at=(current page.center)] {
			\includegraphics[width=\paperwidth]{#1}
		};
	\end{tikzpicture}
\end{frame}
}

\newcommand{\picturepage}[2]{
\begin{frame}[plain]
	\begin{tikzpicture}[remember picture, overlay]
		\node[at=(current page.center)] {
			\includegraphics[width=\paperwidth]{#1}
		};
		\draw<1>[draw=none, fill=black, opacity=0.9] (-1,-5.2) rectangle (current page.south east);
		\node[draw=none,text width=0.96\paperwidth, align=right] at (5.5,-5.5) {\tiny{#2}};
	\end{tikzpicture}
\end{frame}
}

\newcommand{\notepicx}[5]{
\begin{frame}[plain]
	\begin{tikzpicture}[remember picture, overlay]
		\node[at=(current page.center)] {
			\includegraphics[width=\paperwidth]{#1}
		};
		\node[draw=none, fill=black, text width=#5\paperwidth] at ([xshift=#3, yshift=#4] current page.center) {\small{#2}};
	\end{tikzpicture}
\end{frame}
}

\newcommand{\notepic}[4]{
	\notepicx{#1}{#2}{#3}{#4}{0.4}
}

\begin{document}
\title{\sessionnumber}
\subtitle{\modulecode: \moduletitle}

\frame{\titlepage} 

\begin{frame}
	\frametitle{Learning Outcomes}
	
	By the mid-session break, you should be able to:
	
	\begin{itemize}
		\item \textbf{Recognise who} your tutors are
		\item \textbf{Outline what} the Games Academy does
		\item \textbf{Explore} some of the kinds of question that excite game scholars
		\item \textbf{Explain} the key learning outcomes \textbf{and} career paths that the course caters to
		\item \textbf{Recall} the structure of the course
		\item \textbf{Describe} the first-year modules on which you are enrolled on
	\end{itemize}
\end{frame}

\begin{frame}
	\frametitle{Learning Outcomes}
	
	Between the mid-session break and the end of the session, you should become able to:
	
	\begin{itemize}
		\item \textbf{Recall} the assignments for the first semester
		\item \textbf{Contrast} what is expected of students in the higher education context to the compulsory education context
		\item \textbf{Analyse how} to invest sufficient time in both course activities \textbf{as well as} self-regulated deliberate practice to achieve key goals
	\end{itemize}
\end{frame}

\part{Your Tutors}
\frame{\partpage}

\picturepage{mike_and_monica}{Michael Scott working with Monica McGill at an ACM Working Group in Peru}

\picturepage{ed_powley}{Ed Powley presenting with Professor Cowling at The Royal Academy of Engineering's Summer Soir\'{e}e}

\picturepage{al_parker}{Al Parker performing in front of 13 screens and 13 P2 cameras}

\picturepage{games_academy_team}{Other Members of Staff in the Games Academy}

\part{The Games Academy}
\frame{\partpage}

% Meta-Makers

% EU Investment - AIR

\notepic{cam01-fr0300}{\textbf{World-Leading} Research in \textbf{Digital Games} and \textbf{Digital Games Technology}}{2.5cm}{3cm}

\notepicx{painting_fool_sadness}{Hold more than \textbf{2 million} of funds for research in \textbf{Artificial Intelligence}, \textbf{Procedural Content Generation}, and \textbf{Transmedial Aesthetics}}{-4.5cm}{-2cm}{0.25}

\notepic{tanya}{Lead By \textbf{World-Renowned Researchers}}{-2.5cm}{-3.5cm}

\notepic{simon}{Lead By \textbf{World-Renowned Researchers}}{-2.5cm}{-3.5cm}

\notepic{launchpad}{Striving Towards a \textbf{First-Class Educational Provision} that Prepares Students for \textbf{Careers} in the \textbf{Creative Industries}}{-2.5cm}{3cm}

\fullbleed{cornwall_technation_1}

\fullbleed{cornwall_technation_2}

\notepic{prospectus_alex_034}{Undergraduate Courses in \textbf{Digital Games}}{-2.5cm}{-3.5cm}

\notepic{prospectus_alex_064}{Undergraduate Courses in \textbf{Computing for Games}}{-2.5cm}{-3.5cm}

\notepic{1_studio_wide}{Postgraduate Courses in \textbf{Games Entrepreneurship}}{-2.5cm}{-3.5cm}

\notepic{mike_filming_ma}{Distance-Learning Courses in \textbf{Creative App Development}}{-2.5cm}{-3.5cm}

\notepicx{studio_games_showoff}{A Innovative \textbf{Interdisciplinary} Approach To Education}{2.5cm}{-3.5cm}{0.4}

%\notepic{orange_helicopter}{Our Staff Are \textbf{Indie Game Developers}}{-2.5cm}{-3.5cm}

\notepic{space_caves}{Our Staff Are \textbf{Indie Game Developers}}{-2.5cm}{-3.5cm}

\notepic{warm_gun}{Our Staff Are \textbf{Indie Game Developers}}{-2.5cm}{-3.5cm}

\notepicx{deal_with_the_devil}{We Work Closely with \textbf{Cornwall's Largest Game Studios}}{-4.5cm}{-2.25cm}{0.2}

\notepicx{rising-storm-2}{We Work Closely with \textbf{Cornwall's Largest Game Studios}}{-4.5cm}{-2.25cm}{0.2}

\notepic{legends}{We Attract \textbf{Industry Legends} as Visiting Lecturers}{3cm}{-2.5cm}

\part{The Meta-Game}
\frame{\partpage}

\begin{frame}
	\frametitle{The Games Meta-Game}
	
	Setup:
	
	\begin{itemize}
		\item Self-organise into groups of 3-4 players
		\item You will each receive two sets of card: game cards and question cards.
		\item While you are waiting for your cards, identify the youngest player. They will be the first critic.
		\item All actions are clockwise from the critic.
	\end{itemize}
\end{frame}

\begin{frame}
	\frametitle{The Games Meta-Game}
	
	Instructions:
	
	\begin{enumerate}
		\item 	\textbf{Question}: The critic draws a question card. 
		\item 	\textbf{Answer}: The \textit{remaining players} (i.e., not the critic!) submit their best game card, to answer the question, face-up.
		\item 	\textbf{Justification}: The \textit{remaining players} justify the game card they have selected.
		\item 	\textbf{Selection}: The critic selects the most suitable game card answering the question. That player `wins' the round, keeping the question card as a scoring token
			and becomes the next critic.	
		\item 	\textbf{Repeat} from step 1, for approximately 20 minutes.	
	\end{enumerate}
\end{frame}
   
\part{Careers in the Games Industry}
\frame{\partpage}

\begin{frame}
	\frametitle{Careers for Computing Professionals}
	
	It is important to note that:
	
	\begin{itemize}
		\item Games are complex and therefore require significant knowledge and skills to produce \pause
		\item They bring together art, storytelling, design and computing \pause
		\item Roles are therefore quite diverse and specialised \pause
		\item Each role requires very specific skills, mastered in considerable depth \pause
		\item Teamwork is essential (though there are many ways of working)
	\end{itemize}
\end{frame}

\begin{frame}
	\frametitle{Careers for Computing Professionals}
	
	Computing professionals tend to:
	
	\begin{itemize}
		\item Deal with the technical side of games development \pause
		\item Be specialists, consultants, analysts, or technical leaders \pause
		\item Be people who are comfortable with mathematics and science \pause
		\item Keep up with the fast-paced field of computer technology \pause
		\item Have a science degree rather than an arts degree, with an ability to conduct independent research
		\item Experts in programming and software engineering
	\end{itemize}
\end{frame}

\begin{frame}
	\frametitle{Careers for Computing Professionals}
	
	There is a wide range of technical roles in game studios:
	
	\begin{columns}
		\begin{column}{0.5\textwidth}
			\begin{itemize}
				\item Technical Director / CTO / Lead
				\item Gameplay Programmer
				\item Engine Programmer
				\item Physics Programmer
				\item AI Programmer
				\item Network Programmer
				\item Graphics Programmer
			\end{itemize}
		\end{column}
		\begin{column}{0.5\textwidth}
			\begin{itemize}
				\item Tools Programmer
				\item UX / UI Programmer
				\item Middleware / Technology Developer
				\item Porting Programmer
				\item Level Scripter
				\item Audio Engineer
				\item Data Scientist
			\end{itemize}
		\end{column}
	\end{columns}
\end{frame}

\begin{frame}
	\frametitle{What About Other Careers?}
	
	Naturally, your degree does not pre-determine your career path:
	
	\begin{itemize}
		\item 	\textbf{Design}: designers who can actually prototype their designs can test their designs and quickly become better at designing \pause
		\item 	\textbf{Art}: art for digital games is indeed digital and technical artists are in high demand \pause
		\item 	\textbf{Management}: insight into how software developers practice their craft will make you better at managing them in a studio context 
			(and perhaps even garner some respect) \pause
		\item 	\textbf{Administration \& Commerce}: the games industry isn't just about development, there is a huge range of other career paths,
			such as human resources and IT
	\end{itemize}
\end{frame}

\fullbleed{t-shape}

\begin{frame}
	\frametitle{Learning Objectives}
	
	The learning objectives of the course are:
	
	\begin{itemize}
		\item \textbf{Technical Development Practice}: leverage professional practices and technical skills to craft creative software \pause
		\item \textbf{Communication}: communicate effectively with stakeholders in writing, verbally, and through adherence to standards and conventions in documentation \pause
		\item \textbf{Critical Evaluation}: reflect critically on, and evaluate, the quality of working methods and solutions
	\end{itemize}
\end{frame}

\begin{frame}
	\frametitle{Learning Objectives}
	
	The learning objectives of the course are:
	
	\begin{itemize}
		\item \textbf{Research}: engage in activities that may create new knowledge, present that knowledge in an academic format, and apply it to practice\pause
		\item \textbf{Enterprise \& Innovation}: provide opportunities for enterprise through innovation, invention, and creativity\pause
		\item \textbf{Professionalism}: set goals, manage workloads to meet deadlines, work efficiently and effectively in teams, and accommodate change
	\end{itemize}
\end{frame}

\part{Course Map}
\frame{\partpage}

\fullbleed{course_map_1}

\fullbleed{course_map_2}

\fullbleed{course_map_3}

\part{First Year Modules}
\frame{\partpage}

\begin{frame}
	\frametitle{COMP110: Principles of Computing}
	
	This module is designed to introduce you to the basic principles of computing and programming in the context of digital games.
	
	\vspace{2em}
	
	Your learning will complement the other modules through providing a broad foundation on the different methods and techniques which will help you to be able to construct computer programs and able to use relevant scholarly sources. 

\end{frame}

\begin{frame}
	\frametitle{COMP120: Tinkering}
	
	This module is designed to help you learn different ways of engaging with code using practical and exploratory methods. 
	
	\vspace{2em}
	
	You will learn the value of taking a creative approach to computing and become acquainted with some of the principles behind Creative Computing. 

\end{frame}

\begin{frame}
	\frametitle{COMP150: Game Dev Practice}
	
	This module introduces you to the founding principles and processes of professional game development. 
	
	\vspace{2em}
	
	You gain an understanding of the way that the different components of game development come together to make playable games and how those components are organised through the development pipeline. You also gain a `first-principles' understanding of how games are designed with a target market in mind and have a strong underlying concept.

\end{frame}

\begin{frame}
	\frametitle{COMP130: Game Architecture}
	
	This module helps you to understand the ways in which the architecture of games shape the types of computing solutions that one might build. 
	
	\vspace{2em}
	
	Your contextual understanding of game architecture is then brought into sharper focus practically through practical worksheet and research tasks.

\end{frame}

\begin{frame}
	\frametitle{COMP140: Hacking}
	
	The module allows you to develop further a creative approach to computing within the context of building solutions used to develop games. 
	
	\vspace{2em}
	
	You will begin to bring different elements together, taking existing code from multiple sources and learn ways and methods for bringing these together in synthesis in order to build more creative and robust solutions. 

\end{frame}

\begin{frame}
	\frametitle{COMP160: Software Engineering}
	
	This module helps you build on your experience of game development by engaging in depth with the principles of professional software engineering. 
	
	\vspace{2em}
	
	You will learn the importance of reuse and scalability when creating solutions and how to identify recurring problems within a specific domain. You'll demonstrate this both by applying existing design patterns and by creating your own reusable solutions. 

\end{frame}

\part{Coffee Break}
\frame{\partpage}

\begin{frame}
	\frametitle{Coffee Break}
	
	Please return at 5 minutes past the hour.

\end{frame}

\part{Assignments}
\frame{\partpage}

\begin{frame}
	\frametitle{Assignment Structure}
	
	Each Semester, you will complete \textbf{six} assignment `tracks':
	
	\begin{itemize}
		\item Collaborative Game Development Project
		\item Academic Essay
		\item 2x Small Programming Projects
		\item Small Portfolio Pieces and/or Worksheets
		\item Research Journal
		\item Continuing Personal Development Tasks and Reflective Report
	\end{itemize}

\end{frame}

\begin{frame}
	\frametitle{Assignments}
	
	Live Demo
	
	\vspace{3em}
	
	All assignment briefs can be found on:
	
	\vspace{0.5em}
	
	\indent \url{learningspace.falmouth.ac.uk}
	
	\vspace{0.5em}
	
	Read them very carefully!
	
\end{frame}


\part{Time Management}
\frame{\partpage}

\begin{frame}
	\frametitle{Expectation}
	
	600-hours per semester. Including contact time.
	
	Only $1/3$ of is contact time.
	
	40 hours per week. Over 15 weeks. 
	
	12 weeks of sessions.

\end{frame}

\begin{frame}
	\frametitle{Activity: Planning Your Time}
	
	Each Semester, you will complete \textbf{six} assignment `tracks':

\end{frame}


\end{document}
