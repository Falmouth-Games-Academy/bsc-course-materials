\part{Live coding: applied OOP}
\frame{\partpage}

\begin{frame}{Introduction}
    \begin{center}
        \url{https://github.com/Falmouth-Games-Academy/comp150-live-coding}
    \end{center}
    \begin{itemize}
        \item Clone this repository and load it into Visual C++
            \begin{itemize}
                \item Note the instructions in \texttt{readme.md} with regard to copying dlls
            \end{itemize}
        \item A simple game, but implemented as one long function
        \item Let's improve it!
    \end{itemize}
\end{frame}

\begin{frame}{Resource Acquisition Is Initialisation (RAII)}
    \begin{itemize}
        \item A common C++ idiom for handling \textbf{allocation and deallocation of resources} \pause
        \item Create a class, allocate in the \textbf{constructor}, deallocate in the \textbf{destructor} \pause
        \item If the instance is created on the \textbf{stack}, the destructor is called \textbf{automatically}
            when the instance goes out of \textbf{scope}
            --- no need to remember to deallocate things
    \end{itemize}
\end{frame}

\begin{frame}[fragile]{Accessor methods}
    \begin{lstlisting}
private:
    int health;
    
public:
    int getHealth() { return health; }
    void setHealth(int h) { health = h; }
    \end{lstlisting}
    \pause
    \begin{itemize}
        \item A.k.a. \textbf{getters} and \textbf{setters} \pause
        \item Allow finer control over access to data in a class \pause
        \item E.g.\ could have a \textbf{public} getter and a \textbf{private} setter \pause
        \item E.g.\ could have a setter that \textbf{validates} the new value
    \end{itemize}
\end{frame}

