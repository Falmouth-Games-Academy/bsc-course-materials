
\begin{frame}{Data types in GLSL}
	\begin{itemize}
		\pause\item \lstinline{bool}, \lstinline{int}, \lstinline{float}: just like in C++
		\pause\item \lstinline{vec2}, \lstinline{vec3}, \lstinline{vec4}: \textbf{vectors} of \lstinline{float}s
			\begin{itemize}
				\pause\item Vectors in the mathematical sense, not the \lstinline[language=C++]{std::vector} sense
			\end{itemize}
		\pause\item \lstinline{mat2}, \lstinline{mat3}, \lstinline{mat4}: \textbf{square matrices} of \lstinline{float}s
		\pause\item \lstinline{mat2x3}, \lstinline{mat3x2}, \lstinline{mat4x2} etc: \textbf{rectangular matrices} of \lstinline{float}s
		\pause\item \textbf{Arrays} of constant size e.g.\ \lstinline{float myArray[10]}
		\pause\item There's no such thing as \textbf{pointers} in GLSL (hooray!)
	\end{itemize}
\end{frame}

\begin{frame}{Vectors}
	\begin{itemize}
		\pause\item An \textbf{$n$-dimensional vector} is formed of $n$ numbers
		\pause\item E.g.\ 2-dimensional vectors:
			$$ (1, 2) \qquad (-2.7, 0) \qquad (3.4, -12.7) $$
		\pause\item E.g.\ 3-dimensional vectors:
			$$ (1, 2, 0) \qquad (-9, 6, 3.7) \qquad (2.1, 2.1, 2.1) $$
		\pause\item Used to represent \textbf{points} or \textbf{directions} in $n$ dimensions
		\pause\item Also used to represent e.g.\ colours in RGB(A) space
	\end{itemize}
\end{frame}

\begin{frame}[fragile]{Constructing vectors in GLSL}
	\begin{lstlisting}
vec2 a = vec2(1.2, 3.4);
vec3 b = vec3(1); // same as vec3(1, 1, 1)
vec3 c = vec3(a, 5.6); // same as vec3(1.2, 3.4, 5.6)
	\end{lstlisting}
\end{frame}

\begin{frame}[fragile]{Vector maths}
	\pause Most operations work \textbf{component-wise}:
	\begin{lstlisting}
vec2 a = vec2(1, 2);
vec2 b = vec2(3, 4);
vec2 c = a + b; // c == vec2(4, 6);
vec2 d = a * b; // d == vec2(3, 8);
	\end{lstlisting}
	\pause Can also multiply a \textbf{vector} by a \textbf{scalar}:
	\begin{lstlisting}
vec2 e = 3.1 * a; // e == vec2(3.1, 6.2)
	\end{lstlisting}
\end{frame}

\begin{frame}[fragile]{Accessing components}
	\pause Can access the components of a vector as \lstinline{.x}, \lstinline{.y}, \lstinline{.z}, \lstinline{.w}:
	\pause\begin{lstlisting}
vec4 a = vec4(1, 2, 3, 4);
float b = a.y; // b == 2
float c = a.z; // c == 3
a.x = 5;       // a == vec4(5, 2, 3, 4)
a.w = a.y;     // a == vec4(5, 2, 3, 2)
	\end{lstlisting}
	\pause Can also use \lstinline{r g b a} (for colours) and \lstinline{s t p q} (for texture coordinates)
\end{frame}

\begin{frame}[fragile]{Swizzling}
	\pause Can access multiple components in one go:
	\pause\begin{lstlisting}
vec4 a = vec4(1, 2, 3, 4);
vec2 b = a.xy;    // b == vec2(1, 2)
vec3 c = a.zyz;   // c == vec3(3, 2, 3)
a.xw = vec2(5,6); // a == vec4(5, 2, 3, 6)
a.xyzw = a.wzyx;  // a == vec4(6, 3, 2, 5)
	\end{lstlisting}
	\begin{itemize}
		\pause \item\textbf{Can} use the same component twice in the \textbf{right-hand side} of an assignment 
		\pause \item\textbf{Cannot} use the same component twice in the \textbf{left-hand side} of an assignment 
		\pause \item Swizzling is generally \textbf{faster} than the equivalent code without swizzling
		\pause \item Can also use  \lstinline{r g b a} or \lstinline{s t p q}, but can't mix them
			(e.g.\ \lstinline{.gbr} is valid but \lstinline{.gzx} is not)
	\end{itemize}
\end{frame}
