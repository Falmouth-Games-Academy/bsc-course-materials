\part{COMP120 Refresher}
\frame{\partpage}

\begin{frame}{Image Processing Techniques}
	\begin{itemize}
		\item\pause Some of the algorithms you learned in COMP120 can be applied in GLSL:
		\begin{itemize}
			\item Colour Replacement
			\item Luminance calculation  
			\item Colour Correction
			\item Black \& White and Sepia Tone
			\item Edge Detection
		\end{itemize}
		\pause\item Key difference: we access the data via \textbf{texture coordinates}, rather than array indices.
	\end{itemize}
\end{frame}

\begin{frame}{Modifying Textures}
	\begin{itemize}
		\item\pause A Texture Lookup uses the texture coordinates passed to the fragment shader
		\item\pause Additionally, the fragment shader only processes a single fragment at a time
		\item\pause If we want to access a pixel adjacent to the current one, we can offset the current texture coordinates - this is especially useful for edge detection
		\item\pause GLSL has lots of inbuilt functions (e.g distance) which can aid in creating post-processing effects
	\end{itemize}
\end{frame}