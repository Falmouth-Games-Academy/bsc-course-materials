\part{COMP120 Refresher}
\frame{\partpage}

\begin{frame}{Overview}
	\begin{itemize}
		\item\pause Some of the algorithm you learned in COMP120 can be applied to GLSL
		\item\pause The only difference is we don't use an array to access the pixel data
		\item\pause Remember that a Texture Lookup uses the texture coordinates passed to the fragment shader
		\item\pause Additionally, the fragment shader only processes a one fragment at a time
	\end{itemize}
\end{frame}

\begin{frame}{COMP120 Techniques}
	\begin{itemize}
		\item Colour Replacement
		\item Luminance calculation  
		\item Colour Correction
		\item Black \& White and Sepia Tone
		\item Edge Detection
	\end{itemize}
\end{frame}

\begin{frame}{Implementing these effects}
	\begin{itemize}
		\item\pause Remember we can't access a pixel array, we use the texture coordinates
		\item\pause If we want to access a pixel adjacent to the current one, we can offset the current texture coordinates
		\item\pause This is especially useful for edge detection
		\item\pause Lastly, GLSL has lots of inbuilt functions (e.g distance) which can aid in creating post-processing effects
	\end{itemize}
\end{frame}