\part{Context and applications}
\frame{\partpage}

\begin{frame}{Why are graphics and simulation important?}
	\begin{itemize}
		\pause\item More than 50\% of the cortex, the surface of the brain, is devoted to processing visual information (David R. Williams, William G. Allyn Professor of Medical Optics at the University of Rochester; reference \href{William G. Allyn Professor of Medical Optics}{here}).
		\pause\item Most computer output appears on the screen.
		\pause\item Enables intuitive observation of data, representing:
		\begin{itemize}
			\pause\item locations, orientations and dimensions of objects in space
			\pause\item statistical or experimental results
		\end{itemize}
	\end{itemize}
\end{frame}

\begin{frame}{Applications}
	\begin{itemize}
		\pause\item Entertainment: games, virtual reality, films/TV
		\pause\item Architecture and CAD
		\pause\item Engineering
		\pause\item Science and medicine
	\end{itemize}
\end{frame}

\begin{frame}{Challenges}
	\begin{itemize}		
		\pause\item Accuracy vs. speed: complex calculations take time...
		\pause\item Solution: use approximations, considering:
		\begin{itemize}
			\pause\item Which objects do we really need to see?
			\pause\item How much detail do we need to see them in?
			\pause\item What can be sacrificed without compromising the end result (too much)?
		\end{itemize}
		\pause\item Optimisations are possible at each stage of the graphics pipeline
		\pause\item Different applications require different balances of trade-offs; in games, speed is paramount (but detail is desirable!)
	\end{itemize}
\end{frame}
