\part{Creating the model, view, projection matrix in GLM}
\frame{\partpage}

\begin{frame}{The model matrix}
	\pause Exactly what we've been doing so far today...
\end{frame}

\begin{frame}[fragile]{The view matrix}
	\pause Need to translate and rotate the scene so that the ``camera'' is at $(0,0,0)$ and looking in the negative $z$ direction
	\pause\begin{lstlisting}
glm::mat4 view = glm::lookAt(
  glm::vec3(2, 0, 2),    // eye
  glm::vec3(0, 0, 0),    // centre
  glm::vec3 up(0, 1, 0)  // up
);
	\end{lstlisting}
	\begin{itemize}
		\pause\item \lstinline{eye} is the position of the camera
		\pause\item \lstinline{centre} is a point for the camera to look at
		\pause\item \lstinline{up} is which direction is ``up'' for the camera (usually the positive $y$-axis)
	\end{itemize}
\end{frame}

\begin{frame}[fragile]{The projection matrix}
	\pause\begin{lstlisting}
glm::mat4 projection = glm::perspective(
	glm::radians(45.0f), // field of view
	4.0f / 3.0f,         // aspect ratio
	0.1f,                // near clip plane
	100.0f               // far clip plane
);
	\end{lstlisting}
	\begin{itemize}
		\pause\item \textbf{Field of view (FOV)}: how ``wide'' or ``narrow'' the view is
		\pause\item \textbf{Aspect ratio}: should be \lstinline{screenWidth / screenHeight}
		\pause\item \textbf{Near and far clip planes}: fragments that fall outside this range of distances from the camera are not drawn
	\end{itemize}
	\pause Also available: \lstinline{glm::ortho} for orthographic projection
\end{frame}