\part{Scene graphs}
\frame{\partpage}

\begin{frame}{Coordinate spaces}
	\begin{center}
		\begin{tabular}{cl}
			\pause Model space \\
			\pause $\downarrow$ & Model matrix \\
			\pause World space \\
			\pause $\downarrow$ & View matrix \\
			\pause Camera space \\
			\pause $\downarrow$ & Projection matrix \\
			\pause Screen space
		\end{tabular}
	\end{center}
\end{frame}

\begin{frame}{Rule of thumb}
	\begin{itemize}
		\pause\item When performing calculations, \textbf{do not mix} vectors from \textbf{different coordinate spaces}
		\pause\item E.g.\ when performing lighting calculations, ensure your fragment position, normal, light direction, eye direction are all
			in the \textbf{same} space
	\end{itemize}
\end{frame}

\begin{frame}{Scene graph}
	\begin{itemize}
		\pause\item It is often useful to organise objects into a \textbf{hierarchy}
		\pause\item Each node in the hierarchy has its own model matrix
		\pause\item Transformations stack: object is affected by its own transformation,
			and that of its parent,
			and that of its grandparent,
			and so on
		\pause\item The model matrix is the \textbf{product} of model matrices for the node and its ancestors
	\end{itemize}
\end{frame}
