\part{GLSL}
\frame{\partpage}

\begin{frame}{OpenGL Shading Language (GLSL)}
	\begin{itemize}
		\pause\item Used for writing \textbf{shaders} for OpenGL applications
		\pause\item C-like syntax
		\pause\item GLSL compiler is part of the \textbf{graphics driver}
			on the \textbf{end user's machine}
			\begin{itemize}
				\pause\item Yes, you need to ship your shader source code with your game!
			\end{itemize}
	\end{itemize}
\end{frame}

\begin{frame}{Basic vertex shader}
	\lstinputlisting[language=GLSL]{basicVert.glsl}
\end{frame}

\begin{frame}{Basic vertex shader}
	\lstinline{\#version 330 core}
	\begin{itemize}
		\pause\item Tells the compiler to use OpenGL 3.3 core functionality
	\end{itemize}
\end{frame}

\begin{frame}{Basic vertex shader}
	\lstinline{layout(location = 0) in vec3 vertexPos;}
	\begin{itemize}
		\pause\item Specifies \textbf{input values} to the vertex shader
		\pause\item Corresponds with layout of \textbf{vertex buffers} in C++ program
	\end{itemize}
\end{frame}

\begin{frame}{Basic vertex shader}
	\lstinline{void main()}
	\begin{itemize}
		\pause\item Every shader program must define a \lstinline{void main()} function
	\end{itemize}
\end{frame}

\begin{frame}{Basic vertex shader}
	\lstinline{gl_Position=vec4(vertexPos,1.0f);}
	\begin{itemize}
		\pause\item \lstinline{gl_Position} is one of many \textbf{built-in} variables with special meaning
		\pause\item See \url{https://www.opengl.org/wiki/Built-in_Variable_(GLSL)}
	\end{itemize}
\end{frame}

\begin{frame}{Basic fragment shader}
	\lstinputlisting[language=GLSL]{basicFrag.glsl}
\end{frame}

\begin{frame}{Basic fragment shader}
	\lstinline{out vec3 color;}
	\begin{itemize}
		\pause\item By convention, fragment shader should have \textbf{one output}, namely the fragment colour
		\pause\item Doesn't have to be named \lstinline{color} --- could be any other non-reserved identifier
	\end{itemize}
\end{frame}

\begin{frame}{Programming in GLSL}
	\begin{itemize}
		\pause\item \lstinline{if} statements, \lstinline{for} loops, \lstinline{while} loops, \lstinline{do while} loops, \lstinline{switch} statements,
			\lstinline{break}, \lstinline{continue}, \lstinline{return} all work the same as C++
		\pause\item \lstinline{// Single-line comments} and \lstinline{/* Multi-line comments */} work the same too
		\pause\item Function definitions and declarations are similar to C++, except that parameters must be declared as
			\lstinline{in}, \lstinline{out} or \lstinline{inout}
		\pause\item Recursion is \textbf{forbidden}
		\pause\item No \texttt{\#include} --- splitting a shader into multiple files is not easy...
		\pause\item No \lstinline[language=C++]{class}
	\end{itemize}
\end{frame}
