\part{Your first OpenGL program}
\frame{\partpage}

\begin{frame}{Shaders}
	\begin{itemize}
		\pause\item The vertex processor and fragment processor are \textbf{programmable}
		\pause\item Programs for these units are called \textbf{shaders}
		\pause\item \textbf{Vertex shader}: responsible for geometric transformations, deformations, and projection
		\pause\item \textbf{Fragment shader}: responsible for the visual appearance of the surface
		\pause\item Vertex shader and fragment shader are separate programs,
			but the vertex shader can pass arbitrary values through to the fragment shader
	\end{itemize}

\begin{frame}{SDL and OpenGL}
	\begin{itemize}
		\pause\item OpenGL only handles rendering of graphics
		\pause\item We need something else to handle windows, events, audio etc
		\pause\item We will use \textbf{SDL} (which you have used before in COMP140)
	\end{itemize}
\end{frame}

\begin{frame}{Live coding}
	\begin{center}
		\url{https://github.com/Falmouth-Games-Academy/comp220-code-examples}
	\end{center}
\end{frame}

\begin{frame}{Live coding - basics}
	\begin{center}
		\url{http://headerphile.com/sdl2/opengl-part-1-sdl-opengl-awesome/}
	\end{center}
\end{frame}

\begin{frame}{Our first triangle}
	\begin{center}
		\url{http://www.opengl-tutorial.org/beginners-tutorials/tutorial-2-the-first-triangle/}
	\end{center}
\end{frame}

