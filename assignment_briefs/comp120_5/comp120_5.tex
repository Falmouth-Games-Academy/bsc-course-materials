\documentclass{../fal_assignment}
\graphicspath{ {../} }

\usepackage{enumitem}
\usepackage[T1]{fontenc} % http://tex.stackexchange.com/a/17858
\usepackage{url}
\usepackage{todonotes}

\title{Semester Two Reflective Report}
\author{Dr Michael Scott}

\begin{document}

\maketitle
%\begin{marginquote}
%    ``Students come into programming classes with a broad range of backgrounds ---
%    some have experience in several programming languages, others have never programmed before in their life.
%    
%    Being able to engage with the community and support each other is important.
%    Upload your code to GitHub and receive feedback from experienced peers.
%    Review your peers' work yourself and really consider what `quality' actually means
%    and what `good' source code looks like.
%    Debate, argue, and question others about it ---
%    an open and sustained discourse is an excellent way to learn ---
%    for both beginners and adepts!''
%\end{marginquote}
\marginpicture{MakeyMakey.jpg}{
    The \emph{MaKey~MaKey} allows a multitude of materials to be used to create videogame controllers.
}
\section*{Introduction}

In this assignment, you are required to critically reflect on your progress across the Semester and, in particular, review the strengths and weaknesses of the key skills that are influencing the quality of your work. You will also develop plans of action to overcome your weaknesses and thereby improve the quality of your work in future assignments.

Such reflection and planning is an extremely important part of learning games development, and in particular computer programming. They are key components in a technique known as deliberate practice, which research has shown to be very successful at nurturing expertise in software engineering. Everyone that properly adopts this technique eventually succeeds, despite the challenging nature of the subject. It is, therefore, very important that reflection and planning are not dismissed as an afterthought at the end of the course. For this reason, this assessment has two components, A and B, as follows:

\begin{enumerate}[label=(\alph*)]
    \item A series of brief weekly reports that must:
    	\begin{enumerate}[label=\roman*.]
    		\item \textbf{describe} your progress;
    		\item \textbf{assess} any problems or issues that you have encountered;
    		\item and then \textbf{outline} some specific actions to take to overcome these problems.
	\end{enumerate}
    \item A final 500-word report that must:
    	\begin{enumerate}[label=\roman*.]
    		\item \textbf{identify three} key skills that you consider weaknesses;
    		\item \textbf{assess} your application of \textbf{each} of these skills, \textit{describing how} they affected the quality of your submissions \textbf{and suggesting why} they became challenges;
    		\item and then \textbf{identify how} to improve \textbf{each} of these skills, with reference to SMART actions.
	\end{enumerate}
\end{enumerate}

Part A consists of \textbf{multiple formative submissions} with deadlines at the end of each week across the Semester. This work will be assessed on a \textbf{threshold} basis. The threshold is set at 15\%. This means that 15\% of the total marks available for the coursework overall are awarded on a pass or fail basis. In other words, satisfactory submissions will be awarded 15\%. However, unsatisfactory submissions will receive 0\%.

The following criteria are used to determine a pass or fail for each submission in Part A:

\begin{enumerate}[label=(\alph*)]
	\item Progress has been described with adequate detail;
	\item Problems and issues have been clearly explained and assessed;
	\item There is evidence of reflection;
	\item At least one SMART action, appropriate for resolving the issue, has been outlined;
\end{enumerate}

You will receive ongoing feedback through your fortnightly tutorials and will have the opportunity to revise the work prior to the final deadline based on this feedback. It is especially important that work that is not yet satisfactory is revised based on the feedback provided as \textbf{all reports must be satisfactory to pass}.

Part B is a single summative submission and will be assessed on a criterion-referenced basis. This submission is expected to take students from the threshold of 15\% (F) up to the maximum of 100\% (A*). This means that 85\% of the total marks available for the coursework overall will be awarded.

The following criteria are used to allocate marks:

\begin{enumerate}[label=(\alph*)]
	\item Appropriateness and Specificity of Selection of Key Skills;
	\item Adequacy of Self-Appraisal in Relation to Key Skills;
	\item Depth of Reflection on Key Skills;
	\item Appropriateness of Plan for the Future;
	\item Quality of Academic Writing;
\end{enumerate}

\section*{Submission Instructions}

\subsection*{Part A}

Part A must be completed as a formative submission on GitHub. Fork the GitHub project at the following URL:

\indent \url{https://github.com/Falmouth-Games-Academy/comp110-evaluation}

Write your weekly reports in the \texttt{readme.md} file. Also use this repository for any other digital assets you create (e.g.\ images), checking them in regularly as you work on your projects. For the reports, images should be embedded directly in the \texttt{readme.md} file. Videos should be uploaded to a video sharing site (e.g.\ YouTube, Vimeo, Vine) and linked from the \texttt{readme.md} file.

You will need to show your most recent weekly reports to your tutor during each of your fortnightly tutorials, at which point they will be signed-off.

\subsection*{Part B}

Part B must be completed as a single PDF document, prepared in LaTeX. The LaTeX source files should be hosted on GitHub in the same repository as Part A. The single PDF document must be submitted to the LearningSpace by the final submission deadline as shown on LearningSpace. Please note that the LearningSpace will only accept a single PDF document.

You will receive formal feedback three weeks after the submission deadline shown on LearningSpace.

\begin{marginquote}
    ``Remember, learning to program can take a surprising amount of time \& effort --- students may get there at different rates, but all students who put in the time \& effort get there eventually. Making good use of [reflection and deliberate practice] are an essential part of this process.''
    
    --- Professor Quintin Cutts
\end{marginquote}
\marginpicture{guitarhero}{
    Rhythm games such as \emph{Guitar Hero} and \emph{Rock Band} are excellent examples of games
    which make use of unique input devices to enhance gameplay.
}
\section*{Additional Guidance}

Reflection is taking time to examine thoughts, feelings, beliefs, values, attitudes and assumptions in the context of a specific topic, situation, problem, issue, or process. Part of reflection is relating these varied understandings to your experiences of or within the context. Then analysing that relation to work out how and why an understanding arose. Further to this is projecting that relationship into the future, drawing on the how and the why, to identify future activities. Thus, combining these leads to a plan on how to develop your knowledge and improve your skills.

A common mistake made by beginners to reflective writing is to use many words to describe the context and/or the experience. Avoid this. The description of an experience is not particularly important. It is the analysis and evaluation of that experience which is important because this will reveal interesting insights about yourself and your actions which will be useful when planning more effective actions in the future.

Computing knowledge and, in particular, programming skills require a significant investment in time and energy to develop. However, quality of practice is more important than quantity of practice. Your readiness, the suitability of the form of practice, and the relevance of the skill being practised are all important factors which affect quality. Given that you must work to tight deadlines, it is very important that time and energy is used effectively to optimise progress. A strategy that has shown success at achieving this optimisation is deliberate practice.

A continuous cycle of reflection and planning forms the cornerstone of deliberate practice. Unfortunately, however, another common mistake that students make is treating this reflection as an afterthought. This is a problem because rather than focusing on the development process, there is often too much attention on the end-product: the actual submitted work. However, it is important to consider that software development is a process. It is the experience of that process and how to change that process that is important. This is what will lead to higher quality submissions in the future and forms the underlying purpose of this assignment. Subsequently, failing to record experiences and failing to engage in regular reflection will not only slow your progress substantially but will also result in a poorly focused report.

So what is deliberate practice? Briefly, it is practice which is: conscious and intentional; designed with your current skill level in mind to force an exerted effort but avoid frustration; provides relevant and measurable feedback to track progress; and follows a repeatable structure.

A common mistake when planning such practice is being too general. It is, therefore, important to consider SMART actions: specific; measurable; achievable; relevant; and time-bound. Also note, problem solving and designing are particularly important programming skills and approaches to developing skills in these areas can be relevant.

\begin{marginquote}
    ``The first 90 percent of the code accounts for the first 90 percent of the development time.
    
    ``The remaining 10 percent of the code accounts for the other 90 percent of the development time.''
    
    --- Tom Cargill
    
    \marginquoterule
    
    ``Hofstadter's Law:
    
    ``It always takes longer than you expect, even when you take into account Hofstadter's Law.''
    
    --- Douglas Hofstadter
\end{marginquote}
\marginpicture{fishing}{
    The \emph{Dreamcast Fishing Controller}, released as a peripheral for the game \emph{Sega Bass Fishing}.
    Even peripherals which appeal to only a small audience can enjoy moderate commercial success.
}
\section*{Additional Resources}

\begin{itemize}
    \item Ericsson, K.A., Krampe, R.T., and Tesch-Romer, C. (1993)The Role of Deliberate Practice in the Acquisition of Expert Performance. Psychological Review, 100(3), 363-406.
    \item Bolton, G.E.J. (2014) Reflective Practice: Writing and Professional Development. SAGE Publications: London.
\end{itemize}

\begin{markingrubric}
%
    \criterion{Satisfactory Preparation of Weekly Reports}{15\%}
        \grade\fail 	At least one weekly report has not been submitted, is incomplete, or is unsatisfactory.
        \grade 
        \grade 
        \grade 
        \grade 
        \grade 		All weekly reports have been signed-off by your tutor by the deadline.
%
    \criterion{Appropriateness, Specificity, and Relevance of Selection of Key Skills}{10\%}
        \grade\fail 	Less than two appropriate key skills are mentioned.
        \grade 		At least two appropriate key skills are mentioned.
        \grade 		At least three appropriate key skills are mentioned.
        \grade 		At least three appropriate key skills are mentioned.
        \par 		At least two of the key skills are both specific and relevant.
        \grade 		At least three appropriate key skills are mentioned.
        \par 		At least three of the key skills are both specific and relevant.
        \grade 		At least three appropriate key skills are mentioned.
        \par 		At least two of the key skills are both specific and a priority.
%
    \criterion{Adequacy of Self-Criticism in Relation to Key Skills}{20\%}
        \grade\fail 	No self-criticism is made.
        \grade 		Little self-criticism is made.
        \grade 		Some self-criticism is made.
        \grade 		Much self-criticism is made.
        \grade 		A significant level of self-criticism is made.
            \par 		Some of the self-criticism is accurate and pertinent.
        \grade 		An exception level of self-criticism is made.
            \par 		Much of the self-criticism is accurate and pertinent.
%
    \criterion{Depth of the Reflection on the Application of Skills}{20\%}
        \grade\fail 	No reflection is evident.
        \grade 		Little reflection is evident.
        \grade 		Some reflection is evident.
        \grade 		Much reflection is evident.
        \par 		Some depth of insight is demonstrated.
        \grade 		Significant reflection is evident.
        \par 		Much depth of insight is demonstrated.
        \grade 		Exemplary reflection is evident.
        \par 		Significant depth of insight is demonstrated.
%
    \criterion{Appropriateness of Plan for Future Development}{20\%}
        \grade\fail 	No appropriate plans are proposed.
        \grade 		At least one generally appropriate plan is proposed.
        \grade 		At least two specific and achievable plans are proposed. 
        \grade 		At least three specific and achievable plans are proposed. 
        \par 		At least two of the plans are also relevant.
        \grade 		At least three specific, relevant, and achievable plans are proposed. 
        \par 		At least two of the plans are also measurable and time-bound.
        \grade 		At least three specific, measurable, achievable, relevant, and time-bound plans are proposed. 
%
    \criterion{Appropriateness of Reflective Writing Style}{5\%}
        \grade\fail 	Demonstrates no evidence of ability in reflective writing.
        \grade 		Demonstrates evidence of little ability in reflective writing.
        \grade 		Demonstrates evidence of some ability in reflective writing.  
        \grade 		Demonstrates evidence of partial mastery of reflective writing.
        \grade 		Demonstrates evidence of mastery in reflective writing.
        \grade 		Demonstrates significant evidence of mastery in reflective writing.
%
    \criterion{Appropriateness of Spelling and Grammar}{5\%}
        \grade\fail 	Substantial spelling and/or grammar errors.
        \grade 		Many spelling and/or grammar errors.
        \grade 		Some spelling and/or grammar errors.  
        \grade 		Few spelling and/or grammar errors.
        \grade 		Nearly no spelling and/or grammar errors.
        \grade 		No spelling and/or grammar errors.
%
    \criterion{Appropriateness of Essay Structure}{5\%}
        \grade\fail 	There is no structure, or the structure is unclear.
        \grade 		There is little structure.
        \grade 		There is some structure.
        \par 		A few sentences and paragraphs are well constructed.
        \grade 		There is much structure.
        \par 		Some sentences and paragraphs are well constructed.
        \par 		There is a clear introduction and conclusion.
        \grade 		There is much structure, highlighting the key skills.
        \par 		Most sentences and paragraphs are well constructed.
        \par 		There is a clear and well-constructed introduction and conclusion.
        \grade 		There is much structure, highlighting the key skills.
        \par 		All sentences and paragraphs are well constructed.
        \par 		There is a clear and well-constructed introduction and conclusion.
\end{markingrubric}

\end{document}