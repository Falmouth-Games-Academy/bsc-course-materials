\documentclass{../fal_assignment}
\graphicspath{ {../} }

\usepackage{enumitem}

\title{Interface Hacking}
\author{Dr Edward Powley}

\begin{document}

\maketitle
%\begin{marginquote}
%    ``Students come into programming classes with a broad range of backgrounds ---
%    some have experience in several programming languages, others have never programmed before in their life.
%    
%    Being able to engage with the community and support each other is important.
%    Upload your code to GitHub and receive feedback from experienced peers.
%    Review your peers' work yourself and really consider what `quality' actually means
%    and what `good' source code looks like.
%    Debate, argue, and question others about it ---
%    an open and sustained discourse is an excellent way to learn ---
%    for both beginners and adepts!''
%\end{marginquote}
\marginpicture{MakeyMakey.jpg}{
    The \emph{MaKey~MaKey} allows a multitude of materials to be used to create videogame controllers.
}
\section*{Introduction}

In this assignment, you are required to design and prototype a novel game controller device.
Your prototype should function as an input device, either for the game you developed in COMP130 last semester,
or for the game you are developing in COMP150 this semester.
Your prototype should use the \emph{MaKey~MaKey} platform, which you will be provided with in class,
to convert user actions into game inputs.

Computing for Games combines technical and creative skills in equal parts.
All of your assignments involve a mixture of the two;
in this assignment the emphasis is more on creativity.
You will build upon the technical skills you have learned so far,
combined with your own creativity and innovation,
to produce a unique creative artefact.

This assignment is formed of three parts:

\begin{enumerate}[label=\Alph*.]
    \item Prepare \textbf{at least five} weekly reports that must:
        \begin{enumerate}[label=\roman*.]
            \item Document your iterative design process
            \item Document the stages of your prototyping efforts.
                As well as text, you are strongly encouraged to make use of photographs and videos as appropriate.
        \end{enumerate}
    \item Do a thing
    \item Integrate the controller into your game, and prepare a practical demonstration.
\end{enumerate}

\section*{Submission Instructions}

\subsection*{Part A}

Formative submission: fork the GitHub project at TODO, and write your weekly reports in the \texttt{readme.md} file.
Images should be embedded in the markdown file.
If you use videos, upload them to a video sharing site (e.g.\ YouTube, Vimeo, Vine) and add a link to them.

Summative submission: create a .zip archive containing the contents of your GitHub project.
Upload the zip to LearningSpace.
If you have used videos, compress them in .avi or .mp4 format and add them to the zip.
\textbf{Images and videos that are not uploaded to LearningSpace will not be considered for marking.}

\subsection*{Part B}

Blah

\subsection*{Part C}

No formal submission is required.
Please attend the demo session (date and time to be confirmed in class)
with your prototype and an executable of your game,
and be prepared to discuss it with your tutors and peers.

\begin{marginquote}
    ``Alive with new thinking, buzzing with opportunity, connected with the best in the business,
    Falmouth University is the perfect place to start shaping your creative career.
    Thousands of people from around the globe come to us every year,
    graduating to become the brightest stars in art, design, media, performance and writing industries.

    ``Voted the number one institution worldwide across seven categories in the latest International Student Barometer poll,
    Falmouth has forged its position as one of the most highly regarded creative arts institutions across the globe.''
    
    --- Falmouth University website
\end{marginquote}
\marginpicture{guitarhero}{
    Rhythm games such as \emph{Guitar Hero} and \emph{Rock Band} are excellent examples of games
    which make use of unique input devices to enhance gameplay.
}
\clearpage
\begin{marginquote}
    ``Alive with new thinking, buzzing with opportunity, connected with the best in the business,
    Falmouth University is the perfect place to start shaping your creative career.
    Thousands of people from around the globe come to us every year,
    graduating to become the brightest stars in art, design, media, performance and writing industries.

    ``Voted the number one institution worldwide across seven categories in the latest International Student Barometer poll,
    Falmouth has forged its position as one of the most highly regarded creative arts institutions across the globe.''
    
    --- Falmouth University website
\end{marginquote}
\marginpicture{fishing}{
    The \emph{Dreamcast Fishing Controller}, released as a peripheral for the game \emph{Sega Bass Fishing}.
    Even peripherals which appeal to only a small audience can enjoy moderate commercial success.
}
\section*{Additional Guidance}

Falmouth University is nationally and internationally renowned as an arts institution.
Despite the fact that you are studying for a Bachelor of Science degree in a technical discipline,
you are still expected to strive for the same level of innovation and creative flair
as your fellow students on courses outside the Games Academy.
This assignment is an opportunity to flex your creative muscles, without getting too bogged down
in the technical aspects of writing functional and maintainable code.



We have given you some of the materials you need: a MaKey~MaKey kit, crocodile clip leads and conductive paint.
You will need to add your own materials to produce a working physical prototype.
You do not need to go to great expense for this: a ``Blue Peter'' style prototype made from
household items is fine, as is something made out of Play-Doh, Lego bricks, etc.
However you should still choose your materials carefully, as overly flimsy construction will
lose you marks for the functionality criterion.

You have been issued with version 1.2 of the MaKey~MaKey kit.
Earlier versions of the kit were based on the Arduino platform, and thus were easily reprogrammable at the firmware level
(as you may discover in the course of your background research).
Unfortunately this version is not.

Do not be constrained by the fact that the layout of the MaKey~MaKey board is reminiscent of a
classic NES-style game controller.
Instead, treat the pads as six on-off inputs to be interpreted by your game in whatever way is appropriate,
and remember that a further 12 inputs are available through the pins on the back of the MaKey~MaKey.

\section*{Additional Resources}



\begin{markingrubric}
    \criterion{Design of the solution}{15\%}
        \grade\fail F
        \grade D
        \grade C
        \grade B
        \grade A
        \grade Astar
    \criterion{Development journal}{10\%}
        \grade\fail The development journal is incomplete or not submitted.
        \grade D
        \grade C
        \grade B
        \grade A
        \grade A$^*$
    \criterion{Innovation and creative flair}{30\%}
        \grade\fail Demonstrates no evidence of innovation and/or creativity.
            \par The brief has not been followed, or the provided hardware has not been added to in any way.
        \grade Demonstrates evidence of emerging innovation and/or creativity.
            \par The solution is purely derivative of existing devices on the market.
        \grade Demonstrates evidence of progressing innovation and/or creativity.
            \par The solution is mostly derivative, with some attempts at innovation.
        \grade Demonstrates evidence of partial mastery of innovative and creative practice.
        \grade Demonstrates some evidence of mastery of innovative and creative practice.
        \grade Demonstrates much evidence of mastery of innovative and creative practice.
    \criterion{Functionality of physical prototype}{20\%}
        \grade\fail A physical prototype is not produced, or the prototype is completely non-functional.
        \grade The physical prototype is barely functional.
            \par There are serious technical and/or constructional flaws.
        \grade The physical prototype is somewhat functional.
            \par There are obvious technical and/or constructional flaws.
        \grade The physical prototype is mostly functional.
            \par There are minor technical and/or constructional flaws.
        \grade The physical prototype is functional.
            \par There are superficial technical and/or constructional flaws.
        \grade The physical prototype is functional.
            \par The technical execution and physical construction are flawless.
    \criterion{Sophistication: software, electronics, physical construction}{10\%}
        \grade\fail The solution lacks even a basic level of sophistication in any of the three areas.
        \grade The solution is basic and unsophisticated in all three areas.
            \par Little insight has been demonstrated in any area.
        \grade The solution is moderately sophisticated in one of the areas, but lacking in the other two.
            \par Emerging insight has been demonstrated in at least one of the areas.
        \grade The solution is moderately sophisticated in two of the noted areas, but lacking in the third.
            \par Much insight has been demonstrated in at least one of the areas.
        \grade The solution combines somewhat sophisticated software, electronics and physical construction.
            \par Significant insight has been demonstrated in at least two of these areas.
        \grade The solution combines highly sophisticated software, electronics and physical construction.
            \par Exemplary insight has been demonstrated in all three areas.
    \criterion{Maintainability}{10\%}
        \grade\fail F
        \grade D
        \grade C
        \grade B
        \grade A
        \grade A$^*$
    \criterion{Professional practice}{5\%}
        \grade\fail GitHub has not been used.
        \grade Material has been checked into GitHub only a few times,
            and only when required for formative or summative submission.
        \grade Material has seldom been checked into GitHub.
        \grade Material has quite regularly been checked into GitHub.
        \grade Material has regularly been checked into GitHub.
        \grade Material has regularly been checked into GitHub.
            \par There is evidence of engagement (e.g.\ voluntarily reviewing peers' code)
                within the Falmouth Games Academy
\end{markingrubric}

\end{document}
