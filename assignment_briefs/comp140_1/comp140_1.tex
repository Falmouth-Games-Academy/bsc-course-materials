\documentclass{../fal_assignment}
\graphicspath{ {../} }

\usepackage{enumitem}

\title{Interface Hacking}
\author{Dr Edward Powley}

\begin{document}

\maketitle
%\begin{marginquote}
%    ``Students come into programming classes with a broad range of backgrounds ---
%    some have experience in several programming languages, others have never programmed before in their life.
%    
%    Being able to engage with the community and support each other is important.
%    Upload your code to GitHub and receive feedback from experienced peers.
%    Review your peers' work yourself and really consider what `quality' actually means
%    and what `good' source code looks like.
%    Debate, argue, and question others about it ---
%    an open and sustained discourse is an excellent way to learn ---
%    for both beginners and adepts!''
%\end{marginquote}
\marginpicture{MakeyMakey.jpg}{
    The \emph{MaKey~MaKey} allows a multitude of materials to be used to create videogame controllers.
}
\section*{Introduction}

In this assignment, you are required to design and prototype a novel game controller device.
Your prototype should function as an input device, either for the game you developed in COMP130 last semester,
or for the game you are developing in COMP150 this semester.
Your prototype should use a hardware platform such as \emph{MaKey~MaKey}, \emph{Arduino}, \emph{Raspberry~Pi} etc,
to convert user actions into game inputs.

Computing for Games combines technical and creative skills in equal parts.
All of your assignments involve a mixture of the two;
in this assignment the emphasis is more on creativity.
You will build upon the technical skills you have learned so far,
combined with your own creativity and innovation,
to produce a unique creative artefact.

This assignment is formed of three parts:

\begin{enumerate}[label=\Alph*.]
    \item \textbf{Design} a novel game controller device.
    \item \textbf{Build} a prototype of your game controller.
        Prepare \textbf{at least five} weekly reports (one per sprint),
        including images and short videos as appropriate, that document your iterative design and prototyping process.
    \item \textbf{Integrate} the controller into your game, and prepare a practical demonstration.
\end{enumerate}

\section*{Submission Instructions}

\subsection*{Part A (Formative)}

On Trello, create a Task Board that defines the high concept and key requirements (in terms of components and user stories) of the controller. Arrange a meeting with your tutor and show the Trello Task Board to them.

\subsection*{Part A (Summative)}

Take screenshots of your Trello Task Board, and include them in the summative submission
of your weekly reports (Part~B).

\subsection*{Part B (Formative)}

Fork the GitHub project at TODO, and write your weekly reports in the \texttt{readme.md} file.
Also use this repository for any other digital materials you create
(e.g.\ source code for firmware or software, illustration files, circuit diagrams etc.).

You are strongly encouraged to make use of images (sketches, diagrams, photographs, screenshots),
code snippets and short videos to document your design and prototyping process.
Images and code snippets should be embedded directly in the \texttt{readme.md} file.
Videos should be uploaded to a video sharing site (e.g.\ YouTube, Vimeo, Vine)
and linked from the \texttt{readme.md} file.

\subsection*{Part B (Summative)}

Compress the contents of your GitHub repository as a zip file,
and upload it to the appropriate submission queue on LearningSpace.
If you have used images and videos, ensure that they are included in the zip file.
Videos should be compressed in \texttt{avi} or \texttt{mp4} format.

\textbf{Materials that are not included in the zip file will not be considered for marking.}
In particular, please remember to add any videos to your zip file before submission,
as otherwise they will not be marked.

If you include video material, its \textbf{total combined length} must not exceed \textbf{three minutes}.
Exceeding this limit will be subject to the same penalties as detailed in the course's word count policy,
available on LearningSpace.
There is \textbf{no limit} on words or static images for this assignment.

\subsection*{Part C (Formative)}

No formal submission is required.
Attend the demo session (date and time to be confirmed in class)
with your prototype and an executable of your game,
and be prepared to discuss it with your tutors and peers.

\subsection*{Part C (Summative)}

Physical submission?

\begin{marginquote}
    ``Lorem ipsum dolor sit amet, consectetur adipiscing elit. Donec bibendum orci arcu, vitae interdum odio accumsan sit amet. Proin sed leo in magna consequat semper. Vivamus nec consequat nulla, id dignissim orci. Integer est neque, efficitur at sodales eget, scelerisque non justo. Donec dictum lorem est, quis euismod nibh accumsan eu. Duis enim sem, dignissim in laoreet sed, volutpat aliquet justo.
    
    ``In rhoncus elementum est dignissim volutpat. Quisque eget sem vehicula, placerat ipsum vel, posuere sem. Proin porta posuere facilisis. Nullam eu auctor nunc. Suspendisse potenti. Suspendisse at scelerisque eros, sodales interdum ligula. Pellentesque habitant morbi tristique senectus et netus et malesuada fames ac turpis egestas. Curabitur consequat imperdiet libero, sed consectetur felis consectetur at. Nulla nec libero lacus.''
\end{marginquote}
\marginpicture{guitarhero}{
    Rhythm games such as \emph{Guitar Hero} and \emph{Rock Band} are excellent examples of games
    which make use of unique input devices to enhance gameplay.
}
\clearpage
\begin{marginquote}
    ``Alive with new thinking, buzzing with opportunity, connected with the best in the business,
    Falmouth University is the perfect place to start shaping your creative career.
    Thousands of people from around the globe come to us every year,
    graduating to become the brightest stars in art, design, media, performance and writing industries.

    ``Voted the number one institution worldwide across seven categories in the latest International Student Barometer poll,
    Falmouth has forged its position as one of the most highly regarded creative arts institutions across the globe.''
    
    --- Falmouth University website
\end{marginquote}
\marginpicture{fishing}{
    The \emph{Dreamcast Fishing Controller}, released as a peripheral for the game \emph{Sega Bass Fishing}.
    Even peripherals which appeal to only a small audience can enjoy moderate commercial success.
}
\section*{Additional Guidance}

Falmouth University is nationally and internationally renowned as an arts institution.
Despite the fact that you are studying for a Bachelor of Science degree in a technical discipline,
you are still expected to strive for the same level of innovation and creative flair
as your fellow students on courses outside the Games Academy.
This assignment is an opportunity to flex your creative muscles, without getting too bogged down
in the technical aspects of writing functional and maintainable code.



We have given you some of the materials you need: a MaKey~MaKey kit, crocodile clip leads and conductive paint.
You will need to add your own materials to produce a working physical prototype.
You do not need to go to great expense for this: a ``Blue Peter'' style prototype made from
household items is fine, as is something made out of Play-Doh, Lego bricks, etc.
However you should still choose your materials carefully, as overly flimsy construction will
lose you marks for the functionality criterion.
You may also wish to connect electronic components such as LEDs, buzzers, photoresistors etc to the MaKey~MaKey,
or even use a different, more flexible hardware platform such as Arduino\footnote{
Note that the MaKey~MaKey kits provided in class are version~1.2, which, unlike earlier versions, is not based on Arduino.
Any tutorials you may find online for reprogramming the MaKey~MaKey firmware are unfortunately not applicable
to this version.
}.

Do not be constrained by the fact that the layout of the MaKey~MaKey board is reminiscent of a
classic NES-style game controller.
Instead, treat the pads as six on-off inputs to be interpreted by your game in whatever way is appropriate,
and remember that a further 12 inputs are available through the pins on the back of the MaKey~MaKey.

\section*{Additional Resources}



\begin{markingrubric}
    \criterion{Design of the solution}{15\%}
        \grade\fail User stories are not provided.
        \grade D
        \grade C
        \grade B
        \grade A
        \grade Astar
    \criterion{Weekly reports}{10\%}
        \grade\fail Fewer than three weekly reports are submitted,
            and/or the weekly reports are inappropriate.
        \grade At least three weekly reports are submitted.
            \par The reports document the design and prototyping process in a basic manner.
            \par The reports do not use images or video.
        \grade At least four weekly reports are submitted.
            \par The reports document the design and prototyping process in a satisfactory manner.
            \par At least one relevant image or video is included per report.
        \grade At least five weekly reports are submitted.
            \par TODO
        \grade At least five weekly reports are submitted.
            \par TODO
        \grade At least five weekly reports are submitted.
            \par The reports document the design and prototyping process in an exemplary manner.
            \par The reports use images and video in an exemplary manner.
    \criterion{Innovation and creative flair}{30\%}
        \grade\fail Demonstrates no evidence of innovation and/or creativity.
            \par The brief has not been followed.
        \grade Demonstrates evidence of emerging innovation and/or creativity.
            \par The solution is purely derivative of existing products, .
        \grade Demonstrates evidence of progressing innovation and/or creativity.
            \par The solution is mostly derivative, with some attempts at innovation.
        \grade Demonstrates evidence of partial mastery of innovative and creative practice.
            \par The solution is an interesting and somewhat innovative product.
        \grade Demonstrates some evidence of mastery of innovative and creative practice.
            \par The solution is a novel and innovative product.
        \grade Demonstrates much evidence of mastery of innovative and creative practice.
            \par The solution is a unique and innovative product.
    \criterion{Functionality of physical prototype}{20\%}
        \grade\fail A physical prototype is not produced, or the prototype is completely non-functional.
        \grade The physical prototype is barely functional.
            \par There are serious technical and/or constructional flaws.
        \grade The physical prototype is somewhat functional.
            \par There are obvious technical and/or constructional flaws.
        \grade The physical prototype is mostly functional.
            \par There are minor technical and/or constructional flaws.
        \grade The physical prototype is functional.
            \par There are superficial technical and/or constructional flaws.
        \grade The physical prototype is functional.
            \par The technical execution and physical construction are flawless.
    \criterion{Sophistication: \par Software \par Electronics \par Physical construction}{10\%}
        \grade\fail The solution lacks even a basic level of sophistication in any of the three areas.
        \grade The solution is basic and unsophisticated in all three areas.
            \par Little insight has been demonstrated in any area.
        \grade The solution is moderately sophisticated in one of the areas, but lacking in the other two.
            \par Emerging insight has been demonstrated in at least one of the areas.
        \grade The solution is moderately sophisticated in two of the noted areas, but lacking in the third.
            \par Much insight has been demonstrated in at least one of the areas.
        \grade The solution combines somewhat sophisticated software, electronics and physical construction.
            \par Significant insight has been demonstrated in at least two of these areas.
        \grade The solution combines highly sophisticated software, electronics and physical construction.
            \par Exemplary insight has been demonstrated in all three areas.
    \criterion{Maintainability}{10\%}
        \grade\fail F
            \par Source code is unmaintainable.
        \grade D
            \par Source code is not well structured.
                Names are unclear. Comments are not used.
        \grade C
            \par Source code is not well structured.
                Names are sometimes unclear. Comments are used.
        \grade B
            \par Source code is reasonably well structured.
                Names are mostly clear. Comments are reasonably effective.
        \grade A
            \par Source code is well structured, in an attempt to achieve high cohesion and low coupling.
                Names are clear. Comments are effective.
        \grade A$^*$
            \par Source code is extremely well structured, with high cohesion and low coupling.
                Names are very clear. Comments are exemplary.
    \criterion{Professional practice}{5\%}
        \grade\fail GitHub has not been used.
        \grade Material has only been checked into GitHub when required for review or submission.
        \grade Material has been checked into GitHub at least once per sprint.
        \grade Material has been checked into GitHub several times per sprint.
        \grade Material has been checked into GitHub several times per sprint.
            \par Commit messages are clear and concise.
        \grade Material has been checked into GitHub several times per sprint.
            \par Commit messages are clear and concise.
            \par There is evidence of engagement (e.g.\ voluntarily reviewing peers' code)
                within the Academy.
\end{markingrubric}

\end{document}
