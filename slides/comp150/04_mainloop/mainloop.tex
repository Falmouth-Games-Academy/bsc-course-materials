\part{The main loop}
\frame{\partpage}

\begin{frame}{Basic program architecture}
    \begin{itemize}
        \item CPUs execute sequences of instructions
    \end{itemize}
\end{frame}

\begin{frame}{The basic main loop}
    The most basic main game loop does \textbf{three} things:
    \begin{enumerate}
        \item Handle \textbf{input}
            \begin{itemize}
                \item Mouse, keyboard, joypad etc.
                \item Operating system events (minimise, close, alt+tab etc.)
            \end{itemize}
        \item \textbf{Update} the state of the game
            \begin{itemize}
                \item Physics, collision detection, AI etc.
            \end{itemize}
        \item \textbf{Render} the game to the screen
    \end{enumerate}
    It does these \textbf{once per frame} (typically 30 or 60 times per second)
\end{frame}

\begin{frame}[fragile]{The basic main loop}
    \begin{lstlisting}
bool running = true;

while (running)
{
    handleInput();
    update();
    render();
}
    \end{lstlisting}
\end{frame}

\begin{frame}{Handling input}
    There are two ways of handling input in a game:
    \begin{itemize}
        \item By handling events
            \begin{itemize}
                \item \lstinline{SDL_PollEvent}
                \item See \url{https://wiki.libsdl.org/SDL_EventType} for a list of event types
            \end{itemize}
        \item By querying state
            \begin{itemize}
                \item \lstinline{SDL_GetKeyboardState}, \lstinline{SDL_GetMouseState},
                    \lstinline{SDL_GameControllerGetAxis}, etc.
            \end{itemize}
    \end{itemize}
    What's the difference?
    \begin{itemize}
        \item Event: ``The space bar was [pressed / released]''
        \item State: ``The space bar is [down / up] right now''
    \end{itemize}
\end{frame}

\begin{frame}{Updating the game state}
    \begin{itemize}
        \item Generally this is where your \textbf{game logic} is implemented
        \item I.e.\ anything not directly related to input or graphics
        \item What goes in here depends on the game...
    \end{itemize}
\end{frame}

\begin{frame}{Rendering}
    \begin{itemize}
        \item This is where you draw the \textbf{current state of the game} to the screen
        \item Also draw any \textbf{heads-up display (HUD)} elements, e.g.\ score, lives, mini-map, etc.
        \item Graphical effects (animations, particles) may be handled \textbf{either} in the render step
            \textbf{or} in the update step (but be consistent)
        \item In frameworks like SDL, you generally \textbf{redraw everything} on every frame
        \item Rendering in SDL is \textbf{double buffered}
            \begin{itemize}
                \item \lstinline{SDL_Render*} actually draws to an \textbf{off-screen buffer}
                \item \lstinline{SDL_RenderPresent} displays the off-screen buffer on screen
            \end{itemize}
    \end{itemize}
\end{frame}

\begin{frame}{Screen refresh rate}
    \begin{itemize}
        \item Old CRT monitors worked by scanning an electron beam down the screen
            \begin{itemize}
                \item \url{https://www.youtube.com/watch?v=lRidfW_l4vs}
            \end{itemize}
        \item Hence the term \textbf{(vertical) refresh rate}
        \item Refresh rate is measured in \textbf{cycles per second} i.e.\ \textbf{Hz}
        \item Other monitor technologies work differently, but still refresh the screen at regular intervals
        \item We generally want to sync it up so that
    \end{itemize}
    \begin{center}
        \textbf{one display refresh = one main loop iteration}
    \end{center}
    \begin{itemize}
        \item If the main loop runs too slowly, we get ``lag''
        \item If the main loop runs too quickly, we waste resources on drawing things faster than the display can show them
    \end{itemize}
\end{frame}

\begin{frame}{Limiting the frame rate}
    \begin{itemize}
        \item If the renderer was created with the \lstinline{SDL_RENDERER_PRESENTVSYNC} flag,
            \lstinline{SDL_RenderPresent} waits for the next vertical blank
        \item This limits the game's frame rate to the refresh rate of the device
        \item However, refresh rates can vary
            \begin{itemize}
                \item Older TVs: $\sim$ 30Hz
                \item HDTVs and standard monitors: 60Hz
                \item High-end ``gaming'' monitors: 120Hz or higher
            \end{itemize}
    \end{itemize}
\end{frame}

\begin{frame}{Socrative \texttt{6E8NSW3IN}}
    Why might updating the game state once per frame be undesirable?
	\begin{itemize}
		\item In pairs.
		\item Discuss for 2-minutes.
		\item \textbf{Suggest} an undesirable effect that might result from updating the game state exactly once per frame.
	\end{itemize}
\end{frame}

\begin{frame}[fragile]{Measuring elapsed time}
    \begin{lstlisting}
bool running = true;
Uint32 lastTime = SDL_GetTicks();

while (running)
{
    Uint32 currentTime = SDL_GetTicks();
    Uint32 deltaTime = currentTime - lastTime;
    // deltaTime is the number of milliseconds since the last update
    
    handleInput();
    update(deltaTime);
    render(deltaTime);
    
    lastTime = currentTime;
}
    \end{lstlisting}
\end{frame}

