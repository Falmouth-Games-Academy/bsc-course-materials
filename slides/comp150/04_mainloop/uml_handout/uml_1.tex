\documentclass{scrartcl}

\usepackage[hidelinks]{hyperref}
\usepackage[none]{hyphenat}
\usepackage{setspace}
\usepackage{enumitem}
\setlist{nosep} % Make enumerate / itemize lists more closely spaced
\doublespace

\begin{document}

\title{UML Worksheet I}
\author{COMP150: Game Development Practice}
\date{}

\maketitle

Unified Modelling Language (UML) is a way of communicating the design of software using diagrams. It is a notation that built upon the work of Grady Booch, James Rumbaugh, Ivar Jacobson, and the Rational Software Corporation. It was originally developed to support the object-oriented paradigm, although has since been extended to accommodate a diverse range of projects. According to the Object Management Group (OMG), UML is the international standard for software modelling.

\section{In-Class Task}

In today's in-class task you will learn how to draw \textbf{UML Use-Case} and \textbf{UML Class} diagrams. To complete this you will:

\begin{itemize}
	\item \textbf{Organise} yourselves into your COMP150 project teams.
	\item \textbf{Watch} the video tutorial at \url{https://www.youtube.com/watch?v=OkC7HKtiZC0}.
	\item \textbf{Read} \url{http://www.tutorialspoint.com/uml/uml_use_case_diagram.htm}.
	\item \textbf{Draw} a UML Use-Case diagram to model ONE part (e.g. AI agent) of your game.
	\item \textbf{Watch} the video tutorial at \url{https://www.youtube.com/watch?v=3cmzqZzwNDM}.
	\item \textbf{Read} \url{http://www.tutorialspoint.com/uml/uml_class_diagram.htm}.
	\item \textbf{Draw} a UML Class diagram to model the intended outcome of the FIRST sprint.
\end{itemize}

\vspace{1ex}

Use the white boards to draw your diagrams.
 
Alternatively, use Gliffy: \url{https://www.gliffy.com/uses/uml-software/}

\section{Submission}

 This task is not assessed, but will help you on your COMP150 project. Add the diagram to your weekly report for your COMP150 project. This will be reviewed at the next Sprint Review.

\end{document}