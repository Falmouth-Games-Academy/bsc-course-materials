% Uncomment this line for on-screen presentation
\documentclass[xcolor={dvipsnames}]{beamer}\usepackage{etoolbox}\newtoggle{printable}\togglefalse{printable}

% Uncomment this line for printable slides (disable animations and don't waste ink)
%\documentclass[handout, xcolor={dvipsnames}]{beamer}\usepackage{etoolbox}\newtoggle{printable}\toggletrue{printable}

% Adjust these for the path of the theme and its graphics, relative to this file
%\usepackage{beamerthemeFalmouthGamesAcademy}
\usepackage{../../beamerthemeFalmouthGamesAcademy}
\graphicspath{ {../../} }

% Default language for code listings
\lstset{language=C++,
		morekeywords={each,in}
}

\begin{document}
\title{Heuristic Analysis}   
\subtitle{COMP140: Creative Computing Hacking}

\frame{\titlepage} 

\begin{frame}{Lecture Objectives}
	Today's lecture will build upon the practical design of your game controller, focusing on:
	
	\begin{itemize}
		\item Practical guidelines on one design evaluation technique: heuristic analysis
	\end{itemize}
	
	This will be followed up by a practical in which you will identify heuristics and apply them to a peer's game interface.
\end{frame}

\begin{frame}{Important Notice}
	\begin{columns}[onlytextwidth]
		\begin{column}{0.45\textwidth}
			\includegraphics[height=22ex]{MakeyMakey.jpg}
		\end{column}
		\begin{column}{0.45\textwidth}
			Remember to bring your \textit{Makey Makey} kit and associated materials to these lectures for practical 
			support toward the end of each of these sessions.
		\end{column}
	\end{columns}
\end{frame}

%\part{Input, Output, and Interaction Styles}
\frame{\partpage}

\begin{frame}{Learning Outcomes}
	In this section you will learn how to...
	
	\begin{itemize}
		\item \textbf{Explain} the role of input and output in systems design
		\item \textbf{List} and \textbf{describe} a variety of input and output devices, giving examples of situations where each may be appropriate
		\item \textbf{Explain} what interaction styles are, while \textbf{critically evaluating} their respective advantages and disadvantages
		\item \textbf{Discuss} the role of direct manipulation in interacting with current computer systems
	\end{itemize}
\end{frame}

\begin{frame}{Further Reading}
	\begin{itemize}
		\item Shneiderman, B. (1998) \textit{Designing the User Interface: Strategies for Effective Human-Computer Interaction}. 3rd Edition. Addison Wesley.
	\end{itemize}
\end{frame}

\begin{frame}{Input and Output Technologies}
	\begin{itemize}
		\item The cognitive approach is currently the dominant framework (or paradigm) for HCI (Perry, 2006).
		\item Players are characterised as `information processors', in which information undergoes a series of ordered processes
		in the player's mind.
		\item This worldview draws a comparison between the human brain and a computer; we can therefore model player activity in the same
		way that we model computer processing.
	\end{itemize}
\end{frame}

\begin{frame}[fragile]{Socrative \texttt{JBYPC3BBY}}
	\begin{itemize}
		\item In pairs.
		\item Quietly discuss what you think is meant by the term `cognition' for 2-minutes.
		\item \textbf{Explain} cognition in your own words.
	\end{itemize}
\end{frame}

\begin{frame}{Interaction Styles}
	\begin{itemize}
		\item The cognitive approach is currently the dominant framework (or paradigm) for HCI (Perry, 2006).
		\item Players are characterised as `information processors', in which information undergoes a series of ordered processes
		in the player's mind.
		\item This worldview draws a comparison between the human brain and a computer; we can therefore model player activity in the same
		way that we model computer processing.
	\end{itemize}
\end{frame}

\begin{frame}[fragile]{Socrative \texttt{JBYPC3BBY}}
	\begin{itemize}
		\item In pairs.
		\item Quietly discuss what you think is meant by the term `cognition' for 2-minutes.
		\item \textbf{Explain} cognition in your own words.
	\end{itemize}
\end{frame}

%\part{Prototyping}
\frame{\partpage}

\begin{frame}{Learning Outcomes}
	In this section you will learn how to...
	
	\begin{itemize}
		\item \textbf{Explain} the role of prototyping in game interface design
		\item \textbf{Compare} different approaches to prototyping
		\item \textbf{Select} an appropriate prototyping method for particular usability challenges
	\end{itemize}
\end{frame}

\begin{frame}{Further Reading}
	\begin{itemize}
		\item Jensen, S. (2002) \textit{The Simplicity Shift}. Cambridge University Press.
	\end{itemize}
\end{frame}

\begin{frame}{The Value of Prototyping}
	\begin{itemize}
		\item The cognitive approach is currently the dominant framework (or paradigm) for HCI (Perry, 2006).
		\item Players are characterised as `information processors', in which information undergoes a series of ordered processes
		in the player's mind.
		\item This worldview draws a comparison between the human brain and a computer; we can therefore model player activity in the same
		way that we model computer processing.
	\end{itemize}
\end{frame}

\begin{frame}[fragile]{Socrative \texttt{JBYPC3BBY}}
	\begin{itemize}
		\item In pairs.
		\item Quietly discuss what you think is meant by the term `cognition' for 2-minutes.
		\item \textbf{Explain} cognition in your own words.
	\end{itemize}
\end{frame}

\begin{frame}{Searching the Design Space}
	\begin{itemize}
		\item The cognitive approach is currently the dominant framework (or paradigm) for HCI (Perry, 2006).
		\item Players are characterised as `information processors', in which information undergoes a series of ordered processes
		in the player's mind.
		\item This worldview draws a comparison between the human brain and a computer; we can therefore model player activity in the same
		way that we model computer processing.
	\end{itemize}
\end{frame}

\begin{frame}[fragile]{Socrative \texttt{JBYPC3BBY}}
	\begin{itemize}
		\item In pairs.
		\item Quietly discuss what you think is meant by the term `cognition' for 2-minutes.
		\item \textbf{Explain} cognition in your own words.
	\end{itemize}
\end{frame}

\begin{frame}{Approaches to Prototype Development}
	\begin{itemize}
		\item The cognitive approach is currently the dominant framework (or paradigm) for HCI (Perry, 2006).
		\item Players are characterised as `information processors', in which information undergoes a series of ordered processes
		in the player's mind.
		\item This worldview draws a comparison between the human brain and a computer; we can therefore model player activity in the same
		way that we model computer processing.
	\end{itemize}
\end{frame}

\begin{frame}[fragile]{Socrative \texttt{JBYPC3BBY}}
	\begin{itemize}
		\item In pairs.
		\item Quietly discuss what you think is meant by the term `cognition' for 2-minutes.
		\item \textbf{Explain} cognition in your own words.
	\end{itemize}
\end{frame}

\begin{frame}{Best Practices and Pitfalls}
	\begin{itemize}
		\item The cognitive approach is currently the dominant framework (or paradigm) for HCI (Perry, 2006).
		\item Players are characterised as `information processors', in which information undergoes a series of ordered processes
		in the player's mind.
		\item This worldview draws a comparison between the human brain and a computer; we can therefore model player activity in the same
		way that we model computer processing.
	\end{itemize}
\end{frame}

\begin{frame}[fragile]{Socrative \texttt{JBYPC3BBY}}
	\begin{itemize}
		\item In pairs.
		\item Quietly discuss what you think is meant by the term `cognition' for 2-minutes.
		\item \textbf{Explain} cognition in your own words.
	\end{itemize}
\end{frame}

\part{Heuristic Analysis}
\frame{\partpage}

\begin{frame}{Learning Outcomes}
	In this section you will learn how to...
	
	\begin{itemize}
		\item \textbf{Explain} what heuristic analysis is
		\item \textbf{Recognise} key heuristics for game interfaces
		\item \textbf{Describe} the application of heuristic analysis to game interfaces
	\end{itemize}
\end{frame}

\begin{frame}{Further Reading}
	\begin{itemize}
		\item Nielsen, J. (1993) \textit{Usability Evaluation}. Academic Press.
		\item Pinelle D., Wong N., and Stach, T. (2008) `Heuristic Evaluation for Games: Usability Principles for Video Game Design'. In \textit{Proceedings of the SIGCHI Conference on Human Factors in Computing Systems}. ACM, pp. 1453-1462. 
	\end{itemize}
\end{frame}

\begin{frame}{Usability Evaluation}
	\begin{itemize}
		\item Experts use their knowledge of users and technology to review software usability.
		\item Expert critiques can be formal or informal reports.
		\item Heuristic evaluation is a review guided by a set of heuristics.
		\item A `heuristic' is a mental shortcut that allows people to solve problems and make judgments quickly 
		and efficiently. These rule-of-thumb strategies shorten decision-making time and allow people to function 
		without constantly stopping to think about their next course of action.
	\end{itemize}
\end{frame}

\begin{frame}[fragile]{Socrative \texttt{JBYPC3BBY}}
	\begin{itemize}
		\item In pairs.
		\item Quietly discuss why it is important to `evaluate' a game controller design for 2-minutes.
		\item \textbf{Explain why} evaluation is important in your own words.
	\end{itemize}
\end{frame}


\begin{frame}{Heuristic Analysis}
	\begin{itemize}
		\item Previously, published usability guidelines had hundreds or thousands of rules.
		\item Tended to be inflexible and many rules were context-specific.
		\item HCI scholars desired more fundamental and elegant principles.
	\end{itemize}
\end{frame}

\begin{frame}{Heuristic Analysis}
	\begin{itemize}
		\item In 1990s Jacob Nielsen conducted empirical analyses of key usability challenges.
		\item From this, he and other scholars derived sets of heuristics.
		\item Heuristics tend to be adaptable as technology changes and new evidence is found.
	\end{itemize}
\end{frame}

\begin{frame}{Heuristic Analysis}
Nielsen's (1990) Original Heuristics:

	\begin{columns}[onlytextwidth]
		\begin{column}{0.45\textwidth}
			\begin{itemize}
				\item Simple and Natural Dialogue.
				\item Speak the User's Language.
				\item Minimize memory load.
				\item Consistency.
				\item Feedback.
			\end{itemize}
		\end{column}
		\begin{column}{0.45\textwidth}				
			\begin{itemize}
				\item Clearly Marked Exits.
				\item Shortcuts.
				\item Good Error Messages.
				\item Prevent Errors.
				\item Help and Documentation.
			\end{itemize}
		\end{column}
	\end{columns}
\end{frame}

\begin{frame}{Heuristic Analysis}
Nielsen's (1993) Revised Heuristics:

	\begin{columns}[onlytextwidth]
		\begin{column}{0.45\textwidth}
			\begin{itemize}
				\item Visibility of System Status.
				\item Match Between System and the Real World.
				\item User Control and Freedom.
				\item Consistency.
				\item Error Prevention.
			\end{itemize}
		\end{column}
		\begin{column}{0.45\textwidth}				
			\begin{itemize}
				\item Recognition Over Recall.
				\item Flexibility and Efficiency.
				\item Aesthetic and Minimalist Design.
				\item Help Users Diagnose and Recover from Errors.
				\item Help and Documentation.
			\end{itemize}
		\end{column}
	\end{columns}
\end{frame}

\begin{frame}{Heuristic Analysis}
Shneiderman \textit{et al} (2010) 8 ``Golden'' Rules:

	\begin{columns}[onlytextwidth]
		\begin{column}{0.45\textwidth}
			\begin{itemize}
				\item Strive for Consistency.
				\item Cater to Universal Usability
				\item Offer Informative Feedback.
				\item Design Dialogs to Yield Closure.
			\end{itemize}
		\end{column}
		\begin{column}{0.45\textwidth}				
			\begin{itemize}
				\item Prevent Errors.
				\item Permit Easy Reversal of Actions.
				\item Support Internal Locus of Control.
				\item Reduce Short-Term Memory Load.
			\end{itemize}
		\end{column}
	\end{columns}
\end{frame}

\begin{frame}[fragile]{Socrative \texttt{JBYPC3BBY}}
	\begin{itemize}
		\item In pairs.
		\item Quietly discuss how game controller design for 10-minutes.
		\item \textbf{List ONE} adapted heuristic \textbf{and explain why} it should be included.
	\end{itemize}
\end{frame}

\begin{frame}{Heuristic Analysis}
Pinelle \textit{et al's} (2008) Game Heuristics:

	\begin{columns}[onlytextwidth]
		\begin{column}{0.45\textwidth}
			\begin{itemize}
				\item Consistent Responses to Player's Actions.
				\item Permit Customisation of Game Settings.
				\item Predictable and Reasonable Agent Behaviours.
			\end{itemize}
		\end{column}
		\begin{column}{0.45\textwidth}				
			\begin{itemize}
				\item Provide Manageable Controls with Appropriate Sensitivity and Responsiveness.
				\item Provide Clear Information on Game Status.
				\item Provide Instruction, Training, and Help.
			\end{itemize}
		\end{column}
	\end{columns}
\end{frame}

\begin{frame}{Heuristic Analysis}
Pinelle \textit{et al's} (2008) Game Heuristics:

	\begin{columns}[onlytextwidth]
		\begin{column}{0.45\textwidth}
			\begin{itemize}
				\item Unobstructed Views of Information Needed to Inform Player Actions.
				\item Permit Skipping of Non-Playable Content.
				\item Intuitive and Customizable Input Mappings.
			\end{itemize}
		\end{column}
		\begin{column}{0.45\textwidth}				
			\begin{itemize}
				\item Provide Easily Interpreted Visual Representations
				\item Minimize the Need for Micromanagement.
			\end{itemize}
		\end{column}
	\end{columns}
\end{frame}

\begin{frame}{Heuristic Analysis Method}
	\begin{itemize}
		\item Heuristic evaluation is referred to as `discount' evaluation when 5 evaluators are used
		\item Empirical evidence suggests that on average 5 evaluators identify 75-80\% of usability problems.
	\end{itemize}
\end{frame}



\begin{frame}{Heuristic Analysis Method}
	\begin{itemize}
		\item Select set of heuristics
		\item Brief evaluators
	\end{itemize}
\end{frame}

\begin{frame}{Heuristic Analysis Method}
	\begin{itemize}
		\item Conduct evaluation:
			\begin{itemize}
				\item Each of the 5 evaluators works and takes notes separately.
				\item Each evaluator takes one overall pass of the system to get a feel for it.
				\item Each evaluator takes a second pass to focus on specific features.
				\item If done well, should takes approximately 1-2 hours.
			\end{itemize}
	\end{itemize}
\end{frame}

\begin{frame}{Heuristic Analysis Method}
	\begin{itemize}
		\item Conduct debriefing session:
			\begin{itemize}
				\item The 5 evaluators work together to identify and prioritise problems.
				\item They then report their results collectively to the designer.
				\item If done well, should takes approximately 1 hour.
			\end{itemize}
	\end{itemize}
\end{frame}

\begin{frame}{Heuristic Analysis Method}
Advantages of this approach include (Budd, 2007):

	\begin{itemize}
		\item Few ethical and practical issues as players not involved.
		\item Computing professionals are readily available.
		\item Excellent cost-benefit ratio.
	\end{itemize}
\end{frame}

\begin{frame}{Heuristic Analysis Method}
Disadvantages of this approach include (Budd, 2007):

	\begin{itemize}
		\item Variable quality as best experts require in-depth knowledge of genre and target players.
		\item Important problems are sometimes missed.
		\item Many trivial problems are often identified and over-emphasised.
		\item Experts have biases.
		\item Experts disagree with each other.
	\end{itemize}
\end{frame}

% Move this to next week
%\part{Variables and types}
\frame{\partpage}

\begin{frame}[fragile]{Function definitions}
    \begin{itemize}
        \item We have already seen an example of a function definition
    \end{itemize}
    \begin{lstlisting}
int main()
{
    std::cout << "Hello, world!" << std::endl;
    return 0;
}
    \end{lstlisting}
    \begin{itemize}
        \item The function \lstinline{main} takes no parameters, and returns a value of type \lstinline{int}
    \end{itemize}
\end{frame}

\begin{frame}[fragile]{Function signatures}
    \begin{itemize}
        \item The \textbf{signature} of a function defines its return type, name, and parameters
    \end{itemize}
    \begin{lstlisting}
double foo(std::string x, int y, bool z)
    \end{lstlisting}
    \pause
    \begin{itemize}
        \item This function takes three parameters: \pause
        \lstinline{x} of type \lstinline{std::string}, \pause
        \lstinline{y} of type \lstinline{int}, \pause
        and \lstinline{z} of type \lstinline{bool} \pause
        \item It returns a value of type \lstinline{double}
    \end{itemize}
\end{frame}

\begin{frame}[fragile]{Functions without return values}
    \begin{itemize}
        \item It is possible to define a function which does not return a value, using the \lstinline{void} keyword
        in place of its return type
    \end{itemize}
    \pause
    \begin{lstlisting}
void printNumber(int n)
{
    std::cout << n << std::endl;
}
    \end{lstlisting}
\end{frame}

\begin{frame}[fragile]{Pass by value}
    \begin{itemize}
        \item Function parameters are passed \textbf{by value}:
        the function receives \textbf{copies} of the original variables
    \end{itemize}
    \pause
    \begin{lstlisting}
void changeName(std::string name)
{
    name = "Ed";
}

int main()
{
    std::string name = "Mike";
    std::cout << name << std::endl;
    changeName();
    std::cout << name << std::endl;
}
    \end{lstlisting}
\end{frame}

\begin{frame}[fragile]{Pass by reference}
    \begin{itemize}
        \item Parameters can be passed \textbf{by reference} using the \lstinline{&}, allowing the function to modify them
    \end{itemize}
    \pause
    \begin{lstlisting}
void changeName(std::string& name)
{
    name = "Ed";
}

int main()
{
    std::string name = "Mike";
    std::cout << name << std::endl;
    changeName();
    std::cout << name << std::endl;
}
    \end{lstlisting}
\end{frame}

\begin{frame}[fragile]{Constant references}
    \begin{lstlisting}
void greet(std::string name)
{
    std::cout << "Hi " << name << std::endl;
}
    \end{lstlisting}
    \begin{itemize}
        \item Pass by value --- the string will be copied in order to be passed in
        \item More efficient to pass a reference, and mark it \lstinline{const} to prevent accidental modification
    \end{itemize}
    \begin{lstlisting}
void greet(const std::string& name)
{
    std::cout << "Hi " << name << std::endl;
}
    \end{lstlisting}
    \begin{itemize}
        \item (this is only worthwhile for large data structures like strings and vectors, not for basic data types)
    \end{itemize}
\end{frame}



\part{Practical Activity}
\frame{\partpage}

\begin{frame}{Heuristic Analysis Task}
	\begin{itemize}
		\item \textbf{Review} the heuristics at \url{https://www.nngroup.com/articles/ten-usability-heuristics/}.
		\item Self-organise into pairs.
		\item \textbf{Setup} your game and novel game controller.
		\item \textbf{Demonstrate} the prototype to a peer.
		\item \textbf{Conduct} a heurstic analysis of your peer's game interface, following the guidence at:
		\url{https://www.nngroup.com/articles/how-to-conduct-a-heuristic-evaluation/}
	\end{itemize}
\end{frame}

\begin{frame}{Coursework Progress}
	\begin{itemize}
		\item \textbf{Complete} the heuristic analysis.
		\item \textbf{Complete} the final version of the prototype game controller. 
		\item \textbf{Prepare} for final submission on LearningSpace. 
	\end{itemize}
\end{frame}

% -------------------------------------------------------

%\part{The compiler}
%\frame{\partpage}
%
%\begin{frame}
%	\frametitle{The build process}
%	\includegraphics[height=\textwidth,angle=90]{compiler_sketch}
%\end{frame}

\end{document}
