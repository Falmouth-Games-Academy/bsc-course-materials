\part{Heuristic Analysis}
\frame{\partpage}

\begin{frame}{Learning Outcomes}
	In this section you will learn how to...
	
	\begin{itemize}
		\item \textbf{Explain} what heuristic analysis is
		\item \textbf{Recognise} key heuristics for game interfaces
		\item \textbf{Describe} the application of heuristic analysis to game interfaces
	\end{itemize}
\end{frame}

\begin{frame}{Further Reading}
	\begin{itemize}
		\item Nielsen, J. (1993) \textit{Usability Evaluation}. Academic Press.
		\item Pinelle D., Wong N., and Stach, T. (2008) `Heuristic Evaluation for Games: Usability Principles for Video Game Design'. In \textit{Proceedings of the SIGCHI Conference on Human Factors in Computing Systems}. ACM, pp. 1453-1462. 
	\end{itemize}
\end{frame}

\begin{frame}{Usability Evaluation}
	\begin{itemize}
		\item Experts use their knowledge of users and technology to review software usability.
		\item Expert critiques can be formal or informal reports.
		\item Heuristic evaluation is a review guided by a set of heuristics.
		\item A `heuristic' is a mental shortcut that allows people to solve problems and make judgments quickly 
		and efficiently. These rule-of-thumb strategies shorten decision-making time and allow people to function 
		without constantly stopping to think about their next course of action.
	\end{itemize}
\end{frame}

\begin{frame}[fragile]{Socrative \texttt{JBYPC3BBY}}
	\begin{itemize}
		\item In pairs.
		\item Quietly discuss why it is important to `evaluate' a game controller design for 2-minutes.
		\item \textbf{Explain why} evaluation is important in your own words.
	\end{itemize}
\end{frame}


\begin{frame}{Heuristic Analysis}
	\begin{itemize}
		\item Previously, published usability guidelines had hundreds or thousands of rules.
		\item Tended to be inflexible and many rules were context-specific.
		\item HCI scholars desired more fundamental and elegant principles.
	\end{itemize}
\end{frame}

\begin{frame}{Heuristic Analysis}
	\begin{itemize}
		\item In 1990s Jacob Nielsen conducted empirical analyses of key usability challenges.
		\item From this, he and other scholars derived sets of heuristics.
		\item Heuristics tend to be adaptable as technology changes and new evidence is found.
	\end{itemize}
\end{frame}

\begin{frame}{Heuristic Analysis}
Nielsen's (1990) Original Heuristics:

	\begin{columns}[onlytextwidth]
		\begin{column}{0.45\textwidth}
			\begin{itemize}
				\item Simple and Natural Dialogue.
				\item Speak the User's Language.
				\item Minimize memory load.
				\item Consistency.
				\item Feedback.
			\end{itemize}
		\end{column}
		\begin{column}{0.45\textwidth}				
			\begin{itemize}
				\item Clearly Marked Exits.
				\item Shortcuts.
				\item Good Error Messages.
				\item Prevent Errors.
				\item Help and Documentation.
			\end{itemize}
		\end{column}
	\end{columns}
\end{frame}

\begin{frame}{Heuristic Analysis}
Nielsen's (1993) Revised Heuristics:

	\begin{columns}[onlytextwidth]
		\begin{column}{0.45\textwidth}
			\begin{itemize}
				\item Visibility of System Status.
				\item Match Between System and the Real World.
				\item User Control and Freedom.
				\item Consistency.
				\item Error Prevention.
			\end{itemize}
		\end{column}
		\begin{column}{0.45\textwidth}				
			\begin{itemize}
				\item Recognition Over Recall.
				\item Flexibility and Efficiency.
				\item Aesthetic and Minimalist Design.
				\item Help Users Diagnose and Recover from Errors.
				\item Help and Documentation.
			\end{itemize}
		\end{column}
	\end{columns}
\end{frame}

\begin{frame}{Heuristic Analysis}
Shneiderman \textit{et al} (2010) 8 ``Golden'' Rules:

	\begin{columns}[onlytextwidth]
		\begin{column}{0.45\textwidth}
			\begin{itemize}
				\item Strive for Consistency.
				\item Cater to Universal Usability
				\item Offer Informative Feedback.
				\item Design Dialogs to Yield Closure.
			\end{itemize}
		\end{column}
		\begin{column}{0.45\textwidth}				
			\begin{itemize}
				\item Prevent Errors.
				\item Permit Easy Reversal of Actions.
				\item Support Internal Locus of Control.
				\item Reduce Short-Term Memory Load.
			\end{itemize}
		\end{column}
	\end{columns}
\end{frame}

\begin{frame}[fragile]{Socrative \texttt{JBYPC3BBY}}
	\begin{itemize}
		\item In pairs.
		\item Quietly discuss how game controller design for 10-minutes.
		\item \textbf{List ONE} adapted heuristic \textbf{and explain why} it should be included.
	\end{itemize}
\end{frame}

\begin{frame}{Heuristic Analysis}
Pinelle \textit{et al's} (2008) Game Heuristics:

	\begin{columns}[onlytextwidth]
		\begin{column}{0.45\textwidth}
			\begin{itemize}
				\item Consistent Responses to Player's Actions.
				\item Permit Customisation of Game Settings.
				\item Predictable and Reasonable Agent Behaviours.
			\end{itemize}
		\end{column}
		\begin{column}{0.45\textwidth}				
			\begin{itemize}
				\item Provide Manageable Controls with Appropriate Sensitivity and Responsiveness.
				\item Provide Clear Information on Game Status.
				\item Provide Instruction, Training, and Help.
			\end{itemize}
		\end{column}
	\end{columns}
\end{frame}

\begin{frame}{Heuristic Analysis}
Pinelle \textit{et al's} (2008) Game Heuristics:

	\begin{columns}[onlytextwidth]
		\begin{column}{0.45\textwidth}
			\begin{itemize}
				\item Unobstructed Views of Information Needed to Inform Player Actions.
				\item Permit Skipping of Non-Playable Content.
				\item Intuitive and Customizable Input Mappings.
			\end{itemize}
		\end{column}
		\begin{column}{0.45\textwidth}				
			\begin{itemize}
				\item Provide Easily Interpreted Visual Representations
				\item Minimize the Need for Micromanagement.
			\end{itemize}
		\end{column}
	\end{columns}
\end{frame}

\begin{frame}{Heuristic Analysis Method}
	\begin{itemize}
		\item Heuristic evaluation is referred to as `discount' evaluation when 5 evaluators are used
		\item Empirical evidence suggests that on average 5 evaluators identify 75-80\% of usability problems.
	\end{itemize}
\end{frame}



\begin{frame}{Heuristic Analysis Method}
	\begin{itemize}
		\item Select set of heuristics
		\item Brief evaluators
	\end{itemize}
\end{frame}

\begin{frame}{Heuristic Analysis Method}
	\begin{itemize}
		\item Conduct evaluation:
			\begin{itemize}
				\item Each of the 5 evaluators works and takes notes separately.
				\item Each evaluator takes one overall pass of the system to get a feel for it.
				\item Each evaluator takes a second pass to focus on specific features.
				\item If done well, should takes approximately 1-2 hours.
			\end{itemize}
	\end{itemize}
\end{frame}

\begin{frame}{Heuristic Analysis Method}
	\begin{itemize}
		\item Conduct debriefing session:
			\begin{itemize}
				\item The 5 evaluators work together to identify and prioritise problems.
				\item They then report their results collectively to the designer.
				\item If done well, should takes approximately 1 hour.
			\end{itemize}
	\end{itemize}
\end{frame}

\begin{frame}{Heuristic Analysis Method}
Advantages of this approach include (Budd, 2007):

	\begin{itemize}
		\item Few ethical and practical issues as players not involved.
		\item Computing professionals are readily available.
		\item Excellent cost-benefit ratio.
	\end{itemize}
\end{frame}

\begin{frame}{Heuristic Analysis Method}
Disadvantages of this approach include (Budd, 2007):

	\begin{itemize}
		\item Variable quality as best experts require in-depth knowledge of genre and target players.
		\item Important problems are sometimes missed.
		\item Many trivial problems are often identified and over-emphasised.
		\item Experts have biases.
		\item Experts disagree with each other.
	\end{itemize}
\end{frame}