\part{Mental Representation \& Conceptual Models}
\frame{\partpage}

\begin{frame}{Learning Outcomes}
	In this section you will learn how to...
	
	\begin{itemize}
		\item \textbf{Define} what is meant by the term `mental model'
		\item \textbf{Explain} the place of mental models in HCI
		\item \textbf{Explain} what a metaphor is
		\item \textbf{Describe} the use of metaphor in HCI
		\item \textbf{Describe} the use of mental models in HCI
	\end{itemize}
\end{frame}

\begin{frame}{Further Reading}
	\begin{itemize}
		\item Lackoff, G. and Johnson, M. (1980) \textit{Metaphors We Live By}. UCP.
		\item Norman, D. and Draper, S. (1986) \textit{User-Centred Systems Design: New Perspectives on Human-Computer Interaction}. LEA.
		\item Norman, D. (1983) `Some Observations on Mental Models' in Gentner and Stevens (eds) \textit{User-Centred Systems Design: New Perspectives on Human-Computer
		 Interaction}. Erlbaum.
	\end{itemize}
\end{frame}

\begin{frame}{Human Memory Storage}
	\begin{itemize}
		\item The cognitive approach is currently the dominant framework (or paradigm) for HCI (Perry, 2006).
		\item Players are characterised as `information processors', in which information undergoes a series of ordered processes
		in the player's mind.
		\item This worldview draws a comparison between the human brain and a computer; we can therefore model player activity in the same
		way that we model computer processing.
	\end{itemize}
\end{frame}

\begin{frame}[fragile]{Socrative \texttt{JBYPC3BBY}}
	\begin{itemize}
		\item In pairs.
		\item Quietly discuss what you think is meant by the term `cognition' for 2-minutes.
		\item \textbf{Explain} cognition in your own words.
	\end{itemize}
\end{frame}

\begin{frame}{A Cognitive Economy}
	\begin{itemize}
		\item The cognitive approach is currently the dominant framework (or paradigm) for HCI (Perry, 2006).
		\item Players are characterised as `information processors', in which information undergoes a series of ordered processes
		in the player's mind.
		\item This worldview draws a comparison between the human brain and a computer; we can therefore model player activity in the same
		way that we model computer processing.
	\end{itemize}
\end{frame}

\begin{frame}[fragile]{Socrative \texttt{JBYPC3BBY}}
	\begin{itemize}
		\item In pairs.
		\item Quietly discuss what you think is meant by the term `cognition' for 2-minutes.
		\item \textbf{Explain} cognition in your own words.
	\end{itemize}
\end{frame}

\begin{frame}{Mental Models in Psychology}
	\begin{itemize}
		\item The cognitive approach is currently the dominant framework (or paradigm) for HCI (Perry, 2006).
		\item Players are characterised as `information processors', in which information undergoes a series of ordered processes
		in the player's mind.
		\item This worldview draws a comparison between the human brain and a computer; we can therefore model player activity in the same
		way that we model computer processing.
	\end{itemize}
\end{frame}

\begin{frame}[fragile]{Socrative \texttt{JBYPC3BBY}}
	\begin{itemize}
		\item In pairs.
		\item Quietly discuss what you think is meant by the term `cognition' for 2-minutes.
		\item \textbf{Explain} cognition in your own words.
	\end{itemize}
\end{frame}

\begin{frame}{Metaphor in the Interface}
	\begin{itemize}
		\item The cognitive approach is currently the dominant framework (or paradigm) for HCI (Perry, 2006).
		\item Players are characterised as `information processors', in which information undergoes a series of ordered processes
		in the player's mind.
		\item This worldview draws a comparison between the human brain and a computer; we can therefore model player activity in the same
		way that we model computer processing.
	\end{itemize}
\end{frame}

\begin{frame}[fragile]{Socrative \texttt{JBYPC3BBY}}
	\begin{itemize}
		\item In pairs.
		\item Quietly discuss what you think is meant by the term `cognition' for 2-minutes.
		\item \textbf{Explain} cognition in your own words.
	\end{itemize}
\end{frame}

\begin{frame}{Conceptual Models in the Interface}
	\begin{itemize}
		\item The cognitive approach is currently the dominant framework (or paradigm) for HCI (Perry, 2006).
		\item Players are characterised as `information processors', in which information undergoes a series of ordered processes
		in the player's mind.
		\item This worldview draws a comparison between the human brain and a computer; we can therefore model player activity in the same
		way that we model computer processing.
	\end{itemize}
\end{frame}

\begin{frame}[fragile]{Socrative \texttt{JBYPC3BBY}}
	\begin{itemize}
		\item In pairs.
		\item Quietly discuss what you think is meant by the term `cognition' for 2-minutes.
		\item \textbf{Explain} cognition in your own words.
	\end{itemize}
\end{frame}