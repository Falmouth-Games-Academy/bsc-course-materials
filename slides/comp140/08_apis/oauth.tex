\part{User authentication with OAuth}
\frame{\partpage}

\begin{frame}[fragile]{User authentication}
    \begin{lstlisting}[language=Python]
ACCESS_KEY = '...'
ACCESS_SECRET = '...'

auth = tweepy.OAuthHandler(CONSUMER_KEY, CONSUMER_SECRET)
auth.set_access_token(ACCESS_KEY, ACCESS_SECRET)
    \end{lstlisting}\pause
    \begin{itemize}
        \item OK for creating a bot that only ever tweets on its own account \pause
        \item Not suitable for writing a game component that allows users to use
            their own Twitter accounts, i.e.\ tweets on the user's behalf
    \end{itemize}
\end{frame}

\begin{frame}{OAuth}
    \begin{itemize}
        \item Twitter (and many other web services) use \textbf{OAuth} \pause
        \item Allows you (the user) to use a third-party app without giving it your account password \pause
        \item OAuth is yet another API to get to grips with... \pause
        \item Tweepy has a Twitter-specific wrapper for OAuth
    \end{itemize}
\end{frame}

\begin{frame}[fragile]{Using OAuth}
    \begin{lstlisting}[language=Python]
import tweepy
import webbrowser

CONSUMER_KEY = '...'
CONSUMER_SECRET = '...'

auth = tweepy.OAuthHandler(CONSUMER_KEY, CONSUMER_SECRET)

print "Opening Twitter website -- please log in"
webbrowser.open(auth.get_authorization_url())
verifier = raw_input("Verification code:")
auth.get_access_token(verifier)

api = tweepy.API(auth)
api.update_status("Hello, world!")
    \end{lstlisting}
\end{frame}

\begin{frame}{Staying logged in}
    \begin{itemize}
        \item Store \lstinline{auth.access_token} and \lstinline{auth.access_token_secret} along with your
            application's saved data \pause
        \item Now these can be reloaded and passed to \lstinline{auth.set_access_token},
            just like in our first bot example
    \end{itemize}
\end{frame}
