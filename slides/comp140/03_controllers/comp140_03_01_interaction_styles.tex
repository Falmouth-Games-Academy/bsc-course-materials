\part{Input, Output, and Interaction Styles}
\frame{\partpage}

\begin{frame}{Learning Outcomes}
	In this section you will learn how to...
	
	\begin{itemize}
		\item \textbf{Explain} the role of input and output in systems design
		\item \textbf{List} and \textbf{describe} a variety of input and output devices, giving examples of situations where each may be appropriate
		\item \textbf{Explain} what interaction styles are, while \textbf{critically evaluating} their respective advantages and disadvantages
		\item \textbf{Discuss} the role of direct manipulation in interacting with current computer systems
	\end{itemize}
\end{frame}

\begin{frame}{Further Reading}
	\begin{itemize}
		\item Shneiderman, B. (1998) \textit{Designing the User Interface: Strategies for Effective Human-Computer Interaction}. 3rd Edition. Addison Wesley.
	\end{itemize}
\end{frame}

\begin{frame}{Input and Output Technologies}
	\begin{itemize}
		\item \textbf{input}: the process that occurs as data from the players mind (or from the environment) is transformed into data that computers can use.
		\item \textbf{output}: the process of re-representing computer data into a form the player can perceive, comprehend, and make use of.
	\end{itemize}
\end{frame}

\begin{frame}{Input and Output Technologies}
	When developing and/or selecting an input device for a game, designers are faced with several design trade-offs:
	
	\begin{itemize}
		\item no single optimal device for all tasks 
		\item form of data (e.g. selection verus alphanumeric)
		\item variety of players each with different characteristics
	\end{itemize}
\end{frame}

\begin{frame}{Activity}
	You are making a 2D-platformer for blind players.
	
	\begin{itemize}
		\item Self-organise into pairs.
		\item \textbf{Discuss} the challenges of this scenario on Slack.
		\item \textbf{Research} the characteristics of 2D platformers and blind players, \textbf{posting} your key findings on Slack.
		\item \textbf{Design} and/or \textbf{identify} appropriate an input device to support players' jump accuracy.
	\end{itemize}
	
	\vspace{2ex}
	
	Time: 10-minutes.
\end{frame}

\begin{frame}[fragile]{Socrative \texttt{JBYPC3BBY}}
	\begin{itemize}
		\item \textbf{Summarise} your choice and/or design of input device.
	\end{itemize}
\end{frame}

\begin{frame}{Input and Output Technologies}
	Similar challenges arise when designing and/or selecting output devices for a game:
	
	\begin{itemize}
		\item no single optimal device for all tasks 
		\item form of data (e.g. attention verus conversant)
		\item no single optimal sense to engage
		\item variety of players each with different characteristics
	\end{itemize}
\end{frame}

\begin{frame}{Activity}
	You are making an RTS game for deaf players.
	
	\begin{itemize}
		\item Self-organise into pairs.
		\item \textbf{Discuss} the challenges of this scenario on Slack.
		\item \textbf{Research} the characteristics of RTS games and game players, \textbf{posting} your key findings on Slack.
		\item \textbf{Design} and/or \textbf{identify} appropriate an output device to direct players' attention to units in conflict.
	\end{itemize}
	
	\vspace{2ex}
	
	Time: 10-minutes.
\end{frame}

\begin{frame}[fragile]{Socrative \texttt{JBYPC3BBY}}
	\begin{itemize}
		\item \textbf{Summarise} your choice and/or design of output device.
	\end{itemize}
\end{frame}

\begin{frame}{Interaction Styles}
	\begin{itemize}
		\item \textbf{interaction style}: a term used to describe different approaches to communication between players and computer games.
		\item Includes things such as: 
		\begin{itemize}
			\item command-line entry;
			\item menu;
			\item form-fill;
			\item natural language;
			\item WIMP;
			\item direct manipulation.
		\end{itemize}
	\end{itemize}
\end{frame}

\begin{frame}[fragile]{Socrative \texttt{JBYPC3BBY}}
	\begin{itemize}
		\item In pairs.
		\item Quietly discuss what you think is meant by the acronym `WIMP' for 2-minutes.
		\item \textbf{State} the meaning of WIMP.
	\end{itemize}
\end{frame}

\begin{frame}{Command Line Entry}
	\begin{columns}[onlytextwidth]
		\begin{column}{0.45\textwidth}
			Advantages:
	
			\begin{itemize}
				\item Functionally powerful.
				\item Quick to Use.
			\end{itemize}
		\end{column}
		\begin{column}{0.45\textwidth}
			Disadvantages:
	
			\begin{itemize}
				\item Requires player to remember commands and syntax.
				\item Little feedback, \textit{or} far too verbose.
			\end{itemize}
		\end{column}
	\end{columns}	
\end{frame}

\begin{frame}{Menus}
	\begin{columns}[onlytextwidth]
		\begin{column}{0.45\textwidth}
			Advantages:
	
			\begin{itemize}
				\item Facilittes information navigation.
				\item Restricts potential actions---safe for novices.
				\item Reduces memory load---knowledge in the world.
			\end{itemize}
		\end{column}
		\begin{column}{0.45\textwidth}
			Disadvantages:
	
			\begin{itemize}
				\item Restricts functionality and freedom.
				\item Can be made too complex---difficult to find functions.
			\end{itemize}
		\end{column}
	\end{columns}	
\end{frame}

\begin{frame}{Form Fill-In}
	\begin{columns}[onlytextwidth]
		\begin{column}{0.45\textwidth}
			Advantages:
	
			\begin{itemize}
				\item Paper as a metaphor.
				\item Simple and intuitive.
			\end{itemize}
		\end{column}
		\begin{column}{0.45\textwidth}
			Disadvantages:
	
			\begin{itemize}
				\item Minimally interactive.
				\item Requires an effective supporting layout.
			\end{itemize}
		\end{column}
	\end{columns}	
\end{frame}

\begin{frame}{Natural Language}
	\begin{columns}[onlytextwidth]
		\begin{column}{0.45\textwidth}
			Advantages:
	
			\begin{itemize}
				\item Intuitive and potentially powerful.
				\item Works effectively for simple interactions.
			\end{itemize}
		\end{column}
		\begin{column}{0.45\textwidth}
			Disadvantages:
	
			\begin{itemize}
				\item Technological limitations---accents for example.
				\item Ambiguitiy in language interpretation.
				\item Unsuitable for `twitch' contexts.
			\end{itemize}
		\end{column}
	\end{columns}	
\end{frame}

\begin{frame}[fragile]{Socrative \texttt{JBYPC3BBY}}
	\begin{itemize}
		\item In pairs.
		\item Quietly discuss the advantages of`WIMP' for 2-minutes.
		\item \textbf{State TWO} advantages of WIMP.
	\end{itemize}
\end{frame}

\begin{frame}[fragile]{Socrative \texttt{JBYPC3BBY}}
	\begin{itemize}
		\item In pairs.
		\item Quietly discuss the disadvantages of`WIMP' for 2-minutes.
		\item \textbf{State TWO} disadvantges of WIMP.
	\end{itemize}
\end{frame}

\begin{frame}{Direct Manipulation}
	``The central tenets of direct manipulation are visibility of the objects of interest,
	actions being performed through the rapid, reversible, incremental behaviours and
	actions performed directly on screen objects'' 
	
	\vspace{2ex}
	
	(Perry, 2006, p. 33)
\end{frame}

\begin{frame}{Direct Manipulation}
	Examples:
	
	\begin{itemize}
		\item Interactive Page Animations
		\item Desktop Icons
		\item Scrollbars
	\end{itemize}
\end{frame}

\begin{frame}{Direct Manipulation}
	Direct manipulation aims to address two interface challenges (from Normal and Draper, 1986):
	
	\begin{itemize}
		\item Gulf of Execution
		\item Guld of Evaluation
	\end{itemize}
\end{frame}

\begin{frame}{Gulf of Execution}	
	\begin{itemize}
		\item \textbf{Gulf of execution}: the `distance' between a player's goal and the means of achieving it.
		\item ``One measure of this gulf is how well the system allows the person do the intended actions directly, without extra effort'' (Norman, 1988, p. 51)
	\end{itemize}
\end{frame}

\begin{frame}{Gulf of Evaluation}
	\begin{itemize}
		\item \textbf{Gulf of evaluation}: the `distance' between the state of the game and the player's ability to assess it through perceiving representations.
		\item ``The gulf is small when the system provides information about its state in a form that is easy to get, is easy to interpret, and matches the way the person thinks of the system'' (Norman, 1988, p. 51)
	\end{itemize}
\end{frame}

\begin{frame}{Direct Manipulation}
	Shneiderman (1998) suggests several advantages:
	
	\begin{itemize}
		\item Novices learn functionality quickly  \pause
		\item Experienced users can define new functions and features  \pause
		\item Casual or intermittent users can retain operational concepts  \pause
		\item Built-in constraints mean that all user actions are legal
	\end{itemize}
\end{frame}

\begin{frame}{Direct Manipulation}
	Shneiderman (1998) suggests several advantages:
	
	\begin{itemize}
		\item Change is incremental, with immediate feedback to players \pause
		\item Reduces anxiety because the system is comprehensible and actions reversible \pause
		\item Gain confidence through mimesis and predicting actions
	\end{itemize}
\end{frame}

\begin{frame}[fragile]{Socrative \texttt{JBYPC3BBY}}
	\begin{itemize}
		\item In pairs.
		\item Quietly discuss examples of direct manipulation in games for 2-minutes.
		\item \textbf{State TWO} examples of direct manipulation.
	\end{itemize}
\end{frame}