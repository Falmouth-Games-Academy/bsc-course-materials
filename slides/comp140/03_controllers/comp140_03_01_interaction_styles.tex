\part{Input, Output, and Interaction Styles}
\frame{\partpage}

\begin{frame}{Learning Outcomes}
	In this section you will learn how to...
	
	\begin{itemize}
		\item \textbf{Explain} the role of input and output in systems design
		\item \textbf{List} and \textbf{describe} a variety of input and output devices, giving examples of situations where each may be appropriate
		\item \textbf{Explain} what interaction styles are, while \textbf{critically evaluating} their respective advantages and disadvantages
		\item \textbf{Discuss} the role of direct manipulation in interacting with current computer systems
	\end{itemize}
\end{frame}

\begin{frame}{Further Reading}
	\begin{itemize}
		\item Shneiderman, B. (1998) \textit{Designing the User Interface: Strategies for Effective Human-Computer Interaction}. 3rd Edition. Addison Wesley.
	\end{itemize}
\end{frame}

\begin{frame}{Input and Output Technologies}
	\begin{itemize}
		\item The cognitive approach is currently the dominant framework (or paradigm) for HCI (Perry, 2006).
		\item Players are characterised as `information processors', in which information undergoes a series of ordered processes
		in the player's mind.
		\item This worldview draws a comparison between the human brain and a computer; we can therefore model player activity in the same
		way that we model computer processing.
	\end{itemize}
\end{frame}

\begin{frame}[fragile]{Socrative \texttt{JBYPC3BBY}}
	\begin{itemize}
		\item In pairs.
		\item Quietly discuss what you think is meant by the term `cognition' for 2-minutes.
		\item \textbf{Explain} cognition in your own words.
	\end{itemize}
\end{frame}

\begin{frame}{Interaction Styles}
	\begin{itemize}
		\item The cognitive approach is currently the dominant framework (or paradigm) for HCI (Perry, 2006).
		\item Players are characterised as `information processors', in which information undergoes a series of ordered processes
		in the player's mind.
		\item This worldview draws a comparison between the human brain and a computer; we can therefore model player activity in the same
		way that we model computer processing.
	\end{itemize}
\end{frame}

\begin{frame}[fragile]{Socrative \texttt{JBYPC3BBY}}
	\begin{itemize}
		\item In pairs.
		\item Quietly discuss what you think is meant by the term `cognition' for 2-minutes.
		\item \textbf{Explain} cognition in your own words.
	\end{itemize}
\end{frame}