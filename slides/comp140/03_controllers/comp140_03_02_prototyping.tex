\part{Prototyping}
\frame{\partpage}

\begin{frame}{Learning Outcomes}
	In this section you will learn how to...
	
	\begin{itemize}
		\item \textbf{Explain} the role of prototyping in game interface design
		\item \textbf{Compare} different approaches to prototyping
		\item \textbf{Select} an appropriate prototyping method for particular usability challenges
	\end{itemize}
\end{frame}

\begin{frame}{Further Reading}
	\begin{itemize}
		\item Jensen, S. (2002) \textit{The Simplicity Shift}. Cambridge University Press.
	\end{itemize}
\end{frame}

\begin{frame}{The Value of Prototyping}
	\begin{itemize}
		\item Jenson (2002) describes the value of prototyping: ``to fail---and fail fast!''
		\item Only by learning from our mistakes can we develop truly usable designs. 
		\item It is very rare, if ever, we get things right the first time and this is especially true of complex systems---such as those in interaction design.
	\end{itemize}
\end{frame}

\begin{frame}[fragile]{Socrative \texttt{JBYPC3BBY}}
	\begin{itemize}
		\item In pairs.
		\item Quietly discuss how prototyping game designs has help you develop those designs, for 2-minutes.
		\item \textbf{Explain} the benefits of prototyping your own words.
	\end{itemize}
\end{frame}

\begin{frame}{Searching the Design Space}
	Design can be conceptualised as serching for an \textit{acceptable} design within an infinite design space. This perspective provokes
	several ideas:

	\begin{itemize}
		\item Designers may not search effectively
		\item Designers may not recognise an \textit{acceptable} design
		\item Designers may converge on a local maxima in the design space: a bad design
	\end{itemize}
\end{frame}

\begin{frame}{Searching the Design Space}
	Prototyping, therefore, becomes:

	\begin{itemize}
		\item An effective search and evaluation method
		\item A means to communicate design information
	\end{itemize}
\end{frame}

\begin{frame}{Searching the Design Space}
	``In practice, prototyping allows designers to \textbf{conceptualise} their products, to better understand the kinds of task that the users do and to
	support them with the appropriate technology''
	
	\vspace{2ex}
	
	(Perry, 2006, p. 50)
\end{frame}

\begin{frame}{Searching the Design Space}
	``Importantly, prototyping forces the [game designers] to visualise all of the steps in [game] software (even beyond the interface), and how well
	the interface will operate in practice.''
	
	\vspace{2ex}
	
	(Perry, 2006, p. 50)
\end{frame}

\begin{frame}[fragile]{Socrative \texttt{JBYPC3BBY}}
	\begin{itemize}
		\item In pairs.
		\item Quietly discuss the consequence of \textbf{not} prototyping and play-testing for 2-minutes.
		\item \textbf{Explain ONE} of these consequences.
	\end{itemize}
\end{frame}

\begin{frame}{Approaches to Prototype Development}
	Prototypes can be used at a number of levels:

	\begin{itemize}
		\item \textbf{Game conceptualisation}: developing the game concept into a game design.
		\item \textbf{Task-level prototyping}: how a particular game mechanic and/or task for the player meshes with player expectation and their attempts to
		fulfill their goals.
		\item \textbf{Menues and HUDs}: the form and placement of data input and output in specific contexts.
	\end{itemize}
\end{frame}

\begin{frame}{Approaches to Prototype Development}
	Different methods facilitate these levels:

	\begin{itemize}
		\item \textbf{Requirements animation}: demonstrating potential functionality and use-cases as animations that can be easily assessed by players. \pause
		\item \textbf{Rapid prototyping}: intensively collecting information on requirements and modelling them as small prototypes that can be easily assessed by players.
	\end{itemize}
\end{frame}

\begin{frame}{Approaches to Prototype Development}
	Different methods facilitate these levels:

	\begin{itemize}
		\item \textbf{Evolutionary prototyping}: developing an initial model that is evaluated and adapted until it `evolves' into an improved end-product.\pause
		\item \textbf{Incremental prototyping}: step-wise development of large prototypes, such as vertical slices, in phases to avoid delays between specification and delivery.
	\end{itemize}
\end{frame}

\begin{frame}{Full Prototype}
	\begin{itemize}
		\item complete version of the intended system
		\item may be a model or a roughly assembled throw-away
	\end{itemize}
\end{frame}

\begin{frame}{Paper Prototype}
	\begin{itemize}
		\item no functionality
		\item used to talk through a design and demonstrate interfaces
	\end{itemize}
\end{frame}

\begin{frame}{Horizontal Prototype}
	\begin{itemize}
		\item complete coverage of the all interface elements
		\item little to no functionality
	\end{itemize}
\end{frame}

\begin{frame}{Vertical Prototype}
	\begin{itemize}
		\item incomplete coverage of the interface elements
		\item high level of functionality in restricted areas
	\end{itemize}
\end{frame}

\begin{frame}{Low Fidelity Prototype}
	\begin{itemize}
		\item little resemblence to the final `look and feel'
		\item cheap and fast to develop
	\end{itemize}
\end{frame}

\begin{frame}{High Fidelity Prototype}
	\begin{itemize}
		\item much resemblence to the final `look and feel' (may even be better i.e., pre-rendered vs real-time in engine)
		\item expensive and time-consuming
	\end{itemize}
\end{frame}

\begin{frame}{`Wizard of Oz' Prototype}
	\begin{itemize}
		\item no functionality at all---simulated through intervention by a hidden person
		\item requires operator to have key knowledge of system states and interactions
	\end{itemize}
\end{frame}

\begin{frame}[fragile]{Socrative \texttt{JBYPC3BBY}}
	\begin{itemize}
		\item In pairs.
		\item Quietly discuss which prototyping appeach is appropriate for early-stage battle interface design for an RPG for 5-minutes.
		\item Prototyping methods are not exclusionary (e.g. could combine high-fidelity with vertical for a pitch).
		\item \textbf{State ONE} prototyping methods \textbf{and justify} your answer.
	\end{itemize}
\end{frame}

\begin{frame}[fragile]{Socrative \texttt{JBYPC3BBY}}
	\begin{itemize}
		\item In pairs.
		\item Quietly discuss which prototyping appeach is appropriate for \textit{late}-stage battle interface design for an RPG for 5-minutes.
		\item Prototyping methods are not exclusionary (e.g. could combine high-fidelity with vertical for a pitch).
		\item \textbf{State ONE} prototyping methods \textbf{and justify} your answer.
	\end{itemize}
\end{frame}

\begin{frame}{Best Practices and Pitfalls}
	Where is prototyping most effective?

	\begin{itemize}
		\item Poorly defined or uncertain requirements
		\item Cost of system rejection is high
		\item High-stakes (think, e.g. nuclear power station)
		\item Assess impact of changes through requirements to implementation (i.e., contract scope)
	\end{itemize}
\end{frame}

\begin{frame}{Best Practices and Pitfalls}
	What are the limitations?

	\begin{itemize}
		\item May introduce uneccessary constraints early in design process
		\item Development has little, if any, direct input by stakeholders (i.e., expert is implementing the prototype)
		\item Consumes time---a trade-off with production schedule
		\item Little data on safety, reliability, response time, and so on---may lead to impractical designs
	\end{itemize}
\end{frame}

\begin{frame}[fragile]{Socrative \texttt{JBYPC3BBY}}
	\begin{itemize}
		\item In pairs.
		\item Quietly discuss which situations are appropriate for prototyping a game component for 2-minutes.
		\item \textbf{Explain ONE} such situation.
	\end{itemize}
\end{frame}