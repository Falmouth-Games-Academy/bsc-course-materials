\part{Intellectual Property}
\frame{\partpage}

\begin{frame}{Learning Outcomes}
	In this section you will learn how to...
	
	\begin{itemize}
		\item \textbf{Distinguish} between the structure of civil law \textbf{and} criminal law
		\item \textbf{Explain} the importance of ownership issues and contracts in game development
		\item \textbf{Explain} what copyright, moral rights, trademarks, design rights, trade secrets, and patents are.
		\item \textbf{Discuss} the role of licensing in intellectual property usage and management
	\end{itemize}
\end{frame}

\begin{frame}{Civil vs Criminal Law}
	\begin{itemize}
		\item \textbf{criminal law}: the body of law dealing with crimes and their punishment. Law sets out all the things which are considered 
		unacceptable, and which will render someone liable for prosecution.
		\vspace{2ex}
		\item \textbf{civil law}: the body of law dealing with disputes between individuals, organisations, and other bodies. It is a very 
		complicated system which tries to set out rules to cover all the sorts of situation that may arise in life, and provides for disputes 
		to be decided by a Judge if the parties are unable to sort it out themselves.
	\end{itemize}
\end{frame}

\begin{frame}{Civil vs Criminal Law}
	\begin{itemize}
		\item Businesses can engage in criminal acts. Such acts include: fraud, industrial espionage, and tax evasion.
		\vspace{2ex}
		\item Most acts, however, are civil in nature. Such acts include: failure to pay bills; breaches of contract; and misuse of intellectual property.
	\end{itemize}
\end{frame}

\begin{frame}{Civil vs Criminal Law}
The reasons why is because UK law has many different sources, some of which include:

	\begin{itemize}
		\item \textbf{statute}: legislation from UK Parliaments and its devolved counterparts.
		\vspace{1ex}
		\item \textbf{`common law'}: principles established through historic cases.
		\vspace{1ex}
		\item \textbf{EU}: regulations and directives set by Europe for all EU citizens.
	\end{itemize}
\end{frame}

\begin{frame}{Civil vs Criminal Law}
	\begin{itemize}
		\item An interesting characteristic of the law in England is the `doctrine of judicial precedents': the judgements of the 
		courts are a binding source of law for future cases. 
		\vspace{1ex}
		\item Judges are bound by the judgements of courts of a higher jurisdiction (although not those of the lower courts).
		\vspace{1ex}
		\item Often appeals processes are based on making a distinction from such historic cases or on the basis of lower courts misinterpreting the law.
	\end{itemize}
\end{frame}

\begin{frame}{Civil vs Criminal Law}
	\begin{itemize}
		\item In civil courts, the claimant assembles their case, with the standard of proof being the ‘balance of probabilities’: the case must be more likely
		to be correct than incorrect.
		\vspace{2ex}
		\item In criminal courts, the prosecution have the burden of proof, and the standard is much higher: they have to prove their 
		case `beyond reasonable doubt'.
	\end{itemize}
\end{frame}

\begin{frame}{Intellectual Property}
	\begin{itemize}
		\item IP law is designed to reward and motivate the contributions of human intellect.
		\item This is achieved by granting certain rights which are commercially valuable.
		\item Essentially, the right to make profit from the actualisation and application of your ideas.
	\end{itemize}
\end{frame}

\begin{frame}{Intellectual Property}
	\begin{itemize}
		\item IP law is covered by the civil courts.
		\item Do not breach IP law, as you may be sued.
		\item Even if, you are not making money yourself, you may be denying profit to the defendant. 
		Even if you are not worth suing because you have no money, they may remedy the situation with
		 injunctions and take-down notices.
		\item IP owners have a legal responsibility to actively protect their works, or risk losing their rights.
	\end{itemize}
\end{frame}

\begin{frame}[fragile]{Socrative \texttt{JBYPC3BBY}}
\textit{Alice is a computer programmer. In her free time, she programs a game. Alice, however, is not an artist.
She asks Brian, a friend, for help. Brian agrees and contributes artwork to the game. No compensation is discussed
and no written agreement exists.}
\vspace{2ex}
	\begin{itemize}
		\item \textbf{Who} owns the game overall? 
		\item In pairs, discuss who owns the game overall for 2 minutes.
		\item\textbf{Select} your choice.
	\end{itemize}
\end{frame}

\begin{frame}{Ownership \& Contracts}
	\begin{itemize}
		\item When more than one person contributes to a work and no agreement exists, co-authors are joint owners
		and everyone retains an equal share in the entire work (Ibrahim, 2009).
		\item In an employer/employee context, the employer always owns the copyright (Ibrahim, 2009). This is implicit
		within your contract of employment.
		\item In a client/worker context, an intellectual property transaction is set out as a contract. Typically, the IP belongs
		to the creator of a work, but in the case of contract work it will likely be transferred to another party as part of the contract.
	\end{itemize}
\end{frame}

\begin{frame}{Ownership \& Contracts}
Contracts require several things to be valid (Ibrahim, 2009):

	\begin{itemize}
		\item Capacity
		\item Mutual Assent
		\item Legal Purpose
		\item Bargained-for Consideration
		\item A Signed-Writing*
	\end{itemize}
\end{frame}

\begin{frame}{Copyright}
	\begin{itemize}
		\item Copyright protects a work from being: translated, copied, publicly performed/transmitted/broadcast, and adapted.
		\item Any time someone copies, performs, or displays a copyrighted work without
		permission, they commit copyright infringement.
		\item Specifically, it protects only expressions, and not ideas.
		\item Copyright has a limited duration, after which work goes into the public domain.
	\end{itemize}
\end{frame}

\begin{frame}{Copyright}
`Fair dealing' exceptions exist:

	\begin{itemize}
		\item Non-commercial research
		\item Private study
		\item Criticism, Review, and Reporting
		\item Teaching
		\item Assistance for Disabled People
		\item Time-shifting
		\item Parody
		\item Orphan Works
	\end{itemize}
\end{frame}

\begin{frame}{Moral Rights}
	\begin{itemize}
		\item Protect the personal interests of the author of a copyrighted work.
		\item Moral rights include:
		\begin{itemize}
			\item Right to be identified
			\item Right to object to derogatory treatment
			\item Right to object to false attribution
			\item Right to privacy
		\end{itemize}
	\end{itemize}
\end{frame}

\begin{frame}{Trademark}
	\begin{itemize}
		\item Trademarks identify the source and/or quality of a product or service.
		\item Protects branding and brand names, so long as its use does not become diluted.
		\item Registration of a trademark is advised as it strengthens its validity, but 
		protection is automatic once brand names are used in commercial transactions.
		\item Most registered trademarks are internationally-binding through the Madrid Protocol.
	\end{itemize}
\end{frame}

\begin{frame}{Design Rights}
	\begin{itemize}
		\item Protects the (3D) shape of a product. Enables a design to be distinctive.
		\item Often used to protect particular user interface elements which are novel.
		\item Does not extend to other aesthetic properties.
		\item Right applies to UK/EU inventors.
	\end{itemize}
\end{frame}

\begin{frame}{Trade Secrets}
	\begin{itemize}
		\item Trade secrets are secrets that have commercial value.
		\item For a valid trade secret to exist, the company claiming a trade secret must
		clearly determine what it is and will take steps to keep it concealed.
		\item Often, takes the form of Non-Disclosure Agreements (NDAs).
		\item Breach of an NDA can result in substantial damages being awarded.
	\end{itemize}
\end{frame}

\begin{frame}{Patents}
	\begin{itemize}
		\item Protects inventions, processes, methods, and products
		\item Criteria include: novel, inventive (not obvious to a skilled person), and capable of industrial application.
		\item Must be registered formally and examined.
		\item Difficult and expensive to obtain, but provides a monopoly over the invention for a period of time.
		\item In exchange for public disclosure, protects the idea behind the invention.
		\item Patents can be territorial or international via the Paris Convention.
	\end{itemize}
\end{frame}

\begin{frame}{Licenses}
	\begin{itemize}
		\item A licensing agreement is a partnership between an intellectual property owner (licensor) and another 
		who is authorized to use such rights (licensee) in exchange for an agreed payment, be that a fee or a royalty.
		\item Includes: technology license; end-user license; trademark franchising agreement; copyright license agreement; etc.
		\item Licensing can broaden the reach of IP into different markets, without the holder incurring risks.
		\item Sometimes there are databases endorsing a `license of right' which means the IP holder grants permission
		to anyone who asks, usually for a fee or a set royalty agreement.
	\end{itemize}
\end{frame}

\part{Practical Activity}
\frame{\partpage}

\begin{frame}[fragile]{Socrative \texttt{JBYPC3BBY}}
	\begin{itemize}
		\item \textbf{Self-organise} into your COMP150 groups.
		\item \textbf{Select} a well-known intellectual property: Mass Effect; Star Craft; Mario; Crazy Taxi.
		\item \textbf{Discuss} on Slack for 15 minutes the legal implications of using any of their IP.
		\vspace{2ex}
		\item \textbf{Summarise FIVE} key IP laws that prevent you from using elements from the IP.
	\end{itemize}
\end{frame}