% Uncomment this line for on-screen presentation
\documentclass[xcolor={dvipsnames}]{beamer}\usepackage{etoolbox}\newtoggle{printable}\togglefalse{printable}

% Uncomment this line for printable slides (disable animations and don't waste ink)
%\documentclass[handout, xcolor={dvipsnames}]{beamer}\usepackage{etoolbox}\newtoggle{printable}\toggletrue{printable}

% Adjust these for the path of the theme and its graphics, relative to this file
%\usepackage{beamerthemeFalmouthGamesAcademy}
\usepackage{../../beamerthemeFalmouthGamesAcademy}
\graphicspath{ {../../} }

% Default language for code listings
\lstset{language=C++,
        morekeywords={each,in}
}

\begin{document}
\title{Transition to C++ III}   
\subtitle{COMP110: Principles of Computing}

\frame{\titlepage} 

\begin{frame}{Learning outcomes}
    In this session you will learn how to...
    \begin{itemize}
        \item Define your own \textbf{classes} in C++
        \item Use \textbf{pointers}, and allocate objects on the \textbf{heap}
        \item Use \textbf{typecasting} to convert values from one type to another
        \item Use the \textbf{CImg} library to write basic GUI applications and image processing algorithms
    \end{itemize}
\end{frame}

\part{Object-oriented programming in C++}
\frame{\partpage}

\begin{frame}[fragile]{OOP refresher}
    \begin{itemize}
        \item A \textbf{class} is a collection of \textbf{fields} (data) and \textbf{methods} (functions) \pause
        \item Fields and methods may be \textbf{public} (accessible everywhere),
            \textbf{protected} (accessible in the class and classes that inherit from it)
            or \textbf{private} (accessible in the class only) \pause
        \item Classes may \textbf{inherit} fields and methods from other classes \pause
        \item Subclasses may \textbf{override} methods which they inherit --- this gives rise to \textbf{polymorphism}
    \end{itemize}
\end{frame}

\begin{frame}[fragile]{Class declarations}
    \begin{lstlisting}
class MyClass
{
public:
    void doMethod(int x)
    {
        std::cout << x << std::endl;
    }

private:
    int field = 7;
};
    \end{lstlisting}
\end{frame}

\begin{frame}[fragile]{Fields and methods}
    \begin{itemize}
        \item Fields and methods are declared in the class declaration, just like variables and functions \pause
        \item Class declaration is split into sections by access type (\textbf{public}, \textbf{protected}, \textbf{private})
    \end{itemize}
\end{frame}

\begin{frame}[fragile]{Overloading}
    \begin{itemize}
        \item Functions and methods can be defined with the \textbf{same name} but \textbf{different parameters}
    \end{itemize}
    \pause
    \begin{lstlisting}
    double getVectorLength(double x, double y)
    {
        return sqrt(x * x + y * y);
    }

    double getVectorLength(Vector v)
    {
        return sqrt(v.x * v.x + v.y * v.y);
    }
    \end{lstlisting}
\end{frame}

\begin{frame}[fragile]{Constructors and destructors}
	\begin{lstlisting}
class MyClass
{
public:
    MyClass()
    {
    }
    
    ~MyClass()
    {
    }
};
	\end{lstlisting}
    \begin{itemize}
        \item The \textbf{constructor} is executed when the class is instantiated
        \item The \textbf{destructor} is executed when the instance is freed
    \end{itemize}
\end{frame}

\begin{frame}[fragile]{Constructors}
    \begin{itemize}
        \item The constructor name matches the class name \pause
        \item Constructors can take parameters \pause
        \item The constructor can be overloaded, i.e.\ can have several constructors with different parameters
    \end{itemize}
    \pause
	\begin{lstlisting}
class MyClass
{
public:
    // Parameterless constructor
    MyClass() { }
    
    // Constructor with parameters
    MyClass(int x, double y) { }
};
	\end{lstlisting}
\end{frame}

\begin{frame}[fragile]{Destructors}
    \begin{itemize}
        \item The destructor name is the class name prefixed with $\sim$ (tilde) \pause
        \item Destructors \textbf{cannot} take parameters
    \end{itemize}
\end{frame}

\begin{frame}[fragile]{Modular program design}
    \begin{itemize}
        \item Method \textbf{declarations} go in the class declaration \pause
        \item Method \textbf{definitions} look like function definitions,
            with the function name replaced with \lstinline{ClassName::methodName} \pause
        \item Method definitions can also go inline into the class declaration
        \begin{itemize}
            \item Best used for short (1 or 2 line) methods
        \end{itemize}
        \pause
        \item Good practice: Put class declaration in \texttt{ClassName.h}, and method definitions in \texttt{ClassName.cpp}
    \end{itemize}
\end{frame}

\begin{frame}[fragile]{Example: Circle.h}
    \begin{lstlisting}
#pragma once

class Circle
{
public:
    Circle(double radius);
    
    double getArea();

private:
    double radius;
};
    \end{lstlisting}
\end{frame}

\begin{frame}[fragile]{Example: Circle.cpp}
    \begin{lstlisting}
#include "stdafx.h"
#include "Circle.h"

Circle::Circle(double radius)
    : radius(radius)
{
}

double Circle::getArea()
{
    return M_PI * radius * radius;
}
    \end{lstlisting}
\end{frame}

\begin{frame}[fragile]{Inheritance}
    \begin{lstlisting}
class Shape
{
public:
    virtual double getArea();
};

class Circle : Shape
{
public:
    virtual double getArea()
    {
        return M_PI * radius * radius;
    }
};
    \end{lstlisting}
    \pause
    \begin{itemize}
        \item Methods to be overridden must be marked \textbf{virtual}
    \end{itemize}
\end{frame}

\begin{frame}[fragile]{Pure virtual methods}
    \begin{itemize}
        \item \textbf{Abstract classes} should never be instantiated --- they only exist to serve as a base class \pause
        \item \textbf{Abstract methods} are not defined in the base class, and must be overridden in the subclass \pause
        \item In C++, abstract methods are called \textbf{pure virtual} \pause
        \item Having at least one pure virtual method \textbf{automatically} makes the class abstract
    \end{itemize}
    \pause
    \begin{lstlisting}
class Shape
{
public:
    virtual double getArea() = 0;
};
    \end{lstlisting}
\end{frame}

\begin{frame}[fragile]{Virtual destructors}
    \begin{itemize}
        \item If your class is intended to be subclassed, you should declare the \textbf{destructor} to be \lstinline{virtual}
    \end{itemize}
    \pause
    \begin{lstlisting}
class Shape
{
public:
    virtual ~Shape();
};
    \end{lstlisting}
    \pause
    \begin{itemize}
        \item Why?
        \begin{itemize}
            \item \footnotesize\url{http://stackoverflow.com/questions/461203/when-to-use-virtual-destructors}
            \item \footnotesize\url{http://programmers.stackexchange.com/questions/284561/when-not-to-use-virtual-destructors}
        \end{itemize}
    \end{itemize}
\end{frame}

\begin{frame}[fragile]{Instantiation}
    \begin{itemize}
        \item To instantiate with a \textbf{parameterless constructor}, just declare a variable
    \end{itemize}
    \begin{lstlisting}
MyClass myInstance;
    \end{lstlisting}
    \pause
    \begin{itemize}
        \item To instantiate with a \textbf{constructor with parameters}, add the parameters in parentheses
    \end{itemize}
    \begin{lstlisting}
MyClass myInstance(27);
    \end{lstlisting}
    \pause
    \begin{itemize}
        \item This allocates the instance on the \textbf{stack} \pause
        \item The instance is destroyed (and the destructor is called) when the variable goes \textbf{out of scope}
    \end{itemize}
\end{frame}

\begin{frame}[fragile]{Instantiation: C++ vs Python}
    \pause
    \begin{lstlisting}[language=Python]
# Python
myInstance = MyClass()
myOtherInstance = myInstance
    \end{lstlisting}
    \pause
    \begin{itemize}
        \item \lstinline{myInstance} is a \textbf{reference} to an instance \pause
        \item \lstinline{myOtherInstance} is a reference to the \textbf{same} instance \pause
    \end{itemize}
    \begin{lstlisting}
// C++
MyClass myInstance;
MyClass myOtherInstance = myInstance;
    \end{lstlisting}
    \pause
    \begin{itemize}
        \item \lstinline{myInstance} \textbf{is} an instance \pause
        \item \lstinline{myOtherInstance} is a \textbf{different} instance --- usually a \textbf{copy} of \lstinline{myInstance}
            (but it depends on how \lstinline{MyClass} is defined)
    \end{itemize}
\end{frame}

\begin{frame}[fragile]{Accessing members}
    \begin{itemize}
        \item Use dot (\lstinline{.}) notation, similar to Python
    \end{itemize}
    \pause
    \begin{lstlisting}
Circle myCircle(10);
double area = myCircle.getArea();
    \end{lstlisting}
\end{frame}

%\part{Pointers}
\frame{\partpage}

\begin{frame}[fragile]{Pointers}
    \begin{itemize}
        \item A \textbf{pointer} is the address of a memory location \pause
        \item If \lstinline{T} is a type, \lstinline{T*} is the type ``pointer to \lstinline{T}'' \pause
        \item \lstinline{&} is the \textbf{address-of} operator: gets a pointer to something \pause
        \item \lstinline{*} is the \textbf{dereference} operator: gets the thing the pointer points to
    \end{itemize}
\end{frame}

\begin{frame}[fragile]{Allocating objects on the heap}
    \begin{itemize}
        \item Objects can be allocated on the heap using the \lstinline{new} keyword \pause
        \item \lstinline{new} gives a pointer to the new instance
    \end{itemize}
    \pause
    \begin{lstlisting}
// To use a parameterless constructor
MyClass* myInstance = new MyClass;

// To use a constructor with parameters
MyClass* myOtherInstance = new MyClass(1, 2, 3);
    \end{lstlisting}
\end{frame}

\begin{frame}[fragile]{Deleting objects from the heap}
    \begin{itemize}
        \item Objects instantiated with \lstinline{new} must be deleted using \lstinline{delete} \pause
    \end{itemize}
    \begin{lstlisting}
delete myInstance;
    \end{lstlisting}
    \pause
    \begin{itemize}
        \item Forgetting to do this is a \textbf{memory leak} \pause
        \item Deleting something \textbf{twice} is bad \pause
        \item Trying to \textbf{dereference a deleted pointer} is bad
    \end{itemize}
\end{frame}

\begin{frame}[fragile]{Addressing and dereferencing}
    \begin{lstlisting}
int a = 7;

// Address-of operator
int* b = &a;

// Dereferencing
int c = *b;
    \end{lstlisting}
    \pause
    \begin{itemize}
        \item \lstinline{&} gets the \textbf{address} of a variable, i.e.\ a pointer to it
        \item \lstinline{*} \textbf{dereferences} the pointer, i.e.\ looks up the thing it points to
    \end{itemize}
\end{frame}

\begin{frame}[fragile]{Socrative \texttt{6E8NSW3IN}}
    \begin{lstlisting}
int a = 7;
int* b = &a;
int c = *b;
    \end{lstlisting}
    Suppose that the variables are assigned to the following memory addresses:
    \begin{center}
        \begin{tabular}{r|ccc}
            \textbf{Variable} & {\lstinline!a!} & {\lstinline!b!} & {\lstinline!c!} \\ \hline
%            \textbf{Variable} & a & b & c \\
            \textbf{Address} & 1000 & 1004 & 1008
        \end{tabular}
    \end{center}
    \pause
    \begin{enumerate}
        \item What is the value of \lstinline{a}? \pause
        \item What is the value of \lstinline{b}? \pause
        \item What is the value of \lstinline{c}?
    \end{enumerate}
\end{frame}

\begin{frame}[fragile]{Dereferencing for objects}
    \begin{itemize}
        \item This would work...
    \end{itemize}
    \begin{lstlisting}
Circle* myCircle = new Circle(10);
double area = (*myCircle).getArea();
    \end{lstlisting}
    \pause
    \begin{itemize}
        \item \lstinline{->} is a shorthand for dereferencing and accessing a member \pause
        \item The code below is equivalent to the code above, but clearer
    \end{itemize}
    \begin{lstlisting}
Circle* myCircle = new Circle(10);
double area = myCircle->getArea();
    \end{lstlisting}
\end{frame}

\begin{frame}[fragile]{Null pointer}
    \begin{itemize}
        \item Pointers can have a special value \lstinline{nullptr} \pause
        \item This signifies the pointer doesn't point to anything \pause
    \end{itemize}
    \begin{lstlisting}
MyClass* notAnInstance = nullptr;
    \end{lstlisting}
    \pause
    \begin{itemize}
        \item Similar to \lstinline[language=Python]{None} in Python \pause
        \item You may also see \lstinline{NULL} used instead of \lstinline{nullptr} ---
            the meaning is the same
    \end{itemize}
\end{frame}

%\begin{frame}[fragile]{Dangling pointers}
    %\begin{lstlisting}
%delete myInstance;
    %\end{lstlisting}
    %\pause
    %\begin{itemize}
        %\item The instance that \lstinline{myInstance} pointed to has been deleted, but \lstinline{myInstance} still holds its old address \pause
        %\item \lstinline{myInstance} is a \textbf{dangling pointer}, and dereferencing it would certainly be a bug \pause
        %\item Get into the habit of setting pointers to \lstinline{nullptr} after deleting them \pause
        %\begin{itemize}
            %\item Dereferencing a null pointer is still a bug, but will generally crash your program straight away rather than causing obscure errors
        %\end{itemize}
    %\end{itemize}
%\end{frame}

\begin{frame}[fragile]{Polymorphism}
    \begin{itemize}
        \item Can have a pointer to a \textbf{base class} which is actually an instance of a \textbf{derived class}
    \end{itemize}
    \begin{lstlisting}
class Shape { ... };
class Circle : Shape { ... };

Shape* myShape = new Circle(10);
std::cout << myShape.getArea() << std::endl;
    \end{lstlisting}
\end{frame}

%\part{Type conversion}
\frame{\partpage}

\begin{frame}[fragile]{Numeric type conversion}
    \begin{itemize}
        \item Most conversions happen automatically on assignment \pause
        \item Conversions that can ``lose'' data will give compiler warnings
    \end{itemize}
    \pause
    \begin{lstlisting}
int a = 27;
double b = a;    // OK
double c = 12.3;
int d = c;       // Warning
    \end{lstlisting}
\end{frame}

\begin{frame}[fragile]{Explicit type conversion}
    \begin{itemize}
        \item Common pitfall: an \lstinline{int} divided by an \lstinline{int} is an \lstinline{int}
    \end{itemize}
    \pause
    \begin{lstlisting}
int a = 3;
int b = 2;
double fraction = a / b;
std::cout << fraction << std::endl; // Prints 1.0
    \end{lstlisting}
    \pause
    \begin{itemize}
        \item The code calculates \lstinline{a / b} as an \lstinline{int}, then converts the result to a \lstinline{double} \pause
        \item Need to \textbf{cast} the \lstinline{int}s to \lstinline{double}s to get the desired result
    \end{itemize}
    \pause
    \begin{lstlisting}
double fraction = static_cast<double>(a) / static_cast<double>(b);
std::cout << fraction << std::endl; // Prints 1.5
    \end{lstlisting}
\end{frame}

\begin{frame}[fragile]{Static cast}
    \begin{itemize}
        \item \lstinline{static_cast<Type>(value)} is for type conversions that the compiler can verify are valid \pause
        \item E.g. converting between basic (numeric) types \pause
        \item E.g. converting (pointer to derived class) to (pointer to base class)
    \end{itemize}
    \begin{lstlisting}
Circle* myCircle = new Circle(1);
Shape* myShape = static_cast<Shape*>(myCircle);
    \end{lstlisting}
\end{frame}

\begin{frame}[fragile]{Dynamic casting}
    \begin{itemize}
        \item Some casts can fail at runtime \pause
        \item E.g. converting (pointer to base class) to (pointer to derived class) \pause
        \item E.g. we have a \lstinline{Shape*} and want to convert it to a \lstinline{Circle*},
            but what if it's actually a \lstinline{Square*}? \pause
        \item \lstinline{dynamic_cast<Circle*>(myPointer)} will convert the pointer if possible,
            otherwise it will evaluate to \lstinline{nullptr}
    \end{itemize}
\end{frame}

\begin{frame}[fragile]{Other types of cast}
    \begin{itemize}
        \item \lstinline{reinterpret_cast<>()} reinterprets the bytes in memory as a different type
        \begin{itemize}
            \item This is dangerous, and only useful in certain specialised circumstances
        \end{itemize}
        \pause
        \item \textbf{C-style casts} can behave like \lstinline{static_cast} or \lstinline{reinterpret_cast} depending on context
        \begin{itemize}
            \item Syntax: \lstinline{(Type)value} \pause
            \item Also dangerous, but often used for converting between basic (numeric) types
        \end{itemize}
    \end{itemize}
\end{frame}

\begin{frame}[fragile]{C-style casts}
    \begin{lstlisting}
double fraction = static_cast<double>(a) / static_cast<double>(b);
    \end{lstlisting}
    \pause
    \begin{itemize}
        \item You may see this written as
    \end{itemize}
    \begin{lstlisting}
double fraction = (double)a / (double)b;
    \end{lstlisting}
    \pause
    \begin{itemize}
        \item This is more concise, but many C++ programmers consider it bad style
    \end{itemize}
\end{frame}

\begin{frame}[fragile]{Converting to and from strings}
    \begin{itemize}
        \item You \textbf{can't} use typecasting to convert values to and from strings \pause
        \item Instead, use \lstinline{stringstream} \pause
        \item There are many examples online
    \end{itemize}
\end{frame}


%\part{Live coding: Image generation}
\frame{\partpage}

\begin{frame}[fragile]{CImg setup}
    \begin{enumerate}
        \item \label{i:cimg_newproj} Open Visual C++ 2015 and create a new ``Win32 Console Application'' (under Templates $\to$ Visual C++ $\to$ Win32)
        \item Open a web browser to \url{http://cimg.eu/download.shtml} and download the ``Standard Package''
        \item Find the \texttt{CImg.h} file inside the downloaded zip, and copy it to the project folder created in Step~\ref{i:cimg_newproj} (next to the other \texttt{.cpp} and \texttt{.h} files)
        \item Add the following to the bottom of \texttt{stdafx.h}:
    \end{enumerate}
    \begin{lstlisting}
#include "CImg.h"
using namespace cimg_library;
    \end{lstlisting}
\end{frame}


\part{Live coding: Generating Images}
\frame{\partpage}

% -------------------------------------------------------

%\part{The compiler}
%\frame{\partpage}
%
%\begin{frame}
%	\frametitle{The build process}
%	\includegraphics[height=\textwidth,angle=90]{compiler_sketch}
%\end{frame}

\end{document}
