\part{Object-oriented programming in C++}
\frame{\partpage}

\begin{frame}[fragile]{OOP refresher}
    \begin{itemize}
        \item A \textbf{class} is a collection of \textbf{fields} (data) and \textbf{methods} (functions)
        \item Fields and methods may be \textbf{public} (accessible everywhere), \textbf{protected} (accessible in the class and classes that inherit from it) or \textbf{private} (accessible in the class only)
        \item Classes may \textbf{inherit} fields and methods from other classes
        \item Subclasses may \textbf{override} methods which they inherit --- this gives rise to \textbf{polymorphism}
    \end{itemize}
\end{frame}

\begin{frame}[fragile]{Class declarations}
    \begin{lstlisting}
class MyClass
{
public:
    void doMethod(int x)
    {
        std::cout << x << std::endl;
    }

private:
    int field = 7;
};
    \end{lstlisting}
\end{frame}

\begin{frame}[fragile]{Fields and methods}
    \begin{itemize}
        \item Fields and methods are declared in the class declaration, just like variables and functions
        \item Class declaration is split into sections by access type (\textbf{public}, \textbf{protected}, \textbf{private})
    \end{itemize}
\end{frame}

\begin{frame}[fragile]{Overloading}
    \begin{itemize}
        \item Functions and methods can be defined with the \textbf{same name} but \textbf{different parameters}
    \end{itemize}
    \begin{lstlisting}
    double getVectorLength(double x, double y)
    {
        return sqrt(x * x + y * y);
    }

    double getVectorLength(Vector v)
    {
        return sqrt(v.x * v.x + v.y * v.y);
    }
    \end{lstlisting}
\end{frame}

\begin{frame}[fragile]{Constructors and destructors}
    \begin{itemize}
        \item The \textbf{constructor} is executed when the class is instantiated
        \item The \textbf{destructor} is executed when the instance is freed
    \end{itemize}
	\begin{lstlisting}
class MyClass
{
public:
    MyClass()
    {
    }
    
    ~MyClass()
    {
    }
};
	\end{lstlisting}
\end{frame}

\begin{frame}[fragile]{Constructors}
    \begin{itemize}
        \item The constructor name matches the class name
        \item Constructors can take parameters
        \item The constructor can be overloaded, i.e.\ can have several constructors with different parameters
    \end{itemize}
	\begin{lstlisting}
class MyClass
{
public:
    // Parameterless constructor
    MyClass() { }
    
    // Constructor with parameters
    MyClass(int x, double y) { }
};
	\end{lstlisting}
\end{frame}

\begin{frame}[fragile]{Destructors}
    \begin{itemize}
        \item The destructor name is the class name prefixed with $\sim$ (tilde)
        \item Destructors \textbf{cannot} take parameters
    \end{itemize}
\end{frame}

\begin{frame}[fragile]{Modular program design}
    \begin{itemize}
        \item Method \textbf{declarations} go in the class declaration
        \item Method \textbf{definitions} look like function definitions,
            with the function name replaced with \lstinline{ClassName::methodName}
        \item Method definitions can also go inline into the class declaration
        \begin{itemize}
            \item Best used for short (1 or 2 line) methods
        \end{itemize}
        \item Good practice: Put class declaration in \texttt{ClassName.h}, and method definitions in \texttt{ClassName.cpp}
    \end{itemize}
\end{frame}

\begin{frame}[fragile]{Example: Circle.h}
    \begin{lstlisting}
#pragma once

class Circle
{
public:
    Circle(double radius);
    
    double getArea();

private:
    double radius;
};
    \end{lstlisting}
\end{frame}

\begin{frame}[fragile]{Example: Circle.cpp}
    \begin{lstlisting}
#include "stdafx.h"
#include "Circle.h"

Circle::Circle(double radius)
    : radius(radius)
{
}

double Circle::getArea()
{
    return M_PI * radius * radius;
}
    \end{lstlisting}
\end{frame}

\begin{frame}[fragile]{Inheritance}
    \begin{lstlisting}
class Shape
{
public:
    virtual double getArea();
};

class Circle : Shape
{
public:
    virtual double getArea()
    {
        return M_PI * radius * radius;
    }
};
    \end{lstlisting}
    \begin{itemize}
        \item Methods to be overridden must be marked \textbf{virtual}
    \end{itemize}
\end{frame}

\begin{frame}[fragile]{Pure virtual methods}
    \begin{itemize}
        \item \textbf{Abstract classes} should never be instantiated --- they only exist to serve as a base class
        \item \textbf{Abstract methods} are not defined in the base class, and must be overridden in the subclass
        \item In C++, abstract methods are called \textbf{pure virtual}
        \item Having at least one pure virtual method \textbf{automatically} makes the class abstract
    \end{itemize}
    \begin{lstlisting}
class Shape
{
public:
    virtual double getArea() = 0;
};
    \end{lstlisting}
\end{frame}

\begin{frame}[fragile]{Instantiation}
    \begin{itemize}
        \item To instantiate with a \textbf{parameterless constructor}, just declare a variable
    \end{itemize}
    \begin{lstlisting}
MyClass myInstance;
    \end{lstlisting}
    \begin{itemize}
        \item To instantiate with a \textbf{constructor with parameters}, add the parameters in parentheses
    \end{itemize}
    \begin{lstlisting}
MyClass myInstance(27);
    \end{lstlisting}
    \begin{itemize}
        \item This allocates the instance on the \textbf{stack}
        \item The instance is destroyed (and the destructor is called) when the variable goes \textbf{out of scope}
    \end{itemize}
\end{frame}

\begin{frame}[fragile]{Accessing members}
    \begin{itemize}
        \item Use dot (\lstinline{.}) notation, similar to Python
    \end{itemize}
    \begin{lstlisting}
Circle myCircle(10);
double area = myCircle.getArea();
    \end{lstlisting}
\end{frame}
