\part{Test driven development}
\frame{\partpage}

\begin{frame}[fragile]{Unit testing}
    \begin{itemize}
        \item A \textbf{unit test} or \textbf{test case} is a piece of code that verifies a unit (e.g.\ a function or class) of a program \pause
        \item E.g.\ verifies that a function called with a particular set of parameters returns the expected result \pause
        \item The following might be unit tests for a \lstinline!factorial! function:
        \begin{itemize}
            \item \lstinline!factorial(1) == 1!
            \item \lstinline!factorial(2) == 2!
            \item \lstinline!factorial(3) == 6!
            \item \lstinline!factorial(4) == 24!
        \end{itemize}
    \end{itemize}
\end{frame}

\begin{frame}{Why do unit testing?}
    \begin{itemize}
        \item Can find problems that normal testing misses \pause
        \item \textbf{Bottom-up} testing --- if the \textbf{parts} work properly, it's easier to make the \textbf{whole} work properly \pause
        \item When code is \textbf{changed}, can verify that nothing was broken
    \end{itemize}
\end{frame}

\begin{frame}{Caveats}
    \begin{itemize}
        \item Have to spend time writing tests \pause
            \begin{itemize}
                \item Not really a drawback --- good unit tests will probably \textbf{save} more time in debugging than it takes to write them \pause
            \end{itemize}
        \item Can give a false sense of security \pause
            \begin{itemize}
                \item Unit tests can't cover 100\% of a complex program --- \textbf{not a substitute} for other forms of testing
            \end{itemize}
    \end{itemize}
\end{frame}

\begin{frame}{Test driven development (TDD)}
    \begin{itemize}
        \item A development process that advocates writing the unit tests \textbf{first} \pause
        \item Repeat the following three steps: \pause
            \begin{enumerate}
                \item \textbf{Red}: create a new test case, which should initially \textbf{fail} \pause
                \item \textbf{Green}: write code to make the new test \textbf{succeed} (without causing the other test cases to fail) \pause
                \item \textbf{Refactor}: \textbf{improve} the code, ensuring that all tests still \textbf{succeed}
            \end{enumerate}
    \end{itemize}
\end{frame}

\begin{frame}{Why TDD?}
    \begin{itemize}
        \item All the benefits of \textbf{unit testing}, plus... \pause
        \item Often easier to convert a \textbf{user story} into test cases rather than directly into code \pause
        \item Writing the bare minimum of code to make the test ``green''
            lets you \textbf{focus on user stories}, not on \textbf{over-generalisation} or \textbf{non-essential functionality} \pause
            \begin{itemize}
                \item \textbf{KISS}: Keep It Simple, Stupid
                \item \textbf{YAGNI}: You Aren't Gonna Need It
            \end{itemize}
    \end{itemize}
\end{frame}

\begin{frame}{Red}
    \begin{itemize}
        \item Create a new test case, which should initially \textbf{fail} \pause
        \item Write only enough code to allow the test case to compile and run,
            e.g.\ write a \textbf{stub} function \pause
        \item What if the test succeeds? \pause
            \begin{itemize}
                \item Maybe you already implemented that feature? \pause
                \item Maybe the test case is wrong? \pause
                \item Maybe your unit testing code is broken?
            \end{itemize}
    \end{itemize}
\end{frame}

\begin{frame}{Green}
    \begin{itemize}
        \item Add the \textbf{bare minimum} of code to make the new test case succeed \pause
            \begin{itemize}
                \item \textbf{K}eep \textbf{I}t \textbf{S}imple, \textbf{S}tupid! \pause
            \end{itemize}
        \item Verify that \textbf{all} unit tests now succeed \pause
        \item What if old tests now fail? \pause
            \begin{itemize}
                \item Fix it \pause
                \item \textbf{Or} revert and start again --- can be faster than debugging \pause
                \item (you \textbf{did} commit before you started, right?)
            \end{itemize}
    \end{itemize}
\end{frame}

\begin{frame}{Refactor}
    \begin{itemize}
        \item E.g. remove duplication, improve names, add documentation, apply design patterns, ... \pause
        \item To generalise or not to generalise? \pause
        \item \textbf{Do} generalise if it makes the code \textbf{simpler} \pause
        \item \textbf{Don't} generalise because you ``might'' need it later \pause
            \begin{itemize}
                \item \textbf{Y}ou \textbf{A}ren't \textbf{G}onna \textbf{N}eed \textbf{I}t!
                \item Wait until it \textbf{is} needed in another cycle \pause
            \end{itemize}
        \item Verify that \textbf{all} unit tests still succeed
    \end{itemize}
\end{frame}

