\part{Your first C++ program}
\frame{\partpage}

\begin{frame}
	\frametitle{Project setup}
	\begin{itemize}
		\item Open \textbf{Visual Studio 2015} from the Start menu
		\item Click \textbf{New Project}
		\item Choose \textbf{Templates $\to$ Visual~C++ $\to$ Win32 $\to$ Win32~Console~Application}
		\item Choose an appropriate name and location, and click \textbf{OK}
		\item Click \textbf{Finish}
		\item If asked about source control, click \textbf{Cancel}
	\end{itemize}
\end{frame}

\begin{frame}[fragile]
	\frametitle{The code}
%	\begin{itemize}
%		\item Edit \texttt{$\langle$YourApplicationName$\rangle$.cpp} to match the following:
%	\end{itemize}
	\begin{lstlisting}
// ConsoleApplication1.cpp : Defines the entry point for the console application.

#include "stdafx.h"

int main()
{
    std::cout << "Hello, world!" << std::endl;
    return 0;
}
	\end{lstlisting}
	\begin{itemize}
		\item Add the following line to the end of \texttt{stdafx.h}:
	\end{itemize}
	\begin{lstlisting}
#include <iostream>
	\end{lstlisting}
\end{frame}

\begin{frame}[fragile]{Running it}
	\begin{itemize}
		\item Click \includegraphics[height=2ex]{run_button.png}, or press \textbf{F5} \pause
		\item It worked, but the window disappeared before we could see it! \pause
		\item Solution 1: click \textbf{Debug $\to$ Start~Without~Debugging}, or press \textbf{Ctrl~+~F5}
		\item Solution 2: click in the left margin next to the \lstinline{return 0;} line to set a \textbf{breakpoint} ---
			a red circle should appear. Then click \includegraphics[height=2ex]{run_button.png}
	\end{itemize}
\end{frame}

\begin{frame}[fragile]
	\frametitle{Comments}
	\begin{lstlisting}
// ConsoleApplication1.cpp : Defines the entry point for the console application.
	\end{lstlisting}
	\pause
	\begin{itemize}
		\item \lstinline{//} denotes a single-line comment \pause
		\item Equivalent of \lstinline[language=Python]{#} in Python \pause
		\item \textcolor{Gray}{$\hookleftarrow$} denotes a line too long to fit on the slide ---
			in your program this should be a single line \pause
		\item Multi-line comments, delimited by \lstinline{/* */}, are also available
	\end{itemize}
	\begin{lstlisting}
/* This is an example of a multi-line comment
   More comment text
   Even more comment text */
	\end{lstlisting}
\end{frame}

\begin{frame}[fragile]
	\frametitle{The \#include directive}
	\begin{lstlisting}
#include "stdafx.h"
	\end{lstlisting}
	\begin{lstlisting}
#include <iostream>
	\end{lstlisting}
	\pause
	\begin{itemize}
		\item \lstinline{#include} imports definitions from a \textbf{header file} \pause
		\item Similar to \lstinline[language=Python]{import} in Python \pause
		\item \lstinline{#include "..."} (quotes) is used for headers in the current project \pause
		\item \lstinline{#include <...>} (angle brackets) is used for external libraries \pause
		\item \texttt{stdafx.h} is the \textbf{precompiled header} file --- for faster compilation, external library headers should be included here rather than in the main \texttt{.cpp} file
	\end{itemize}
\end{frame}

\begin{frame}[fragile]
	\frametitle{Entry point}
	\begin{lstlisting}
int main()
	\end{lstlisting}
	\pause
	\begin{itemize}
		\item All code must be inside a function \pause
		\item The \textbf{entry point} of an application is (almost) always named \lstinline{main} \pause
		\begin{itemize}
			\item Some types of Windows GUI application use a different name for the entry point \pause
		\end{itemize}
		\item \lstinline{int} means the function returns a value of integer type \pause
		\item \lstinline{()} means the function takes no parameters
	\end{itemize}
\end{frame}

\begin{frame}[fragile]
	\frametitle{Blocks and semicolons}
	\begin{lstlisting}
{
    ...;
    ...;
}
	\end{lstlisting}
	\pause
	\begin{itemize}
		\item Curly braces are used to denote blocks \pause
		\item All statements in C++ end with a semicolon \lstinline{;} \pause
		\item Unlike Python, C++ ignores whitespace (indentation and line breaks) \pause
		\item ... but whitespace is important for readability, so use it anyway
	\end{itemize}
\end{frame}

\begin{frame}[fragile]
	\frametitle{Writing to the console}
	\begin{lstlisting}
    std::cout << "Hello, world!" << std::endl;
	\end{lstlisting}
	\pause
	\begin{itemize}
		\item Equivalent of Python's \lstinline[language=Python]{print} statement \pause
		\item \lstinline{std} is the \textbf{namespace} containing most of the C++ standard library \pause
		\item \lstinline{std::cout} is the console output stream \pause
		\item \lstinline{std::endl} is the end-of-line character \pause
		\item To use \lstinline{std::cout} and \lstinline{std::endl}, it is necessary to
			\lstinline{#include <iostream>} \pause
		\item \lstinline{<<} is the \textbf{insertion operator} --- used to write values to a stream
	\end{itemize}
\end{frame}

\begin{frame}[fragile]
	\frametitle{Exit code}
	\begin{lstlisting}
    return 0;
	\end{lstlisting}
	\pause
	\begin{itemize}
		\item Returning 0 from \lstinline{main} tells the OS that the program completed successfully \pause
		\item Mainly useful for writing tools to be used in DOS/Windows batch scripts or Linux shell scripts ---
			for our purposes, \lstinline{main} will almost always return 0
	\end{itemize}
\end{frame}
