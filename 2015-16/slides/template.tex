% Uncomment this line for on-screen presentation
\documentclass[xcolor={dvipsnames}]{beamer}\usepackage{etoolbox}\newtoggle{printable}\togglefalse{printable}

% Uncomment this line for printable slides (disable animations and don't waste ink)
%\documentclass[handout, xcolor={dvipsnames}]{beamer}\usepackage{etoolbox}\newtoggle{printable}\toggletrue{printable}

% Adjust these for the path of the theme and its graphics, relative to this file
\usepackage{beamerthemeFalmouthGamesAcademy}
%\usepackage{../../beamerthemeFalmouthGamesAcademy}
%\graphicspath{ {../../} }

% Default language for code listings
\lstset{language=C++}

\begin{document}
\title{Title of lecture}   
\subtitle{COMP110: Principles of Computing}

\frame{\titlepage} 

\begin{frame}
	\frametitle{Learning outcomes}
	By the end of this session you will
	\begin{itemize}
		\item Understand a thing
		\item Understand another thing
		\item Be convinced that \LaTeX\ makes better-looking slides than PowerPoint
	\end{itemize}
\end{frame}

\begin{frame}
	\frametitle{Title of slide}
	\begin{itemize}
		\item Point number 1 \pause
		\item Point number 2 \pause
		\begin{itemize}
			\item ``pause'' is optional at the end of items
			\item Or it can be included \pause
			\item Like this
		\end{itemize}
	\end{itemize}
\end{frame}

\part{Part heading}
\frame{\partpage}

\begin{frame}
	\frametitle{Pseudocode}
	\begin{algorithmic}
		\Procedure{Euclid}{$a,b$}\Comment{The g.c.d. of a and b}
			\State $r\gets a\bmod b$
			\While{$r\not=0$}\Comment{We have the answer if r is 0}
				\State $a\gets b$
				\State $b\gets r$
				\State $r\gets a\bmod b$
			\EndWhile\label{euclidendwhile}
			\State \textbf{return} $b$\Comment{The gcd is b}
		\EndProcedure
	\end{algorithmic}
\end{frame}

% NB: [fragile] is required for slides containing lstlisting environments
% Tab characters do not work inside lstlisting -- use spaces instead

\begin{frame}[fragile]
	\frametitle{Code listing: Python}
	\begin{lstlisting}[language=Python]
def factorial(n):
    if type(n) is not int or n < 0:
        raise ValueError("n must be a nonnegative integer")
    else if n <= 1:
        return 1
    else:
        # Recursive call
        return n * factorial(n-1)
	\end{lstlisting}
\end{frame}

\begin{frame}[fragile]
	\frametitle{Code listing: C++}
	\begin{lstlisting}
// My first C++ program

#include <iostream>

int main(int argc, char** argv)
{
    std::cout << "Hello, world!" << std::endl;
    return 0;
}
	\end{lstlisting}
	
	You can also refer to code in text, as in \lstinline{this->example()}.
\end{frame}

\end{document}
