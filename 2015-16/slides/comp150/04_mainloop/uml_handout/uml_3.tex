\documentclass{scrartcl}

\usepackage[hidelinks]{hyperref}
\usepackage[none]{hyphenat}
\usepackage{setspace}
\usepackage{enumitem}
\setlist{nosep} % Make enumerate / itemize lists more closely spaced
\doublespace

\begin{document}

\title{UML Worksheet III}
\author{COMP150: Game Development Practice}
\date{}

\maketitle

Unified Modelling Language (UML) is a way of communicating the design of software using diagrams. It is a notation that built upon the work of Grady Booch, James Rumbaugh, Ivar Jacobson, and the Rational Software Corporation. It was originally developed to support the object-oriented paradigm, although has since been extended to accommodate a diverse range of projects. According to the Object Management Group (OMG), UML is the international standard for software modelling.

\section{In-Class Task}

In today's in-class task you will learn how to draw \textbf{UML Sequence} and \textbf{UML Interaction Overview} diagrams. To complete this you will:

\begin{itemize}
	\item \textbf{Organise} yourselves into your COMP150 project teams.
	\item \textbf{Watch} the video tutorial at \url{https://www.youtube.com/watch?v=cxG-qWthxt4}.
	\item \textbf{Read} \url{http://agilemodeling.com/artifacts/sequenceDiagram.htm}.
	\item \textbf{Draw} a UML Sequence diagram to model ONE component (e.g. sprite control).
	\item \textbf{Watch} the video tutorial at \url{https://www.youtube.com/v/CJUc2crmrMs?version=3\&start=108\&end=163\&autoplay=1}.
	\item \textbf{Read} \url{http://agilemodeling.com/artifacts/interactionOverviewDiagram.htm}.
	\item \textbf{Draw} a UML Interaction Overview diagram to model your main loop.
\end{itemize}

\vspace{0.5ex}

Use the white boards to draw your diagrams. Alternatively, use Gliffy.

\section{Submission}

 This task is not assessed, but will help you on your COMP150 project. Add the diagram to your weekly report for your COMP150 project. This will be reviewed at the next Sprint Review.

\end{document}