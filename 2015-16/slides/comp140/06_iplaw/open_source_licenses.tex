\part{Open Source Software Licenses}
\frame{\partpage}

\begin{frame}{Learning Outcomes}
	In this section you will learn how to...
	
	\begin{itemize}
		\item \textbf{Recognise} the philosophy of open-source licensing and its benefits
		\item \textbf{Differentiate} between the Creative Commons, MIT, Apache, and GNU General Public Licenses
		\item \textbf{Suggest} the most appropriate license for a given circumstance
	\end{itemize}
	
	This section has been adapted from the talk `A Lawyer Looks at the Open Source Revolution' by Robert W. Gomulkiewicz.
\end{frame}

\begin{frame}{Open Source}
	\begin{itemize}
		\item Source = software in source code form
		\item Open = freedom to:
		\begin{itemize}
			\item View the source code
			\item Run the software for any purpose
			\item Modify the software in any way
			\item Distribute the software and any modifications
		\end{itemize}
		\item Software Development Model
		\item Philosophy --- Share and Collaborate
		\item Licensing Model
	\end{itemize}
\end{frame}

\begin{frame}{Open Source}
	\begin{itemize}
		\item Opposite of `proprietary' (`commercial') software:
		\begin{itemize}
			\item Hold the source code as a trade secret
			\item Distribute software as a binary
			\item Limited licensing for derivative works
			\item Buggy, and difficult to extend
		\end{itemize}
	\end{itemize}
\end{frame}

\begin{frame}{Open Source}
Notable names in Open Source:

	\begin{itemize}
		\item Richard Stallman (Free Software Foundation)
		\item Eric Raymond (The Cathedral and the Bazaar)
		\item Linus Torvalds (Linux)
		\item Bruce Perens (Open Source Definition)
	\end{itemize}
\end{frame}

\begin{frame}{Open Source}
\textit{``Given enough eyeballs, all bugs are shallow''}
--- Eric Raymond
\end{frame}

\begin{frame}{Open Source}
Advantages of open-source:

	\begin{itemize}
		\item Scratching an itch to fix or extend something
		\item Forking and Pull-Requests
		\item Peer Review
		\item Centralized decision-making
	\end{itemize}
\end{frame}

\begin{frame}{Open Source}
	\begin{itemize}
		\item Software has always traditionally been shared by scientists and hobbyists
		\item The Internet and WWW makes sharing and collaboration very efficient
		\item Watershed: Netscape licensed Communicator under an open source license
		\item Linux+Apache became the most popular web server
		\item Ever-increasing adoption
	\end{itemize}
\end{frame}

\begin{frame}{Open Source}
As a business model:

	\begin{itemize}
		\item ``Think `free speech', not `free beer''' --- Richard Stallman
		\item Branded distributions
		\item Sell hardware, give away software
		\item Sell services and support
		\item Dual versions, dual licensing
		\item Value added software
		\item Sell sponsorships, ads, and T-shirts
	\end{itemize}
\end{frame}

\begin{frame}{Open Source}
Free and open is not:

	\begin{itemize}
		\item Public domain
		\item Copyright `first sale'
		\item Shareware or freeware
	\end{itemize}
\end{frame}

\begin{frame}{Open Source}
Licenses (and IP law) make it work:

	\begin{itemize}
		\item Control and limitations over use
		\item Risk shifting
		\item To remain free (and worthwhile), software must be copyrighted and licensed
	\end{itemize}
\end{frame}

\begin{frame}{Creative Commons}
Key terms:

	\begin{itemize}
		\item Flexible. Can be customised.
		\item BY = Attribution.
		\item SA = Share Alike.
		\item ND = No Derivatives.
		\item NC = Non-Commercial.
	\end{itemize}
\end{frame}

\begin{frame}{BSD Licence}
	\begin{itemize}
		\item License grant:  unlimited use, modification, distribution
		\item No warranties; disclaimer of consequential damages
		\item No endorsement
		\item Attribution required.
	\end{itemize}
\end{frame}

\begin{frame}{MIT Licence}
	\begin{itemize}
		\item commercial use
		\item can modify
		\item can distribute
		\item can sub-license
		\item private use
		\item cannot hold liable
		\item must include copyright notice
		\item must include copy of license in distribution
	\end{itemize}
\end{frame}

\begin{frame}{Apache Licence}
	\begin{itemize}
		\item freely download and use Apache software, in whole or in part, for personal, company
		 internal, or commercial purposes;
		\item use Apache software in packages or distributions that you create.
		\item must redistribute any piece of Apache-originated software with proper attribution;
		\item must not use any marks owned by The Apache Software Foundation in any way that 
		might state or imply that the Foundation endorses your distribution;
	\end{itemize}
\end{frame}

\begin{frame}{Apache Licence}
	\begin{itemize}
		\item must not use any marks owned by The Apache Software Foundation in any way that
		 might state or imply that you created the Apache software in question.
		\item must include a copy of the license in any redistribution you may make that includes Apache software;
		\item must provide clear attribution to The Apache Software Foundation for any distributions that include Apache software.
	\end{itemize}
\end{frame}

\begin{frame}{GNU General Public Licence}
	\begin{itemize}
		\item Unlimited right to run program
		\item Unlimited access to source code
		\item Unlimited right to distribute verbatim copies
		\item May create derivatives IF you agree to make the derivatives “free”
		\item License is “viral”
		\item No warranties; disclaimer of consequential damages
	\end{itemize}
\end{frame}

\part{Practical Activity}
\frame{\partpage}

\begin{frame}[fragile]{Socrative \texttt{JBYPC3BBY}}
In your teams:
	\begin{itemize}
		
		\item \textbf{Research} the SCO litigation case.
		\item \textbf{Discuss} the litigation for 10 minutes on Slack.
	
		\item \textbf{State ONE} fact about the case for EACH member in your group.
	\end{itemize}
\end{frame}

\begin{frame}[fragile]{Socrative \texttt{JBYPC3BBY}}
In your teams:
	\begin{itemize}
		
		\item \textbf{Research} the licenses in more depth.
		\item \textbf{Discuss} which license is most suited to your COMP150 game for 10 minutes on Slack.
	
		\item \textbf{Suggest ONE} license \textbf{and justify why} you chose it.
	\end{itemize}
\end{frame}