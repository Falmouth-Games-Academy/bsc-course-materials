\part{The Twitter API}
\frame{\partpage}

\begin{frame}{REST}
    \begin{itemize}
        \item REST = \textbf{Representational State Transfer} \pause
        \item Many web services provide a \textbf{REST API} \pause
            \begin{itemize}
                \item Including your COMP110 client/server coding task! \pause
            \end{itemize}
        \item An API is \textbf{RESTful} if it is: \pause
            \begin{itemize}
                \item Based on a \textbf{client-server model}: clients send requests to a server \pause
                \item \textbf{Stateless}: each request from the client is self-contained \pause
                \item \textbf{Cacheable}: the API makes clear which responses can and cannot be cached \pause
                \item \textbf{Layered}: the ``server'' may actually be a cluster of machines \pause
                \item Uses a \textbf{uniform interface}: e.g.\ HTTP requests, URLs, XML, JSON, ...
            \end{itemize}
    \end{itemize}
\end{frame}

\begin{frame}{Twitter API}
    \begin{itemize}
        \item Twitter provides a REST API \pause
        \item Example: to post a tweet \pause
            \begin{itemize}
                \item (see documentation at \url{https://dev.twitter.com/rest/reference/post/statuses/update}) \pause
                \item The client makes an \textbf{HTTP POST request} to
                    \url{https://api.twitter.com/1.1/statuses/update.json?status=Hello+world} \pause
                \item The \textbf{HTTP request header} contains authentication information for the app
                    and for the user \pause
                \item The \textbf{response} from the server is a JSON document containing information about the
                    posted tweet
            \end{itemize}
    \end{itemize}
\end{frame}

\begin{frame}{Libraries}
    \begin{itemize}
        \item Working with REST APIs directly through HTTP requests can be cumbersome \pause
        \item For most popular web services, there are many \textbf{libraries} (official and third party)
            which wrap the REST APIs in a more programmer-friendly interface \pause
        \item For Twitter: \url{https://dev.twitter.com/overview/api/twitter-libraries} \pause
        \item For today's live coding, I will be using the \textbf{Tweepy} library for Python
    \end{itemize}
\end{frame}

