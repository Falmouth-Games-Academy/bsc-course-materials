% Adjust these for the path of the theme and its graphics, relative to this file
%\usepackage{beamerthemeFalmouthGamesAcademy}
\usepackage{../../beamerthemeFalmouthGamesAcademy}
\usepackage{multimedia}
\graphicspath{ {../../} }

% Default language for code listings
\lstset{language=C++,
        morekeywords={each,in,nullptr}
}

% For strikethrough effect
\usepackage[normalem]{ulem}
\usepackage{wasysym}

\usepackage{pdfpages}

% http://www.texample.net/tikz/examples/state-machine/
\usetikzlibrary{arrows,automata}

\newcommand{\modulecode}{COMP260}\newcommand{\moduletitle}{Distributed Systems}\newcommand{\sessionnumber}{5}

\newcommand{\fullbleed}[1]{
\begin{frame}[plain]
	\begin{tikzpicture}[remember picture, overlay]
		\node[at=(current page.center)] {
			\includegraphics[width=\paperwidth]{#1}
		};
	\end{tikzpicture}
\end{frame}
}

\newcommand{\picturepage}[2]{
\begin{frame}[plain]
	\begin{tikzpicture}[remember picture, overlay]
		\node[at=(current page.center)] {
			\includegraphics[width=\paperwidth]{#1}
		};
		\draw<1>[draw=none, fill=black, opacity=0.9] (-1,-5.2) rectangle (current page.south east);
		\node[draw=none,text width=0.96\paperwidth, align=right] at (5.5,-5.5) {\tiny{#2}};
	\end{tikzpicture}
\end{frame}
}

\newcommand{\notepicx}[5]{
\begin{frame}[plain]
	\begin{tikzpicture}[remember picture, overlay]
		\node[at=(current page.center)] {
			\includegraphics[width=\paperwidth]{#1}
		};
		\node[draw=none, fill=black, text width=#5\paperwidth] at ([xshift=#3, yshift=#4] current page.center) {\small{#2}};
	\end{tikzpicture}
\end{frame}
}

\newcommand{\notepic}[4]{
	\notepicx{#1}{#2}{#3}{#4}{0.4}
}

\setbeamertemplate{navigation symbols}{}

\begin{document}
\title{\sessionnumber: Research Methods}
\subtitle{\modulecode: \moduletitle}

\frame{\titlepage} 

\begin{frame}
	\frametitle{Learning Outcomes}
	\begin{itemize}
		\item \textbf{Compare} research methods
		\item \textbf{Distinguish} between qualitative and quantitative measurement
		\item \textbf{Assess} the suitability of a research design for a given question
		\item \textbf{Design} a suitable market research task
	\end{itemize}
\end{frame}

\picturepage{research_egg}{Saunders, Lewis, and Thornhill, 2012}

\begin{frame}
	\frametitle{Philosophy and Practice}
	\begin{itemize}
		\item \textbf{Ontology}---your view of the nature of reality; assumptions on what the world is and how the world works
		\item \textbf{Epistemology}---your view of the nature of knowledge; beliefs on what constitutes knowledge and acceptable evidence
		\item \textbf{Axiology}---your view of the role of people in research; how professional researchers should conduct themselves
		\item \textbf{Approach to Inference}---they ways in which you approach sense-making and form conclusions; notably, deduction, induction, and/or abduction
	\end{itemize}
\end{frame}

\begin{frame}
	\frametitle{Well-Known Philosophies}
	\begin{itemize}
		\item \textbf{Positivism} \\
		Knowledge derived from logical and mathematical treatments and reports of sensory experience are the exclusive source of all authoritative knowledge
		\item \textbf{Interpretivism} \\
		The social realm may not be subject to the same determinism and laws as the natural world; knowledge is constructed from understanding the interpretations that social actions have for the people being studied
	\end{itemize}
\end{frame}

\begin{frame}
	\frametitle{Mary's Example}
	
	Mary wants to understand how permanent loss in persistent game worlds influences the stress-levels, interest, and purchasing behaviour of players. 
	To this end, she starts planning a research project.
	
	\vspace{1em}
	
	However, many approaches are possible:
	
	\begin{itemize}
		\item An observational study or experiment exploring measurable indicators of stress, while tracking in-game behaviour and events
		\item An interview study where the researcher investigates workers’ perspective on stress and its influence on them
	\end{itemize}
\end{frame}

\begin{frame}
	\frametitle{Mary's Example}
	
	So, what are you philosophical positions?
	
	\vspace{1em}
	
	\textbf{Discuss} with a partner for \textbf{5} minutes
	
	\begin{itemize}
		\item Focus on ontology, epistemology, axiology, and approach to inference
		\item Probe whether your is positivist, interpretivist, or neither
	\end{itemize}
	
	\textbf{Discuss} openly with the class for \textbf{10} minutes
	
	\begin{itemize}
		\item Debate the `best' philosophies
		\item Identify how many people in the class are positivist, interpretivist, or neither
	\end{itemize}
\end{frame}

\begin{frame}
	\frametitle{Research Design}
	
	Research design is a \textbf{coherent framework} that outlines each procedure involved in all stages of the research, from the hypothesis to the analysis.

	\begin{itemize}
		\item Complex maze of decisions to navigate
		\item Fundamentally based on your own philosophy
		\item Conventionally, market researchers typically assume an empirical position---that knowledge should be drawn from observations
		\item Four key decisions:
		\begin{itemize}
			\item Mono or Multi Method
			\item Qualitative or Quantitative or Mixed Measurement
			\item Longitudinal or Cross-Sectional
			\item Objectivist, Subjectivist, or Critical Modes of Collection
		\end{itemize}
	\end{itemize}
\end{frame}

\begin{frame}
	\frametitle{Methodology}
	\begin{itemize}
		\item \textbf{Mono-Method} \\
		Use a single data collection method
		\item \textbf{Multi-Method} \\
		More than one data collection/analysis method
	\end{itemize}
\end{frame}

\begin{frame}
	\frametitle{Types of Data}
	\begin{itemize}
		\item \textbf{Quantiative} \\
		Numerical data
		\item \textbf{Qualitative} \\
		Systematic coding and decomposition of meanings from observations
	\end{itemize}
\end{frame}

\begin{frame}
	\frametitle{Time-Frame}
	\begin{itemize}
		\item \textbf{Longitudinal} \\
		Repeated over a long period of time; trends in the market
		\item \textbf{Cross-Sectional} \\
		Study of a particular situation at a particular time; the status-quo
	\end{itemize}
\end{frame}

\begin{frame}
	\frametitle{Data Collection Methods}
	
	\begin{itemize}
		\item \textbf{Objectivist} \\
		Test; Questionnaire; Experiment; Quasi-Experiment; Structured Observation (e.g., Market Data)
		\item \textbf{Subjectivist} \\
		Interview; Focus Group; Ethnography; General Observation
		\item \textbf{Critical} \\
		Content Analysis
	\end{itemize}
\end{frame}

\picturepage{methodological_choice}{Saunders, Lewis, and Thornhill, 2012}

\begin{frame}
	\frametitle{Activity}
	
	In pairs, review and analyse the following proposal:
	
	\vspace{1em}
	
	\url{https://www.academia.edu/4420526/Research_Proposal_Video_Game_Preferences_Empathy_trait}
	
	\vspace{1em}
	
	Pay careful attention to the underlying philosophy, methodology, and its credibility.
	
	\vspace{1em}
	
	(20 minutes)
	
\end{frame}

\begin{frame}
	\frametitle{Activity}
	
	As a class, discuss the content of the proposal. Could we draw useful insight from this research?
	
	\vspace{1em}
	
	(10 minutes)
	
\end{frame}

\begin{frame}
	\frametitle{Be Critical}
	
	\begin{itemize}
		\item Ensure sources of data are credible
		\item Assure validity and reliability in measurement
		\item Apply reasoning appropriate to your philosophy
		\item If applying statistical analyses, assure assumptions hold
		\item Respect limitations
	\end{itemize}
	
	Market research done wrong can be disastrous:
	
	\vspace{1em}
	
	\url{https://www.qualtrics.com/blog/coca-cola-market-research/}
	
\end{frame}

\end{document}
