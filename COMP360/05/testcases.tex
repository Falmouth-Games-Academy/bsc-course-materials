% !TeX root = ./2020-21-COMP360-05-workshop-materials_screen.tex
\part{Test cases}
\frame{\partpage}

\begin{frame}{Three questions}
    \begin{itemize}
        \pause\item What makes a good test case?
        \pause\item What makes a good bug report?
        \pause\item What does a unit test describe?
        \pause\item The answer to these three questions is very similar...
    \end{itemize}
\end{frame}

\begin{frame}{Test case}
    \pause A good test case should include:
    \begin{itemize}
        \pause\item Any assumptions, initial conditions, prerequisites
        \pause\item A set of steps to follow
        \pause\item The expected result
        \pause\item (When the test is carried out) The actual result
    \end{itemize}
\end{frame}

\begin{frame}{Bug report}
    \pause A good bug report should include:
    \begin{itemize}
        \pause\item Any assumptions, initial conditions, prerequisites
        \pause\item A set of steps to follow
        \pause\item The expected result
        \pause\item The actual result (the bug being that this differs from the expected result)
    \end{itemize}
\end{frame}

\begin{frame}{Unit test}
    \pause A good unit test should include:
    \begin{itemize}
        \pause\item Code to set up any assumptions, initial conditions, prerequisites
        \pause\item A function to execute (a set of steps)
        \pause\item An assertion checking the expected result matches the actual result
    \end{itemize}
\end{frame}

\begin{frame}{How they fit together}
    \begin{itemize}
        \pause\item A test case is general --- can fit at unit, integration, system or acceptance testing level
        \pause\item A unit test is essentially an automated test case
        \pause\item A bug report is suggestive of a new test case that should be added to the test plan
    \end{itemize}
\end{frame}
