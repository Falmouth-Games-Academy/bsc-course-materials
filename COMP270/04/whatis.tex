\part{What is PCG?}
\frame{\partpage}

\begin{frame}{What is procedural content generation (PCG)?}
	\begin{itemize}
		\item<3-> \textbf{Procedural}: by computer program or algorithm,
			with little or no direct input from designer or user
		\item<2-> \textbf{Content}: levels, maps, art, animations, stories, items,
			quests, music, weapons, vehicles, characters, ...
		\item<1-> \textbf{Generation}: creating stuff
	\end{itemize}
\end{frame}

\begin{frame}{Types of PCG}
	\begin{itemize}
		\pause\item \textbf{Online}
			\begin{itemize}
				\pause\item Generate content at run-time
				\pause\item Part of the game
			\end{itemize}
		\pause\item \textbf{Offline}
			\begin{itemize}
				\pause\item Generate content at design-time
				\pause\item Tool for developers
			\end{itemize}
	\end{itemize}
\end{frame}

\begin{frame}{PCG $\neq$ randomness}
	\begin{itemize}
		\pause\item Many PCG systems use random numbers, but randomness in itself is not PCG
		\pause\item Can have PCG without randomness, e.g.\ based on fractals or simulations
		\pause\item Randomness in PCG is generally \textbf{constrained} to produce desired content
		\pause\item Shuffling a deck of cards for a game of Solitaire is \textbf{not} PCG!
	\end{itemize}
\end{frame}

\begin{frame}{Not to be confused with...}
	\begin{itemize}
		\pause\item \textbf{Procedural Rhetoric} / \textbf{Procedurality} (Bogost)
		\pause\item ``the art of persuasion through rule-based representations and interactions,
			rather than the spoken word, writing, images, or moving pictures''
		\pause\item There: ``procedural'' = ``rule-based''
		\pause\item Here: ``procedural'' = ``algorithmic''
	\end{itemize}
\end{frame}

\begin{frame}{Why PCG?}
	\begin{itemize}
		\pause\item More content for less development effort
		\pause\item Decrease development costs
		\pause\item Increase replayability
		\pause\item Decrease storage requirements
		\pause\item Allow game mechanics based on unseen content
	\end{itemize}
\end{frame}

\begin{frame}{PCG approaches}
	\begin{itemize}
		\pause\item Combining hand-authored blocks
		\pause\item Noise functions
		\pause\item Fractals
		\pause\item L-Systems
		\pause\item Simulation
		\pause\item Evolutionary algorithms
		\pause\item Constraint solving
		\pause\item Machine learning
		\pause\item ...
	\end{itemize}
\end{frame}

\begin{frame}{Further reading}
	Noor Shaker, Julian Togelius and Mark J. Nelson.
	\textit{Procedural Content Generation in Games: A textbook and an overview of current research}.
	Springer, 2016.
	
	Available online: \url{http://pcgbook.com}
\end{frame}
