\part{Module introduction}
\frame{\partpage}

\begin{frame}{Aim}
\begin{center}
To empower you to leverage mathematics and mathematical modelling in the design and implementation of real-time 3D worlds and simulations.
\end{center}
\end{frame}

\begin{frame}{Description}
On this module, you learn the fundamental mathematics involved in the design, development and maintenance of real-time 3D worlds and simulations. In doing so, you will leverage mathematics practically to generate and manipulate worlds and simulations relevant to a range of creative computing contexts. Indicatively, content spans topics such as linear algebra (vectors, matrices and quaternions), geometry, trigonometry, 3D transformation, collision detection, Newtonian mechanics, numerical control, calculus, and efficiency and optimisation of numerical methods.
\end{frame}

\begin{frame}{Learning Outcome}
	\begin{itemize}
		\item SOLVE
		\item Apply knowledge of algorithms, data structures, and mathematics to solve well-defined problems.
		\item Assessment criteria category: PROCESS
	\end{itemize}
\end{frame}

\begin{frame}{Topic schedule}
	\begin{center}
		On LearningSpace
	\end{center}
\end{frame}

\begin{frame}{Timetable}
	\begin{center}
		\url{http://mytimetable.falmouth.ac.uk}
	\end{center}
\end{frame}

\begin{frame}{Assignments}
	\begin{itemize}
		\pause\item Assignment 1: worksheet tasks
			\begin{itemize}
				\pause\item \textbf{Four} worksheets --- roughly 2 weeks each
			\end{itemize}
		\pause\item See LearningSpace for assignment briefs and worksheets
		\pause\item See MyFalmouth for deadlines
	\end{itemize}
\end{frame}

\begin{frame}{Worksheet A}
	\begin{itemize}
		\item B\'ezier curves
		\item Due \textbf{Monday week 4 (14th October)}
	\end{itemize}
\end{frame}
