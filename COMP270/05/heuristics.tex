\part{Heuristics for search}
\frame{\partpage}

\begin{frame}{Depth limiting}
	\begin{itemize}
		\pause\item Standard minimax needs to search all the way to \textbf{terminal} (game over) states
		\pause\item \textbf{Depth limiting} is a common technique to apply minimax to larger games
		\pause\item Still evaluate terminal states as $+1$ / $0$ / $-1$
		\pause\item For nonterminal states at depth $d$, apply a heuristic evaluation instead of searching deeper
		\pause\item Evaluation is a number between $-1$ and $+1$, estimating the probable outcome of the game
	\end{itemize}
\end{frame}

\begin{frame}{1-ply search}
	\begin{itemize}
		\pause\item Case $d=1$
		\pause\item For each move, evaluate the state resulting from playing that move
		\pause\item This is computationally fast
		\pause\item Often easier to design a ``which state is better'' heuristic than to directly design a ``which move to play'' heuristic
	\end{itemize}
\end{frame}

\begin{frame}{Move ordering}
	\begin{itemize}
		\pause\item Minimax can \textbf{stop early} if it sees a value of $+1$ for maximising player or $-1$
			for minimising player
		\pause\item Modifications to minimax algorithm (e.g.\ \textbf{alpha-beta pruning}) lead to more of this
		\pause\item Thus ordering moves from \textbf{best to worst} means faster search
		\pause\item How do we know which moves are ``best'' and ``worst''? Use a heuristic!
	\end{itemize}
\end{frame}

\begin{frame}{Designing heuristics}
	\begin{itemize}
		\pause\item The \textbf{playing strength} of depth limited minimax depends heavily on the design of the \textbf{heuristic}
		\pause\item Good heuristic design requires \textbf{in-depth knowledge} of the tactics and strategy of the game
		\pause\item Next time we will look at what we can do if we don't possess such knowledge
	\end{itemize}
\end{frame}

