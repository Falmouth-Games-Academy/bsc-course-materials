\part{Integrated Development Environment (IDE)}
\frame{\partpage}

\begin{frame}{Using an IDE}
\begin{itemize}
	\item You \textit{could} just write code in Notepad, but...
	\item An \textbf{Integrated Development Environment (IDE)} is an application providing several
	useful features for programmers, including:
	\begin{itemize}
		\item A ``run'' button
		\item Management of multi-file projects
		\item Syntax highlighting
		\item Autocompletion
		\item Navigation
		\item Language and API documentation
		\item Debugging
		\item Profiling
		\item Version control
	\end{itemize}
\end{itemize}
\end{frame}

\begin{frame}[fragile]{Setting up your own PC}
\begin{itemize}
	\item Python 3.6.7
	\begin{itemize}
		\item \url{https://www.python.org/}
		\item Python 2.x and Python 3.x are (slightly) different programming languages; we are using 3.x (for now)
		\item Python is included with Mac OSX and most Linux distributions, but needs to be installed separately on Windows
	\end{itemize}

	\item PyGame 1.9.6
	\begin{itemize}
	\item We use \texttt{PyGame} as our framework for media computation and game development
	\item Library version must accord with language version
	\item Insteall on your PC using \texttt{pip}
	\end{itemize}
\end{itemize}

 \begin{lstlisting}
 pip install pygame==1.9.6
 \end{lstlisting}

\end{frame}

\begin{frame}{Setting up your own PC}
\begin{itemize}
\item PyCharm 19.1.2
\begin{itemize}
	\item \url{https://www.jetbrains.com/student/}
	\item Register with your \texttt{falmouth.ac.uk} email address to obtain PyCharm Professional Edition for free
	\item Or, use the free open-source entitled `Community Edition'
	\item Runs on Windows, Mac and Linux
\end{itemize}
\end{itemize}
\end{frame}

\begin{frame}[fragile]{PyCharm in the Lab}
	\begin{itemize}
		\item You have to license your account to use PyCharm
		\item Run PyCharm and select \textbf{License server}
		\item In the \textbf{License server address} enter the following:
	\end{itemize} 
	 \begin{lstlisting}
	http://trlicefal.fal.ac.uk
 	\end{lstlisting}
	\begin{itemize}
		\item This will be added to your user profile and (hopefully) you will not need to do this again
	\end{itemize} 
\end{frame}

\begin{frame}{Getting started with PyCharm}
\begin{itemize}
\item Create a new project (from the start-up wizard or from the File menu)
\item We want a ``Pure Python'' project
\item Right-click the project in the panel on the left, and choose ``New $\to$ Python File''
\item Write some code!
\item Setup the run configurations
\item First run: click ``Run $\to$ Run...'' and choose the Python file
\item Subsequent runs: click the $\blacktriangleright$ button
\end{itemize}
\end{frame}
