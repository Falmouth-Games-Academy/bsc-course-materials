% Adjust these for the path of the theme and its graphics, relative to this file
%\usepackage{beamerthemeFalmouthGamesAcademy}
\usepackage{../../../beamerthemeFalmouthGamesAcademy}
\usepackage{multimedia}
\graphicspath{ {../../../} }

% Default language for code listings
\lstset{language=Python,
        morekeywords={each,in,nullptr}
}

% For strikethrough effect
\usepackage[normalem]{ulem}
\usepackage{wasysym}

\usepackage{pdfpages}

% http://www.texample.net/tikz/examples/state-machine/
\usetikzlibrary{arrows,automata}

\newcommand{\modulecode}{COMP260}\newcommand{\moduletitle}{Distributed Systems}\newcommand{\sessionnumber}{5}

\setbeamertemplate{navigation symbols}{}

\newcommand{\fullbleed}[1]{
\begin{frame}[plain]
	\begin{tikzpicture}[remember picture, overlay]
		\node[at=(current page.center)] {
			\includegraphics[width=\paperwidth]{#1}
		};
	\end{tikzpicture}
\end{frame}
}

\newcommand{\picturepage}[2]{
\begin{frame}[plain]
	\begin{tikzpicture}[remember picture, overlay]
		\node[at=(current page.center)] {
			\includegraphics[width=\paperwidth]{#1}
		};
		\draw<1>[draw=none, fill=black, opacity=0.9] (-1,-5.2) rectangle (current page.south east);
		\node[draw=none,text width=0.96\paperwidth, align=right] at (5.5,-5.5) {\tiny{#2}};
	\end{tikzpicture}
\end{frame}
}

\newcommand{\notepicx}[5]{
\begin{frame}[plain]
	\begin{tikzpicture}[remember picture, overlay]
		\node[at=(current page.center)] {
			\includegraphics[width=\paperwidth]{#1}
		};
		\node[draw=none, fill=black, text width=#5\paperwidth] at ([xshift=#3, yshift=#4] current page.center) {\small{#2}};
	\end{tikzpicture}
\end{frame}
}

\newcommand{\notepic}[4]{
	\notepicx{#1}{#2}{#3}{#4}{0.4}
}

\begin{document}
\title{\sessionnumber: Tinkering Graphics II}
\subtitle{\modulecode: \moduletitle}

\frame{\titlepage} 

\begin{frame}
	\frametitle{Learning Outcomes}
	\begin{itemize}
		\item \textbf{Explain how} conditional logic can manipulate the output of a computer program
		\item \textbf{Apply} mathematical knowledge to \textbf{write} computer programs that manipulate pixels in a surface
		\item \textbf{Trace} existing computer programs
	\end{itemize}
\end{frame}

\begin{frame}
	\frametitle{Distance Between Colors}
	
	Sometimes we need to measure when something is `close enough':
	
	\begin{itemize}		
		\item Distance between two points in the Cartesian coordinate system:
		\item $\sqrt{(x_{1} - x_{2})^2 + (y_{1} - y_{2})^2}$
		\item Distance between two colours in the RGB colour representation system:
		\item $\sqrt{(red_{1} - red_{2})^2 + (green_{1} - green_{2})^2 + (blue_{1} - blue_{2})^2}$
	\end{itemize}
\end{frame}

\begin{frame}
	\frametitle{Activity \#1: Color Distance}
	
	In pairs:
	
	\vspace{2em}
	
	\begin{itemize}		
		\item Setup a basic project in PyGame
		\item Use the distance equation from the previous slide to write a function which accepts a two colours and returns the distance
		\item Test your solution
		\item Then, post your solution on Slack
		\item 10 minutes
	\end{itemize}
\end{frame}

\begin{frame}[fragile]
	\frametitle{Numeric Return Values}
	
\begin{lstlisting}
def distance(colour_base, colour_comparitor):
    return  sqrt( (colour_base.r - colour_comparitor.r)**2 + (colour_base.g - colour_comparitor.g)**2 + (colour_base.b - colour_comparitor.b)**2 )
\end{lstlisting}

Note: This source code excerpt will not work in PyGame.

\end{frame}

\begin{frame}[fragile]
	\frametitle{Expected Output: Color Distance}
	
\begin{lstlisting}
>>> print distance(WHITE, BLACK)
441.6729559300637
>>> print distance(WHITE, PINK)
113.13708498984761
>>> print distance(BLACK, PINK)
355.3519382246282
>>> print distance(MAGENTA, PINK)
192.41881404893857
\end{lstlisting}

\end{frame}

\begin{frame}
	\frametitle{Activity \#2: Colour Tolerance}
	
	In pairs:
	
	\vspace{2em}
	
	\begin{itemize}		
		\item Setup a basic project in PyGame
		\item Implement the function \texttt{closeEnough((colour, colour), tolerance)} that returns a boolean value
		\item Test your solution
		\item Then, post your solution on Slack
		\item 10 minutes
	\end{itemize}
\end{frame}

\begin{frame}[fragile]
	\frametitle{Boolean Return Values}
	
\begin{lstlisting}
def close_enough(colour_base, colour_comparitor):
    if distance(colour_base, colour_comparitor) < 50.0:
        return True
    else:
        return False
\end{lstlisting}

Note: This source code excerpt will not work in PyGame.

\end{frame}

\begin{frame}[fragile]
	\frametitle{Tolerance-based Pixel Manipulation}
	
\begin{lstlisting}
def turnRed():
  brown = makeColor(42,25,15)
  file="/Users/guzdial/Desktop/mediasources/katieFancy.jpg"
  picture=loadPicture(file)
  for pixel in getPixels(picture):
    color = getColor(pixel)
    if distance(color, BROWN) < 50.0:
      red=getRed(pixel)*2
      green=getGreen(pixel)
      blue=getBlue(pixel)
      setColor(pixel,makeColor(red,green,blue))
  return picture

\end{lstlisting}

Note: This source code excerpt will not work in PyGame.

\end{frame}

\begin{frame}
	\frametitle{Red Eye}
	
	\begin{itemize}		
		\item When the flash of the camera catches the eye just right (especially with light colored eyes), we get bounce back from the back of the retina.
		\item This results in `red eye'
		\item We can replace the “red” with a color of our choosing
		\item First, we figure out where the eyes are (x,y)
	\end{itemize}
\end{frame}

\fullbleed{jenny}

\begin{frame}
	\frametitle{Activity \#3: Red Eye}
	
	In pairs:
	
	\vspace{2em}
	
	\begin{itemize}		
		\item Setup a basic project in PyGame
		\item Refer to the following documentation
		\begin{itemize}
			\item \url{http://www.pygame.org/docs/ref/rect.html}
		\end{itemize}
		\item Implement the function: \texttt{removeRedEye(picture, area, colour)}
		\item Test your solution
		\item Then, post your solution on Slack
		\item 20 minutes
	\end{itemize}
\end{frame}

\fullbleed{jenny_nored}

\begin{frame}
	\frametitle{Posterization}
	
	\begin{itemize}		
		\item Posterization is simply reducing the number of colours in an image
		\item We look for a range of colours, then map them to a single colour, e.g:
		\begin{itemize}
			\item If red is between 63 and 128, set it to 95
			\item If green is less than 64, set it to 31
		\end{itemize}	
		\item The end result is that a bunch of different colours, get set to a few colours
		\item Beware of naive solutions with a large number of `if' statements
	\end{itemize}
\end{frame}

\fullbleed{ben}

\begin{frame}
	\frametitle{Calculating Luminance in RGB}
	
	To do this, we may need to determine the luminance of a pixel:
	
	\begin{itemize}		
		\item Luminance is the overall brightness of a pixel
		\item In RGB, it is the \textit{mean} average value of each component:
		\begin{itemize}
			\item $lum = (red + green + blue) / 3$
		\end{itemize}	
	\end{itemize}
\end{frame}



\begin{frame}
	\frametitle{Activity \#5: Black and White}
	
	In pairs:
	
	\vspace{2em}
	
	\begin{itemize}		
		\item Setup a basic project in PyGame
		\item Refer to the following documentation
		\item Implement the function: \texttt{makeGreyscale(picture, colourCount)}
		\item Test your solution
		\item Then, post your solution on Slack
		\item 20 minutes
	\end{itemize}
\end{frame}

\begin{frame}[fragile]
	\frametitle{Source Code: Black and White}
	
\begin{lstlisting}
def blackAndWhitePosterize(picture):
  for pixel in getPixels(picture):
    red = getRed(pixel)
    green = getGreen(pixel)
    blue = getBlue(pixel)
    luminance = (red + green + blue) / 3
    if luminance < 64:
      setColor(pixel, BLACK)
    else:
      setColor(pixel, WHITE)
\end{lstlisting}

Note: This source code excerpt will not work in PyGame.

\end{frame}

\begin{frame}
	\frametitle{Sepia Tone}
	
	\begin{itemize}		
		\item Pictures that are sepia-toned have a yellowish tint to them that we associate with older pictures.
		\item It's not directly a matter of simply increasing the yellow in the picture, because it's not a one-to-one correspondence:
		\begin{itemize}
			\item Instead, colors in different ranges get mapped to other colours.
			\item We can create such a mapping using IF statements
		\end{itemize}	
		\item The end result is that a bunch of different colours, get set to a few colours
		\item Beware of naive solutions with a large number of `if' statements
	\end{itemize}
\end{frame}

\fullbleed{field_normal}

\fullbleed{field_sepia}

\begin{frame}
	\frametitle{Sepia Tone}
	
	\begin{itemize}		
		\item First, we're calling greyScaleNew (the one with weights).
		\item We then manipulate the red (increasing) and the blue (decreasing) channels to bring out more yellows and oranges.
		\begin{itemize}
			\item It's perfectly okay to have one function calling another.
			\item Why are we doing the comparisons on the red? Why not?  After greyscale conversion, all channels are the same!
		\end{itemize}	
		\item The end result is that a bunch of different colours, get set to a few colours
		\item Why these values? Trial-and-error: Tinker the values!

	\end{itemize}
\end{frame}

\begin{frame}[fragile]
	\frametitle{Source Code: Sepia (1)}
	
\begin{lstlisting}
def sepiaTint(picture):
  #Convert image to greyscale
  makeGreyscale(picture)

  #loop through picture to tint pixels
  for p in getPixels(picture):
    red = getRed(p)
    blue = getBlue(p)

    #tint shadows
    if (red < 63):
      red = red*1.1
      blue = blue*0.9
...
\end{lstlisting}

Note: This source code excerpt will not work in PyGame.

\end{frame}

\begin{frame}[fragile]
	\frametitle{Source Code: Sepia (2)}
	
\begin{lstlisting}
...
   #tint midtones
    if (red > 62 and red < 192):
      red = red*1.15
      blue = blue*0.85

    #tint highlights
    if (red > 191):
      red = red*1.08
      if (red > 255):
        red = 255

      blue = blue*0.93

    #set the new color values
    setBlue(p, blue)
    setRed(p, red)


\end{lstlisting}

Note: This source code excerpt will not work in PyGame.

\end{frame}

\begin{frame}
	\frametitle{Activity \#6: Sepia Tone}
	
	In pairs:
	
	\vspace{2em}
	
	\begin{itemize}		
		\item Setup a basic project in PyGame
		\item Refer to the following documentation
		\item Refactor the function: \texttt{sepiaTint(picture) to use constants rather than literals}
		\item Tinker with the values of the constants to test your solution
		\item Then, post your solution on Slack
		\item 20 minutes
	\end{itemize}
\end{frame}

\end{document}
