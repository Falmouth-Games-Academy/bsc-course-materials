% Adjust these for the path of the theme and its graphics, relative to this file
%\usepackage{beamerthemeFalmouthGamesAcademy}
\usepackage{../../beamerthemeFalmouthGamesAcademy}
\usepackage{multimedia}
\graphicspath{ {../../} }

% Default language for code listings
\lstset{language=Python,
        morekeywords={each,in,nullptr}
}

% For strikethrough effect
\usepackage[normalem]{ulem}
\usepackage{wasysym}

\usepackage{pdfpages}

% //www.texample.net/tikz/examples/state-machine/
\usetikzlibrary{arrows,automata}

\newcommand{\modulecode}{COMP260}\newcommand{\moduletitle}{Distributed Systems}\newcommand{\sessionnumber}{5}

\setbeamertemplate{navigation symbols}{}

\newcommand{\fullbleed}[1]{
\begin{frame}[plain]
	\begin{tikzpicture}[remember picture, overlay]
		\node[at=(current page.center)] {
			\includegraphics[width=\paperwidth]{#1}
		};
	\end{tikzpicture}
\end{frame}
}

\newcommand{\picturepage}[2]{
\begin{frame}[plain]
	\begin{tikzpicture}[remember picture, overlay]
		\node[at=(current page.center)] {
			\includegraphics[width=\paperwidth]{#1}
		};
		\draw<1>[draw=none, fill=black, opacity=0.9] (-1,-5.2) rectangle (current page.south east);
		\node[draw=none,text width=0.96\paperwidth, align=right] at (5.5,-5.5) {\tiny{#2}};
	\end{tikzpicture}
\end{frame}
}

\newcommand{\notepicx}[5]{
\begin{frame}[plain]
	\begin{tikzpicture}[remember picture, overlay]
		\node[at=(current page.center)] {
			\includegraphics[width=\paperwidth]{#1}
		};
		\node[draw=none, fill=black, text width=#5\paperwidth] at ([xshift=#3, yshift=#4] current page.center) {\small{#2}};
	\end{tikzpicture}
\end{frame}
}

\newcommand{\notepic}[4]{
	\notepicx{#1}{#2}{#3}{#4}{0.4}
}

\begin{document}
\title{\sessionnumber: Tinkering Graphics I}
\subtitle{\modulecode: \moduletitle}

\frame{\titlepage} 

\begin{frame}
	\frametitle{Learning Outcomes}
	By the end of this workshop, you should be able to:	
	\begin{itemize}
		\item \textbf{Apply} knowledge of colour models to \textbf{write} code that manipulates pixels in a surface
		\item \textbf{Use} functions, arguments, and basic data structures such as arrays
	\end{itemize}
\end{frame}

\begin{frame}
	\frametitle{Activity \#1a -- Setup}
	
	In pairs:
	
	\vspace{2em}
	
	\begin{itemize}		
		\item Launch a basic Python project in PyCharm
		\item Import PyGame, setup a main window, and define a game-loop which renders a white background
		\item Refer to the following documentation:
		\begin{itemize}
			\item \url{www.pygame.org/docs/tut/tom_games2.html}
		\end{itemize}
	\end{itemize}
\end{frame}

\begin{frame}[fragile]
	\frametitle{Activity \#1a -- Setup}
	
\begin{lstlisting}
import pygame

pygame.init()

main_window = pygame.display.set_mode((800,600))

running = True
while running:
    for event in pygame.event.get():
        if event.type == pygame.QUIT:
            crashed = False

    main_window.fill((255,255,255))
    pygame.display.update()

pygame.quit()
\end{lstlisting}

Note: This is a PyGame example.

\end{frame}
	
\begin{frame}
	\frametitle{Activity \#1b -- Setup}
	
	In pairs:
	
	\vspace{2em}
	
	\begin{itemize}		
		\item Render a green \texttt{Surface} in the top corner of the window
		\item Define a function to manipulate a single pixel in the \texttt{Surface} using a \texttt{PixelArray}
		\item Refer to the following documentation:
		\begin{itemize}
			\item \url{www.pygame.org/docs/ref/surface.html}
			\item \url{www.pygame.org/docs/ref/pixelarray.html}
		\end{itemize}
	\end{itemize}
\end{frame}

\begin{frame}[fragile]
	\frametitle{Activity \#1b -- Setup}

Firstly, create a new \texttt{Surface} and fill it with green:
	
\begin{lstlisting}
my_surface = pygame.Surface((200,200))
my_surface.fill((0,255,0))
\end{lstlisting}

\vspace{0.5em}

Then, blit the green \texttt{Surface} onto the main window at the origin:

\begin{lstlisting}
    main_window.fill((255,255,255))
    main_window.blit(my_surface, (0,0))
    pygame.display.update()
\end{lstlisting}

Note: This is a PyGame example.

\end{frame}

\begin{frame}[fragile]
	\frametitle{Activity \#1b -- Setup}

Finally, define a new function using the \texttt{def} keyword to manipulate a single pixel in the \texttt{Surface}:
	
\begin{lstlisting}
def set_pixel(surf, x_pos, y_pos, colour):
    px_array = pygame.PixelArray(surf)
    px_array[x_pos,y_pos] = colour
    del px_array

set_pixel(my_surface, 100, 100, (0,0,0))
\end{lstlisting}

Note: This is a PyGame example.

\end{frame}

\begin{frame}
	\frametitle{Activity \#2 -- Less Red}
	
	In pairs:
	
	\vspace{2em}
	
	\begin{itemize}		
		\item Define a function to load an image file to a \texttt{Surface}
		\item Then, define a function to reduce it's redness
		\item Refer to the following documentation:
		\begin{itemize}
			\item \url{https://www.pygame.org/docs/ref/image.html}
		\end{itemize}
	\end{itemize}
\end{frame}

\begin{frame}[fragile]
	\frametitle{Activity \#2 -- Less Red}

\begin{lstlisting}
my_surface = pygame.image.load('test.jpg')
\end{lstlisting}

\vspace{0.5em}

	
\begin{lstlisting}
def decreaseRed(pict):
  pixelMatrix = getPixels(pict)
  for pixel in pixelMatrix:
    value = getRed(pixel)
    setRedPixel(pixel, value * 0.5)
\end{lstlisting}

Note: Not all of this source code excerpt will work in PyGame.

\end{frame}

\begin{frame}
	\frametitle{Activity \#3 -- Swap Channel}
	
	In pairs:
	
	\vspace{2em}
	
	\begin{itemize}
		\item Define a function that turns all of the red values of pixels into blue values...
		\item ...and all of the blue values into red values
	\end{itemize}
\end{frame}

\begin{frame}[fragile]
	\frametitle{Activity \#3 -- Swap Channel}
	
\begin{lstlisting}
def swapRedBlueChannels(pict):
  pixelMatrix = getPixels(pict)
  for pixel in pixelMatrix:
    red_value = getRed(pixel)
    blue_value = getBlue(pixel)
    setRedPixel(pixel, blue_value)
    setBluePixel(pixel, red_value)
\end{lstlisting}

Note: This source code excerpt will not work in PyGame.

\end{frame}

\begin{frame}
	\frametitle{Activity \#4 -- Greyscale}
	
	In pairs:
	
	\vspace{2em}
	
	\begin{itemize}
		\item Define a function that loads an image and turns it to greyscale
		\item Consider the following calculation:
		\begin{itemize}
			\item $New Pixel Value = \frac{\Sigma Current Channel Value}{Number Of Channels}$
		\end{itemize}
	\end{itemize}
\end{frame}

\begin{frame}[fragile]
	\frametitle{Activity \#4 -- Greyscale}
	
\begin{lstlisting}
def loadGrayscale(file):
  pixelMatrix = getPixels(makePicture(file))
  for pixel in pixelMatrix:
    red = getRed(p)
    green = getGreen(p)
    blue = getBlue(p)
    
    pixelValue = (red+green+blue)/3
    
    setRedPixel(pixel,pixelValue)
    setGreenPixel(pixel, pixelValue)
    setBluePixel(pixel, pixelValue)
\end{lstlisting}

Note: This source code excerpt will not work in PyGame.

\end{frame}

\begin{frame}
	\frametitle{Activity \#5 -- Negative}
	
	In pairs:
	
	\vspace{2em}
	
	\begin{itemize}
		\item Define a function that loads an image and turns it to its negative
		\item Consider the following calculation:
		\begin{itemize}
			\item $New Channel Value = 255 - Current Channel Value$
		\end{itemize}
	\end{itemize}
\end{frame}

\begin{frame}[fragile]
	\frametitle{Activity \#5 -- Negative}
	
\begin{lstlisting}
def neg(picture):
  pixelMatrix = getPixels(makePicture(file))
  for pixel in pixelMatrix:
    red = getRed(p)
    green = getGreen(p)
    blue = getBlue(p)
        
    setRedPixel(pixel,255-red)
    setGreenPixel(pixel, 255-green)
    setBluePixel(pixel, 255-blue)
\end{lstlisting}

Note: This source code excerpt will not work in PyGame.

\end{frame}

\begin{frame}
	\frametitle{Activity \#6 -- Sunset}
	
	In pairs:
	
	\vspace{2em}
	
	\begin{itemize}
		\item Define a function that loads an image and produces several images as output, descreasing luminance
		\item Refer to the following documentation:
		\begin{itemize}
			\item \url{//www.pygame.org/docs/ref/time.html}
		\end{itemize}
	\end{itemize}
\end{frame}

\begin{frame}[fragile]
	\frametitle{Activity \#6 -- Sunset}

\begin{lstlisting}
def decreaseRed(picture, amount):
  for p in getPixels(picture):
    value=getRed(p)
    setRed(p,value*amount)

amount = 0.1 #tinker with this value
wait_time = 50 #tinker with this value    
    
for i in range(10):
  decreaseRed(picture, amount)
  decreaseGreen(picture, amount)
  decreaseBlue(picture, amount)
  wait(50)
\end{lstlisting}

Note: This source code excerpt will not work in PyGame.

\end{frame}

\begin{frame}
	\frametitle{Activity \#7 -- Top-Copy}
	
	In pairs:
	
	\vspace{2em}
	
	\begin{itemize}
		\item Define a function that copies the top half of a picture to its bottom half
		\item Refer to the following documentation:
		\begin{itemize}
			\item \url{https://docs.python.org/3.7/tutorial/introduction.html\#lists}
		\end{itemize}
	\end{itemize}
\end{frame}

\begin{frame}[fragile]
	\frametitle{Activity \#7 -- Top-Copy}

\begin{lstlisting}
def copyHalf(picture):
 pixels = getPixels(picture)
 for index in range(0,len(pixels)/2):
   sourcePixel = pixels[index]
   sourceRGBValue = getColor(sourcePixel)
   destinationPixel = pixels[index + len(pixels)/2]
   setColor(destinationPixel,sourceRGBValue)
 repaint(picture)
\end{lstlisting}

Note: This source code excerpt will not work in PyGame.

\end{frame}

\end{document}
