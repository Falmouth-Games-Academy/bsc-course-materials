% Adjust these for the path of the theme and its graphics, relative to this file
%\usepackage{beamerthemeFalmouthGamesAcademy}
\usepackage{../../beamerthemeFalmouthGamesAcademy}
\usepackage{multimedia}
\graphicspath{ {../../} }

\usepackage{textcomp}

% Default language for code listings
\lstset{language=Python, upquote=true,
        morekeywords={each,in,nullptr}
}

% For strikethrough effect
\usepackage[normalem]{ulem}
\usepackage{wasysym}

\usepackage{pdfpages}

% http://www.texample.net/tikz/examples/state-machine/
\usetikzlibrary{arrows,automata}

\newcommand{\modulecode}{COMP140 GAM160}\newcommand{\moduletitle}{Hacking Hardware/Advanced Programming}\newcommand{\sessionnumber}{Session 6}

\setbeamertemplate{navigation symbols}{}

\newcommand{\fullbleed}[1]{
\begin{frame}[plain]
	\begin{tikzpicture}[remember picture, overlay]
		\node[at=(current page.center)] {
			\includegraphics[width=\paperwidth]{#1}
		};
	\end{tikzpicture}
\end{frame}
}

\newcommand{\picturepage}[2]{
\begin{frame}[plain]
	\begin{tikzpicture}[remember picture, overlay]
		\node[at=(current page.center)] {
			\includegraphics[width=\paperwidth]{#1}
		};
		\draw<1>[draw=none, fill=black, opacity=0.9] (-1,-5.2) rectangle (current page.south east);
		\node[draw=none,text width=0.96\paperwidth, align=right] at (5.5,-5.5) {\tiny{#2}};
	\end{tikzpicture}
\end{frame}
}

\newcommand{\notepicx}[5]{
\begin{frame}[plain]
	\begin{tikzpicture}[remember picture, overlay]
		\node[at=(current page.center)] {
			\includegraphics[width=\paperwidth]{#1}
		};
		\node[draw=none, fill=black, text width=#5\paperwidth] at ([xshift=#3, yshift=#4] current page.center) {\small{#2}};
	\end{tikzpicture}
\end{frame}
}

\newcommand{\notepic}[4]{
	\notepicx{#1}{#2}{#3}{#4}{0.4}
}

\begin{document}
\title{\sessionnumber: Python, Pairs, \& PyGame}
\subtitle{\modulecode: \moduletitle}

\frame{\titlepage} 

\begin{frame}
	\frametitle{Learning Outcomes}
	\begin{itemize}
		%\item \textbf{Analyse} the role of computing professionals in the games industry
		%Comp110 - 02 Material and Tinkering Text
		\item \textbf{Explain} the role and basic functions of the IDE
		\item\textbf{Interpret} some basic Python code
		\item \textbf{Apply} pair programming practices to solve a simple problem
	\end{itemize}
\end{frame}

\part{Integrated Development Environment (IDE)}
\frame{\partpage}

\begin{frame}{Using an IDE}
\begin{itemize}
	\item You \textit{could} just write code in Notepad, but...
	\item An \textbf{Integrated Development Environment (IDE)} is an application providing several
	useful features for programmers, including:
	\begin{itemize}
		\item A ``run'' button
		\item Management of multi-file projects
		\item Syntax highlighting
		\item Autocompletion
		\item Navigation
		\item Language and API documentation
		\item Debugging
		\item Profiling
		\item Version control
	\end{itemize}
\end{itemize}
\end{frame}

\begin{frame}[fragile]{Setting up your own PC}
\begin{itemize}
	\item Python 3.6.7
	\begin{itemize}
		\item \url{https://www.python.org/}
		\item Python 2.x and Python 3.x are (slightly) different programming languages; we are using 3.x (for now)
		\item Python is included with Mac OSX and most Linux distributions, but needs to be installed separately on Windows
	\end{itemize}

	\item PyGame 1.9.6
	\begin{itemize}
	\item We use \texttt{PyGame} as our framework for media computation and game development
	\item Library version must accord with language version
	\item Insteall on your PC using \texttt{pip}
	\end{itemize}
\end{itemize}

 \begin{lstlisting}
 pip install pygame==1.9.6
 \end{lstlisting}

\end{frame}

\begin{frame}{Setting up your own PC}
\begin{itemize}
\item PyCharm 19.1.2
\begin{itemize}
	\item \url{https://www.jetbrains.com/student/}
	\item Register with your \texttt{falmouth.ac.uk} email address to obtain PyCharm Professional Edition for free
	\item Or, use the free open-source entitled `Community Edition'
	\item Runs on Windows, Mac and Linux
\end{itemize}
\end{itemize}
\end{frame}

\begin{frame}[fragile]{PyCharm in the Lab}
	\begin{itemize}
		\item You have to license your account to use PyCharm
		\item Run PyCharm and select \textbf{License server}
		\item In the \textbf{License server address} enter the following:
	\end{itemize} 
	 \begin{lstlisting}
	http://trlicefal.fal.ac.uk
 	\end{lstlisting}
	\begin{itemize}
		\item This will be added to your user profile and (hopefully) you will not need to do this again
	\end{itemize} 
\end{frame}

\begin{frame}{Getting started with PyCharm}
\begin{itemize}
\item Create a new project (from the start-up wizard or from the File menu)
\item We want a ``Pure Python'' project
\item Right-click the project in the panel on the left, and choose ``New $\to$ Python File''
\item Write some code!
\item Setup the run configurations
\item First run: click ``Run $\to$ Run...'' and choose the Python file
\item Subsequent runs: click the $\blacktriangleright$ button
\end{itemize}
\end{frame}


\part{Basic Python programs}
\frame{\partpage}

\begin{frame}[fragile]{Your first Python program}
	\begin{lstlisting}
print "Hello, world!"
	\end{lstlisting}
\end{frame}

\begin{frame}[fragile]{Your second Python program}
	\begin{lstlisting}
print "This is a very long line of code which had to be split to fit on the slide, but you should type it as a single line."
	\end{lstlisting}
\end{frame}

\begin{frame}[fragile]{Assigning to variables}
	\begin{lstlisting}
a = 7
print a
	\end{lstlisting}
\end{frame}

% Assignment
% Conditional
% For loop
% While loop
% While 1
% Show PyCharm bouncing ball example -- think about this between now and Michael's session next Wednesday


\part{Professional Practice}
\frame{\partpage}

\begin{frame}
	\frametitle{Pair Programming}
		
	Pair programming is an agile software development technique in which two programmers work together 
	at one workstation. 
	\\~\\
	One, the driver, writes code while the other, the observer or navigator, reviews 
	each line of code as it is typed in. 
	\\~\\
	The two programmers switch roles frequently.
	
\end{frame}

\begin{frame}
	\frametitle{Pair Programming}
	
	Watch the video at:
	
	\vspace{1.5em}
		
	\url{https://www.youtube.com/watch?v=ET3Q6zNK3Io}
	
	\vspace{1em}
		
	(5 minutes)
	
\end{frame}

\begin{frame}
	\frametitle{Pair Programming}
	
	Review the guidelines at:
	
	\vspace{1.5em}
		
	\url{http://www.pairprogramming.co.uk/}
	
	\vspace{1em}
		
	(5 minutes)
	
\end{frame}

\begin{frame}
	\frametitle{Pair Programming}
	
	Watch the video at:
	
	\vspace{1.5em}
		
	\url{https://www.youtube.com/watch?v=ONnYCT_LJio}
	
	\vspace{1em}
		
	(5 minutes)
	
\end{frame}

\begin{frame}
	\frametitle{PASS Challenge}
	
	\begin{itemize}
		\item In pairs
		\item \textbf{Implement} the code excerpt
		\item \textbf{Fix} the errors in the code excerpt
		\item \textbf{Modify} the code excerpt to incorporate functions and arguments
		\item \textbf{Post} your solution to the \lstinline{\#comp120} slack channel
	\end{itemize}
	
	You can learn more about functions and arguments at:
	
	\vspace{1em}
	
	 \url{https://docs.python.org/3/tutorial/controlflow.html\#defining-functions}
	
	\vspace{1em}
	
	(20 minutes)
	
\end{frame}

\begin{frame}[fragile]
	\frametitle{PASS Challenge}
		
	The function:

	\begin{lstlisting}
		def madlib()
	\end{lstlisting}
	
	\vspace{1.5em}
	
	Should become:
	
	\begin{lstlisting}
		def madlib(name, pet, verb, snack)
	\end{lstlisting}
	
\end{frame}

\begin{frame}[fragile]
	\frametitle{PASS Challenge}
	
	\begin{lstlisting}
def madlib():
	name = 'Link'
	pet = 'Spyro'
	verb = 'ate'
	snack = 'doughnuts'
	line1 = 'once upon a time,' + name + ' walked'
	line2 = ' with ' + pet + ', a trained dragon.'
	line3 = 'Suddenly, ' + pet + ' announced,'
	line4 = 'I really want some ' + snack + '!'
	line5 = name + ' complained. Where am I going to get that?'
	line6 = 'Then ' + name + 'found a wizard's wand.'
	line 7 = 'With a wave of the wand, '
	line8 = pet + ' got ' + snack + '. '
	line9 = 'Perhaps surprisingly, ' + pet + ' ' + verb + '  ' + snack
	print line1 + line2 + line3 + line4
	print line5 + line6 + line7 + line8 + line9
\end{lstlisting}
	
\end{frame}

\begin{frame}[fragile]
	\frametitle{PASS Challenge Stretch Goal}
	
	\begin{itemize}
		\item In pairs
		\item \textbf{Incorporate} your code into the PyGame framework
		\item \textbf{Post} your solution to the \lstinline{#comp120} slack channel
		\item You will likely need to search the PyGame library documentation and StackOverflow:
	\end{itemize}
	
	\vspace{2em}
	\url{www.pygame.org/docs/ref/pygame.html}
	
	\vspace{1em}
	\url{stackoverflow.com/questions/tagged/pygame}
	
\end{frame}

\begin{frame}[fragile]
\frametitle{PASS Challenge Stretch Goal}
	
\begin{lstlisting}
import pygame
pygame.init()
screen = pygame.display.set_mode((640, 480))
#TODO: setup string, madlib, and font

done = False
while not done:
    for event in pygame.event.get():
        if event.type == pygame.QUIT:
            done = True
        if event.type == pygame.KEYDOWN and event.key == pygame.K_ESCAPE:
            done = True

    screen.fill((255, 255, 255))
    #TODO: display text on the screen
    pygame.display.flip()
\end{lstlisting}
	
\end{frame}

\end{document}