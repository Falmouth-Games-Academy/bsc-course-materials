\part{The PyCharm IDE}
\frame{\partpage}

\begin{frame}{Integrated Development Environment (IDE)}
\begin{itemize}
	\pause\item You \textit{could} just write code in Notepad, but...
	\pause\item An \textbf{Integrated Development Environment (IDE)} is an application providing several
	useful features for programmers, including:
	\begin{itemize}
		\pause\item A ``run'' button
		\pause\item Management of multi-file projects
		\pause\item Syntax highlighting
		\pause\item Autocompletion
		\pause\item Navigation
		\pause\item Language and API documentation
		\pause\item Debugging
		\pause\item Profiling
		\pause\item Version control
	\end{itemize}
\end{itemize}
\end{frame}

\begin{frame}{Setting up your own PC}
\begin{itemize}
\pause\item Python 2.7
\begin{itemize}
	\pause\item \url{https://www.python.org/}
	\pause\item Python 2.7 is included with Mac OSX and most Linux distributions, but needs to be installed separately on Windows
	\pause\item Python 2.x and Python 3.x are (slightly) different programming languages; we are using 2.x (for now)
\end{itemize}
\pause\item PyCharm
\begin{itemize}
	\pause\item \url{https://www.jetbrains.com/student/}
	\pause\item Register with your \texttt{falmouth.ac.uk} email address to obtain PyCharm Professional Edition for free
	\pause\item Runs on Windows, Mac and Linux
	\pause\item Other Python IDEs are available
\end{itemize}
\end{itemize}
\end{frame}

\begin{frame}{PyCharm in the Lab}
	\begin{itemize}
		\pause\item You have to license your account to use PyCharm
		\pause\item Run PyCharm and select \textbf{License server}
		\pause\item In the \textbf{License server address} enter the following \textbf{http://trlicefal.fal.ac.uk}
		\pause\item This will be added to your user profile and you will not need to do this again
	\end{itemize} 
\end{frame}

\begin{frame}{Getting started with PyCharm}
\begin{itemize}
\pause\item Create a new project (from the start-up wizard or from the File menu)
\pause\item We want a ``Pure Python'' project
\pause\item Right-click the project in the panel on the left, and choose ``New $\to$ Python File''
\pause\item Write some code!
\pause\item First run: click ``Run $\to$ Run...'' and choose the Python file
\pause\item Subsequent runs: click the $\blacktriangleright$ button
\end{itemize}
\end{frame}
