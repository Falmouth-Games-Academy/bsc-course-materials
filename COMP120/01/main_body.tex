% Adjust these for the path of the theme and its graphics, relative to this file
%\usepackage{beamerthemeFalmouthGamesAcademy}
\usepackage{../../beamerthemeFalmouthGamesAcademy}
\usepackage{multimedia}
\graphicspath{ {../../} }

\usepackage{textcomp}

% Default language for code listings
\lstset{language=Python, upquote=true,
        morekeywords={each,in,nullptr}
}

% For strikethrough effect
\usepackage[normalem]{ulem}
\usepackage{wasysym}

\usepackage{pdfpages}

% http://www.texample.net/tikz/examples/state-machine/
\usetikzlibrary{arrows,automata}

\newcommand{\modulecode}{COMP260}\newcommand{\moduletitle}{Distributed Systems}\newcommand{\sessionnumber}{5}

\setbeamertemplate{navigation symbols}{}

\newcommand{\fullbleed}[1]{
\begin{frame}[plain]
	\begin{tikzpicture}[remember picture, overlay]
		\node[at=(current page.center)] {
			\includegraphics[width=\paperwidth]{#1}
		};
	\end{tikzpicture}
\end{frame}
}

\newcommand{\picturepage}[2]{
\begin{frame}[plain]
	\begin{tikzpicture}[remember picture, overlay]
		\node[at=(current page.center)] {
			\includegraphics[width=\paperwidth]{#1}
		};
		\draw<1>[draw=none, fill=black, opacity=0.9] (-1,-5.2) rectangle (current page.south east);
		\node[draw=none,text width=0.96\paperwidth, align=right] at (5.5,-5.5) {\tiny{#2}};
	\end{tikzpicture}
\end{frame}
}

\newcommand{\notepicx}[5]{
\begin{frame}[plain]
	\begin{tikzpicture}[remember picture, overlay]
		\node[at=(current page.center)] {
			\includegraphics[width=\paperwidth]{#1}
		};
		\node[draw=none, fill=black, text width=#5\paperwidth] at ([xshift=#3, yshift=#4] current page.center) {\small{#2}};
	\end{tikzpicture}
\end{frame}
}

\newcommand{\notepic}[4]{
	\notepicx{#1}{#2}{#3}{#4}{0.4}
}

\begin{document}
\title{\sessionnumber: Computing Professionals}
\subtitle{\modulecode: \moduletitle}

\frame{\titlepage} 

\begin{frame}
	\frametitle{Learning Outcomes}
	\begin{itemize}
		\item \textbf{Analyse} the role of computing professionals in the games industry
		%Comp110 - 02 Material and Tinkering Text
		\item Explain the role and basic functions of the IDE
		\item Produce some basic Python programs
		\item \textbf{Apply} pair programming practices to solve simple problems
	\end{itemize}
\end{frame}

\part{Professional Roles}
\frame{\partpage}

\begin{frame}
	\frametitle{TwitterFall Activities}
		
	\begin{itemize}
		\item Self-organise into small groups of 3-4
		\item Load a Twitter app, or login to Twitter on a PC
		\item Conduct research on the given topic
		\item Post a tweet when you find something interesting
	\end{itemize}

	\begin{itemize}
		\item Please use the hashtag for the module (i.e., \lstinline{\#comp120})
		\item Also please ensure you use the @ symbol to open and continue discussions
	\end{itemize}
\end{frame}

\begin{frame}
	\frametitle{TwitterFall Activity \#1}
	
	\textbf{Answer} the follow question:
	
	\vspace{1em}
	
	``What do computing professionals do, \textit{generally}?''
	
	\vspace{1em}
	
	You have:
	
	\begin{itemize}
		\item 10 minutes to conduct research and tweet to \lstinline{\#comp120}
		\item 5 minutes to debrief
	\end{itemize}
\end{frame}

\begin{frame}
	\frametitle{TwitterFall Activity \#2}
	
	\textbf{Answer} the follow question:
	
	\vspace{1em}
	
	``What do computing professionals do, \textit{in games}?''
	
	\vspace{1em}
	
	You have:
	
	\begin{itemize}
		\item 10 minutes to conduct research and tweet to \lstinline{\#comp120}
		\item 5 minutes to debrief
	\end{itemize}
\end{frame}

\begin{frame}
	\frametitle{TwitterFall Activity \#3}
	
	\textbf{Answer} the follow question:
	
	\vspace{1em}
	
	``What career options are available to graduates with B.Sc. degrees in computing?''
	
	\vspace{1em}
	
	You have:
	
	\begin{itemize}
		\item 10 minutes to conduct research and tweet to \lstinline{\#comp120}
		\item 5 minutes to debrief
	\end{itemize}
\end{frame}

\part{Professional Development}
\frame{\partpage}

\begin{frame}
	\frametitle{Continuing Professional Development}
	
	\begin{itemize}
		\item Games industry is fast-moving
		\item Learning does not end at school and university
		\item A goal of this course is to facilitate your development as self-regulated learners
		\item Gradually, more independence across each year of study
		\item This is a science degree, which means you will become a producer of knowledge, not just a consumer of knowledge!
	\end{itemize}
\end{frame}

\begin{frame}
	\frametitle{Continuing Professional Development}
	
	\begin{itemize}
		\item It isn't easy!
		\item Many of you will encounter programming anxiety
		\item Some will experience a sense of fear or a sense of hopelessness --- it is more common than you think
		\item Some will need more support than others --- this isn't a bad thing
		\item Everyone who puts in the time and effort will eventually achieve mastery
	\end{itemize}
\end{frame}


\part{Professional Practice}
\frame{\partpage}

\begin{frame}
	\frametitle{Pair Programming}
		
	Pair programming is an agile software development technique in which two programmers work together 
	at one workstation. 
	\\~\\
	One, the driver, writes code while the other, the observer or navigator, reviews 
	each line of code as it is typed in. 
	\\~\\
	The two programmers switch roles frequently.
	
\end{frame}

\begin{frame}
	\frametitle{Pair Programming}
	
	Watch the video at:
	
	\vspace{1.5em}
		
	\url{https://www.youtube.com/watch?v=ET3Q6zNK3Io}
	
	\vspace{1em}
		
	(5 minutes)
	
\end{frame}

\begin{frame}
	\frametitle{Pair Programming}
	
	Review the guidelines at:
	
	\vspace{1.5em}
		
	\url{http://www.pairprogramming.co.uk/}
	
	\vspace{1em}
		
	(10 minutes)
	
\end{frame}

\begin{frame}
	\frametitle{Pair Programming}
	
	Watch the video at:
	
	\vspace{1.5em}
		
	\url{https://www.youtube.com/watch?v=ONnYCT_LJio}
	
	\vspace{1em}
		
	(5 minutes)
	
\end{frame}

\begin{frame}
	\frametitle{Pair Programming Challenge}
	
	\begin{itemize}
		\item In pairs
		\item \textbf{Implement} the code excerpt
		\item \textbf{Fix} the errors in the code excerpt
		\item \textbf{Modify} the code excerpt to incorporate functions and arguments
		\item \textbf{Post} your solution to the \lstinline{\#comp120} slack channel
	\end{itemize}
	
	You can learn more about functions and arguments at:
	
	\vspace{1em}
	
	 \url{https://docs.python.org/3/tutorial/controlflow.html\#defining-functions}
	
	\vspace{1em}
	
	(20 minutes)
	
\end{frame}

\begin{frame}[fragile]
	\frametitle{Pair Programming Challenge}
	
	The function:

	\begin{lstlisting}
		def madlib()
	\end{lstlisting}
	
	\vspace{1.5em}
	
	Should become:
	
	\begin{lstlisting}
		def madlib(name, pet, verb, snack)
	\end{lstlisting}
	
\end{frame}


\begin{frame}[fragile]
	\frametitle{Pair Programming Challenge}
	
	\begin{lstlisting}
def madlib():
	name = 'Mike'
	pet = 'Spyro'
	verb = 'ate'
	snack = 'doughnuts'
	line1 = 'once upon a time,' + name + ' walked'
	line2 = ' with ' + pet + ', a trained dragon.'
	line3 = 'Suddenly, ' + pet + ' announced,'
	line4 = 'I really want some ' + snack + '!'
	line5 = name + ' complained. Where am I going to get that?'
	line6 = 'Then ' + name + 'found a wizard's wand.'
	line 7 = 'With a wave of the wand, '
	line8 = pet + ' got ' + snack + '. '
	line9 = 'Perhaps surprisingly, ' + pet + ' ' + verb + '  ' + snack
	print line1 + line2 + line3 + line4
	print line5 + line6 + line7 + line8 + line9

\end{lstlisting}
	
\end{frame}

\part{Coffee Break}

\end{document}