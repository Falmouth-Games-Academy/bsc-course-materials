\newcommand{\codeslide}[2]{
	\begin{columns}
		\begin{column}{0.48\textwidth}
			\lstinputlisting{#1}
		\end{column}
		
		\begin{column}{0.48\textwidth}
			\begin{center}
				\colorbox{white}{
					\color{black}
					\begin{tabular}{|c|c|}
						\hline
						\textbf{Variable} & \textbf{Value} \\\hline
						#2
					\end{tabular}
				}
			\end{center}
		\end{column}
	\end{columns}
}

\newcommand{\trow}[1]{ & \\ \texttt{#1} & \\ & \\\hline}

\part{Basic C\# programs}

\frame{\partpage}

\begin{frame}[fragile]{Your first C\# program}
\begin{lstlisting}
using System;

namespace Test
{
    class MainClass
    {
        public static void Main(string[] args)
        {
            Console.WriteLine("Hello World!");
        }
    }
}
\end{lstlisting}
\end{frame}

\begin{frame}
	\frametitle{C\# Terminology}
	\small
	\begin{itemize}
	\item \textbf{Using} The using directive creates an alias for a namespace or import types defined in other namespaces.
	\item \textbf{nameSpace} A namespace is designed to keep one set of names separate from another. Consequently class names declared in one namespace do not conflict with the same class names declared in another.
	\item \textbf{Class} A class defines the kinds of data and the functionality objects will have. A class enables you to create your custom types by grouping variables of other types, methods, and events.
	\item \textbf{public static void Main} It is the first method which gets invoked whenever an application started and it is present in every C\# executable file.
	\end{itemize}
\end{frame}
	


\begin{frame}[fragile]{Your second C\# program}
\begin{lstlisting}
Console.WriteLine("This is a very long line of code which
had to be split to fit on the slide, but you should type 
it as a single line.")
Console.WriteLine("This is the second line of code.")
\end{lstlisting}
\end{frame}

\begin{frame}{Assigning to variables}
\codeslide{assign0.cs}{\trow{a}}
\end{frame}

\begin{frame}{Remember!}
\begin{itemize}
\item A program is a \textbf{sequence of instructions}
\item The C\# interpreter executes the \textbf{first line} of your program, then the \textbf{second line},
and so on
\item When it reaches the end of the file, it \textbf{stops}
\end{itemize}
\end{frame}

\begin{frame}{Socrative - FALCOMPMIKE}
Login to Socrative!

\vspace{1em}

\url{https://b.socrative.com/login/student/}

\end{frame}

\begin{frame}{Reassigning variables (1)}
\codeslide{assign1.cs}{\trow{a}\trow{b}}
\end{frame}

\begin{frame}{Reassigning variables (2)}
\codeslide{assign2.cs}{\trow{a}\trow{b}}
\end{frame}

\begin{frame}{Reassigning variables (3)}
\codeslide{assign3.cs}{\trow{big}\trow{small}}
\end{frame}

\begin{frame}{Reassigning variables (4)}
\codeslide{assign4.cs}{\trow{a}\trow{b}}
\end{frame}

\begin{frame}{Reassigning variables (5)}
\codeslide{assign5.cs}{\trow{a}\trow{b}\trow{c}}
\end{frame}

\begin{frame}{Reading Input}
\lstinputlisting{input.cs}
\begin{itemize}
\item \lstinline{Console.ReadLine()} reads a \textbf{string} (a sequence of characters---text) from the command line
\item \lstinline{Int16.Parse(...)} parses(converts) a \textbf{string} into an \textbf{integer} (a number)
\end{itemize}
\end{frame}

\begin{frame}{Conditionals (1)}
\codeslide{cond1.cs}{\trow{a}\trow{b}}
\end{frame}

\begin{frame}{Indentation}
\begin{itemize}
\item Like many other programming languages, \textbf{indentation is not essential but useful} in C\#
\item C\# uses indentation to denote the \textbf{block of code} inside a conditional, loop, function etc.
\item Microsoft recommends \textbf{4 spaces} for indentation
\begin{itemize}
\item Some programmers use a tab character
\item \textbf{Never} mix tabs and spaces in the same file!
\end{itemize}
\end{itemize}

\scriptsize\url{https://docs.microsoft.com/en-us/dotnet/csharp/programming-guide/inside-a-program/coding-conventions}
\end{frame}

\begin{frame}{Conditionals (2)}
\codeslide{cond2.cs}{\trow{a}\trow{b}}
\end{frame}

\begin{frame}{Conditionals}
 An \lstinline{if} statement can have:
\begin{itemize}
\item \textbf{Zero or more} \lstinline{else if} clauses
\item \textbf{An optional} \lstinline{else} clause
\end{itemize}
 In that order!
\end{frame}

\begin{frame}{Mathematical operators}
\begin{itemize}
\item \lstinline{+} add
\item \lstinline{-} subtract
\item \lstinline{*} multiply
\item \lstinline{/} divide
\item \lstinline{**} power
\end{itemize}
 Order of operations: \textbf{BIDMAS}
\begin{itemize}
\item \uline{B}rackets first
\item Then \uline{i}ndices (powers)
\item Then \uline{d}ivision and \uline{m}ultiplication (left to right)
\item Then \uline{a}ddition and \uline{s}ubtraction (left to right)
\end{itemize}
\end{frame}

\begin{frame}{Comparison operators}
\begin{itemize}
\item \lstinline{<} less than
\item \lstinline{<=} less than or equal to
\item \lstinline{>} greater than
\item \lstinline{>=} greater than or equal to
\item \lstinline{==} equal to
\item \lstinline{!=} not equal to
\end{itemize}
 Note the difference between \lstinline{=} and \lstinline{==}
\begin{itemize}
\item \lstinline{a = b} means ``make \lstinline{a} be equal to \lstinline{b}''
\item \lstinline{a == b} means ``is \lstinline{a} equal to \lstinline{b}?''
\end{itemize}
\end{frame}

\begin{frame}{For loops and ranges}
\lstinputlisting{for0.cs}
\begin{itemize}
\item \lstinline{for} contains 3 statements: \textbf{variable, condition} and \textbf{increment}
\item Initially the  \textbf{variable} is set to a value and the  \textbf{incrementer} increases the value until the  \textbf{condition} is met
\item The \lstinline{for} loop iterates through the items in a sequence \textbf{in order}. As the loop iterates the variable is increased each time: $0, 1, 2, 3, 4$
\item Note: \lstinline{i < 5} \textbf{does not include} $5$ as the condition is met at 4 so the loop stops.
\end{itemize}
\end{frame}

\begin{frame}{For loops (1)}
\codeslide{for1.cs}{\trow{a}\trow{b}\trow{i}}
\end{frame}

\begin{frame}{For loops (2)}
\codeslide{for2.cs}{\trow{a}\trow{b}\trow{i}}
\end{frame}

\begin{frame}{While loops}
The \lstinline{while} loop keeps executing while the condition is \textbf{true}

\codeslide{while1.cs}{\trow{a}}
\end{frame}

\begin{frame}{Looping forever}
\lstinputlisting{while2.cs}
\end{frame}

\begin{frame}{Summary}
 We have seen some basic code constructions in Python
\begin{itemize}
\item \lstinline{Console.WriteLine()} and \lstinline{Console.ReadLine()} for command-line input and output
\item Variable assignment using \lstinline{=}
\item \lstinline{if} statements for choosing whether or not to execute a block of code
\item \lstinline{for} loops to execute a block of code a specified number of times
\item \lstinline{while} loops to execute a block of code until a condition is no longer true
\end{itemize}
 These are enough to write some simple programs, but you will see several more in coming weeks...
\end{frame}

% Show PyCharm bouncing ball example? -- think about this between now and Michael's session next Wednesday