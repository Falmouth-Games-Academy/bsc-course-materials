\newcommand{\codeslide}[2]{
	\begin{columns}
		\begin{column}{0.48\textwidth}
			\lstinputlisting{#1}
		\end{column}
		\pause
		\begin{column}{0.48\textwidth}
			\begin{center}
				\colorbox{white}{
					\color{black}
					\begin{tabular}{|c|c|}
						\hline
						\textbf{Variable} & \textbf{Value} \\\hline
						#2
					\end{tabular}
				}
			\end{center}
		\end{column}
	\end{columns}
}

\newcommand{\trow}[1]{ & \\ \texttt{#1} & \\ & \\\hline}

\part{Basic Python programs}
\frame{\partpage}

\begin{frame}[fragile]{Your first Python program}
\begin{lstlisting}
print "Hello, world!"
\end{lstlisting}
\end{frame}

\begin{frame}[fragile]{Your second Python program}
\begin{lstlisting}
print "This is a very long line of code which had to be split to fit on the slide, but you should type it as a single line."
print "This is the second line of code."
\end{lstlisting}
\end{frame}

\begin{frame}{Assigning to variables}
\codeslide{assign0.py}{\trow{a}}
\end{frame}

\begin{frame}{Remember!}
\begin{itemize}
\pause\item A program is a \textbf{sequence of instructions}
\pause\item The Python interpreter executes the \textbf{first line} of your program, then the \textbf{second line},
and so on
\pause\item When it reaches the end of the file, it \textbf{stops}
\end{itemize}
\end{frame}

\begin{frame}{Reassigning variables (1)}
\codeslide{assign1.py}{\trow{a}\trow{b}}
\end{frame}

\begin{frame}{Reassigning variables (2)}
\codeslide{assign2.py}{\trow{a}\trow{b}}
\end{frame}

\begin{frame}{Reassigning variables (3)}
\codeslide{assign3.py}{\trow{big}\trow{small}}
\end{frame}

\begin{frame}{Reassigning variables (4)}
\codeslide{assign4.py}{\trow{a}\trow{b}}
\end{frame}

\begin{frame}{Reassigning variables (5)}
\codeslide{assign5.py}{\trow{a}\trow{b}\trow{c}}
\end{frame}

\begin{frame}{Reading input}
\pause \lstinputlisting{input.py}
\begin{itemize}
\pause\item \lstinline{raw_input()} reads a \textbf{string} as text from the command line
\pause\item \lstinline{int(...)} converts a \textbf{string} into an \textbf{integer} (a number)
\end{itemize}
\end{frame}

\begin{frame}{Conditionals (1)}
\codeslide{cond1.py}{\trow{a}\trow{b}}
\end{frame}

\begin{frame}{Indentation}
\begin{itemize}
\pause\item Unlike many other programming languages, \textbf{indentation has meaning} in Python!
\pause\item Python uses indentation to denote the \textbf{block of code} inside a conditional, loop, function etc.
\pause\item PEP-8 recommends \textbf{4 spaces} for indentation
\begin{itemize}
\pause\item Some programmers use a tab character
\pause\item \textbf{Never} mix tabs and spaces in the same file!
\pause\item PyCharm inserts 4 spaces by default when you press the tab key;
other IDEs and text editors can be configured to do this
\end{itemize}
\end{itemize}
\end{frame}

\begin{frame}{Conditionals (2)}
\codeslide{cond2.py}{\trow{a}\trow{b}}
\end{frame}

\begin{frame}{Conditionals}
\pause An \lstinline{if} statement can have:
\begin{itemize}
\pause\item \textbf{Zero or more} \lstinline{elif} clauses
\pause\item \textbf{An optional} \lstinline{else} clause
\end{itemize}
\pause In that order!
\end{frame}

\begin{frame}{Mathematical operators}
\begin{itemize}
\pause\item \lstinline{+} add
\pause\item \lstinline{-} subtract
\pause\item \lstinline{*} multiply
\pause\item \lstinline{/} divide
\pause\item \lstinline{**} power
\end{itemize}
\pause Order of operations: \textbf{BIDMAS}
\begin{itemize}
\pause\item \uline{B}rackets first
\pause\item Then \uline{i}ndices (powers)
\pause\item Then \uline{d}ivision and \uline{m}ultiplication (left to right)
\pause\item Then \uline{a}ddition and \uline{s}ubtraction (left to right)
\end{itemize}
\end{frame}

\begin{frame}{Comparison operators}
\begin{itemize}
\pause\item \lstinline{<} less than
\pause\item \lstinline{<=} less than or equal to
\pause\item \lstinline{>} greater than
\pause\item \lstinline{>=} greater than or equal to
\pause\item \lstinline{==} equal to
\pause\item \lstinline{!=} not equal to
\end{itemize}
\pause Note the difference between \lstinline{=} and \lstinline{==}
\begin{itemize}
\pause\item \lstinline{a = b} means ``make \lstinline{a} be equal to \lstinline{b}''
\pause\item \lstinline{a == b} means ``is \lstinline{a} equal to \lstinline{b}?''
\end{itemize}
\end{frame}

\begin{frame}{For loops and ranges}
\lstinputlisting{for0.py}
\begin{itemize}
\pause\item \lstinline{xrange(n)} is the \textbf{sequence}
$0, 1, 2, \dots, n-1$
\pause\item So \lstinline{xrange(5)} is the \textbf{sequence}
$0, 1, 2, 3, 4$
\pause\item Note: \lstinline{xrange(n)} \textbf{does not include} $n$
\pause\item The \lstinline{for} loop iterates through the items in a sequence \textbf{in order}
\pause\item Can also use \lstinline{range} instead of \lstinline{xrange},
but \lstinline{range} is less efficient
\begin{itemize}
\item Homework (advanced): what is the difference between \lstinline{range} and \lstinline{xrange}?
\end{itemize}
\end{itemize}
\end{frame}

\begin{frame}{For loops (1)}
\codeslide{for1.py}{\trow{a}\trow{b}\trow{i}}
\end{frame}

\begin{frame}{For loops (2)}
\codeslide{for2.py}{\trow{a}\trow{b}\trow{i}}
\end{frame}

\begin{frame}{While loops}
The \lstinline{while} loop keeps executing while the condition is \textbf{true}

\codeslide{while1.py}{\trow{a}}
\end{frame}

\begin{frame}{Looping forever}
\lstinputlisting{while2.py}
\end{frame}

\begin{frame}{Summary}
\pause We have seen some basic code constructions in Python
\begin{itemize}
\pause\item \lstinline{print} and \lstinline{raw_input} for command-line input and output
\pause\item Variable assignment using \lstinline{=}
\pause\item \lstinline{if} statements for choosing whether or not to execute a block of code
\pause\item \lstinline{for} loops to execute a block of code a specified number of times
\pause\item \lstinline{while} loops to execute a block of code until a condition is no longer true
\end{itemize}
\pause These are enough to write some simple programs, but you will see several more in coming weeks...
\end{frame}

% Show PyCharm bouncing ball example? -- think about this between now and Michael's session next Wednesday