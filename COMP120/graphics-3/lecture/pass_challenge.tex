\begin{frame}[fragile]
	\frametitle{PASS Challenge}
		
	Review the following python setup code:

	\begin{lstlisting}
import random
randomNumber = random.randrange(0,100)
print("Random number has been generated")
 	\end{lstlisting}
		
\end{frame}

\begin{frame}[fragile]
	\frametitle{PASS Challenge}
		
	Review the following pythoin game code:

	\begin{lstlisting}
guessed = False
while guessed==False:
    userInput = int(input("Your guess pleas: "))
    if userInput==randomNumber:
        guessed = True
        print("Well done!")
    elif userInput>100:
        print("Our guess range is between 0 and 100, please try a bit lower")
    elif userInput<0:
        print("Our guess range is between 0 and 100, please try a bit higher")
    elif userInput>randomNumber:
        print("Try one more time, a bit lower")
    elif userInput < randomNumber:
        print("Try one more time, a bit higher")
print("You win!")
	\end{lstlisting}
		
\end{frame}

\begin{frame}
	\frametitle{PASS Challenge}
	
	\begin{itemize}
		\item In pairs
		\item \textbf{Implement} the code excerpt
		\item \textbf{Refactor} the code to improve readability
		\item \textbf{Improve} overall maintainability of the code, breaking it down into functions
		\item \textbf{Note} the principles which make the revised version better
	\end{itemize}
	
	You can learn more about PyGame \texttt{random} at:
	
	\vspace{1em}
	
	 \url{docs.python.org/3.6/library/random.html}
	
	\vspace{1em}
	
	(40 minutes)
	
\end{frame}