% Adjust these for the path of the theme and its graphics, relative to this file
%\usepackage{beamerthemeFalmouthGamesAcademy}
\usepackage{../../../beamerthemeFalmouthGamesAcademy}
\usepackage{multimedia}
\graphicspath{ {../../../} }

% Default language for code listings
\lstset{language=Python,
        morekeywords={each,in,nullptr}
}

% For strikethrough effect
\usepackage[normalem]{ulem}
\usepackage{wasysym}

\usepackage{pdfpages}

% http://www.texample.net/tikz/examples/state-machine/
\usetikzlibrary{arrows,automata}

\newcommand{\modulecode}{COMP260}\newcommand{\moduletitle}{Distributed Systems}\newcommand{\sessionnumber}{5}

\setbeamertemplate{navigation symbols}{}

\newcommand{\fullbleed}[1]{
\begin{frame}[plain]
	\begin{tikzpicture}[remember picture, overlay]
		\node[at=(current page.center)] {
			\includegraphics[width=\paperwidth]{#1}
		};
	\end{tikzpicture}
\end{frame}
}

\newcommand{\picturepage}[2]{
\begin{frame}[plain]
	\begin{tikzpicture}[remember picture, overlay]
		\node[at=(current page.center)] {
			\includegraphics[width=\paperwidth]{#1}
		};
		\draw<1>[draw=none, fill=black, opacity=0.9] (-1,-5.2) rectangle (current page.south east);
		\node[draw=none,text width=0.96\paperwidth, align=right] at (5.5,-5.5) {\tiny{#2}};
	\end{tikzpicture}
\end{frame}
}

\newcommand{\notepicx}[5]{
\begin{frame}[plain]
	\begin{tikzpicture}[remember picture, overlay]
		\node[at=(current page.center)] {
			\includegraphics[width=\paperwidth]{#1}
		};
		\node[draw=none, fill=black, text width=#5\paperwidth] at ([xshift=#3, yshift=#4] current page.center) {\small{#2}};
	\end{tikzpicture}
\end{frame}
}

\newcommand{\notepic}[4]{
	\notepicx{#1}{#2}{#3}{#4}{0.4}
}

\begin{document}
\title{\sessionnumber: Tinkering Graphics III}
\subtitle{\modulecode: \moduletitle}

\frame{\titlepage} 

\begin{frame}
	\frametitle{Learning Outcomes}
	\begin{itemize}
		\item \textbf{Explain how} to duplicate parts of an image
		\item \textbf{Explain how} to identify the spaces and features of an image
		\item \textbf{Apply} mathematical knowledge to \textbf{write} computer programs that manipulate the spaces and features of an image
		\item \textbf{Implement} executable code that can `tinker' graphics
	\end{itemize}
\end{frame}

\begin{frame}
	\frametitle{Mirroring}
	
	\begin{itemize}		
		\item Mirroring is averaging across pixels
		\item A mirror will:
		\begin{itemize}
			\item Define a dimension of inflexion
			\item Duplicate pixels from the one side of the inflextion to the other, in reverse order
		\end{itemize}	
	\end{itemize}
\end{frame}

\fullbleed{mirror}

\begin{frame}[fragile]
	\frametitle{Source Code: Mirroring (1)}
	
\begin{lstlisting}
def mirror_vertical(picture):
  width = get_width(picture)
  height = get_height(picture)
  mirror_point = width / 2
  for y in xrange(0, height):
      for x in xrange(0, mirror_point):
          left_pixel = get_pixel(picture, x, y)
          right_pixel = get_pixel(picture, width - x - 1, y)
          set_colour(right_pixel, get_colour(left_pixel))
\end{lstlisting}

\end{frame}

\begin{frame}[fragile]
	\frametitle{Source Code: Mirroring (2)}
	
\begin{lstlisting}
def mirror_vertical(picture):
  width = get_width(picture)
  height = get_height(picture)
  mirror_point = height / 2
  for x in xrange(0, width):
      for y in xrange(0, mirror_point):
          top_pixel = get_pixel(picture, x, y)
          bottom_pixel = get_pixel(picture, x, height - y - 1)
          set_colour(bottom_pixel, get_colour(top_pixel))
\end{lstlisting}

\end{frame}

\begin{frame}
	\frametitle{Activity \#1: Mirroring}
	
	In pairs:
	
	\vspace{2em}
	
	\begin{itemize}		
		\item Use your function to repair the temple
		\item Post your repaired temple on Slack!
	\end{itemize}
\end{frame}

\fullbleed{repair_the_temple}

\begin{frame}
	\frametitle{Edge Detection}
	
	\begin{itemize}		
		\item Blurring is averaging across pixels
		\item Edge detection is looking for \textit{differences} between pixels:
		\begin{itemize}
			\item We draw lines that our eyes see --- where the luminance changes
		\end{itemize}	
		\item If the pixel changes left-to-right, or up-and-down, make a pixel black. Else, white.
	\end{itemize}
\end{frame}

\begin{frame}[fragile]
	\frametitle{Source Code: Edge Detection}
	
\begin{lstlisting}
def drawEdges(picture):
  for x in xrange(0, picture.length - 1):
      for y in xrange(0, picture.height - 1):
          pixelSum = getRed(pixel) + getGreen(pixel) + getBlue(pixel)
          nextPixel = getPixel(picture, x+1, y+1)
          nextPixelSum = getRed(nextPixel)+ getGreen(nextPixel) + getBlue(nextPixel)
          diff = abs(nextPixelSum-pixelSum)
          newColor = makeColor(diff,diff,diff)
          setColor(pixel, newColor)
\end{lstlisting}

Note: This source code excerpt will not work in PyGame.

\end{frame}

\fullbleed{seahorse}

\begin{frame}[fragile]
	\frametitle{Source Code: Edge Detection (1)}
	
\begin{lstlisting}
def drawBetterEdges(picture):
  orig = makePicture(filename)
  makeBw = makePicture(filename)
  
  for x in xrange(0,getWidth(picture)-1):
    for y in xrange(0,getHeight(picture)-1):
      here=getPixel(makeBw,x,y)
      down=getPixel(orig,x,y+1)
      right=getPixel(orig,x+1,y)
      
      hereL=(getRed(here)+getGreen(here)+getBlue(here))/3
      downL=(getRed(down)+getGreen(down)+getBlue(down))/3
      rightL=(getRed(right)+getGreen(right)+getBlue(right))/3
...
\end{lstlisting}

Note: This source code excerpt will not work in PyGame.

\end{frame}

\begin{frame}[fragile]
	\frametitle{Source Code: Edge Detection (2)}
	
\begin{lstlisting}
...
      if abs(hereL-downL)>10 and abs(hereL-rightL)>10:
        setColor(here,black)
      if abs(hereL-downL)<=10 and abs(hereL-rightL)<=10:
        setColor(here,white)
  return makeBw
\end{lstlisting}

Note: This source code excerpt will not work in PyGame.

\end{frame}

\fullbleed{butterfly}

\begin{frame}
	\frametitle{Activity \#2: Edge Detection}
	
	In pairs:
	
	\vspace{2em}
	
	\begin{itemize}		
		\item Setup a basic project in PyGame
		\item Refer to the following documentation
		\item Refactor the function: \texttt{drawBetterEdges(picture)} to use better variable names
		\item Test your solution
	\end{itemize}
\end{frame}

\begin{frame}
	\frametitle{Background Subtraction}
	
	\begin{itemize}		
		\item Let's say that you have a picture of someone, and a picture of the same place (same background) without the someone there,
		could you subtract out the background and leave the picture of the person?
		\item Maybe even change the background?
		\item What we most need to do is to figure out whether the pixel in the Person shot is the same as the in the Background shot.
		\item Will they be the EXACT same colour?  Probably not.
	\end{itemize}
\end{frame}

\begin{frame}[fragile]
	\frametitle{Source Code: Background Subtraction}
	
\begin{lstlisting}
def swapBackground(picture, back, newBack, tolerance):
  for pixel in getPixels(picture):
    x = getX(pixel)
    y = getY(pixel)
    backPixel = getPixel(back, x, y)
    pixelColour = getColour(pixel)
    backColour = getColour(backPixel)
    if (distance(pixelColour,backColour) < tolerance):
      newColour = getColour(getPixel(newBack, x, y))
      setColor(pixel, newColour)
\end{lstlisting}

Note: This source code excerpt will not work in PyGame.

\end{frame}

\fullbleed{katie_nomoon}

\fullbleed{katie_forest}

\begin{frame}
	\frametitle{Problems}
	
	\begin{itemize}		
		\item We've got places where we got pixels swapped that we didn't want to swap
		\item We've got places where we want pixels swapped, but didn't get them swapped
		\begin{itemize}
			\item Shirt stripes
			\item Shadows
			\item etc.
		\end{itemize}	
	\end{itemize}
\end{frame}

\begin{frame}
	\frametitle{Activity \#3: Background Subtraction}
	
	In pairs:
	
	\vspace{2em}
	
	\begin{itemize}		
		\item Setup a basic project in PyGame
		\item Refer to the following documentation:
		\begin{itemize}
			\item \url{http://www.cs.utah.edu/~michael/chroma/}
		\end{itemize}
		\item Implement chroma key as a form of background subtraction
		\item Test your solution
	\end{itemize}
\end{frame}

\begin{frame}[fragile]
	\frametitle{Source Code: Collage}
	
\begin{lstlisting}
def copyBarb():
  # Set up the source and target pictures
  barbf=getMediaPath("barbara.jpg")
  barb = makePicture(barbf)
  canvasf = getMediaPath("7inX95in.jpg")
  canvas = makePicture(canvasf)
  # Now, do the actual copying
  targetX = 0
  for sourceX in range(0,getWidth(barb)):
    targetY = 0
    for sourceY in range(0,getHeight(barb)):
      color = getColor(getPixel(barb,sourceX,sourceY))
      setColor(getPixel(canvas,targetX,targetY), color)
      targetY = targetY + 1
    targetX = targetX + 1
  show(barb)
  show(canvas)
  return canvas
\end{lstlisting}

\end{frame}

\fullbleed{copy_1}
\fullbleed{copy_2}
\fullbleed{copy_3}
\fullbleed{copy_4}
\fullbleed{copy_5}

\begin{frame}
	\frametitle{Activity \#4: Collage}
	
	In pairs:
	
	\vspace{2em}
	
	\begin{itemize}		
		\item Find some smaller images online
		\item Integrate the copy algorithm into your tinkering graphics project
		\item Create a collage of the images you found
	\end{itemize}
\end{frame}

\begin{frame}{Sprite Sheets and Animations}
	Review Al Swigart's pyganim python module: 
	
	\vspace{2em}
	
	\url{http://inventwithpython.com/pyganim/}
	
\end{frame}

\begin{frame}
	\frametitle{Activity \#5: Sprite Sheets}
	
	In pairs:
	
	\vspace{2em}
	
	\begin{itemize}		
		\item Find a sprite sheet online
		\item Integrate pyganim into your tinkering graphics project
		\item Animate something
	\end{itemize}
\end{frame}

\begin{frame}
	\frametitle{What Next?}
	
	If you have implemented all of these algorithms, then use the rest of the workshop to:
	
	\vspace{2em}
	
	\begin{itemize}	
		\item You now have a range of image manipulation algorithms at your disposal
		\item You only need to successfully implement and repurpose a set of these to do well on your Tinkering Graphics assignment!
		\item So, finish implementing the algorithms needed to complete your coursework
		\item Tidy up your code, ready for next session's peer-review activity
		\item Extend the code beyond the brief as appropriate to your particular game
	\end{itemize}
\end{frame}

\end{document}
