% Adjust these for the path of the theme and its graphics, relative to this file
%\usepackage{beamerthemeFalmouthGamesAcademy}
\usepackage{../../beamerthemeFalmouthGamesAcademy}
\usepackage{multimedia}
\graphicspath{ {../../} }

% Default language for code listings
\lstset{language=Python,
        morekeywords={each,in,nullptr}
}

% For strikethrough effect
\usepackage[normalem]{ulem}
\usepackage{wasysym}

\usepackage{pdfpages}

% //www.texample.net/tikz/examples/state-machine/
\usetikzlibrary{arrows,automata}

\newcommand{\modulecode}{COMP260}\newcommand{\moduletitle}{Distributed Systems}\newcommand{\sessionnumber}{5}

\setbeamertemplate{navigation symbols}{}

\newcommand{\fullbleed}[1]{
\begin{frame}[plain]
	\begin{tikzpicture}[remember picture, overlay]
		\node[at=(current page.center)] {
			\includegraphics[width=\paperwidth]{#1}
		};
	\end{tikzpicture}
\end{frame}
}

\newcommand{\picturepage}[2]{
\begin{frame}[plain]
	\begin{tikzpicture}[remember picture, overlay]
		\node[at=(current page.center)] {
			\includegraphics[width=\paperwidth]{#1}
		};
		\draw<1>[draw=none, fill=black, opacity=0.9] (-1,-5.2) rectangle (current page.south east);
		\node[draw=none,text width=0.96\paperwidth, align=right] at (5.5,-5.5) {\tiny{#2}};
	\end{tikzpicture}
\end{frame}
}

\newcommand{\notepicx}[5]{
\begin{frame}[plain]
	\begin{tikzpicture}[remember picture, overlay]
		\node[at=(current page.center)] {
			\includegraphics[width=\paperwidth]{#1}
		};
		\node[draw=none, fill=black, text width=#5\paperwidth] at ([xshift=#3, yshift=#4] current page.center) {\small{#2}};
	\end{tikzpicture}
\end{frame}
}

\newcommand{\notepic}[4]{
	\notepicx{#1}{#2}{#3}{#4}{0.4}
}

\begin{document}
\title{\sessionnumber: Tinkering Graphics I}
\subtitle{\modulecode: \moduletitle}

\frame{\titlepage} 

\begin{frame}
	\frametitle{Learning Outcomes}
	By the end of this workshop, you should be able to:	
	\begin{itemize}
		\item \textbf{Explain how} pictures are digitised into raster images by a computer system
		\item \textbf{Apply} knowledge of colour models to \textbf{write} code that manipulates pixels in a surface
		\item \textbf{Use} functions, arguments, and basic data structures such as arrays
	\end{itemize}
\end{frame}

\begin{frame}
	\frametitle{Light Perception}
	\begin{itemize}
		\item Colour is continuous:
		\begin{itemize}
			\item Visible light is in the wavelengths between 370nm and 730nm
			\item i.e., 0.00000037 --- 0.00000073 meters
		\end{itemize}
		\item However, we \textit{perceive} light around three particular peaks:
		\begin{itemize}
			\item Blue peaks around 425nm
			\item Green peaks around 550nm
			\item Red peaks around 560nm
		\end{itemize}
	\end{itemize}
\end{frame}

\begin{frame}
	\frametitle{Light Perception}
	\begin{itemize}
		\item Our eyes have three types of colour-sensitive photoreceptor cells called `cones' that respond to light wavelengths
		\item Our perception of colour is based on how much of each kind of sensor is responding
		\item An implication of this is perception overlap: we see two kinds of `orange' --- one that's spectral and one that's combinatorial
	\end{itemize}
\end{frame}

\fullbleed{1416_Color_Sensitivity}

\begin{frame}
	\frametitle{Luminance vs Colour}
	\begin{itemize}
		\item Our eyes have another type of photoreceptor cells called `rods' that respond to light intensity
		\item Our perception, however, is actually luminance: a relativistic contrast of \textit{borders} of things (i.e., motion) 
		\begin{itemize}
			\item Luminance is \textit{not} the amount of light, but our perception of the amount of light
			\item Much of our luminance perception is based on comparison to background, not raw values
		\end{itemize}
		\item An implication of this is perception overlap: we see blue as `darker' than red when the intensity is actually the same
	\end{itemize}
\end{frame}

\fullbleed{grayscale}

\begin{frame}
	\frametitle{Resolution}
	\begin{itemize}
		\item We have a limited number of rods and cones in our eyes
		\item This means humans perceive vision in a limited resolution --- yet, we perceive vision as continuous
		\item We take advantage of this human characteristic in computer monitors
	\end{itemize}
\end{frame}

\fullbleed{sony-dsc-29-625x1000}

\begin{frame}
	\frametitle{Pixels}
	\begin{itemize}
		\item We digitize pictures into many little dots
		\item Enough dots and it looks like a continuous whole to our eye
		\item Each element is referred to as a \textit{pixel}
	\end{itemize}
\end{frame}

\begin{frame}
	\frametitle{Pixels}
	
	Pixels must have:
	
	\begin{itemize}
		\item a color
		\item a position
	\end{itemize}
\end{frame}

\begin{frame}
	\frametitle{Pictures and Surfaces}
	
	A picture is a \textit{matrix} of pixels
	
	\begin{itemize}
		\item It is not a continuous line of elements, that is, a one-dimensional \textit{array}
		\item A picture has two dimensions: width and height
		\item It's a two-dimensional \textit{array}
	\end{itemize}
\end{frame}

\fullbleed{pic_array}

\fullbleed{pic_matrix}

\begin{frame}
	\frametitle{Pictures and Surfaces}
	
	\begin{itemize}
		\item (x, y) ---or--- (horizontal, vertical)
		\item (1,0) = 12
		\item (0, 2) = 6
	\end{itemize}
\end{frame}

\begin{frame}
	\frametitle{Encoding Colour}
	
	\begin{itemize}
		\item Each element in the matrix is a pixel, with the matrix defining its position and the value defining  its colour
		\item Computer memory stores numbers, so colour must be encoded into a number:
		\begin{itemize}
			\item CMYK = cyan, magenta, yellow, black
			\item HSB = hue, saturation, brightness
			\item RGBA = red, green, blue, alpha (transparency)
		\end{itemize}
		\item By default, PyGame uses RGBA
	\end{itemize}
\end{frame}

\begin{frame}
	\frametitle{Encoding RGB}
	
	\begin{itemize}
		\item Each component color (red, green, and blue) is encoded as a single byte
		\item Colors go from \lstinline{(0,0,0)} to \lstinline{(255,255,255)}:
		\begin{itemize}
			\item If all three components are the same, the colour is in grey-scale
			\item  \lstinline{(0,0,0)} is black
			\item  \lstinline{(255, 255, 255)} is white
		\end{itemize}
		\item Why \texttt{255}?
	\end{itemize}
\end{frame}

\fullbleed{rgb_matrix}

\begin{frame}
	\frametitle{Encoding Bits}
	
	\begin{itemize}
		\item If we have one bit, we can represent two patterns:
		\begin{itemize}
			\item 0
			\item 1
		\end{itemize}
		
		\item If we have two bits, we can represent two patterns:
		\begin{itemize}
			\item 00
			\item 01
			\item 10
			\item 11
		\end{itemize}
		\item As a general rule: In \textit{n} bits, we can have $2^n$ patterns
		\item One of these patterns will be 0, so the highest value we can represent is: $2^8 - 1$, or 255
	\end{itemize}
\end{frame}

\begin{frame}
	\frametitle{Encoding Bits}
	\begin{itemize}
		\item RGB uses 24-bit color (i.e., $3 * 8$ = 24)
		\begin{itemize}
			\item That's 16,777,216 (224) possible colours
			\item Our eyes cannot discern many colours beyond this
			\item The big issue is the monitor: they can`t reliably reproduce 16 million colours
		\end{itemize}
		\item RGBA uses 32-bit colour
		\begin{itemize}
			\item No additional colour, but offers support for transparency
		\end{itemize}
	\end{itemize}
\end{frame}

\begin{frame}
	\frametitle{Encoding Bits}
	\begin{itemize}
		\item Use this information to estimate the size of a bitmap:
		\begin{itemize}
			\item 320x240x24 = 230,400 bytes
			\item 640x480x32 = 1,228,800 bytes
			\item 1024x768x32 = 3,145,728 bytes
		\end{itemize}
		\item Why do we have smaller numbers here?
	\end{itemize}
\end{frame}

\begin{frame}
	\frametitle{Activity \#1}
	
	In pairs:
	
	\vspace{2em}
	
	\begin{itemize}		
		\item Setup a basic project in PyGame
		\item Refer to the following documentation
		\begin{itemize}
			\item \url{www.pygame.org/docs/ref/surface.html}
			\item \url{www.pygame.org/docs/ref/pixelarray.html}
		\end{itemize}
		\item Manipulate the pixels in a surface using PixelArray
		\item 10 minutes
	\end{itemize}
\end{frame}

\begin{frame}[fragile]
	\frametitle{Make a Picture Less Red}
	
\begin{lstlisting}
def decreaseRed(pict):
  pixelMatrix = getPixels(pict)
  for pixel in pixelMatrix:
    value = getRed(pixel)
    setRedPixel(pixel, value * 0.5)
\end{lstlisting}

Note: This source code excerpt will not work in PyGame.

\end{frame}

\begin{frame}
	\frametitle{Activity \#2}
	
	In pairs:
	
	\vspace{2em}
	
	\begin{itemize}
		\item Define a function that turns all of the red values of pixels into blue values...
		\item ...and all of the blue values into red values
	\end{itemize}
\end{frame}

\begin{frame}[fragile]
	\frametitle{Swap Channels}
	
\begin{lstlisting}
def swapRedBlueChannels(pict):
  pixelMatrix = getPixels(pict)
  for pixel in pixelMatrix:
    red_value = getRed(pixel)
    blue_value = getBlue(pixel)
    setRedPixel(pixel, blue_value)
    setBluePixel(pixel, red_value)
\end{lstlisting}

Note: This source code excerpt will not work in PyGame.

\end{frame}

\begin{frame}
	\frametitle{Activity \#3}
	
	In pairs:
	
	\vspace{2em}
	
	\begin{itemize}
		\item Refer to the following documentation:
		\begin{itemize}
			\item \url{//www.pygame.org/docs/ref/image.html}
		\end{itemize}
		\item Define a function that loads an image and turns it to greyscale
	\end{itemize}
\end{frame}

\begin{frame}[fragile]
	\frametitle{Make a Picture Grey-scale}
	
\begin{lstlisting}
def loadGrayscale(file):
  pixelMatrix = getPixels(makePicture(file))
  for pixel in pixelMatrix:
    red = getRed(p)
    green = getGreen(p)
    blue = getBlue(p)
    
    pixelValue = (red+green+blue)/3
    
    setRedPixel(pixel,pixelValue)
    setGreenPixel(pixel, pixelValue)
    setBluePixel(pixel, pixelValue)
\end{lstlisting}

Note: This source code excerpt will not work in PyGame.

\end{frame}

\begin{frame}
	\frametitle{Activity \#4}
	
	In pairs:
	
	\vspace{2em}
	
	\begin{itemize}
		\item Refer to the following documentation:
		\begin{itemize}
			\item \url{//www.pygame.org/docs/ref/image.html}
		\end{itemize}
		\item Define a function that loads an image and turns it to its negative
	\end{itemize}
\end{frame}

\begin{frame}[fragile]
	\frametitle{Make a Picture Negative}
	
\begin{lstlisting}
def neg(picture):
  pixelMatrix = getPixels(makePicture(file))
  for pixel in pixelMatrix:
    red = getRed(p)
    green = getGreen(p)
    blue = getBlue(p)
        
    setRedPixel(pixel,255-red)
    setGreenPixel(pixel, 255-green)
    setBluePixel(pixel, 255-blue)
\end{lstlisting}

Note: This source code excerpt will not work in PyGame.

\end{frame}

\begin{frame}
	\frametitle{Activity \#5}
	
	In pairs:
	
	\vspace{2em}
	
	\begin{itemize}
		\item Refer to the following documentation:
		\begin{itemize}
			\item \url{//www.pygame.org/docs/ref/time.html}
		\end{itemize}
		\item Define a function that loads an image and animates a sunset
	\end{itemize}
\end{frame}

\begin{frame}[fragile]
	\frametitle{Decrease Luminance Over Time}

\begin{lstlisting}
def decreaseRed(picture, amount):
  for p in getPixels(picture):
    value=getRed(p)
    setRed(p,value*amount)

amount = 0.1 #tinker with this value
wait_time = 50 #tinker with this value    
    
for i in range(10):
  decreaseRed(picture, amount)
  decreaseGreen(picture, amount)
  decreaseBlue(picture, amount)
  wait(50)
\end{lstlisting}

Note: This source code excerpt will not work in PyGame.

\end{frame}

\begin{frame}
	\frametitle{Activity \#6}
	
	In pairs:
	
	\vspace{2em}
	
	\begin{itemize}
		\item Refer to the following documentation:
		\begin{itemize}
			\item \url{https://docs.python.org/2/tutorial/introduction.html\#lists}
		\end{itemize}
		\item Define a function that animates copying the top half of a picture to its bottom half
	\end{itemize}
\end{frame}

\begin{frame}[fragile]
	\frametitle{Copying Part of a Picture}

\begin{lstlisting}
def copyHalf(picture):
 pixels = getPixels(picture)
 for index in range(0,len(pixels)/2):
   sourcePixel = pixels[index]
   sourceRGBValue = getColor(sourcePixel)
   destinationPixel = pixels[index + len(pixels)/2]
   setColor(destinationPixel,sourceRGBValue)
 repaint(picture)
\end{lstlisting}

Note: This source code excerpt will not work in PyGame.

\end{frame}

\end{document}
