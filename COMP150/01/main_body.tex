% Adjust these for the path of the theme and its graphics, relative to this file
%\usepackage{beamerthemeFalmouthGamesAcademy}
\usepackage{../../beamerthemeFalmouthGamesAcademy}
\usepackage{multimedia}
\graphicspath{ {../../} }

% Default language for code listings
\lstset{language=C++,
        morekeywords={each,in,nullptr}
}

% For strikethrough effect
\usepackage[normalem]{ulem}
\usepackage{wasysym}

\usepackage{pdfpages}

% http://www.texample.net/tikz/examples/state-machine/
\usetikzlibrary{arrows,automata}

\newcommand{\modulecode}{COMP260}\newcommand{\moduletitle}{Distributed Systems}\newcommand{\sessionnumber}{5}

\setbeamertemplate{navigation symbols}{}

\newcommand{\fullbleed}[1]{
\begin{frame}[plain]
	\begin{tikzpicture}[remember picture, overlay]
		\node[at=(current page.center)] {
			\includegraphics[width=\paperwidth]{#1}
		};
	\end{tikzpicture}
\end{frame}
}

\newcommand{\picturepage}[2]{
\begin{frame}[plain]
	\begin{tikzpicture}[remember picture, overlay]
		\node[at=(current page.center)] {
			\includegraphics[width=\paperwidth]{#1}
		};
		\draw<1>[draw=none, fill=black, opacity=0.9] (-1,-5.2) rectangle (current page.south east);
		\node[draw=none,text width=0.96\paperwidth, align=right] at (5.5,-5.5) {\tiny{#2}};
	\end{tikzpicture}
\end{frame}
}

\newcommand{\notepicx}[5]{
\begin{frame}[plain]
	\begin{tikzpicture}[remember picture, overlay]
		\node[at=(current page.center)] {
			\includegraphics[width=\paperwidth]{#1}
		};
		\node[draw=none, fill=black, text width=#5\paperwidth] at ([xshift=#3, yshift=#4] current page.center) {\small{#2}};
	\end{tikzpicture}
\end{frame}
}

\newcommand{\notepic}[4]{
	\notepicx{#1}{#2}{#3}{#4}{0.4}
}

\begin{document}
\title{\sessionnumber: From Concepts to Design}
\subtitle{\modulecode: \moduletitle}

\frame{\titlepage} 

\begin{frame}
	\frametitle{Learning Outcomes}
	
	By the end of this session, you should be able to:
	
	\begin{itemize}
		\item \textbf{Explain} the key differences between ideating game concepts and designing game mechanics
		\item \textbf{Recall} functional definitions of the terms \textit{game} and \textit{games design}
		\item \textbf{Recognise} the formal elements of games \textbf{and} how they fit within the MDA model
		\item \textbf{Compare and contrast} general approaches to game development practice
		\item \textbf{Discuss} the roles of prototyping and play-testing in game development practice
	\end{itemize}
\end{frame}

\begin{frame}
	\frametitle{Defining Game}
	
	So, what is a `game'?
	
	\begin{itemize}
		\item Login to Socrative and enter the room: \\ FALCOMPMIKE
		\item In silence, consider your own definition (5 minutes) and then enter it into the system
	\end{itemize}

\end{frame}

\begin{frame}
	\frametitle{Defining Game}
		
	\begin{center}
	\begin{huge}
	A game has ``ends and means'': an objective, an outcome, and a set of rules to get there.
	\end{huge}
	
	\vspace{3em}
	
	(David Parlett, XXXX)
	\end{center}

\end{frame}

\begin{frame}
	\frametitle{Defining Game}
		
	\begin{center}
	\begin{huge}
	A game is an activity involving player decisions, seeking objectives within a ``limiting context'' (i.e., rules).
	\end{huge}
	
	\vspace{3em}
	
	(Clark C. Abt, XXXX)
	\end{center}

\end{frame}

\begin{frame}
	\frametitle{Defining Game}
		
	\begin{center}
	\begin{large}
	A game has six properties, it:
	
		\begin{itemize}
		\item is \textit{free}
		\item is \textit{separate}
		\item has an uncertain outcome
		\item is uproductive
		\item is governed by rules
		\item and is \textit{make believe}
	\end{itemize}
	
	\begin{center}
	\end{large}
	
	\vspace{3em}
	
	(Roger Callois, XXXX)
	\end{center}

\end{frame}

\begin{frame}
	\frametitle{Defining Game}
		
	\begin{center}
	\begin{huge}
	Voluntary effort to overcome unnecessary obstacles
	\end{huge}
	
	\vspace{3em}
	
	(Bernard Suits, XXXX)
	\end{center}

\end{frame}

\begin{frame}
	\frametitle{Defining Game}
		
	\begin{large}
	A game has four properties, they:
	
	\begin{itemize}
		\item are a \textit{closed} and \textit{formal system}
		\item involve interaction
		\item involve conflict
		\item and they offer safety
	\end{itemize}
	
	\end{large}
	
	\vspace{3em}
	
	\begin{center}
	(Chris Crawford, XXXX)
	\end{center}

\end{frame}

\begin{frame}
	\frametitle{Defining Game}
		
	\begin{center}
	\begin{huge}
	A form of art in which the participants, termed players, make decisions in order to manage resources through tokens in the pursuit of a goal
	\end{huge}
	
	\vspace{3em}
	
	(Chris Crawford, XXXX)
	\end{center}

\end{frame}

\begin{frame}
	\frametitle{Defining Game}
		
	\begin{center}
	\begin{huge}
	A system in which players engage in an artificial conflict, defined by rules, that results in a quantifiable outcome
	\end{huge}
	
	\vspace{3em}
	
	(Salen \& Zimmerman, 2004)
	\end{center}

\end{frame}

\begin{frame}
	\frametitle{Defining Game}
		
	\begin{center}
	\begin{huge}
	http://www.gamedefinitions.com/
	\end{huge}
	\end{center}
	
	\vspace{3em}
	
	\begin{small}
	Yes, someone wrote a procedural generator for game definitions...
	\end{small}
	
\end{frame}

\begin{frame}
	\frametitle{Defining Game}
	
	My attempt:
		
	\begin{center}
	\begin{huge}
	A game is a framework that codifies imagination, encapsulating play and performance in a way that permits copying and sharing
	\end{huge}
	
	\vspace{3em}
	
	(Scott, 2011)
	\end{center}

\end{frame}

\begin{frame}
	\frametitle{Defining Game}
	
	There are over 60 different definitions, with considerable consensus of what artefacts may be described as games---but little agreement on an adequete definition (XXXX, 2015).

\end{frame}

\end{document}
