% Adjust these for the path of the theme and its graphics, relative to this file
%\usepackage{beamerthemeFalmouthGamesAcademy}
\usepackage{../../beamerthemeFalmouthGamesAcademy}
\usepackage{multimedia}
\graphicspath{ {../../} }

% Default language for code listings
\lstset{language=C++,
        morekeywords={each,in,nullptr}
}

% For strikethrough effect
\usepackage[normalem]{ulem}
\usepackage{wasysym}
\usepackage[T1]{fontenc}
\usepackage{pdfpages}

% http://www.texample.net/tikz/examples/state-machine/
\usetikzlibrary{arrows,automata}

\newcommand{\modulecode}{COMP260}\newcommand{\moduletitle}{Distributed Systems}\newcommand{\sessionnumber}{5}

\newcommand{\fullbleed}[1]{
\begin{frame}[plain]
	\begin{tikzpicture}[remember picture, overlay]
		\node[at=(current page.center)] {
			\includegraphics[width=\paperwidth]{#1}
		};
	\end{tikzpicture}
\end{frame}
}

\newcommand{\picturepage}[2]{
\begin{frame}[plain]
	\begin{tikzpicture}[remember picture, overlay]
		\node[at=(current page.center)] {
			\includegraphics[width=\paperwidth]{#1}
		};
		\draw<1>[draw=none, fill=black, opacity=0.9] (-1,-5.2) rectangle (current page.south east);
		\node[draw=none,text width=0.96\paperwidth, align=right] at (5.5,-5.5) {\tiny{#2}};
	\end{tikzpicture}
\end{frame}
}

\newcommand{\notepicx}[5]{
\begin{frame}[plain]
	\begin{tikzpicture}[remember picture, overlay]
		\node[at=(current page.center)] {
			\includegraphics[width=\paperwidth]{#1}
		};
		\node[draw=none, fill=black, text width=#5\paperwidth] at ([xshift=#3, yshift=#4] current page.center) {\small{#2}};
	\end{tikzpicture}
\end{frame}
}

\newcommand{\notepic}[4]{
	\notepicx{#1}{#2}{#3}{#4}{0.4}
}

\setbeamertemplate{navigation symbols}{}

\begin{document}
\title{\sessionnumber}
\subtitle{\modulecode: \moduletitle}

\frame{\titlepage} 

\begin{frame}
	\frametitle{Learning Outcomes}
	
	By the end of this session, you should be able to:
	
	\begin{itemize}
		\item \textbf{Recognise who} your tutors are
		\item \textbf{Outline what} the Games Academy offers from a computing perspective
		\item \textbf{Explain} the career paths \textbf{and} key learning objectives that the computing course caters to
		\item \textbf{Suggest} some of the kinds of question that excite game scholars within and around the computing discipline
		\item \textbf{Recall} the structure of the course
		\item \textbf{Describe} the first-year modules upon which you are enrolled
	\end{itemize}
\end{frame}

\begin{frame}
	\frametitle{Learning Outcomes}
	
	By the end of this session, you should be able to:
	
	\begin{itemize}
		\item \textbf{Recall} the assignments for the first semester
		\item \textbf{Contrast} what is expected of students in the higher education context to the compulsory education context
		\item \textbf{Analyse how} to invest sufficient time in both course activities \textbf{as well as} self-regulated deliberate practice to achieve key goals
	\end{itemize}
\end{frame}

\part{Course Tutors}
\frame{\partpage}

\picturepage{mike_and_monica}{Michael Scott pictured with Monica McGill}

\picturepage{ed_powley}{Ed Powley}

\picturepage{brian_mcdonald}{Brian McDonald}

\picturepage{gareth_lewis}{Gareth Lewis}

\picturepage{al_parker}{Al Parker}

\part{The Games Academy}
\frame{\partpage}

% Meta-Makers

% EU Investment - AIR

\notepic{cam01-fr0300}{\textbf{World-Leading} Research in \textbf{Digital Games} and \textbf{Digital Games Technology}}{2.5cm}{3cm}

\notepicx{painting_fool_sadness}{Hold more than \textbf{\pounds{}2 million} of funds for research in \textbf{Artificial Intelligence}, \textbf{Procedural Content Generation}, and \textbf{Transmedial Aesthetics}}{-2.5cm}{3cm}{0.5}

\notepic{tanya}{Lead By \textbf{World-Renowned Researchers}}{-2.5cm}{-3.5cm}

\notepic{simon}{Lead By \textbf{World-Renowned Researchers}}{-2.5cm}{-3.5cm}

\notepic{launchpad}{Striving Towards a \textbf{First-Class Educational Provision} that Prepares Students for \textbf{Careers} in the \textbf{Creative Industries}}{-2.5cm}{3cm}

\fullbleed{52_big}

\fullbleed{gold}

\notepic{prospectus_alex_064}{Undergraduate Courses in \textbf{Computing for Games}}{-2.5cm}{-3.5cm}

\notepic{1_studio_wide}{Postgraduate Courses in \textbf{Games Entrepreneurship}}{-2.5cm}{-3.5cm}

\notepic{mike_filming_ma}{Distance-Learning Courses in \textbf{Creative App Development}}{-2.5cm}{-3.5cm}

%\notepic{orange_helicopter}{Our Staff Are \textbf{Indie Game Developers}}{-2.5cm}{-3.5cm}

\notepic{space_caves}{Our Staff Are \textbf{Indie Game Developers}}{-2.5cm}{-3.5cm}

\notepic{warm_gun}{Our Staff Are \textbf{Indie Game Developers}}{-2.5cm}{-3.5cm}

\notepic{underzone_cropped}{Our Staff Are \textbf{Indie Game Developers}}{-2.5cm}{-3.5cm}

\notepicx{deal_with_the_devil}{We Work Closely with \textbf{Cornwall's Largest Game Studios}}{-4.5cm}{-2.25cm}{0.2}

\notepicx{rising-storm-2}{We Work Closely with \textbf{Cornwall's Largest Game Studios}}{-4.5cm}{-2.25cm}{0.2}

\notepic{steve}{We Attract \textbf{Industry Legends} as Visiting Lecturers}{3.5cm}{2.5cm}

\notepic{ian}{We Attract \textbf{Industry Legends} as Visiting Lecturers}{3cm}{-2.5cm}

\notepic{rex}{And \textbf{Our Graduates} Return to Help Us Out}{-3cm}{-3.5cm}

\fullbleed{uk_creative_industries_valuev4-43}

\fullbleed{cornwall_technation_1}

\fullbleed{cornwall_technation_2}

\part{Computing in the Games Industry}
\frame{\partpage}

\begin{frame}
	\frametitle{Careers for Computing Professionals}
	
	It is important to note that:\pause
	
	\begin{itemize}
		\item Games are complex and therefore require significant knowledge and skills to produce \pause
		\item They bring together art, storytelling, design and computing \pause
		\item Roles are therefore quite diverse and specialised \pause
		\item Each role requires very specific skills, mastered in considerable depth \pause
		\item In small indie studios, you might need to fill multiple roles, including business and design \pause
		\item Knowledge of effective team-working tactics is essential (though there are many ways of working)
	\end{itemize}
\end{frame}

\begin{frame}
	\frametitle{Careers for Computing Professionals}
	
	Computing professionals tend to:
	
	\begin{itemize}
		\item Deal with the technical side of games development \pause
		\item Be specialists, consultants, analysts, or technical leaders \pause
		\item Be people who are comfortable with mathematics and science \pause
		\item Keep up with the fast-paced field of computer technology \pause
		\item Have a science degree rather than an arts degree, but are nonetheless creative \pause
		\item Experts in programming and software engineering, with an ability to conduct independent research
	\end{itemize}
\end{frame}

\begin{frame}
	\frametitle{Careers for Computing Professionals}
	
	There is a wide range of technical roles in game studios:
	
	\begin{columns}
		\begin{column}{0.5\textwidth}
			\begin{itemize}
				\item Technical Director / CTO / Lead
				\item Gameplay Programmer
				\item Engine Programmer
				\item Physics Programmer
				\item AI Programmer
				\item Network Programmer
				\item Graphics Programmer
			\end{itemize}
		\end{column}
		\begin{column}{0.5\textwidth}
			\begin{itemize}
				\item Tools Programmer
				\item UX / UI Programmer
				\item Middleware / Technology Developer
				\item Porting Programmer
				\item Level Scripter
				\item Audio Engineer
				\item Data Scientist
			\end{itemize}
		\end{column}
	\end{columns}
\end{frame}

\begin{frame}
	\frametitle{What About Other Careers?}
	
	A computing degree isn't a tether. You can:  \pause
	
	\begin{itemize}
		\item 	\textbf{Design}: designers who can prototype and implement are in high demand, while the analytical and mathematical skills they apply help them to quickly improve their designs \pause
		\item 	\textbf{Illustrate or Compose}: art for digital games is indeed digital and technical artists are in high demand \pause
		\item 	\textbf{Manage}: insight into how software developers practice their craft will make you better at managing them in a studio context 
			(and perhaps even garner some respect) \pause
		\item 	\textbf{Administrate}: the games industry isn't just about development, there is a huge range of other career paths,
			such as human resources and IT
	\end{itemize}
\end{frame}

\fullbleed{t-shape}

\part{The Meta-Game}
\frame{\partpage}

\begin{frame}
	\frametitle{The Games Meta-Game}
	
	Setup:
	
	\begin{itemize}
		\item Self-organise into groups of 3-4 players
		\item You will each receive two sets of card: game cards and question cards.
		\item While you are waiting for your cards, identify the youngest player. They will be the first critic.
		\item All actions are clockwise from the critic.
	\end{itemize}
\end{frame}

\begin{frame}
	\frametitle{The Games Meta-Game}
	
	Instructions:
	
	\begin{enumerate}
		\item 	\textbf{Question}: The critic draws a question card. 
		\item 	\textbf{Answer}: The \textit{remaining players} (i.e., not the critic!) submit their best game card, to answer the question, face-up.
		\item 	\textbf{Justification}: The \textit{remaining players} justify the game card they have selected.
		\item 	\textbf{Selection}: The critic selects the most suitable game card answering the question. That player `wins' the round, keeping the question card as a scoring token
			and becomes the next critic.	
		\item 	\textbf{Repeat} from step 1, for approximately 20 minutes.	
	\end{enumerate}
\end{frame}
   
\part{Route: Computing Professional}
\frame{\partpage}

\begin{frame}
	\frametitle{Victory Parade}
	
	Within the whole university, out of 51 courses, Computing for Games is: \pause
	
	\begin{itemize}
		\item \textbf{\#1} for Assessment \& Feedback \pause
		\item \textbf{\#2} for Learning Opportunities \pause
		\item \textbf{\#2} for Learning Community \pause
		\item \textbf{\#5} for Student Satisfaction \pause
		\begin{itemize}
			\item I want the course to be \textbf{\#1} so please engage with us and your student reps! \pause
		\end{itemize}
		\item Over 80\% of the modules we offer are in the top-10\% of all modules Falmouth offers, as rated by student evaluations  \pause
		\begin{itemize}
			\item COMP250: Artificial Intelligence in top-1\% \pause
		\end{itemize}	
		\item Over 33\% contact-time \pause
		\item Great student:staff ratio (max 25 in each cohort) \pause
	\end{itemize}
\end{frame}

\begin{frame}
	\frametitle{Course Objectives}
	
	The objectives of this course are to facilitate the development of your: \pause
	
	\begin{itemize}
		\item \textbf{Technical Development Practice}: leverage professional practices and technical skills to craft creative software \pause
		\item \textbf{Communication}: communicate effectively with stakeholders in writing, verbally, and through adherence to standards and conventions in documentation \pause
		\item \textbf{Critical Evaluation}: reflect critically on, and evaluate, the quality of working methods and solutions
	\end{itemize}
\end{frame}

\begin{frame}
	\frametitle{Learning Objectives}
	
	The objectives of this course are to facilitate the development of your:
	
	\begin{itemize}
		\item \textbf{Research}: engage in activities that may create new knowledge, present that knowledge in an academic format, and apply it to practice\pause
		\item \textbf{Enterprise \& Innovation}: provide opportunities for enterprise through innovation, invention, and creativity\pause
		\item \textbf{Professionalism}: set goals, manage workloads to meet deadlines, work efficiently and effectively in teams, and accommodate change
	\end{itemize}
\end{frame}

\begin{frame}
	\frametitle{Philosophy}
	
	We are the only science degree in the entire university and do things a little differently:
	
	\begin{itemize}
		\pause\item Emphasis on developing community, and discourse/peer-review within that community
		\begin{itemize}
			\item Do it together and learn from each other, before doing it alone
			\item Critique each others' work and discuss what constitutes good practice
		\end{itemize}
		\pause\item Emphasis on feed-forward over just feed-back
		\begin{itemize}
			\item Early milestones, earlier start, more learning
			\item Get advice on how to improve your own practice \textit{before} you submit your work
		\end{itemize}
	\end{itemize}
\end{frame}

\begin{frame}
	\frametitle{Philosophy}
	
	We are the only science degree in the entire university and do things a little differently:

	\begin{itemize}
		\item Emphasis on highly structured assignments 
		\begin{itemize}
			\item Formative work across the study block
			\item Guaranteed 40\% pass for successfully completing all in-class activities with basic competence and submitting on-time
			\item Face-to-face feedback and discussion			
		\end{itemize}
		\pause\item Emphasis on continuing personal development
		\begin{itemize}
			\item Personal growth over hitting a benchmark
			\item Journey to professional competency and beyond, rather than hitting a grade
			\item Rubrics and qualitative feedback (at least, at first)			
		\end{itemize}
	\end{itemize}
\end{frame}

\part{Course Map}
\frame{\partpage}

\fullbleed{course_map_1}

\fullbleed{course_map_2}

\fullbleed{course_map_3}

\part{Study Block 1}
\frame{\partpage}

\begin{frame}
	\frametitle{\Large{COMP110: Principles of Computing}}
	
	This module is designed to introduce you to the basic principles of computing and programming in the context of digital games.
	
	\vspace{2em}
	
	Your learning will complement the other modules through providing a broad foundation on the different methods and techniques which will help you to be able to construct computer programs and able to use relevant scholarly sources. 

\end{frame}

\begin{frame}
	\frametitle{\Large{COMP120: Creative Computing --- Tinkering}}
	
	This module is designed to help you learn different ways of engaging with code using practical and exploratory methods. 
	
	\vspace{2em}
	
	You will learn the value of taking a creative approach to computing and become acquainted with some of the principles behind Creative Computing. 

\end{frame}

\begin{frame}
	\frametitle{\Large{COMP150: Game Development Practice}}
	
	This module introduces you to the founding principles and processes of professional game development. 
	
	\vspace{2em}
	
	You gain an understanding of the way that the different components of game development come together to make playable games and how those components are organised through the development pipeline. You also gain a `first-principles' understanding of how games are designed with a target market in mind and have a strong underlying concept.

\end{frame}

\part{Study Block 2}
\frame{\partpage}

\begin{frame}
	\frametitle{\Large{COMP140: Creative Computing --- Codecraft}}
	
	The module allows you to further develop your creativity and a creative approach to computing within the context of digital games technology.
	
	\vspace{2em}
	
	At first, you will complete worksheets to deepen your technical understanding and then apply that, as an individual, to your own open-source game \textit{and} game controller development project. You will collate disparate elements together, hardware and code from multiple sources, combining them via adaptation and integration. This will help you to learn ways and methods for technical exploration and synthesis in order to help you build more creative and robust programs. 

\end{frame}

\begin{frame}
	\frametitle{\Large{COMP130: Architecture \& Engineering}}
	
	This module helps you to understand the ways in which software architecture and engineering practice can shape the types of computing solutions that one might build for games.
	
	\vspace{2em}
	
	You do this by building on your experience of practical game development, engaging in depth with more sophisticated and professional approaches within the context of a collaborative multidisciplinary project. You will learn the importance of clarity, reuse, scalability, and extensibility when sharing solutions with your team, applying design patterns and quality assurance to resolve common challenges.

\end{frame}

\part{Timetable}
\frame{\partpage}

\begin{frame}
	\frametitle{Timetable}
	
	Live Demo
	
	\vspace{3em}
	
	The timetable can be found on:
	
	\vspace{0.5em}
	
	\indent \url{http://mytimetable.falmouth.ac.uk}
	
	\vspace{0.5em}
	
	Check the timetable every day! Sessions can, and often do change. Once you are allocated into groups for your collaborative game development projects, meeting times with tutors will change and extra sessions may appear!
	
	\vspace{0.5em}
	
	 This is a full-time course. Any time you are not scheduled to be with a tutor, you are expected to be working on your projects in the studio.
	
\end{frame}

\part{Assignments}
\frame{\partpage}

\begin{frame}
	\frametitle{Assignment Structure}
	
	\begin{Huge}
		\begin{center}
			\textbf{100\% Coursework}
		\end{center}
	\end{Huge}

\end{frame}

\begin{frame}
	\frametitle{Assignment Structure}
	
	
	Assessments are designed to reflect professional practice:
	
	\begin{itemize}
		\item Items for your Portfolio
		\item Collaborative Games Projects
		\item Pitches
		\item Papers
	\end{itemize}

	Relative importance of each will depend on your career trajectory

\end{frame}

\notepic{prospectus_alex_034}{Collaborative Approach with \textbf{Arts Students}}{-2.5cm}{-3.5cm}

\notepicx{studio_games_showoff}{Follows an \textbf{Incubation Model}: Make Games For Real}{2.5cm}{-3.5cm}{0.4}

\notepicx{new_studio}{\textbf{Studio-based} Course: 9-5 in the Studio Working on Games}{2.5cm}{-3.5cm}{0.45}

\fullbleed{teamphoto_basilisk}

\picturepage{nebula_knights}{\url{https://basiliskstudios.github.io/nebulaknights/trailer.mp4}}

\fullbleed{cropped-drnk_splash-with-logo-trans}

\picturepage{monq}{\url{https://www.youtube.com/embed/KzvZD5-Jmo4}}

\fullbleed{controller_1}

\fullbleed{controller_2}

\fullbleed{controller_4}

\begin{frame}
	\frametitle{Assignment Structure}
	
	Each study block, you will complete \textbf{six} assignment `tracks':
	
	\begin{itemize}
		\item Collaborative Game Development Project
		\item Academic Essay
		\item Small Programming Projects
		\item Worksheets and On-line Quizzes
		\item Research Journal
		\item Reflective Journal \& CPD Report
	\end{itemize}

\end{frame}

\begin{frame}
	\frametitle{Assignments}
	
	Live Demo
	
	\vspace{3em}
	
	All assignment briefs can be found on:
	
	\vspace{0.5em}
	
	\indent \url{learningspace.falmouth.ac.uk}
	
	\vspace{0.5em}
	
	Read them very carefully!
	
	\vspace{0.5em}
	
	This is also where you submit the final ``summative'' versions of your assigned coursework tasks!
	
\end{frame}

\begin{frame}
	\frametitle{Assignments}
	
	You will usually submit your work as a single \texttt{.zip} archive. Please use the following convention:
	
	\begin{large}
		\begin{center}
			\textbf{module\_assignmentNumber\_studentID}
		\end{center}
	\end{large}
	
	For example:
	
	\begin{Large}
		\begin{center}
			\textbf{comp110\_1\_0601210}
		\end{center}
	\end{Large}
	
	\vspace{1em}
	
	We use anonymous marking where possible.

\end{frame}

\begin{frame}
	\frametitle{Assignments}
	
	Staff are \textbf{not allowed} to put deadlines on slides, or even tell you when deadlines are---please don't ask!
	
	\vspace{2em}
	
	All assignment deadlines can be found on:
	
	\vspace{1em}
	
	\indent \url{myfalmouth.falmouth.ac.uk}
	
	\vspace{1em}
	
	Take note of these carefully! A single second late, and your work will be capped at the minimum passing grade.
	
\end{frame}

\begin{frame}
	\frametitle{Assignments}
		
	In the absence of extenuating circumstances (i.e., you are seriously ill and stuck in hospital):
	
	\begin{Large}
		\begin{center}
			\textbf{You MUST submit something \\ for EVERY assigned coursework task!}
		\end{center}
	\end{Large}
	
	In the eyes of university policy, not submitting anything is \textit{the same as withdrawing from your studies}. Even if your work is unfinished, submit it! Even submitting a blank piece of paper is better than not submitting anything! 
	
		\vspace{0.5em}
	
	If you forget to submit, there is a grace period of 5 working days after the deadline. If you fail, you get a second attempt.
	
\end{frame}


\part{Expectations in Higher Education}
\frame{\partpage}

\begin{frame}{Socrative \texttt{FALCOMPMIKE}}
	\textbf{List} THREE key differences between expectations in the higher education and compulsory education contexts.
	
	\begin{itemize}
		\item In pairs.
		\item Discuss for 2-minutes what `expectations' means. Then, discuss how they differ between higher and compulsory education.
		\item \textbf{List} the differences. Avoid overlap.
	\end{itemize}
\end{frame}

\begin{frame}
	\frametitle{Expectations}
	
	Please note the following:
	
	\begin{itemize}
		\item This is a full-time course
		\item You are expected to engage 1200-hours of study per academic year
		\item Approximately $1/3$ of that will be contact time
		\item Approximately $2/3$ of that will be `self-directed study'
		\item This means you are expected to study 40 hours per week, every week within the study block
	\end{itemize}

\end{frame}

\begin{frame}
	\frametitle{Expectation}
	
	Please note that each study block has the following structure:
	
	\begin{itemize}
		\item 5 Weeks - Sessions with Tutors
		\item 1 Week - Self-Directed Studio Practice with Team
		\begin{itemize}
			\item \textbf{NOT} a vacation
		\end{itemize}
		\item 6 Weeks - Further Sessions with Tutors
		\item Vacation Period
		\item 1 Week - Game Demos and Assessments
		\item 2 Weeks - Further Self-Directed Studio Practice with Team
		\begin{itemize}
			\item \textbf{NOT} a vacation
		\end{itemize}
	\end{itemize}
	
	But what actually `counts' as study?

\end{frame}

\begin{frame}{Socrative \texttt{FALCOMPMIKE}}
	\textbf{Give} THREE activities that count as `self-directed study'.
	
	\begin{itemize}
		\item In pairs.
		\item Discuss for 2-minutes what `self-directed study' means. Then, discuss what counts as self-directed study.
		\item \textbf{List} the differences. Avoid overlap.
	\end{itemize}
\end{frame}

\fullbleed{lighthouse}

\begin{frame}
	\frametitle{Activity: Time Management}
	
	Please complete the following activity:
	
	\vspace{2em}
	
	\url{http://www.learnhigher.ac.uk/learning-at-university/time-management/getting-organised/}

\end{frame}

\begin{frame}
	\frametitle{Questions \& Answers}	
	\begin{center}
		Thank you for listening. 
		\\~\\
		Please feel welcome to ask questions or raise concerns.
	\end{center}
\end{frame}

\end{document}
