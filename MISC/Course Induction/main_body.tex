% Adjust these for the path of the theme and its graphics, relative to this file
%\usepackage{beamerthemeFalmouthGamesAcademy}
\usepackage{../../beamerthemeFalmouthGamesAcademy}
\usepackage{multimedia}
\graphicspath{ {../../} }

% Default language for code listings
\lstset{language=C++,
        morekeywords={each,in,nullptr}
}

% For strikethrough effect
\usepackage[normalem]{ulem}
\usepackage{wasysym}
\usepackage[T1]{fontenc}
\usepackage{pdfpages}

% http://www.texample.net/tikz/examples/state-machine/
\usetikzlibrary{arrows,automata}

\newcommand{\fullbleed}[1]{
\begin{frame}[plain]
	\begin{tikzpicture}[remember picture, overlay]
		\node[at=(current page.center)] {
			\includegraphics[width=\paperwidth]{#1}
		};
	\end{tikzpicture}
\end{frame}
}

\newcommand{\picturepage}[2]{
\begin{frame}[plain]
	\begin{tikzpicture}[remember picture, overlay]
		\node[at=(current page.center)] {
			\includegraphics[width=\paperwidth]{#1}
		};
		\draw<1>[draw=none, fill=black, opacity=0.9] (-1,-5.2) rectangle (current page.south east);
		\node[draw=none,text width=0.96\paperwidth, align=right] at (5.5,-5.5) {\tiny{#2}};
	\end{tikzpicture}
\end{frame}
}

\newcommand{\notepicx}[5]{
\begin{frame}[plain]
	\begin{tikzpicture}[remember picture, overlay]
		\node[at=(current page.center)] {
			\includegraphics[width=\paperwidth]{#1}
		};
		\node[draw=none, fill=black, text width=#5\paperwidth] at ([xshift=#3, yshift=#4] current page.center) {\small{#2}};
	\end{tikzpicture}
\end{frame}
}

\newcommand{\notepic}[4]{
	\notepicx{#1}{#2}{#3}{#4}{0.4}
}

\setbeamertemplate{navigation symbols}{}

\begin{document}
\title{Induction}
\subtitle{Computing Subject Area}

\frame{\titlepage} 

\begin{frame}
	\frametitle{Computing Subject Area}
	
	Welcome!
	
	\vspace{1em}
	
	You are here because you have enrolled on one of the following courses:
	
	\vspace{0.2em}
	
	\begin{itemize}
		\item BA(Hons) Game Development: Programming
		\item BSc(Hons) Computer Science
		\item BSc(Hons) Computing for Games
		\item BSc(Hons) Data Science
		\item BSc(Hons) Immersive Computing
		\item BSc(Hons) Robotics
	\end{itemize}
	
	\vspace{1em}
	
	All of these courses have a common first-year focused on computing fundamentals and practical projects, and some have the option for a year of professional practice.

\end{frame}

\begin{frame}
	\frametitle{Computing Subject Area}
	
	The ACM define the `computing professional' as:
	
	\vspace{1em}
	
	Someone belonging to a broad discipline that crosses the boundaries between mathematics, science, engineering, and business. They embody important professional competencies lying at the foundation of goal-oriented activities requiring, benefiting from, or creating computation. Computation being any type of calculation that includes both arithmetical and non-arithmetical steps following a well-defined model, typically an algorithm.
	
	\vspace{1em}
	
	You are here because you want to become a \textbf{computing professional}.

\end{frame}

\begin{frame}
	\frametitle{Computing Subject Area}
	
	The discipline consists of five sub-disciplines:
	
	\begin{itemize}
		\item Computer Engineering
		\item Computer Science
		\item Information Systems
		\item Information Technology
		\item Software Engineering	
	\end{itemize}

	Roles such as \textit{games programmer}, \textit{web developer}, or \textit{roboticist} usually draw on several of these sub-disciplines with different emphases.

\end{frame}

\begin{frame}
	\frametitle{Learning Outcomes}
	
	By the end of this session, you should be able to:
	
	\begin{itemize}
		\item \textbf{Recognise who} your course team is
		\item \textbf{Outline what} the Games Academy offers from a computing perspective
		\item \textbf{Explain} the career paths \textbf{and} key learning objectives that our computing courses cater to
		\item \textbf{Suggest} some of the kinds of question that excite scholars within and around the computing discipline
		\item \textbf{Recall} the structure of the course
	\end{itemize}
\end{frame}

\begin{frame}
	\frametitle{Learning Outcomes}
	
	By the end of this session, you should be able to:
	
	\begin{itemize}
		\item \textbf{Contrast} what is expected of students in the higher education context to the compulsory education context
		\item \textbf{Reflect upon how} to invest sufficient time in both course activities \textbf{as well as} self-regulated deliberate practice to achieve key goals
		\item \textbf{Remember} to go to the DoIT Profiler session (\textit{next!}) to identify your individual learning differences
	\end{itemize}
\end{frame}

\part{Course Tutors}
\frame{\partpage}

\picturepage{doug}{Dr Douglas Brown, Director of the Games Academy}

\picturepage{mike_and_fudge}{Dr Michael Scott, Head of Computing \& Associate Professor of Computer Science Education}

\picturepage{brian_mcdonald}{Brian McDonald, Head of Games}

\picturepage{ed_3}{Dr Ed Powley, Associate Professor of Artificial Intelligence}

\picturepage{rory_summerley_2}{Dr Rory Summerley, Undergraduate Courses Leader}

\picturepage{andy}{Andy Smith, Technicial Facilities Manager}

\picturepage{jeff}{Dr Jeff Howard, Senior Lecturer in Games Design}

\picturepage{tg}{Terry Greer, Senior Lecturer in Games Design}

\picturepage{joe}{Dr Joseph Walton-Rivers, Lecturer in Game Programming}

\picturepage{Imran}{Dr Emre Khaliq, Lecturer in Game Programming}

\picturepage{matt-watkins}{Matt Watkins, Lecturer in Robotics \& Creative Computing}

\picturepage{john_speakman}{John Speakman, Lecturer in Computer Graphics}

\picturepage{sokol}{Sokol Murturi, Lecturer in Computer Science}

\picturepage{gareth_lewis}{Gareth Lewis, Lecturer in Indie Games \& User Experience Design (Online)}

\picturepage{warwick}{Warwick New, Associate Lecturer - Computing}

\picturepage{paul}{Paul Hedley, Associate Lecturer - Game Design \& Programming}

\picturepage{lucy}{Lucy Stent, Research Student Teaching Associate - Integrated Foundation Year}

\picturepage{akex}{Alexander Mitchell, Research Student Teaching Associate - Computing}

\picturepage{archie}{Archie Andrews, Technician (Version Control \& Programming)}

\picturepage{bengreen}{Ben Green, Technician (Hardware \& Robotics)}

\part{The Games Academy}
\frame{\partpage}

% Meta-Makers

% EU Investment - AIR

\notepic{cam01-fr0300}{\textbf{World-Leading} Research in \textbf{Digital Games}, \textbf{Creative Technology} and \textbf{Immersive Experience Design}}{2.5cm}{3cm}

\notepicx{weeva}{Awarded more than \textbf{\pounds{}7 million} in funding for research in areas such as \textbf{Artificial Intelligence}, \textbf{Transmedial Aesthetics}, \textbf{Creative Communities} in the last 7 years}{-2.5cm}{3.5cm}{0.5}

\notepicx{mixed_reality}{And hold funding for several labs for research into \textbf{Immersive Technology Applications and Esports Livestreaming}}{-2.5cm}{-3cm}{0.5}

\notepic{tanya}{Lead By \textbf{World-Renowned Researchers}}{-2.5cm}{-3.5cm}

\notepic{launchpad}{\textbf{World-Class Educational Provision} that Prepares Students for \textbf{Careers} in the \textbf{Creative Industries}}{-2.5cm}{3cm}

\fullbleed{gold}

\fullbleed{top-game-schools}

\notepic{kuba_cunity}{Undergraduate Courses in \textbf{Computing}}{-2.5cm}{-3.5cm}

\notepic{lucy_blueprints}{Undergraduate Courses in \textbf{Games}}{-2.5cm}{-3.5cm}

\notepic{immersive_tech_2}{Undergraduate Courses in \textbf{Immersive Experiences}}{-2.5cm}{-3.5cm}

\notepic{webdev}{Undergraduate Courses in \textbf{Computer Science}}{-2.5cm}{-3.5cm}

\notepic{i-m9z6d5Q-X2}{Undergraduate Courses in \textbf{Data Science*}}{-2.5cm}{-3.5cm}

\notepic{i-gPTdSFb-X3}{Undergraduate Courses in \textbf{Robotics}}{-2.5cm}{-3.5cm}

\notepic{studio_games_showoff}{Postgraduate Courses in \\\textbf{Artificial Intelligence}}{-2.5cm}{-3.5cm}

\notepic{i-WDt2BcB-X2}{Postgraduate Courses in \textbf{Game Programming*}}{-2.5cm}{-3.5cm}

\notepic{1_studio_wide}{Postgraduate Courses in \textbf{Games Entrepreneurship and Incubation}}{-2.5cm}{-3.5cm}

\notepic{mike_filming_ma}{Distance-Learning Courses in \textbf{User Experience Design} and \textbf{Indie Games}}{-2.5cm}{-3.5cm}

\notepic{doing_it_for_real}{Emphasis on \textbf{Doing It For Real}}{-2.5cm}{-3.5cm}

%\notepic{orange_helicopter}{Our Staff Are \textbf{Indie Game Developers}}{-2.5cm}{-3.5cm}

\notepic{space_caves}{Our Staff Are \textbf{Indie Game Developers}}{-2.5cm}{-3.5cm}

\notepic{warm_gun}{Our Staff Are \textbf{Indie Game Developers}}{-2.5cm}{-3.5cm}

\notepic{dwarves}{Our Staff Are \textbf{Indie Game Developers}}{-2.5cm}{-3.5cm}

%\notepic{steve}{We Attract \textbf{Industry Legends} as Visiting Lecturers}{3.5cm}{2.5cm}

%\notepic{ian}{We Attract \textbf{Industry Legends} as Visiting Lecturers}{3cm}{-2.5cm}

\notepic{sg}{We Attract \textbf{Industry Legends} as Visiting Lecturers}{3cm}{2.5cm}

\notepic{tb}{We Attract \textbf{Industry Stars} as Keynote Speakers}{3cm}{-3.5cm}

\notepic{rex}{And \textbf{Our Graduates} Return to Help Us Out}{-3cm}{-3.5cm}

\fullbleed{60a2e06021577f0b7577d29a_COMPCreatechStats_2020-Value}

%\fullbleed{technation_cornwall_2017}

\fullbleed{cornwall_technation_2}

\part{Computing in Creative Industries}
\frame{\partpage}

\begin{frame}
	\frametitle{Careers}
	
	It is important to note that:\pause
	
	\begin{itemize}
		\item Digital media are complex and therefore require significant knowledge and skills to produce \pause
		\item They bring together art, storytelling, design and computing \pause
		\item Roles are therefore quite diverse and specialised \pause
		\item Each role requires very specific skills, mastered in considerable depth \pause
		\item In small indie studios, you might need to fill multiple roles, including business and design \pause
		\item Knowledge of effective team-working tactics is essential (though there are many ways of working)
	\end{itemize}
\end{frame}

\begin{frame}
	\frametitle{Careers}
	
	Computing professionals tend to:
	
	\begin{itemize}
		\item Deal with the technical side of creative development \pause
		\item Be specialists, consultants, analysts, or technical leaders \pause
		\item Be people who are comfortable with mathematics and science \pause
		\item Keep up with the fast-paced field of computer technology \pause
		\item Straddle the arts and sciences, being able to draw together elements from both \pause
		\item Have expertise in software engineering and computer science, with an ability to conduct independent research
	\end{itemize}
\end{frame}

\fullbleed{t-shape}

\begin{frame}
	\frametitle{Other Careers?}
	
	\begin{itemize}
		\item 	\textbf{Design}: designers who can prototype and implement are in high demand, while the analytical and mathematical skills they apply help them to quickly improve their designs \pause
		\item 	\textbf{Illustrate or Compose}: art for digital games is indeed digital and technical artists are in high demand \pause
		\item 	\textbf{Manage}: insight into how software developers practice their craft will make you better at managing them in a studio context 
			(and perhaps even garner some respect) \pause
		\item 	\textbf{Administrate}: the games industry isn't just about development, there is a huge range of other career paths, such as human resources and IT
	\end{itemize}
\end{frame}

\fontsize{9pt}{7.2}\selectfont

\begin{frame}
	\frametitle{Potential Career Trajectories}
	
	This is a sampling of technical roles which our graduates have secured:
	
	\begin{columns}
		\begin{column}{0.5\textwidth}
			\begin{itemize}
				\item AI \& Systems Programmer, Nordcurrent
				\item Augmented Reality App Developer, Ndreams
				\item Back-End Developer, Codices
				\item Chief Technical Officer, Studio Mutiny
				\item Creative Software Developer, Ultrahaptics
				\item Data Management Lead, Pineapple Studios
				\item Data Scientist, Solutionpath
				\item Developer, Antoine Lock
			\end{itemize}
		\end{column}
		\begin{column}{0.5\textwidth}
			\begin{itemize}
				\item DevOps Specialist, SCC Scripting
				\item Doctoral Candidate in AI, Google
				\item Freelance Programmer, Square Enix
				\item Full Stack Web Devloper, Dewsign
				\item Game Designer, Supermassive
				\item Game Designer, Firesprite
				\item Games Programmer, FunGeneration Lab
			\end{itemize}
		\end{column}
	\end{columns}
\end{frame}

\begin{frame}
	\frametitle{Potential Career Trajectories}
	
	This is a sampling of technical roles which our graduates have secured:
	
	\begin{columns}
		\begin{column}{0.5\textwidth}
			\begin{itemize}
				\item Graduate Programmer, Ubisoft
				\item Graduate Programmer, Firesprite
				\item Hardware Engineer, BAE Systems
				\item Indie Game Developer, Knights of Borria
				\item Immersive Technologist, Facebook
				\item IT Support Administrator, Subfero
				\item Junior Game Designer, Rare
				\item Junior Programmer, Mediatonic
			\end{itemize}
		\end{column}
		\begin{column}{0.5\textwidth}
			\begin{itemize}
				\item Lead Programmer, Robot Noodle
				\item Level Designer, King
				\item Producer, Coffee Stain Studios
				\item Python Automation Engineer, Imagination Tech
				\item Software Developer, Bluefruit
				\item Software Engineer, Tempest
				\item Support Analyst for Cloud, SolicitorsOS
			\end{itemize}
		\end{column}
	\end{columns}
\end{frame}

\fontsize{11pt}{7.2}\selectfont

\fullbleed{Slide20}

\fullbleed{Slide21}

\fullbleed{Slide22}

\part{Your Course}
\frame{\partpage}

\begin{frame}
	\frametitle{Student Voice}
			
	\begin{itemize}
		\item I want the course to be \textbf{\#1} in every measure, so please engage with us!
		\item Most of the \texttt{COMP} modules we offered were in the top-10\% of all modules Falmouth offers, as rated by student evaluations 
		\begin{itemize}
			\item COMP250: Artificial Intelligence in top-1\% 
		\end{itemize}	
		\item About 25\% contact-time on all modules
	\end{itemize}
	
	\vspace{1em}
	
	You will soon be asked nominate someone to represent your interests in the student-staff liaison group. There are representatives for each cohort. 
	Establishing a working democracy is vital important to the health of your student experience. You \textit{shape} the course!
	
\end{frame}

\begin{frame}
	\frametitle{You Said, We Did}
	
	Improvements this year based on NSS data: \pause
		
	\begin{itemize}
		\item ``My course has challenged me to achieve my best work'' [-13] 
		\begin{itemize}
			\item Briefs supplemented with more open-ended ``challenges and opportunities'' and new rubrics to show how to access marks and reach higher attainment
		\end{itemize}	
		
		\pause\item ``My course has provided me with opportunities to bring information and ideas together from different topics'' [-1]
		\begin{itemize}
			\item Module leaders now coordinate topics and assignments to better highlight synergies 
		\end{itemize}	
		
	\end{itemize}
\end{frame}

\begin{frame}
	\frametitle{You Said, We Did}
			
	\begin{itemize}
		
		\item ``I have been able to contact staff when I needed to'' [-12]
		\begin{itemize}
			\item New policy on staff contact during term time 
			\item Timetabled meetings with tutors
			\item Technicians have extended studio hours
		\end{itemize}		
		
	\end{itemize}
\end{frame}

\begin{frame}
	\frametitle{You Said, We Did}
		
	\begin{itemize}
	
			\item ``The course is well organised and running smoothly'' [-2]
		\begin{itemize}
			\item The \textit{Making the Curriculum Clearer} project now implemented
			\item Simplified course structure, fewer assignments, and more sharing of modules across the Academy
			\item Now share group project modules - same learning outcomes, same assignment, same weight, same ``studio practice''
		\end{itemize}
		
	\end{itemize}
\end{frame}

\begin{frame}
	\frametitle{Programming Tutors}
	
	In study block 1, each student is allocated a tutor:
	
	\begin{itemize}
		\item Small group meetings each week with your tutor
		\item These are mandatory as they help us to nurture your progress
		\item Run by a member of the course team
		\item There to help you, only a message away
		\item Big help on COMP110 and GAM102, especially for newer programmers
	\end{itemize}
	
	\vspace{1em}
	
	We may juggle the groups once we get to know you all a bit better so we can offer the most appropriate support for you
	
\end{frame}

%\begin{frame}
%	\frametitle{PASS Sessions}
%	
%	Peer assisted study sessions:
%	
%	\begin{itemize}
%		\item To be scheduled
%		\item Run by volunteers who have been successful with the course
%		\item Awesome community
%		\item Great place to get help and support with writing/programming/maths
%	\end{itemize}
%\end{frame}

\begin{frame}
	\frametitle{Course Objectives}
	
	The aim of our courses are to:
	
	\vspace{2em}
	
	\begin{itemize}
		\item To develop confident and daring computing professionals with the knowledge, attitudes, and skills needed to operate as programmers in multidisciplinary teams that produce vibrant and innovative digital products and services.
	\end{itemize}
\end{frame}

\begin{frame}
	\frametitle{Course Objectives}
	
	By the end of this year, you should be confidently able to: 
	
	\begin{itemize}
		\item \textbf{Compute}: Translate technical notation and requirements into executable code. 
		\item \textbf{Solve}: Demonstrate computational thinking and numeracy skills.
		\item \textbf{Advocate}: Recognise legal, social, ethical, professional and sustainability issues in projects.
		\item \textbf{Research}: Report findings using appropriate evidence and conventions.
	\end{itemize}
\end{frame}

\begin{frame}
	\frametitle{Course Objectives}
	
	By the end of this year, you should be confidently able to:
	
	\begin{itemize}
		\item \textbf{Reflect}: Explain professional attributes that are relevant to your goals.
		\item \textbf{Collaborate}: Identify your individual personal responsibility in a diverse team context.
		\item \textbf{Present}: Convey information using relevant presentation techniques.
		\item \textbf{Innovate}: Outline the importance of innovation.
	\end{itemize}
\end{frame}

\begin{frame}
	\frametitle{Assessment Criteria}
	
	These learning outcomes are assessed according to these different assessment criteria categories:
	
	\begin{columns}
		\begin{column}{0.5\textwidth}
			\begin{itemize}
				\item Compute
				\item Solve
				\item Advocate
				\item Research
				\item Reflect
				\item Collaborate
				\item Present
				\item Innovate
			\end{itemize}
		\end{column}
		\begin{column}{0.5\textwidth}
			\begin{itemize}
				\item PROCESS
				\item ANALYSE
				\item INDUSTRY
				\item RESEARCH
				\item ORGANISATION
				\item COLLABORATION
				\item COMMUNICATION
				\item INNOVATION
			\end{itemize}
		\end{column}
	\end{columns}
\end{frame}

\begin{frame}
	\frametitle{Philosophy}
	
	We offer the only science degrees in the Game Academy and do things a little differently:
	
	\begin{itemize}
		\pause\item Emphasis on developing a community of practice that motivates ongoing discourse and peer-review between its members
		\begin{itemize}
			\item Doing hands-on practice
			\item Learning from each other
			\item Critique each others' work and discuss what constitutes good practice
		\end{itemize}
		\pause\item Feed-forward over feed-back:
		\begin{itemize}
			\item Early milestones, earlier start, more learning
			\item Get advice on how to improve your own practice \textit{before} you submit your work
		\end{itemize}
	\end{itemize}
\end{frame}

\begin{frame}
	\frametitle{Philosophy}
	
	\begin{itemize}
		\item Emphasis on structure:
		\begin{itemize}
			\item Formative work across the study block
			\item Straightforward to pass, challenging to master
			\item Face-to-face feedback and discussion in assessment by viva			
		\end{itemize}
		\pause\item Emphasis on continuing personal and professional development:
		\begin{itemize}
			\item Personal growth over hitting a benchmark
			\item Competencies over grades
			\item Qualitative over quantitative			
		\end{itemize}
	\end{itemize}
\end{frame}

\part{Course Maps}
\frame{\partpage}

\fullbleed{Slide9}

\fullbleed{Slide10}

\fullbleed{Slide11}

\fullbleed{Slide12}

\fullbleed{Slide13}

\fullbleed{Slide14}

\fullbleed{Slide15}

\fullbleed{Slide16}

%\fullbleed{webdev-s3}

\part{Study Block One}
\frame{\partpage}

\begin{frame}
	\frametitle{Modules}
	
	Everyone does the same three modules in study block one. These are:
	
	\vspace{0.5em}
	
	\begin{itemize}
		\item COMP101 Principles of Computing
		\item GAM101 Development Principles
		\item GAM102 Digital Creativity

	\end{itemize}
	
	\vspace{1em}
	
	Numeracy is part of several modules in this first stage of the course, but the \bextbf{BA(Hons)} course will \bextbf{not} continue this into the second stage. We only recommend making the switch if you do \textbf{not} feel comfortable with mathematics by the end of the year.
	
\end{frame}

\begin{frame}
	\frametitle{Modules}
	
	There are more detailed module introductions, module welcome talks, module induction talks, and assignment briefs available for you to review on the LearningSpace.
	
	\vspace{0.5em}
	
	These should be available to you next week, if they aren't available already.
	
	\vspace{0.5em}
	
	We will briefly introduce these modules now, but you will need to review further details on LearningSpace.
	
\end{frame}

\begin{frame}
	\frametitle{COMP101 Principles of Computing}
		
	\small{\textbf{Aim:} To help you solve practical problems using basic computing and mathematical theory.}
		\vspace{0.5em}
	
	\small{\textbf{Module Leader:} Associate Professor Ed Powley}	
	
		\vspace{0.5em}
	
\footnotesize{On this module, you will learn the principles of computing, discrete mathematics, statistics, and technical communication (e.g., notation, pseudocode, unified modelling language, etc.). You begin to use core concepts and methods from computer science to solve practical problems and leverage algorithms in your solutions. Particular attention will be drawn to the history of computing, referencing the plurality of voices in the profession and the controversies evoked by algorithmic bias. A series of worksheet tasks will acquaint you with the techniques and methods in a practical way, enabling you to responsibly design, build, and annotate computing solutions.}
		\vspace{0.5em}
	
	\small{\textbf{Assignment:} Worksheet Tasks}	
\end{frame}

\begin{frame}
	\frametitle{GAM101 Development Foundations}
		
	\small{\textbf{Aim:} To get you to practice the foundational collaborative skills required for the successful delivery of digital products and services.}
	
	\vspace{0.5em}
	
	\small{\textbf{Module Leader:} Brian McDonald}
	
	\vspace{0.5em}
	
\footnotesize{On this module, you will gain foundational experience of developing digital products and services in teams. You attain this practically through several small-scale projects following various prototyping and pitching methods. You will work to develop a studio culture that strives to uphold professional values such as integrity, inclusivity, respect, and generosity. You will also apply Agile management techniques to facilitate a healthy approach to scoping and time management to promote positive teamwork. All the while, reflective exercises will help you recognise key professional attributes.}

			\vspace{0.5em}
	
	\small{\textbf{Assignment:} Development Projects with Pitches}	
\end{frame}

\begin{frame}
	\frametitle{GAM102 Digital Creativity}
		
	\small{\textbf{Aim:} To get you feeling more comfortable using digital tools and techniques in creative contexts.}
	
	\vspace{0.5em}
	
	\small{\textbf{Module Leader:} Associate Professor Michael Scott}	
	
		\vspace{0.5em}
	
\footnotesize{On this module, you will learn different ways of engaging with digital creativity through a practical exploration of digital media formats including text, image, and sound. You will play, tinker, experiment with, and extend digital artefacts. You will then integrate your digital artefacts with digital game technologies, notably game engines, to make them interactive in some way. In doing so, you will embrace the principles of rapid iteration and of how to use versioning systems. However, appropriating third-party materials raises moral and legal questions that you will consider and frame within topics such as plagiarism, intellectual property law, licensing rights, representation and media literacy, as well as the maker and open-source movements. }
			\vspace{0.5em}
	
	\small{\textbf{Assignment:} In-Engine Diorama}		
\end{frame}

\part{Timetable}
\frame{\partpage}

\begin{frame}
	\frametitle{Timetable}
	
	The timetable can be found on:
	
	\vspace{0.5em}
	
	\indent \url{http://mytimetable.falmouth.ac.uk}
	
	\vspace{0.5em}
	
	Check the timetable every day! Sessions can, and often do change. Once you are allocated into groups for your collaborative game development projects, meeting times with tutors will change and extra sessions may appear!
	
	\vspace{0.5em}
	
	 The course isn't just the time you're scheduled to be with a tutor, you are expected to engage in self-directed study.
	
\end{frame}

\begin{frame}
	\frametitle{Blended Learning}
		
	Many areas of our provision have improved due to online delivery methods. These include:
	
	\vspace{0.5em}
	
	\begin{itemize}
		\item Tutor meetings in GAM102
		\item Mathematics lectures and support in COMP201
		\item R\&D support and dissertation supervision in COMP302 and COMP303
	\end{itemize}
	
	\vspace{0.5em}
	
	Since module ratings improved year-on-year for these modules, we will continue to use and enhance online delivery methods where they make sense and where they assure continuity in the student journey and a high quality of provision.
	
\end{frame}

\begin{frame}
	\frametitle{Blended Learning}
		
	Many areas of our provision benefit from traditional delivery methods. These include:
	
	\vspace{0.5em}
	
	\begin{itemize}
		\item Workshops in COMP101 and GAM102
		\item Studio practice in GAM190 and other group project modules
		\item Using specialist requipment in robotics, immersive computing, etc.
	\end{itemize}
	
	\vspace{0.5em}
	
	These are studio-based courses and you are expected to convene with members of your team in-person in the studio as timetabled.
	
\end{frame}

\part{Assignments}
\frame{\partpage}

\begin{frame}
	\frametitle{Assignment Structure}
	
	\begin{Huge}
		\begin{center}
			\textbf{100\% Coursework}
		\end{center}
	\end{Huge}

\end{frame}

\begin{frame}
	\frametitle{Assignment Structure}
	
	
	Assessments are designed to reflect professional practice:
	
	\begin{itemize}
		\item Items for your Portfolio
		\item Collaborative Projects
		\item Pitches
		\item Papers
	\end{itemize}

	Relative importance of each will depend on your career trajectory

\end{frame}

\notepic{prospectus_alex_034}{Collaborative Approach with \textbf{Arts Students}}{-2.5cm}{-3.5cm}

\notepicx{lucy_blueprints}{Follows an \textbf{Incubation Model}: Make It For Real}{2.5cm}{-3.5cm}{0.4}

\notepicx{new_studio}{Access to Excellent \\\textbf{Studio Facilities}}{2.5cm}{-3.5cm}{0.33}

\notepicx{expo2}{\textbf{Industry Involvement}: Show-off your work to professionals at our expo}{-2.5cm}{-3.5cm}{0.45}

\picturepage{ppc}{\url{https://dangerzone-ga.itch.io/ppc}}

% add SAI

% add rustbreaker

%\fullbleed{cropped-drnk_splash-with-logo-trans}

%\picturepage{monq}{\url{https://www.youtube.com/embed/KzvZD5-Jmo4}}

%\fullbleed{controller_1}

%\fullbleed{controller_2}

%\fullbleed{controller_4}

\begin{frame}
	\frametitle{Assignments}
	
	Live Demo
	
	\vspace{3em}
	
	All assignment briefs will be found on:
	
	\vspace{0.5em}
	
	\indent \url{learningspace.falmouth.ac.uk}
	
	\vspace{0.5em}
	
	Enjoy freshers week now --- read them very carefully next week!
	
	\vspace{0.5em}
	
	LearningSpace is also where you submit \textbf{ALL} final ``summative'' versions of your assigned coursework tasks! We use \texttt{git} repositories to manage the large size of digital projects.
	
\end{frame}

\begin{frame}
	\frametitle{Assignments}
	
	You will usually submit your work as:
	
	\begin{itemize}
	    \item a link to your \texttt{git} repository
	    \item \textit{or} a single \texttt{.pdf} file
	\end{itemize}
	
	\vspace{1em}
	
	 Please use the following convention:
	
	\begin{large}
		\begin{center}
			\textbf{module-assignmentNumber-studentID}
		\end{center}
	\end{large}
	
	For example:
	
	\begin{Large}
		\begin{center}
			\textbf{comp101-1-2011213}
		\end{center}
	\end{Large}
	
	\vspace{1em}
	
	We use anonymous marking where possible.

\end{frame}

\begin{frame}
	\frametitle{Assignments}
		
	All assignment deadlines can be found on:
	
	\vspace{1em}
	
	\indent \url{myfalmouth.falmouth.ac.uk}
	
	\vspace{1em}
	
	Take note of these carefully! A single second late, and your work will be capped at the minimum passing grade.
	
\end{frame}

\begin{frame}
	\frametitle{Assignments}
		
	In the absence of extenuating circumstances (i.e., you are seriously ill and stuck in hospital):
	
	\begin{Large}
		\begin{center}
			\textbf{You MUST submit something \\ for EVERY assigned coursework task!}
		\end{center}
	\end{Large}
	
	In the eyes of university policy, not submitting anything is usually unrecoverable. Even if your work is unfinished, please submit something! Even submitting a blank piece of paper is better than not submitting anything! 
	
		\vspace{0.5em}
	
	If you forget to submit, there is a grace period of 5 working days after the deadline. There is an EC policy which you can use to remove late submission penalties. 
	
\end{frame}

\begin{frame}
	\frametitle{Extenuating Circumstances}
		
	There is an extenuating circumstances policy which can be used to grant long or short extentions:
	
	\vspace{1em}
	
	\url{https://www.falmouth.ac.uk/sites/default/files/media/downloads/Extenuating\%20Circumstances\%20Policy\%20from\%2019\%20September\%202022_0.pdf}
	
	\vspace{1em}
	
	Academic staff do not have any say over whether ECs are granted or refused! These are determined by an impartial team based on the evidence you submit. 
	
		\vspace{1em}
	
	Make contact with a student advisor if you need help with ECs. 
	
\end{frame}

\begin{frame}
	\frametitle{Retrieval}
		
	If you fail an assignment, you get a second attempt. And, usually, a third attempt.
	
		\vspace{1em}
	
	The second attempt is usually a \textbf{synoptic} assessment that takes place over the summer months, and will usually be a new assignment that is different to the originally set brief.
	
		\vspace{1em}
	
	Third attempts are discretionary, and are usually `trailed' into the next stage of study.
	
\end{frame}

\part{Expectations in Higher Education}
\frame{\partpage}

\begin{frame}{Exercise}

    Go to:
    
    \vspace{0.5em}
    
    \small{\url{https://padlet.com/michaelscott5/xjdz7hngsnvyx35z}}
    
    \vspace{1em}
    
    Let's discuss what `expectations' means, with particular focus on how they differ between higher and compulsory education.
    
    \vspace{0.5em}

	\begin{itemize}
	    \item \textbf{List} key differences between expectations in the higher education and compulsory education contexts;
		\item \textbf{Suggest} what will be expected of you during your time on the course;
		\item 	\textbf{Give} examples of activities that count as `self-directed study'.
	\end{itemize}

\end{frame}

\begin{frame}
	\frametitle{Expectations}
	
	Please note the following:
	
	\vspace{0.5em}
	
	\begin{itemize}
		\item This is a full-time course
		\item You are expected to do 1200 hours of study per academic year
		\item Approximately $1/3$ of that will be contact time
		\item Approximately $2/3$ of that will be `self-directed study'
		\item This is a full-time course---you are expected to study \textbf{40 hours per week}, \textbf{EACH} week, for \textbf{15 weeks} across \textbf{EACH} study block
		\item By virtue of enrolling you have made a committment to make this time available to study---your engagement with your studies is monitored
	\end{itemize}

\end{frame}

\begin{frame}
	\frametitle{Expectation}
	
	Typically, this coming study block has the following structure:
	
	\begin{itemize}
		\item 1 Week - Induction
		\item 5 Weeks - Study Weeks
		\item 1 Week - Reading Week
		\begin{itemize}
			\item \textbf{NOT} a vacation
		\end{itemize}
		\item 6 Weeks - Study Weeks
		\item 3 Weeks - Winter Vacation
		\item 3 Weeks - Study Weeks
		\begin{itemize}
			\item \textbf{NOT} a vacation
		\end{itemize}
	\end{itemize}
	
	Term dates: \url{https://www.falmouth.ac.uk/experience/term-dates/term-dates-2022-23}
	
\end{frame}

\fullbleed{climbing-a-mountain}

\begin{frame}
	\frametitle{Staff Support}
	
	You will receive support from staff during your timetabled sessions. However, if you require further assistance, for programming contact your programming tutor in the first instance, and for other academic queries, contact the relevant module leader. However, please be aware that educating you is only one part of our roles. We also do: 
	
	\begin{itemize}
		\item Research
		\item Scholarship
		\item Knowledge-Exchange
		\item Public Service and Outreach
		\item Consultation
		\item Pastoral Care
		\item Administration
		\item External examining
	\end{itemize}	
	
	Arrange ad-hoc support via email or Teams. We are not an on-demand service, but are happy to host you at our mutual convenience.
	
\end{frame}

\begin{frame}
	\frametitle{Later: Robot Olympics}
	
	Later this week we will hold the Programming Olympiad next week --- they're fun low-stakes activities.
	
	\vspace{1em}
	
	 Look at COMP101 for preparation. 
	
	\vspace{1em}
	
	Your programming tutor will introduce you to your team at the start of the event. Please be mutually respectful  --- you will be meeting with these peers regularly across the study block.
\end{frame}

\begin{frame}
	\frametitle{Next: DoIT Profiler}
	
	There are many individual learning differences and neurodiversity in our community. You may familiarise yourself with these here:
	
	\vspace{1em}
	
	\url{https://studyhub.fxplus.ac.uk/accessibility-inclusion/free}
	
	\vspace{1em}
	
	In the next session, you'll complete the following activity to explore some of your individual learning differences:
	
	\vspace{2em}
	
	\url{https://doitprofiler.net/Account/ClientLogin}
	
	\vspace{2em}
	
	Client code: \texttt{fal15mar}

\end{frame}

\begin{frame}
	\frametitle{Questions \& Answers}	
	\begin{center}
		Thank you for listening. 
		\\~\\
		Please feel welcome to ask questions or raise concerns.
	\end{center}
\end{frame}

%\part{Breakout Groups}
%\frame{\partpage}


%\begin{frame}
%	\frametitle{Icebreaker: SpaceTeam}
	
%	A cooperative shouting game for piloting a spaceship!
	
%	Setup:
	
%	\begin{itemize}
%		\item Download \url{https://spaceteam.ca/}
%		\item If you don't have an Apple or Android mobile phone, use an emulator on a personal device (e.g., BlueStacks)
%		\item Share the room code with the people near you to play together online!
%	\end{itemize}
	
%\end{frame}

%\begin{frame}
%	\frametitle{Icebreaker: Games Meta-Game}
	
%	Setup:
	
%	\begin{itemize}
%		\item Organise into your tutor groups of 4-6 players
%		\item You will each receive two sets of card: game cards and question cards.
%		\item While you are waiting for your cards, identify the youngest player. They will be the first critic.
%		\item All actions are clockwise from the critic.
%	\end{itemize}
%\end{frame}

%\begin{frame}
%	\frametitle{Icebreaker: Games Meta-Game}
	
%	Instructions:
	
%	\begin{enumerate}
%		\item 	\textbf{Question}: The critic draws a question card. 
%		\item 	\textbf{Answer}: The \textit{remaining players} (i.e., not the critic!) submit their best game card, to answer the question, face-up.
%		\item 	\textbf{Justification}: The \textit{remaining players} justify the game card they have selected.
%		\item 	\textbf{Selection}: The critic selects the most suitable game card answering the question. That player `wins' the round, keeping the question card as a scoring token and becomes the next critic.	
%		\item 	\textbf{Repeat} from step 1, for approximately 20 minutes.	
%	\end{enumerate}
%\end{frame}

%\begin{frame}
%	\frametitle{Activity: Time Management}
	
%	Please complete the following activity:
	
%	\vspace{2em}
	
%	\url{http://www.learnhigher.ac.uk/learning-at-university/time-management/getting-organised/}

%\end{frame}



\end{document}
