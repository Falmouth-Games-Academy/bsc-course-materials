% Adjust these for the path of the theme and its graphics, relative to this file
%\usepackage{beamerthemeFalmouthGamesAcademy}
\usepackage{../../beamerthemeFalmouthGamesAcademy}
\usepackage{multimedia}
\graphicspath{ {../../} }

% Default language for code listings
\lstset{language=C++,
        morekeywords={each,in,nullptr}
}

% For strikethrough effect
\usepackage[normalem]{ulem}
\usepackage{wasysym}
\usepackage[T1]{fontenc}
\usepackage{pdfpages}

% http://www.texample.net/tikz/examples/state-machine/
\usetikzlibrary{arrows,automata}

\newcommand{\fullbleed}[1]{
\begin{frame}[plain]
	\begin{tikzpicture}[remember picture, overlay]
		\node[at=(current page.center)] {
			\includegraphics[width=\paperwidth]{#1}
		};
	\end{tikzpicture}
\end{frame}
}

\newcommand{\picturepage}[2]{
\begin{frame}[plain]
	\begin{tikzpicture}[remember picture, overlay]
		\node[at=(current page.center)] {
			\includegraphics[width=\paperwidth]{#1}
		};
		\draw<1>[draw=none, fill=black, opacity=0.9] (-1,-5.2) rectangle (current page.south east);
		\node[draw=none,text width=0.96\paperwidth, align=right] at (5.5,-5.5) {\tiny{#2}};
	\end{tikzpicture}
\end{frame}
}

\newcommand{\notepicx}[5]{
\begin{frame}[plain]
	\begin{tikzpicture}[remember picture, overlay]
		\node[at=(current page.center)] {
			\includegraphics[width=\paperwidth]{#1}
		};
		\node[draw=none, fill=black, text width=#5\paperwidth] at ([xshift=#3, yshift=#4] current page.center) {\small{#2}};
	\end{tikzpicture}
\end{frame}
}

\newcommand{\notepic}[4]{
	\notepicx{#1}{#2}{#3}{#4}{0.4}
}

\setbeamertemplate{navigation symbols}{}

\begin{document}
\title{Induction}
\subtitle{Computing Subject Area}

\frame{\titlepage} 

\begin{frame}
	\frametitle{Computing Subject Area}
	
	Welcome!
	
	\vspace{1em}
	
	You are here because you have enrolled on one of the following courses:
	
	\vspace{0.2em}
	
	\begin{itemize}
		\item BA(Hons) Game Development: Programming
		\item BSc(Hons) Computing for Games
		\item BSc(Hons) Immersive Computing
		\item BSc(Hons) Computer Science
		\item BSc(Hons) Robotics
	\end{itemize}
	
	\vspace{1em}
	
	All of these courses have a common first-year focused on computing fundamentals and practical projects, and some have the option for a year of professional practice.

\end{frame}

\begin{frame}
	\frametitle{Computing Subject Area}
	
	The ACM define the `computing professional' as:
	
	\vspace{1em}
	
	Someone belonging to a broad discipline that crosses the boundaries between mathematics, science, engineering, and business. They embody important professional competencies lying at the foundation of goal-oriented activities requiring, benefiting from, or creating computation. Computation being any type of calculation that includes both arithmetical and non-arithmetical steps following a	well-defined model, typically an algorithm.
	
	\vspace{1em}
	
	You are here because you want to become a \textbf{computing professional}.

\end{frame}

\begin{frame}
	\frametitle{Computing Subject Area}
	
	The discipline consists of five sub-disciplines:
	
	\begin{itemize}
		\item Computer Engineering
		\item Computer Science
		\item Information Systems
		\item Information Technology
		\item Software Engineering	
	\end{itemize}

	Roles such as \textit{games programmer} and \textit{web developer} usually draw on several of these sub-disciplines with different emphases.

\end{frame}

\begin{frame}
	\frametitle{Learning Outcomes}
	
	By the end of this session, you should be able to:
	
	\begin{itemize}
		\item \textbf{Recognise who} your course team is
		\item \textbf{Outline what} the Games Academy offers from a computing perspective
		\item \textbf{Explain} the career paths \textbf{and} key learning objectives that our computing courses cater to
		\item \textbf{Suggest} some of the kinds of question that excite scholars within and around the computing discipline
		\item \textbf{Recall} the structure of the course
	\end{itemize}
\end{frame}

\begin{frame}
	\frametitle{Learning Outcomes}
	
	By the end of this session, you should be able to:
	
	\begin{itemize}
		\item \textbf{Contrast} what is expected of students in the higher education context to the compulsory education context
		\item \textbf{Analyse how} to invest sufficient time in both course activities \textbf{as well as} self-regulated deliberate practice to achieve key goals
		\item \textbf{Recall} the role of the DoIT Profiler in identifying individual learning differences
	\end{itemize}
\end{frame}

\part{Course Tutors}
\frame{\partpage}

%\picturepage{doug}{Douglas Brown, Director of the Games Academy}

\picturepage{doug}{Dr Douglas Brown, Director of the Games Academy}

\picturepage{mike_and_fudge}{Dr Michael Scott, Head of Computing}

\picturepage{brian_mcdonald}{Brian McDonald, Head of Games}

\picturepage{ed_3}{Dr Ed Powley, Associate Professor of Artificial Intelligence}

\picturepage{ed_3}{Dr Rory Summerley, Undergraduate Courses Leader} % need new pic

\picturepage{ed_3}{Andy Smith, Technicial Facilities Manager} % need new pic

%\picturepage{al_parker_2}{Alcwyn Parker, Course Leader for Postgraduate Courses}

%\picturepage{umaima}{Dr Umaima Haider, Lecturer}

\picturepage{joe}{Dr Jeff Howard, Senior Lecturer in Games Design} % need new pic

\picturepage{joe}{Terry Greer, Senior Lecturer in Games Design} % need new pic

\picturepage{joe}{Dr Rogerio Silva, Lecturer in Computer Graphics} % need new pic

\picturepage{joe}{Joseph Walton-Rivers, Lecturer in Game Programming}

\picturepage{joe}{Sokol Murturi, Lecturer in Computer Science}

%\picturepage{kate_without_cat}{Kate Bergel, Lecturer}

\picturepage{matt-watkins}{Matt Watkins, Lecturer in Robotics \& Creative Computing}

\picturepage{paul}{Warwick New, Associate Lecturer - Computing} % need new pic

\picturepage{paul}{Paul Hedley, Associate Lecturer - Game Design \& Programming}

\picturepage{john_speakman}{John Speakman, Research Student Teaching Associate - Computing}

\picturepage{lucy}{Lucy Stent, Research Student Teaching Associate - Computing}

\picturepage{akex}{Alexander Mitchell, Research Student Teaching Associate - Computing}

\picturepage{archie}{Archie Andrews, Technician (Version Control \& Programming)}

\part{The Games Academy}
\frame{\partpage}

% Meta-Makers

% EU Investment - AIR

\notepic{cam01-fr0300}{\textbf{World-Leading} Research in \textbf{Digital Games}, \textbf{Creative Technology} and \textbf{Immersive Experience Design}}{2.5cm}{3cm}

\notepicx{weeva}{Awarded more than \textbf{\pounds{}7 million} in funding for research in areas such as \textbf{Artificial Intelligence}, \textbf{Transmedial Aesthetics}, \textbf{Creative Communities} in the last 7 years}{-2.5cm}{3.5cm}{0.5}

\notepicx{mixed_reality}{And hold funding for several labs for research into \textbf{Immersive Technology Applications}}{-2.5cm}{-3cm}{0.5}

\notepic{tanya}{Lead By \textbf{World-Renowned Researchers}}{-2.5cm}{-3.5cm}

\notepic{launchpad}{\textbf{World-Class Educational Provision} that Prepares Students for \textbf{Careers} in the \textbf{Creative Industries}}{-2.5cm}{3cm}

\fullbleed{gold}

\fullbleed{top-game-schools}

\notepic{kuba_cunity}{Undergraduate Courses in \textbf{Computing}}{-2.5cm}{-3.5cm}

\notepic{lucy_blueprints}{Undergraduate Courses in \textbf{Games}}{-2.5cm}{-3.5cm}

\notepic{immersive_tech_2}{Undergraduate Courses in \textbf{Immersive Computing}}{-2.5cm}{-3.5cm}

\notepic{webdev}{Undergraduate Courses in \textbf{Computer Science}}{-2.5cm}{-3.5cm}

\notepic{webdev}{Undergraduate Courses in \textbf{Robotics}}{-2.5cm}{-3.5cm} % replace with lego olympiad

\notepic{studio_games_showoff}{Postgraduate Courses in \textbf{Artificial Intelligence}}{-2.5cm}{-3.5cm}

\notepic{1_studio_wide}{Postgraduate Courses in \textbf{Games Entrepreneurship and Incubation}}{-2.5cm}{-3.5cm}

\notepic{mike_filming_ma}{Distance-Learning Courses in \textbf{User Experience Design} and \textbf{Indie Games}}{-2.5cm}{-3.5cm}

\notepic{doing_it_for_real}{Emphasis on \textbf{Doing It For Real}}{-2.5cm}{-3.5cm}

%\notepic{orange_helicopter}{Our Staff Are \textbf{Indie Game Developers}}{-2.5cm}{-3.5cm}

\notepic{space_caves}{Our Staff Are \textbf{Indie Game Developers}}{-2.5cm}{-3.5cm}

\notepic{warm_gun}{Our Staff Are \textbf{Indie Game Developers}}{-2.5cm}{-3.5cm}

\notepic{dwarves}{Our Staff Are \textbf{Indie Game Developers}}{-2.5cm}{-3.5cm}

%\notepic{steve}{We Attract \textbf{Industry Legends} as Visiting Lecturers}{3.5cm}{2.5cm}

%\notepic{ian}{We Attract \textbf{Industry Legends} as Visiting Lecturers}{3cm}{-2.5cm}

\notepic{sg}{We Attract \textbf{Industry Legends} as Visiting Lecturers}{3cm}{2.5cm}

% investigate troy baker image

\notepic{rex}{And \textbf{Our Graduates} Return to Help Us Out}{-3cm}{-3.5cm}

\fullbleed{60a2e06021577f0b7577d29a_COMPCreatechStats_2020-Value}

\fullbleed{technation_cornwall_2017}

\fullbleed{cornwall_technation_2}

\part{Computing in Creative Industries}
\frame{\partpage}

\begin{frame}
	\frametitle{Careers for Computing Professionals}
	
	It is important to note that:\pause
	
	\begin{itemize}
		\item Digital media are complex and therefore require significant knowledge and skills to produce \pause
		\item They bring together art, storytelling, design and computing \pause
		\item Roles are therefore quite diverse and specialised \pause
		\item Each role requires very specific skills, mastered in considerable depth \pause
		\item In small indie studios, you might need to fill multiple roles, including business and design \pause
		\item Knowledge of effective team-working tactics is essential (though there are many ways of working)
	\end{itemize}
\end{frame}

\begin{frame}
	\frametitle{Careers for Computing Professionals}
	
	Computing professionals tend to:
	
	\begin{itemize}
		\item Deal with the technical side of creative development \pause
		\item Be specialists, consultants, analysts, or technical leaders \pause
		\item Be people who are comfortable with mathematics and science \pause
		\item Keep up with the fast-paced field of computer technology \pause
		\item Straddle the arts and sciences, being able to draw together elements from both \pause
		\item Have expertise in software engineering and computer science, with an ability to conduct independent research
	\end{itemize}
\end{frame}

\begin{frame}
	\frametitle{What About Other Careers?}
	
	\begin{itemize}
		\item 	\textbf{Design}: designers who can prototype and implement are in high demand, while the analytical and mathematical skills they apply help them to quickly improve their designs \pause
		\item 	\textbf{Illustrate or Compose}: art for digital games is indeed digital and technical artists are in high demand \pause
		\item 	\textbf{Manage}: insight into how software developers practice their craft will make you better at managing them in a studio context 
			(and perhaps even garner some respect) \pause
		\item 	\textbf{Administrate}: the games industry isn't just about development, there is a huge range of other career paths,
			such as human resources and IT
	\end{itemize}
\end{frame}

\fontsize{9pt}{7.2}\selectfont

\begin{frame}
	\frametitle{Potential Career Trajectories}
	
	This is a sampling of technical roles which our graduates have secured:
	
	\begin{columns}
		\begin{column}{0.5\textwidth}
			\begin{itemize}
				\item AI \& Systems Programmer, Nordcurrent
				\item Augmented Reality App Developer, Ndreams
				\item Back-End Developer, Codices
				\item Chief Technical Officer, Studio Mutiny
				\item Creative Software Developer, Ultrahaptics
				\item Data Management Lead, Pineapple Studios
				\item Data Scientist, Solutionpath
				\item Developer, Antoine Lock
			\end{itemize}
		\end{column}
		\begin{column}{0.5\textwidth}
			\begin{itemize}
				\item DevOps Specialist, SCC Scripting
				\item Doctoral Candidate in AI, Google
				\item Freelance Programmer, Square Enix
				\item Full Stack Web Devloper, Dewsign
				\item Game Designer, Supermassive
				\item Game Designer, Firesprite
				\item Games Programmer, FunGeneration Lab
			\end{itemize}
		\end{column}
	\end{columns}
\end{frame}

\begin{frame}
	\frametitle{Potential Career Trajectories}
	
	This is a sampling of technical roles which our graduates have secured:
	
	\begin{columns}
		\begin{column}{0.5\textwidth}
			\begin{itemize}
				\item Graduate Programmer, Ubisoft
				\item Graduate Programmer, Firesprite
				\item Hardware Engineer, BAE Systems
				\item Indie Game Developer, Knights of Borria
				\item Immersive Technologist, Facebook
				\item IT Support Administrator, Subfero
				\item Junior Game Designer, Rare
				\item Junior Programmer, Mediatonic
			\end{itemize}
		\end{column}
		\begin{column}{0.5\textwidth}
			\begin{itemize}
				\item Lead Programmer, Robot Noodle
				\item Level Designer, King
				\item Producer, Coffee Stain Studios
				\item Python Automation Engineer, Imagination Tech
				\item Software Developer, Bluefruit
				\item Software Engineer, Tempest
				\item Support Analyst for Cloud, SolicitorsOS
			\end{itemize}
		\end{column}
	\end{columns}
\end{frame}

\fontsize{11pt}{7.2}\selectfont

\fullbleed{t-shape}

\part{Your Course}
\frame{\partpage}

\begin{frame}
	\frametitle{Student Voice}
			
	\begin{itemize}
		\item I want the course to be \textbf{\#1} in every measure, so please engage with us!
		\item Pre-COVID over 80\% of the \texttt{COMP} modules we offer are in the top-10\% of all modules Falmouth offers, as rated by student evaluations 
		\begin{itemize}
			\item COMP250: Artificial Intelligence in top-1\% 
		\end{itemize}	
		\item About 33\% contact-time on all modules
	\end{itemize}
	
	\vspace{1em}
	
	You will soon be asked nominate someone to represent your interests in the student-staff liaison group. There are representatives for each cohort. 
	Establishing a working democracy is vital important to the health of your student experience. You \textit{shape} the course!
	
\end{frame}

\begin{frame}
	\frametitle{You Said, We Did}
	
	Improvements this year based on NSS data: \pause
		
	\begin{itemize}
		\item ``My course has challenged me to achieve my best work'' [-13] 
		\begin{itemize}
			\item Briefs supplemented with more open-ended ``contracts'' and new rubrics to show how to access marks and reach higher attainment
		\end{itemize}	
		
		\pause\item ``My course has provided me with opportunities to bring information and ideas together from different topics'' [-1]
		\begin{itemize}
			\item Module leaders now coordinate topics and assignments to better highlight synergies 
		\end{itemize}	
		
	\end{itemize}
\end{frame}

\begin{frame}
	\frametitle{You Said, We Did}
			
	\begin{itemize}
		
		\item ``I have been able to contact staff when I needed to'' [-12]
		\begin{itemize}
			\item Forthcoming policy to respond to email and Teams messages within three working days during term time 
			\item Studio screens now show who is on-duty for studio supervision
			\item Technicians have extended studio hours
		\end{itemize}		
		
	\end{itemize}
\end{frame}

\begin{frame}
	\frametitle{You Said, We Did}
		
	\begin{itemize}
	
			\item ``The course is well organised and running smoothly'' [-2]
		\begin{itemize}
			\item The \textit{Making the Curriculum Clearer} project now implemented
			\item Simplified course structure, fewer assignments, and more sharing of modules across the Academy
			\item Now share group project modules - same learning outcomes, same assignment, same weight, same ``studio practice''
		\end{itemize}
		
	\end{itemize}
\end{frame}

\begin{frame}
	\frametitle{Programming Tutors}
	
	In study block 1, each student is allocated a tutor:
	
	\begin{itemize}
		\item Small group meetings each week for each COMP module
		\item These are mandatory as they help us to nurture your progress
		\item Run by a member of the course team
		\item There to help you, only a message away
		\item Big help on COMP110 and COMP120, especially for newer programmers
	\end{itemize}
	
	\vspace{1em}
	
	We may juggle the groups once we get to know you all a bit better so we can offer the most appropriate support for you
	
\end{frame}

\begin{frame}
	\frametitle{PASS Sessions}
	
	Peer assisted study sessions:
	
	\begin{itemize}
		\item To be scheduled
		\item Run by volunteers who have been successful with the course
		\item Awesome community
		\item Great place to get help and support with writing/programming/maths
	\end{itemize}
\end{frame}

\begin{frame}
	\frametitle{Course Objectives}
	
	The aim of our courses are to:
	
	\vspace{2em}
	
	\begin{itemize}
		\item To develop confident and daring computing professionals with the knowledge, attitudes, and skills needed to operate as programmers in multidisciplinary teams that produce vibrant and innovative digital products and services.
	\end{itemize}
\end{frame}

\begin{frame}
	\frametitle{Course Objectives}
	
	By the end of this year, you should be confidently able to: \pause
	
	\begin{itemize}
		\item \textbf{Code}: Translate technical notation into executable code. \pause
		\item \textbf{Architect}: Translate requirements into suitable technical notation. \pause
		\item \textbf{Solve}: Demonstrate computational thinking and numeracy skills.
	\end{itemize}
\end{frame}

\begin{frame}
	\frametitle{Course Objectives}
	
	By the end of this year, you should be confidently able to: \pause
	
	\begin{itemize}
		\item \textbf{Advocate}: Recognise the legal, social, ethical, and professional issues that affect creative projects. \pause
		\item \textbf{Research}: Report on an issue using appropriate sources and academic conventions.\pause
		\item \textbf{Reflect}: Identify professional attributes and illustrate how they are relevant to your practice.
	\end{itemize}
\end{frame}

\begin{frame}
	\frametitle{Learning Objectives}
	
	The objectives of this course are to facilitate the development of your:
	
	\begin{itemize}
		\item \textbf{Collaborate/Utilise}: Define suitable development practices, project management approaches, and version control techniques used in the execution of a collaborative project. \pause
		\item \textbf{Pitch}: Identify your role within a creative studio culture. \pause
		\item \textbf{Deliver/Innovate}: Describe how to create and test prototypes in order to deliver an interesting experience.
	\end{itemize}
\end{frame}

\begin{frame}
	\frametitle{Philosophy}
	
	We offer the only science degrees in the Game Academy and do things a little differently:
	
	\begin{itemize}
		\pause\item Emphasis on developing community, and discourse/peer-review within that community
		\begin{itemize}
			\item Do it together and learn from each other, before doing it alone
			\item Critique each others' work and discuss what constitutes good practice
		\end{itemize}
		\pause\item Emphasis on feed-forward over just feed-back
		\begin{itemize}
			\item Early milestones, earlier start, more learning
			\item Get advice on how to improve your own practice \textit{before} you submit your work
		\end{itemize}
	\end{itemize}
\end{frame}

\begin{frame}
	\frametitle{Philosophy}
	
	\begin{itemize}
		\item Emphasis on highly structured assignments 
		\begin{itemize}
			\item Formative work across the study block
			\item Easy to pass for successfully completing all in-class activities with basic competence and submitting on-time
			\item Face-to-face feedback and discussion in assessment by viva			
		\end{itemize}
		\pause\item Emphasis on continuing personal development
		\begin{itemize}
			\item Personal growth over hitting a benchmark
			\item Journey to professional competency and beyond, rather than hitting a grade
			\item Rubrics and qualitative feedback (at least, at first)			
		\end{itemize}
	\end{itemize}
\end{frame}

\part{Indicative Course Maps}
\frame{\partpage}

\fullbleed{Slide8}

\fullbleed{Slide9}

\fullbleed{Slide10}

\fullbleed{Slide11}

%\fullbleed{webdev-s3}

\part{Study Block One}
\frame{\partpage}

\begin{frame}
	\frametitle{Modules}
	
	You have three modules to complete in study block one. These are:
	
	\vspace{0.5em}
	
	\begin{itemize}
		\item COMP110 Principles of Computing
		\item COMP120 Creative Computing
		\item GAM110 Development Principles
	\end{itemize}
	
	\vspace{1em}
	
	Students on BA(Hons) courses and who have no intention of switching can swap COMP110 Principles of Computing for GAM120 Reading Experiences. You will need to complete a module transfer form - we only recommend making the switch if you do \textbf{not} feel comfortable with mathematics.
	
\end{frame}

\begin{frame}
	\frametitle{Modules}
	
	There are more detailed module introductions, module welcome talks, module induction talks, and assignment briefs available for you to review on the LearningSpace.
	
	\vspace{0.5em}
	
	These should be available to you on Monday, if they aren't available already.
	
	\vspace{0.5em}
	
	We will briefly introduce these modules now, but you will need to watch the videos for further detail.
	
\end{frame}

\begin{frame}
	\frametitle{COMP110 Principles of Computing}
		
	\small{\textbf{Aim:} To enable you to apply basic computing and mathematical theory to solve practical problems.}
		\vspace{0.5em}
	
	\small{\textbf{Module Leader:} Dr Ed Powley}	
	
		\vspace{0.5em}
	
\footnotesize{On this module, you will learn the foundational principles of computing, discrete mathematics, and technical communication (e.g., notation, pseudocode, unified modelling language, etc.). You start the process of learning to use core concepts and methods from computer science to solve practical problems and leverage algorithms in your solutions. You will become acquainted in a practical way with the techniques and methods that help you to work through challenges effectively and efficiently to design, build, and annotate computing solutions with reference to relevant scholarly sources.}
	
\end{frame}

\begin{frame}
	\frametitle{COMP120 Creative Computing}
		
	\small{\textbf{Aim:} To develop your comfort using code and computational techniques to manipulate digital media.}
	
	\vspace{0.5em}
	
	\small{\textbf{Module Leader:} Dr Michael Scott}	
	
		\vspace{0.5em}
	
\footnotesize{On this module, you will learn different ways of engaging with code through a practical exploration of media formats including text, image, and sound. Whilst you work in a variety of pair- and mob- programming formats, you will play, tinker, experiment with, and extend code that will convert artefacts that already exist into something new as a form of appropriation.  In doing so, you will embrace the principal of rapid iteration and work in a creative way.  Engaging with creative computing in this way means that you will not only become acquainted with programming at an introductory-level, but you will also exercise your creativity. However, working in such a manner and producing derivative works raises moral and legal questions that you will consider and frame within topics such as plagiarism, intellectual property law, licensing rights, as well as the maker and open-source movements.}
	
\end{frame}

\begin{frame}
	\frametitle{GAM110 Development Principles}
		
	\small{\textbf{Aim:} To engage with the foundational processes of digital project development in a studio-centred context, as well as the culture it supports.}
	
	\vspace{0.5em}
	
	\small{\textbf{Module Leader:} Terry Greer}
	
	\vspace{0.5em}
	
\footnotesize{On this module, you gain an understanding of the basic principles, terminology, roles, and tools used in the development of digital products and services. Supervised studio practice directs your attention towards the different assets and software components that need to be drawn together to make a working digital product and how they are organised throughout the development pipeline. You will immerse yourself in a studio culture in which you apply ‘agile’ project management methods to facilitate practical development and use version control tools to manage your collaboration. You will also gain a ‘first-principles’ understanding of how to design with a target market in mind and maintain a strong underlying concept.}
	
\end{frame}

\begin{frame}
	\frametitle{GAM120 Reading Experiences}
		
	\textit{Optional switch for BA(Hons)	 students}
		
		\vspace{0.2em}
		
	\small{\textbf{Aim:} To introduce you to the formal characteristics of digital experiences, and the theories and concepts that have been developed for their analysis.}
	
	\vspace{0.5em}
	
	\small{\textbf{Module Leader:} Dr Jeff Howard}
	
	\vspace{0.5em}
	
\footnotesize{In exploring ideas about games and the player experience, this module offers a foundational space to begin to think closely and carefully about the formal nature of digital experiences, their markets and the contexts of their production, the pleasures they offer and what it means to play them. The ideas engaged with are intended to inform and broaden your development practice on other modules. You receive foundational lectures and workshops on researching experiences and methods for doing so, as well as on academic research methods and essay writing. Seminars provide the space for debating the ideas and material encountered, lectures provide orientation and a one-on-one tutorial provides individual feedback on your progress.}
	
\end{frame}

\part{Timetable}
\frame{\partpage}

\begin{frame}
	\frametitle{Timetable}
	
	The timetable can be found on:
	
	\vspace{0.5em}
	
	\indent \url{http://mytimetable.falmouth.ac.uk}
	
	\vspace{0.5em}
	
	Check the timetable every day! Sessions can, and often do change. Once you are allocated into groups for your collaborative game development projects, meeting times with tutors will change and extra sessions may appear!
	
	\vspace{0.5em}
	
	 The course isn't just the time you're scheduled to be with a tutor, you are expected to engage in self-directed study.
	
\end{frame}

\begin{frame}
	\frametitle{COVID-19 Adjustments}
		
	An overview is available at: 
	
	\vspace{0.5em}
	
	\url{https://www.falmouth.ac.uk/experience/new-students/welcome-letters/\#course-updates}
	
\end{frame}

\begin{frame}
	\frametitle{Blended Learning}
		
	Many areas of our provision have improved due to online delivery methods. These include:
	
	\vspace{0.5em}
	
	\begin{itemize}
		\item Academic workshop delivery and worksheet tutorials in COMP110
		\item Mathematics lectures and support in COMP270
		\item R\&D support and dissertation supervision in COMP320/COMP360
	\end{itemize}
	
	\vspace{0.5em}
	
	Since module ratings improved year-on-year for these modules, we will continue to use and enhance online delivery methods where they make sense and where they assure continuity in the student journey and a high quality of provision.
	
\end{frame}

\begin{frame}
	\frametitle{Blended Learning}
		
	Many areas of our provision benefit from traditional delivery methods. These include:
	
	\vspace{0.5em}
	
	\begin{itemize}
		\item Programming and version control setup and support in COMP120
		\item Collaborative development practice in our labs and studios in GAM110
		\item Using specialist requipment in COMP140, ROB210, and VR220
	\end{itemize}
	
	\vspace{0.5em}
	
	We are no longer supportive of hybrid methods as these were clunkly to run, spread the support too thin, and diminished community building aspects of the course.
	
	These are studio-based courses and you are expected to convene with members of your team in-person in the studio as timetabled for studio practice.
	
\end{frame}

\part{Assignments}
\frame{\partpage}

\begin{frame}
	\frametitle{Assignment Structure}
	
	\begin{Huge}
		\begin{center}
			\textbf{100\% Coursework}
		\end{center}
	\end{Huge}

\end{frame}

\begin{frame}
	\frametitle{Assignment Structure}
	
	
	Assessments are designed to reflect professional practice:
	
	\begin{itemize}
		\item Items for your Portfolio
		\item Collaborative Projects
		\item Pitches
		\item Papers
	\end{itemize}

	Relative importance of each will depend on your career trajectory

\end{frame}

\notepic{prospectus_alex_034}{Collaborative Approach with \textbf{Arts Students}}{-2.5cm}{-3.5cm}

\notepicx{lucy_blueprints}{Follows an \textbf{Incubation Model}: Make Games For Real}{2.5cm}{-3.5cm}{0.4}

\notepicx{new_studio}{Access to Excellent \textbf{Studio Facilities} Subject to Safe Working Practices}{2.5cm}{-3.5cm}{0.45}

\notepicx{expo2}{\textbf{Industry Involvement}: Show-off your work to professionals at our expo}{-2.5cm}{-3.5cm}{0.45}

\picturepage{ppc}{\url{https://dangerzone-ga.itch.io/ppc}}

% add SAI

% add rustbreaker

%\fullbleed{cropped-drnk_splash-with-logo-trans}

%\picturepage{monq}{\url{https://www.youtube.com/embed/KzvZD5-Jmo4}}

%\fullbleed{controller_1}

%\fullbleed{controller_2}

%\fullbleed{controller_4}

\begin{frame}
	\frametitle{Assignments}
	
	Live Demo
	
	\vspace{3em}
	
	All assignment briefs will be found on:
	
	\vspace{0.5em}
	
	\indent \url{learningspace.falmouth.ac.uk}
	
	\vspace{0.5em}
	
	Enjoy freshers week. Read them very carefully next week!
	
	\vspace{0.5em}
	
	LearningSpace is also where you submit \textbf{ALL} final ``summative'' versions of your assigned coursework tasks!
	
\end{frame}

\begin{frame}
	\frametitle{Assignments}
	
	You will usually submit your work as:
	
	\begin{itemize}
	    \item a link to your \texttt{git} repository
	    \item \textit{or} a single \texttt{.pdf} file
	    \item \textit{or} a single \texttt{.zip} archive
	\end{itemize}
	
	\vspace{1em}
	
	 Please use the following convention:
	
	\begin{large}
		\begin{center}
			\textbf{module\_assignmentNumber\_studentID}
		\end{center}
	\end{large}
	
	For example:
	
	\begin{Large}
		\begin{center}
			\textbf{comp110\_1\_2011213}
		\end{center}
	\end{Large}
	
	\vspace{1em}
	
	We use anonymous marking where possible.

\end{frame}

\begin{frame}
	\frametitle{Assignments}
		
	All assignment deadlines can be found next week on:
	
	\vspace{1em}
	
	\indent \url{myfalmouth.falmouth.ac.uk}
	
	\vspace{1em}
	
	Take note of these carefully! A single second late, and your work will be capped at the minimum passing grade.
	
\end{frame}

\begin{frame}
	\frametitle{Assignments}
		
	In the absence of extenuating circumstances (i.e., you are seriously ill and stuck in hospital):
	
	\begin{Large}
		\begin{center}
			\textbf{You MUST submit something \\ for EVERY assigned coursework task!}
		\end{center}
	\end{Large}
	
	In the eyes of university policy, not submitting anything is usually unrecoverable. Even if your work is unfinished, please submit something! Even submitting a blank piece of paper is better than not submitting anything! 
	
		\vspace{0.5em}
	
	If you forget to submit, there is a grace period of 5 working days after the deadline. If you fail, you get a second attempt. And, usually, a third attempt.
	
\end{frame}


\part{Expectations in Higher Education}
\frame{\partpage}

\begin{frame}{Exercise}

    Go to:
    
    \vspace{0.5em}
    
    \small{\url{https://padlet.com/michaelscott5/xjdz7hngsnvyx35z}}
    
    \vspace{1em}
    
    Let's discuss what `expectations' means, with particular focus on how they differ between higher and compulsory education.
    
    \vspace{0.5em}

	\begin{itemize}
	    \item \textbf{List} key differences between expectations in the higher education and compulsory education contexts;
		\item \textbf{Suggest} what will be expected of you during your time on the course;
		\item 	\textbf{Give} examples of activities that count as `self-directed study'.
	\end{itemize}

\end{frame}

\begin{frame}
	\frametitle{Expectations}
	
	Please note the following:
	
	\vspace{0.5em}
	
	\begin{itemize}
		\item This is a full-time course
		\item You are expected to do 1200 hours of study per academic year
		\item Approximately $1/3$ of that will be contact time
		\item Approximately $2/3$ of that will be `self-directed study'
		\item This is a full-time course---you are expected to study 40 hours per week, \textbf{EACH} week, across \textbf{EACH} study block
		\item If you can't commit to this---you will likely struggle with the pace of the course and the group work
	\end{itemize}

\end{frame}

\begin{frame}
	\frametitle{Expectation}
	
	Typically, this coming study block has the following structure:
	
	\begin{itemize}
		\item 5 Weeks - Sessions with Tutors
		\item 1 Week - Assessments and Self-Directed Studio Practice with Team
		\begin{itemize}
			\item \textbf{NOT} a vacation
		\end{itemize}
		\item 5 Weeks - Further Sessions with Tutors
		\item 1 Week - Assessments and Self-Directed Studio Practice with Team
		\item Vacation Period
		\item 1 Week - Assessments  and Self-Directed Studio Practice with Team
		\item 2 Weeks - Workshops Festival
		\begin{itemize}
			\item \textbf{NOT} a vacation
		\end{itemize}
	\end{itemize}
	
\end{frame}

\fullbleed{climbing-a-mountain}

\begin{frame}
	\frametitle{Time Management}
	
	TBD
	
\end{frame}

\begin{frame}
	\frametitle{Questions \& Answers}	
	\begin{center}
		Thank you for listening. 
		\\~\\
		Please feel welcome to ask questions or raise concerns.
	\end{center}
\end{frame}

\part{Breakout Groups}
\frame{\partpage}

\begin{frame}
	\frametitle{Breaking Out}
	
	Your programming tutor will have setup a Microsoft Teams `chat' with you so that you can engage with them. Please introduce yourself to them and introduce yourself to the other members of your tutor group. You will be meeting with them regularly throughout the COMP120 module. We have a few icebreaking activities for you to choose from and a few recommended activities
	
	\begin{itemize}
		\item SpaceTeam icebreaker
		\item Games Meta-game icebreaker
		\item Time management advice
		\item DoIT Profiler
	\end{itemize}
\end{frame}

\begin{frame}
	\frametitle{Icebreaker: SpaceTeam}
	
	A cooperative shouting game for piloting a spaceship!
	
	Setup:
	
	\begin{itemize}
		\item Download \url{https://spaceteam.ca/}
		\item If you don't have an Apple or Android mobile phone, use an emulator (e.g., BlueStacks)
		\item Share the room code to play together online!
	\end{itemize}
	
\end{frame}

\begin{frame}
	\frametitle{Icebreaker: Games Meta-Game}
	
	Setup:
	
	\begin{itemize}
		\item Organise into your tutor groups of 4-6 players
		\item You will each receive two sets of card: game cards and question cards.
		\item While you are waiting for your cards, identify the youngest player. They will be the first critic.
		\item All actions are clockwise from the critic.
	\end{itemize}
\end{frame}

\begin{frame}
	\frametitle{Icebreaker: Games Meta-Game}
	
	Instructions:
	
	\begin{enumerate}
		\item 	\textbf{Question}: The critic draws a question card. 
		\item 	\textbf{Answer}: The \textit{remaining players} (i.e., not the critic!) submit their best game card, to answer the question, face-up.
		\item 	\textbf{Justification}: The \textit{remaining players} justify the game card they have selected.
		\item 	\textbf{Selection}: The critic selects the most suitable game card answering the question. That player `wins' the round, keeping the question card as a scoring token and becomes the next critic.	
		\item 	\textbf{Repeat} from step 1, for approximately 20 minutes.	
	\end{enumerate}
\end{frame}

\begin{frame}
	\frametitle{Activity: Time Management}
	
	Please complete the following activity:
	
	\vspace{2em}
	
	\url{http://www.learnhigher.ac.uk/learning-at-university/time-management/getting-organised/}

\end{frame}

\begin{frame}
	\frametitle{Activity: DoIT Profiler}
	
	You \textbf{MUST} complete the following activity:
	
	\vspace{2em}
	
	\url{https://doitprofiler.net/Account/ClientLogin}
	
	\vspace{2em}
	
	Client code: \texttt{fal15mar}

\end{frame}

\end{document}
