\part{Inheritance}
\frame{\partpage}

\begin{frame}{Relationships}
	\begin{itemize}
		\pause\item OOP models three types of real-world relationships: \textbf{is~a}, \textbf{has~a} and \textbf{is~a~type~of}
	\end{itemize}
	\pause
	\begin{columns}
		\begin{column}{0.3\textwidth}
			\includegraphics[width=\textwidth]{donald.png}
		\end{column}
		\begin{column}{0.66\textwidth}
			\begin{itemize}
				\pause\item Donald \textbf{is a} duck
				\pause\item A duck \textbf{has a} bill
				\pause\item A duck \textbf{is a type of} bird
			\end{itemize}
		\end{column}
	\end{columns}
\end{frame}

\begin{frame}[fragile]{Is-a $\to$ Instantiation}
	\begin{itemize}
		\pause\item ``X is a Y'' means ``the specific object X is an object of the type Y''
		\pause\item Is-a is modelled by \textbf{classes} and \textbf{instances}:
		\begin{itemize}
			\pause\item ``Donald is a duck'' 
				$\to$ ``\lstinline{donald} is an \textbf{instance} of the \textbf{class} \lstinline{Duck}''
		\end{itemize}
	\end{itemize}
	\pause
	\begin{lstlisting}
class Duck
{
};

Duck donald;
	\end{lstlisting}
\end{frame}

\begin{frame}{Has-a $\to$ Composition}
	\begin{itemize}
		\pause\item ``X has a Y'' means ``an object of type X \textbf{possesses} an object of type Y''
		\pause\item OOP models this by having a \textbf{field} on X which holds an instance of Y
		\begin{itemize}
			\pause\item ``A duck has a bill''
				$\to$ ``The class \lstinline{Duck} has a \textbf{field} which contains an instance of the class \lstinline{Bill}''
		\end{itemize}
	\end{itemize}
\end{frame}

\begin{frame}[fragile]{Composition in C++}
    \begin{center}
        ``A duck has a bill''
    \end{center}
    \pause
    \begin{columns}
        \begin{column}{0.48\textwidth}
            \begin{itemize}
                \item ``Each instance of class Duck contains \textbf{an instance} of class Bill''
            \end{itemize}
            \begin{lstlisting}
class Bill { ... };

class Duck
{
private:
    Bill bill;
};
            \end{lstlisting}
        \end{column}
        \pause
        \begin{column}{0.48\textwidth}
            \begin{itemize}
                \item \textbf{Or} ``Each instance of class Duck contains \textbf{a pointer} to an instance of class Bill''
            \end{itemize}
            \begin{lstlisting}
class Bill { ... };

class Duck
{
private:
    Bill* bill;
};
            \end{lstlisting}
        \end{column}
    \end{columns}
\end{frame}

\begin{frame}{Composition in C++}
    \begin{columns}
        \begin{column}{0.55\textwidth}
            \begin{itemize}
                \item The contained instance of \lstinline{Bill} is stored \textbf{inside} the
                    instance of \lstinline{Duck} (literally, in memory) \pause
                \item It is constructed when the \lstinline{Duck} instance is constructed,
                    and destroyed when it is destroyed \pause
            \end{itemize}
        \end{column}
        \begin{column}{0.5\textwidth}
            \begin{itemize}
                \item The contained instance of \lstinline{Bill} is stored \textbf{outside} the
                    instance of \lstinline{Duck}, which only stores a \textbf{pointer} \pause
                \item It is usually constructed manually using \lstinline{new},
                    and so must be destroyed manually using \lstinline{delete}
            \end{itemize}
        \end{column}
    \end{columns}
\end{frame}

\begin{frame}{When to use each?}
    \begin{itemize}
        \item Pointers are more versatile \pause
            \begin{itemize}
                \item Allow several pointers to the same instance (e.g.\ several ducks might \textbf{have-a} single pond) \pause
                \item Allow \textbf{circular references} (e.g.\ a duck \textbf{has-a} bill, and a bill \textbf{has-a} duck) \pause
                \item Pointers allow \textbf{polymorphism} (e.g.\ a pointer to a ``duck'' might actually be a pointer to a mallard) \pause
            \end{itemize}
        \item \textbf{But} stored instances are easier to work with \pause
            \begin{itemize}
                \item Destruction is handled automatically \pause
            \end{itemize}
        \item They model slightly different types of \textbf{has-a} relationship \pause
            \begin{itemize}
                \item Instance: \textbf{has-a} in the sense of ``contains'' \pause
                \item Pointer: \textbf{has-a} in the sense of ``is associated with''
            \end{itemize}
    \end{itemize}
\end{frame}

\begin{frame}{Is-a-type-of $\to$ Inheritance}
\begin{itemize}
	\pause\item ``X is a type of Y'' means ``If an object is of type X, then it is \textbf{also} of type Y''
	\begin{itemize}
		\pause\item ``A duck is a type of bird'' $\to$ ``If something is a duck, then it is also a bird''
		\pause\item ``Every duck is a bird''
		\pause\item ``If something is true for all birds, then it must be true for ducks''
	\end{itemize}
	\pause\item In OOP terms, this is called \textbf{inheritance}
\end{itemize}
\end{frame}

\begin{frame}{Inheritance}
\begin{itemize}
	\pause\item Recall: an \textbf{object} is a collection of \textbf{fields} (data) and \textbf{methods} (code)
	\pause\item Recall: the \textbf{class} defines which fields and methods an object possesses
	\pause\item ``X is a type of Y'' $\to$ class \lstinline{X} inherits from class \lstinline{Y}
	\pause\item Class X inherits all of the fields and methods from class Y, as well as any fields and methods of its own
\end{itemize}
\end{frame}

\begin{frame}{When to inherit?}
\begin{itemize}
	\pause\item When modelling an \textbf{is-a-type-of} relationship from the real world
	\pause\item When several classes can \textbf{share} some fields and/or methods
	\begin{itemize}
		\pause\item I.e. to minimise \textbf{code duplication}
	\end{itemize}
	\pause\item When several classes should have methods with the \textbf{same names}, but which do \textbf{different things}
	\begin{itemize}
		\pause\item This is called \textbf{polymorphism} --- more on this later
	\end{itemize}
\end{itemize}
\end{frame}

\begin{frame}[fragile]{Inheritance in C++}
	\begin{lstlisting}
class Shape
{
public:
    float centreX, centreY;
    Shape(float cx, float cy)
    	: centreX(cx), centreY(cy) { }
};

class Circle : public Shape
{
public:
    float radius;
    Circle(float cx, float cy, float r)
        : Shape(cx, cy), radius(r) { }
    float getArea()
    {
    	return 3.14159f * radius * radius;
    }
};
	\end{lstlisting}
\end{frame}

\begin{frame}{Chains of inheritance}
\begin{itemize}
	\pause\item ``A mallard is a type of duck, which is a type of bird, which is a type of vertebrate, which is a type of animal...''
	\pause\item Is-a-type-of is \textbf{transitive}
	\begin{itemize}
		\pause\item If A is-a-type-of B and B is-a-type-of C, then A is-a-type-of C
	\end{itemize}
	\pause\item Likewise: class A inherits from class B, which inherits from class C, ...
	\begin{itemize}
		\pause\item ``Inherits from'' is also transitive
	\end{itemize}
\end{itemize}
\end{frame}

\begin{frame}[fragile]
	\begin{columns}
		\begin{column}{0.5\textwidth}
			\begin{lstlisting}[basicstyle=\tiny\ttfamily]
class A {
public:
	int x;
	A(int x) : x(x) {}
	int foo() { return x*x; }
};

class B: public A {
public:
	int y;
	B(int x, int y) : A(x), y(y) {}
};

class C: public B {
public:
	int z;
	C(int x, int y, int z)
		: B(x, y), z(z) {}
};

class D: public A {
public:
	int y;
	D(int y) : A(20), y(y) {}
	int bar() { return x*x*x; }
};

class E {
public:
	int x;
	E(int x) : x(x) {}
};
			\end{lstlisting}
		\end{column}
		\begin{column}{0.46\textwidth}
			Socrative \texttt{FALCOMPED}
			\pause
			\begin{lstlisting}[basicstyle=\tiny\ttfamily]
void first() {
	C c(10, 20, 30);
	cout << c.z << endl;
}

void second() {
	B b(10, 20);
	cout << b.z << endl;
}

void third() {
	B b(10, 20);
	cout << b.foo() << endl;
}

void fourth() {
	B b(10, 20);
	cout << b.bar() << endl;
}

void fifth() {
	D d(10);
	cout << d.foo() << endl;
}
			\end{lstlisting}
		\end{column}
	\end{columns}
\end{frame}

