\documentclass{scrartcl}

\usepackage[hidelinks]{hyperref}
\usepackage[none]{hyphenat}
\usepackage{setspace}
\usepackage{enumitem}
\setlist{nosep} % Make enumerate / itemize lists more closely spaced
\doublespace

\begin{document}

\title{UML Worksheet II}
\author{Game Architecture and Engineering}
\date{}

\maketitle

Unified Modelling Language (UML) is a way of communicating the design of software using diagrams. It is a notation that built upon the work of Grady Booch, James Rumbaugh, Ivar Jacobson, and the Rational Software Corporation. It was originally developed to support the object-oriented paradigm, although has since been extended to accommodate a diverse range of projects. According to the Object Management Group (OMG), UML is the international standard for software modelling.

\section{In-Class Task}

In today's in-class task you will learn how to draw \textbf{UML Activity} and \textbf{UML State} diagrams. To complete this you will:

\begin{itemize}
	\item \textbf{Organise} yourselves into your COMP130 project teams.
	\item \textbf{Watch} the video tutorial at \url{https://www.youtube.com/watch?v=XFTAIj2N2Lc}.
	\item \textbf{Read} \url{http://agilemodeling.com/artifacts/activityDiagram.htm}.
	\item \textbf{Draw} a UML Activity diagram to model ONE part (e.g. game flow) of your game.
	\item \textbf{Watch} the video tutorial at \url{https://www.youtube.com/watch?v=_6TFVzBW7oo}.
	\item \textbf{Read} \url{http://www.agilemodeling.com/artifacts/stateMachineDiagram.htm}.
	\item \textbf{Draw} a UML State diagram to model ONE part (e.g.an AI agent) of your game.
\end{itemize}

\vspace{1ex}

Use the white boards to draw your diagrams.
 
Alternatively, use Gliffy: \url{https://www.gliffy.com/uses/uml-software/}

\end{document}