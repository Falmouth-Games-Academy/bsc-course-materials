\part{Logic}
\frame{\partpage}

\begin{frame}[fragile]{Logical operations}
	\begin{center}
		\begin{tabular}{|c|c|c|c|c|}
			\hline
			Python & C family & Mathematics & Behaviour tree \\\hline
				  \lstinline[language=Python]{not a}
				& \lstinline[language=C++]{!a}
				& $\neg A$ {\huge\phantom{$I$}} or {\huge\phantom{$I$}} $\overline{A}$
				& Inverter
				\\
				  \lstinline[language=Python]{a and b}
				& \lstinline[language=C++]{a && b}
				& $A \wedge B$
				& Sequence
				\\
				  \lstinline[language=Python]{a or b}
				& \lstinline[language=C++]{a || b}
				& $A \vee B$
				& Selector
				\\\hline
		\end{tabular}
	\end{center}
\end{frame}

\begin{frame}{The laws of thought}
	\begin{itemize}
		\pause\item Let $A$ be a \textbf{proposition} (a statement about the world)
		\pause\item \lstinline{A} is a \textbf{boolean} value, either \textbf{true} or \textbf{false}
		\pause\item The law of \textbf{identity}: \lstinline{A == A} is always true
		\pause\item The law of \textbf{non-contradiction}: \lstinline{A && !A} is always false
			\begin{itemize}
				\pause\item I.e.\ \lstinline{A} cannot be both true and false
			\end{itemize}
		\pause\item The law of the \textbf{excluded middle}: \lstinline{A || !A} is always true;
			\begin{itemize}
				\pause\item I.e.\ \lstinline{A} must be either true or false
			\end{itemize}
	\end{itemize}
\end{frame}

\begin{frame}{Predicates}
	\begin{itemize}
		\pause\item \textbf{Predicates} are propositions with \textbf{parameters}
		\pause\item In programming terms, a predicate is a function that returns a boolean
		\pause\item E.g.\ \lstinline{LivesIn(Bob, Falmouth)} could be a predicate for ``Bob lives in Falmouth''
	\end{itemize}
\end{frame}

\begin{frame}{Quantifiers}
	\begin{itemize}
		\pause\item $P(x)$ is a predicate
		\pause\item $\forall x : P(x)$ means that $P(x)$ is true \textbf{for all} values of $x$
		\pause\item $\exists x : P(x)$ means that \textbf{there exists} at least one value of $x$ such that $P(x)$ is true
	\end{itemize}
\end{frame}

\newcommand{\LivesIn}{\operatorname{LivesIn}}
\newcommand{\Falmouth}{\operatorname{Falmouth}}
\newcommand{\Cornwall}{\operatorname{Cornwall}}
\newcommand{\England}{\operatorname{England}}
\newcommand{\InCornwall}{\operatorname{InCornwall}}
\newcommand{\IsCity}{\operatorname{IsCity}}
\newcommand{\Truro}{\operatorname{Truro}}
\newcommand{\Cider}{\operatorname{Cider}}
\newcommand{\Likes}{\operatorname{Likes}}

\begin{frame}{Implication}
	\begin{itemize}
		\pause\item ``$A$ implies $B$'' means ``if $A$ is true then $B$ is true''
		\pause\item Written as $A \implies B$
		\pause\item E.g.\ if someone lives in Falmouth then they live in Cornwall
		\pause\item $\forall x : \LivesIn(x, \Falmouth) \implies \LivesIn(x, \Cornwall)$
	\end{itemize}
\end{frame}

\begin{frame}{Contrapositive}
	\begin{itemize}
		\pause\item $A \implies B$ is equivalent to $\neg B \implies \neg A$
		\pause\item E.g.\ if someone does not live in Cornwall then we know they don't live in Falmouth
		\pause\item $\forall x : \neg \LivesIn(x, \Cornwall) \implies \neg \LivesIn(x, \Falmouth)$
	\end{itemize}
\end{frame}

\begin{frame}{Equivalence}
	\begin{itemize}
		\pause\item If $A \implies B$ and $B \implies A$ then $A$ and $B$ are \textbf{logically equivalent}
		\pause\item $A$ is true \textbf{if and only if} $B$ is true
		\pause\item Written as $A \iff B$
		\pause\item E.g.\ ``Alice lives in a city in Cornwall'' if and only if ``Alice lives in Truro''
		\pause\item This relies on an extra piece of domain knowledge: Truro is the only city in Cornwall
			\begin{itemize}
				\pause\item $\forall x : \InCornwall(x) \wedge \IsCity(x) \implies x = \Truro$
			\end{itemize}
	\end{itemize}
\end{frame}

\begin{frame}{Implication is transitive}
	\begin{itemize}
		\pause\item If $A \implies B$ and $B \implies C$ then $A \implies C$
		\pause\item E.g.\ if someone lives in Falmouth then they live in Cornwall
		\pause\item And if someone lives in Cornwall then they live in England
		\pause\item Therefore if someone lives in Falmouth then they live in England
	\end{itemize}
\end{frame}

\begin{frame}{Inverting quantifiers}
	\begin{itemize}
		\pause\item ``Everyone who lives in Cornwall likes cider''
		\pause\item $\forall x : \LivesIn(x, \Cornwall) \implies \Likes(x, \Cider)$
		\pause\item What is the \textbf{opposite} of this statement?
		\pause\item $\neg( \forall x : \LivesIn(x, \Cornwall) \implies \Likes(x, \Cider) )$
		\pause\item In logical terms, the opposite is \textbf{not} ``nobody who lives in Cornwall likes cider''
		\pause\item It's ``Not everyone who lives in Cornwall likes cider''
		\pause\item I.e.\ ``There is at least one person living in Cornwall who does not like cider''
		\pause\item $\exists x : \LivesIn(x, \Cornwall) \wedge \neg \Likes(x, \Cider)$
	\end{itemize}
\end{frame}
