\part{The role of PCG in games}
\frame{\partpage}

\begin{frame}{Lessons from No Man's Sky}
	\begin{center}
		\pause\includegraphics[width=0.5\textwidth]{nomanssky_steamreviews}
	\end{center}
	\begin{itemize}
		\pause\item If you overscope, pray that you don't have to cut any features
			that you announced on stage at E3...
		\pause\item PCG is not a substitute for gameplay
		\pause\item PCG is not magic --- it doesn't (by itself) let an indie-sized team produce a AAA game
		\pause\item When talking about scale and PCG, it's easy to set unrealistic expectations
	\end{itemize}	
\end{frame}

\begin{frame}{Big numbers}
	\begin{itemize}
		\pause\item ``Over 18 quintillion planets''
		\pause\item $2^{64} = 18\,446\,744\,073\,709\,551\,616$
		\pause\item What does this number even \textbf{mean}?
		\pause\item What it \textbf{really} means: ``our random number generator uses a 64-bit seed''
		\pause\item They could have said ``a near infinite number of planets''
		\pause\item They could easily have made it ``over 340 undecillion'' planets
			($2^{128} = 340\,282\,366\,920\,938\,463\,463\,374\,607\,431\,768\,211\,456$)
	\end{itemize}
\end{frame}

\begin{frame}{Even bigger numbers}
	\begin{itemize}
		\pause\item There are
			\begin{align*}
		 		52! = 80\,658\,175\,170\,943\,878\,571\,660\,636\,856\,403\,766 &\\
		 		          975\,289\,505\,440\,883\,277\,824\,000\,000\,000\,000 &
		 	\end{align*}
			ways of shuffling a deck of playing cards
		\pause\item When you shuffle a deck, it is almost certain that
			\textbf{no deck of cards in human history} has ever existed in that order
		\pause\item But how \textbf{interesting} is that particular shuffled deck?
		\pause\item How \textbf{different} from another shuffled deck?
	\end{itemize}
\end{frame}

\begin{frame}{Uniqueness}
	``I can easily generate 10,000 bowls of plain oatmeal, with each oat being in a different position 
	and different orientation, and \textit{mathematically speaking} they will all be completely unique.
	But the user will likely just see \textit{a lot of oatmeal}.''
	
	--- Kate Compton
	
	{\tiny\url{http://galaxykate0.tumblr.com/post/139774965871/so-you-want-to-build-a-generator}}
\end{frame}

\begin{frame}{Uniqueness}
	``\thinspace`Every Planet Unique' might mean that each planet has a complex sci-fi backstory rich enough to 
	fill a two-part Star Trek episode.
	It might also mean that, mathematically speaking, there's a rock somewhere on the planet that
	doesn't look like any other rock in the universe.''
	
	--- Michael Cook

	{\tiny\url{http://www.gamesbyangelina.org/2016/08/procedurallanguage/}}
\end{frame}

\begin{frame}{Lessons from Spelunky}
	\begin{center}
		\pause\includegraphics[width=0.5\textwidth]{spelunky_steamreviews}
	\end{center}
	\begin{itemize}
		\pause\item PCG can complement solid game mechanics
		\pause\item PCG can \textbf{enable} new (discovery-based) game mechanics
		\pause\item No need to dazzle the audience with big numbers
	\end{itemize}	
\end{frame}

\begin{frame}{Curation}
	\begin{columns}
		\begin{column}{0.4\textwidth}
			\pause\includegraphics[width=\textwidth]{curation}
		\end{column}
		\begin{column}{0.55\textwidth}
			\begin{itemize}
				\pause\item Human creators constantly ask themselves: \textbf{is this any good?}
				\pause\item Smart PCG should not \textbf{merely generate}: it should also
					\textbf{evaluate}
			\end{itemize}
		\end{column}
	\end{columns}
\end{frame}

\begin{frame}{Authorship}
	\begin{itemize}
		\pause\item In a game with \textbf{emergent narrative}, who is the author?
			Is it the developer, the player, or both?
		\pause\item In a game with \textbf{procedurally-generated content}, who (or what) is the author?
			Is it the developer, the player, the system, or all three?
	\end{itemize}
\end{frame}

\begin{frame}{Authorship}
	``[We] create the systems (including some fixed content), and the choices made at that stage
	are influenced by our preferences, worldviews, talents and flaws, and then the system creates the content.
	The players are exposed to the content and can manipulate it using the tools we (and others) create for them.
	How they use the tools is up to them, and how the content reacts is up to our systems.''
	
	--- Tarn Adams
	
	{\tiny\url{http://www.nullpointer.co.uk/content/interview-dwarf-fortress/}}
\end{frame}

