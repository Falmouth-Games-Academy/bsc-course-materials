% !TeX root = ./2020-21-COMP250-05-session-materials_screen.tex
\part{Minimax search}
\frame{\partpage}

\begin{frame}{Minimax}
	\begin{itemize}
		\pause\item Terminal game states have a \textbf{value}
			\begin{itemize}
				\pause\item E.g.\ $+1$ for a win, $-1$ for a loss, $0$ for a draw
			\end{itemize}
		\pause\item I want to \textbf{maximise} the value
		\pause\item My opponent wants to \textbf{minimise} the value
		\pause\item Therefore I want to \textbf{maximise} the \textbf{minimum} value my opponent can achieve
		\pause\item This is generally only true for \textbf{two-player zero-sum} games
	\end{itemize}
\end{frame}

\begin{frame}{Minimax search}
	\begin{itemize}
		\pause\item Recursively defines a \textbf{value} for non-terminal game states
		\pause\item Consider each possible ``next state'', i.e.\ each possible move
		\pause\item If it's my turn, the value is the \textbf{maximum} value over next states
		\pause\item If it's my opponent's turn, the value is the \textbf{minimum} value over next states
	\end{itemize}
\end{frame}

\begin{frame}{Minimax search -- example}
	\begin{center}
		\texttt{minimax.pptx}
	\end{center}
\end{frame}

\begin{frame}{Minimax search pseudocode}
	\footnotesize
	\begin{algorithmic}
		\Procedure{Minimax}{state} \pause
			\If{state is terminal} \pause
				\State \textbf{return} value of state \pause
			\ElsIf{state.currentPlayer $= 1$} \pause
				\State bestValue $= -\infty$ \pause
				\For{\textbf{each} possible nextState} \pause
					\State $v$ = \Call{Minimax}{nextState} \pause
					\State bestValue = \Call{Max}{bestValue, $v$} \pause
				\EndFor
				\State \textbf{return} bestValue \pause
			\ElsIf{state.currentPlayer $= 2$} \pause
				\State bestValue $= +\infty$
				\For{\textbf{each} possible nextState}
					\State $v$ = \Call{Minimax}{nextState}
					\State bestValue = \Call{Min}{bestValue, $v$}
				\EndFor
				\State \textbf{return} bestValue 
			\EndIf
		\EndProcedure
	\end{algorithmic}
\end{frame}

\begin{frame}{Stopping early}
	\begin{algorithmic}
		\For{\textbf{each} possible nextState} 
			\State $v$ = \Call{Minimax}{nextState}
			\State bestValue = \Call{Max}{bestValue, $v$} 
		\EndFor
	\end{algorithmic}
	\begin{itemize}
		\pause\item State values are always between $-1$ and $+1$
		\pause\item So if we ever have bestValue $= 1$, we can stop early
		\pause\item Similarly when minimising if bestValue $= -1$
	\end{itemize}
\end{frame}

\begin{frame}{Using minimax search}
	\begin{itemize}
		\pause\item To decide what move to play next...
		\pause\item Calculate the minimax value for each move
		\pause\item Choose the move with the maximum score
		\pause\item If there are several with the same score, choose one at random
	\end{itemize}
\end{frame}

\begin{frame}{Minimax and game theory}
	\begin{itemize}
		\pause\item For a \textbf{two-player zero-sum} game
			with \textbf{perfect information} and \textbf{sequential moves}
		\pause\item Minimax search will always find a \textbf{Nash equilibrium}
		\pause\item I.e.\ a minimax player plays \textbf{perfectly}
		\pause\item \textbf{But...}
	\end{itemize}
\end{frame}

\begin{frame}{Minimax for larger games}
	\begin{itemize}
		\pause\item The game tree for noughts and crosses has only a few thousand states
		\pause\item Most games are too large to search fully
			\begin{itemize}
				\pause\item Connect 4 has $\approx 10^{13}$ states
				\pause\item Chess has $\approx 10^{47}$ states
			\end{itemize}
	\end{itemize}
\end{frame}

