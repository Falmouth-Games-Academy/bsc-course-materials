\part{Monte Carlo evaluation}
\frame{\partpage}

\begin{frame}{From last time}
	\begin{itemize}
		\pause\item It is useful to have a \textbf{heuristic evaluation function} for nonterminal states
		\pause\item Allows 1-ply search, depth-limited minimax, \dots
		\pause\item Designing a good heuristic requires in-depth knowledge of the game
		\pause\item What if you don't have such knowledge?
	\end{itemize}
\end{frame}

\begin{frame}{Expected value}
	\begin{itemize}
		\pause\item Let $X$ be a \textbf{random variable}
		\pause\item Let $p(x)$ be the probability that $X$ has value $x$
		\pause\item Then the \textbf{expected value} of $X$ is
			$$ \sum_x x \cdot p(x) $$
	\end{itemize}
\end{frame}

\begin{frame}{Expected value --- example}
	\begin{itemize}
		\pause\item A slot machine pays out:
			\begin{itemize}
				\pause\item $\pounds 1$ with probability $0.05$
				\pause\item $\pounds 5$ with probability $0.03$
				\pause\item $\pounds 10$ with probability $0.02$
				\pause\item Nothing with probability $0.9$
			\end{itemize}
		\pause\item The expected payout is
			$$ 1 \times 0.05 + 5 \times 0.03 + 10 \times 0.02 + 0 \times 0.9
				= 0.4 $$
			i.e.\ $\pounds 0.40$
		\pause\item What this means: if you play the slot machine $N$ times, on average you will win $N \times \pounds 0.40$
	\end{itemize}
\end{frame}

\begin{frame}{``Randomness'' in computing}
	\begin{itemize}
		\pause\item Digital computers are \textbf{deterministic}, so there's no such thing as true randomness
			\begin{itemize}
				\pause\item Cryptographically secure systems use an external source of randomness e.g.\ atmospheric noise, radioactive decay
			\end{itemize}
		\pause\item What we actually have are \textbf{pseudo-random number generators (PRNGs)}
		\pause\item A PRNG is an algorithm which gives an \textbf{unpredictable} sequence of numbers based on a \textbf{seed}
		\pause\item Sequence is \textbf{uniformly distributed}, i.e.\ all numbers have equal probability
		\pause\item Seed is generally based on some source of \textbf{entropy}, e.g.\ system clock, mouse input, electronic noise
	\end{itemize}
\end{frame}

\begin{frame}{Monte Carlo methods}
	\begin{itemize}
		\pause\item In computing, a \textbf{Monte Carlo method} is an algorithm based on \textbf{averaging over random samples}
		\pause\item The \textbf{average} over a large number of samples is a good approximation of the \textbf{expected value}
		\pause\item Used for \textbf{quickly approximating} quantities over \textbf{large domains}
		\pause\item Generally designed to \textbf{converge in the limit}
			\begin{itemize}
				\pause\item An \textbf{infinite} number of samples would give an \textbf{exact} answer
				\pause\item As the \textbf{number of samples} increases, the \textbf{accuracy} of the answer improves
			\end{itemize}
		\pause\item Applications in physics, engineering, finance, weather forecasting, graphics, ...
	\end{itemize}
\end{frame}

\begin{frame}{Monte Carlo evaluation in games}
	\begin{itemize}
		\pause\item Based on \textbf{random rollouts}
	\end{itemize}
	\pause
	\begin{algorithmic}
		\While{$s$ is not terminal}
			\State let $m$ be a random legal move from $s$
			\State update $s$ by playing $m$
		\EndWhile
	\end{algorithmic}
	\begin{itemize}
		\pause\item The \textbf{value} of a rollout is the \textbf{value} of the terminal state it reaches (i.e.\ $1$ for a win, $-1$ for a loss, $0$ for a draw)
		\pause\item Averaging gives the \textbf{expected value} of the initial state
		\pause\item Higher expected value $=$ more chance of winning
	\end{itemize}
\end{frame}

\begin{frame}{Monte Carlo search}
	\begin{itemize}
		\pause\item \textbf{Flat Monte Carlo search}: 1-ply search with Monte Carlo evaluation
		\pause\item How about minimax with $d>1$ and Monte Carlo evaluation?
			\begin{itemize}
				\pause\item Minimax assumes the evaluation is \textbf{deterministic}, but Monte Carlo is not
				\pause\item Not commonly used, mainly because there's something better...
			\end{itemize}
	\end{itemize}
\end{frame}

