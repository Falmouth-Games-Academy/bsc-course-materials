\part{Heuristics for search}
\frame{\partpage}

\begin{frame}{From session 2: Minimax search}
	\footnotesize
	\begin{algorithmic}
		\Procedure{Minimax}{state, currentPlayer} 
			\If{state is terminal} 
				\State \textbf{return} value of state 
			\ElsIf{currentPlayer is maximising} 
				\State bestValue $= -\infty$ 
				\For{\textbf{each} possible nextState} 
					\State $v$ = \Call{Minimax}{nextState, $3 -$ currentPlayer} 
					\State bestValue = \Call{Max}{bestValue, $v$} 
					\If{bestValue $\geq 1$}
						\State \textbf{break}
					\EndIf
				\EndFor
				\State \textbf{return} bestValue 
			\ElsIf{currentPlayer is minimising} 
				\State bestValue $= +\infty$
				\For{\textbf{each} possible nextState}
					\State $v$ = \Call{Minimax}{nextState, $3 -$ currentPlayer}
					\State bestValue = \Call{Min}{bestValue, $v$}
					\If{bestValue $\leq -1$}
						\State \textbf{break}
					\EndIf
				\EndFor
				\State \textbf{return} bestValue 
			\EndIf
		\EndProcedure
	\end{algorithmic}
\end{frame}

\begin{frame}{Minimax for larger games}
	\begin{itemize}
		\pause\item The game tree for noughts and crosses has only a few thousand states
		\pause\item Most games are too large to search fully
			\begin{itemize}
				\pause\item Connect 4 has $\approx 10^{13}$ states
				\pause\item Chess has $\approx 10^{47}$ states
			\end{itemize}
	\end{itemize}
\end{frame}

\begin{frame}{Depth limiting}
	\begin{itemize}
		\pause\item Standard minimax needs to search all the way to \textbf{terminal} (game over) states
		\pause\item \textbf{Depth limiting} is a common technique to apply minimax to larger games
		\pause\item Still evaluate terminal states as $+1$ / $0$ / $-1$
		\pause\item For nonterminal states at depth $d$, apply a heuristic evaluation instead of searching deeper
		\pause\item Evaluation is a number between $-1$ and $+1$, estimating the probable outcome of the game
	\end{itemize}
\end{frame}

\begin{frame}{1-ply search}
	\begin{itemize}
		\pause\item Case $d=1$
		\pause\item For each move, evaluate the state resulting from playing that move
		\pause\item This is computationally fast
		\pause\item Often easier to design a ``which state is better'' heuristic than to directly design a ``which move to play'' heuristic
	\end{itemize}
\end{frame}

%\begin{frame}{Move ordering}
	%\begin{itemize}
		%\pause\item Minimax can \textbf{stop early} if it sees a value of $+1$ for maximising player or $-1$
			%for minimising player
		%\pause\item Modifications to minimax algorithm (e.g.\ \textbf{alpha-beta pruning}) lead to more of this
		%\pause\item Thus ordering moves from \textbf{best to worst} means faster search
		%\pause\item How do we know which moves are ``best'' and ``worst''? Use a heuristic!
	%\end{itemize}
%\end{frame}

\begin{frame}{Designing heuristics}
	\begin{itemize}
		\pause\item The \textbf{playing strength} of depth limited minimax depends heavily on the design of the \textbf{heuristic}
		\pause\item Good heuristic design requires \textbf{in-depth knowledge} of the tactics and strategy of the game
		\pause\item What if we don't have that knowledge? ...
	\end{itemize}
\end{frame}

