\newcommand{\pictureslideb}[3]{
	\begin{frame}{#1}
		\begin{center}
			#3
			
			\vspace{6pt}
			
			\includegraphics[height=0.6\textheight]{#2}
		\end{center}
	\end{frame}
}

\newcommand{\pictureslide}[2]{
	\begin{frame}{#1}
		\begin{center}
			\includegraphics[height=0.6\textheight]{#2}
		\end{center}
	\end{frame}
}

\part{AI in games}
\frame{\partpage}

\begin{frame}{What is AI?}
	\begin{itemize}
		\pause\item Recall COMP280 session 7
		\pause\item Performing tasks by machine (or by software) which would ordinarily require human intelligence 
		\pause\item Making decisions to achieve goals
		\pause\item In games, AI systems break down roughly into two categories:
			\begin{itemize}
				\pause\item Authored behaviours: AI follows (often sophisticated) rules set out by a designer
				\pause\item Computational intelligence: AI behaviour emerges from an algorithmic system
			\end{itemize}
	\end{itemize}
\end{frame}

\pictureslide{Nimrod (Ferranti, 1951)}{nimrod}
\pictureslide{Samuel's Checkers program (IBM, 1962)}{samuel}
\pictureslide{Galaxian (Namco, 1979)}{galaxian}
\pictureslide{Pac-Man (Namco, 1980)}{pacman}
\pictureslide{Deep Blue (IBM, 1997)}{deep_blue}
\pictureslide{Half-Life (Valve, 1998)}{half_life}
\pictureslide{The Sims (Maxis, 2000)}{sims}
\pictureslide{Black \& White (Lionhead, 2001)}{black_white}
\pictureslide{Fa\c{c}ade (Mateas \& Stern, 2005)}{facade}
\pictureslide{Chinook (Schaeffer et al, 2007)}{chinook}
\pictureslide{Left 4 Dead (Valve, 2008)}{left_4_dead}
\pictureslide{Watson (IBM, 2011)}{watson}
\pictureslide{Deep learning for Atari games (DeepMind, 2013)}{deepmind_atari}
\pictureslide{AlphaGo (Google DeepMind, 2016)}{alphago}

\begin{frame}{What will we be covering?}
	\begin{itemize}
		\pause\item Finite state machines
		\pause\item Behaviour trees
		\pause\item Game theory
		\pause\item Planning
		\pause\item Utility-based AI
		\pause\item Game tree search
		\pause\item Procedural content generation
		\pause\item Multi-agent systems
		\pause\item Pathfinding and navigation
		\pause\item Evolutionary algorithms
		\pause\item Artificial neural networks
	\end{itemize}
\end{frame}
