% Adjust these for the path of the theme and its graphics, relative to this file
%\usepackage{beamerthemeFalmouthGamesAcademy}
\usepackage{../../beamerthemeFalmouthGamesAcademy}
\usepackage{multimedia}
\graphicspath{ {../../} }

% Default language for code listings
\lstset{language=C++,
        morekeywords={each,in,nullptr,int32, TCHAR, uint8, int8, uint16, int16,
        uint32, int32, uint64, int64, PTRINT, UObject. AActor, SWidget, FName,
        FString, UClass, USoundCue, UTexture}
}

% For strikethrough effect
\usepackage[normalem]{ulem}
\usepackage{wasysym}
\usepackage{listings}
\usepackage{pdfpages}

% http://www.texample.net/tikz/examples/state-machine/
\usetikzlibrary{arrows,automata}

\newcommand{\modulecode}{COMP260}\newcommand{\moduletitle}{Distributed Systems}\newcommand{\sessionnumber}{5}

\begin{document}
\title{\sessionnumber: Epic Coding Standards}
\subtitle{\modulecode: \moduletitle}

\frame{\titlepage}

\begin{frame}
	\frametitle{Learning outcomes}
	\begin{itemize}
		\item \textbf{Understand} Epic coding standards
		\item \textbf{Compare} different coding standards
		\item \textbf{Apply} Epic coding standard to your own Unreal Projects
	\end{itemize}
\end{frame}

\begin{frame}
  \frametitle{Why Coding Standards?}
  \begin{itemize}
    \item Aids in software maintance
    \item Improves readability and understandability
    \item Adds to the documentation of the project
  \end{itemize}
\end{frame}

\begin{frame}
  \frametitle{Naming Conventions 1}
  \begin{itemize}
    \item First letter of each word in name is capitalised, and no underscores between words
    \begin{itemize}
      \item E.g. Health and UPrimitiveComponent, not lastMouseCoordinates or delta\_coordinates
    \end{itemize}
    \item Type and variable names are nouns
    \item Method names are verbs that describe the effects or describe return value
    \item Names should be clear unambigous, and discriptive. Avoid over-abbreviation.
  \end{itemize}
\end{frame}

\begin{frame}
  \frametitle{Naming Conventions 2}
  \begin{itemize}
    \item All variables should be declared one at a time to allow commments
    \item All functions that return bool should ask a true/false questions
    \item A procedure(a function with no return) should use strong verb followed by an Object
    \item Prefix function parameters passed by reference with \textbf{Out}
  \end{itemize}
\end{frame}

\begin{frame}
  \frametitle{Naming Conventions 3}
  \begin{itemize}
    \item Type names are prefixed with an additional upper-case letter, to distinguish them from variable names.
    E.g. \textbf{FSkin} for type, and \textbf{Skin} is an instance of \textbf{FSkin}
    \begin{itemize}
      \item Template classes are prefixed by T
      \item Classes that inherit from UObject are prefixed by U
      \item Classes that inheirt from AActor are prefixed by A
      \item Classes that inheirt from SWidget are prefixed by S
      \item Classes that are Interfaces are prexfixed by I
      \item Enums are prexfixed by E
      \item Boolean variables must be prefixed by b (e.g. bIsDead or bHasFallen)
      \item Most other classes are prefixed by F
    \end{itemize}
  \end{itemize}
\end{frame}

\begin{frame}[fragile]
  \frametitle{Naming Convention Examples}
  \begin{lstlisting}
    float TeaWeight;

    int32 TeaCount;

    bool bDoesTeaStink;

    FName TeaName;

    FString TeaFriendlyName;

    UClass* TeaClass;

    USoundCue* TeaSound;

    UTexture* TeaTexture;

    bool IsTeaFresh(UTea Tea)
  \end{lstlisting}
\end{frame}

\begin{frame}
  \frametitle{Portable Aliases for C++ Types}
  \begin{table}
      \begin{tabular}{ | l | l | l | }
        \hline
        \textbf{Unreal Type} & \textbf{C++ Type} & \textbf{Size} \\
        \hline
        bool & bool or BOOL & Never assume size \\
        \hline
        TCHAR & TCHAR or char & Never assume size \\
        \hline
        uint8 & unsigned bytes & 1 byte \\
        \hline
        int8 & bytes & 1 byte \\
        \hline
        uint16 & unsigned short & 2 bytes \\
        \hline
        int16 & short & 2 bytes \\
        \hline
        uint32 & unsigned int & 4 bytes \\
        \hline
        int32 & int & 4 bytes \\
        \hline
        uint64 & unsigned long & 8 bytes \\
        \hline
        int64 & long & 8 bytes \\
        \hline
        float & float & 4 bytes \\
        \hline
        double & double & 8 bytes \\
        \hline
        PTRINT & void* & Never assume size \\
        \hline
      \end{tabular}
    \end{table}
\end{frame}

\begin{frame}[fragile]
  \frametitle{Comments - Write self documenting code}
  \begin{lstlisting}
    // Bad:
    t = s + 1 - b;

    // Good:
    TotalLeaves=SmallLeaves+LargeLeaves-SmallAndLargeLeaves;
  \end{lstlisting}
\end{frame}

\begin{frame}[fragile]
  \frametitle{Commets - Write Useful Commets}
  \begin{lstlisting}
    // Bad:
    // Increment Leaves
    ++Leaves;

    // Good:
    // we know there is another tea leaf
    ++Leaves;
  \end{lstlisting}
\end{frame}

\begin{frame}[fragile]
  \frametitle{Comments - Do not comment bad code - rewrite it}
  \begin{lstlisting}
    // Bad:
    // total number of leaves is sum of
    // small and large leaves less the
    // number of leaves that are both
    t = s + 1 - b;

    // Good:
    TotalLeaves=SmallLeaves+LargeLeaves-SmallAndLargeLeaves;
  \end{lstlisting}
\end{frame}

\begin{frame}[fragile]
  \frametitle{Comments - Do not contradict code}
  \begin{lstlisting}
    // Bad:
    // never increment Leaves!
    ++Leaves;

    //Good:
    // we know there is another leaf
    ++Leaves;
  \end{lstlisting}
\end{frame}

\begin{frame}
  \frametitle{Comments - Formatting}
  \begin{itemize}
    \item Unreal uses a system based on JavaDocs to extract comments to build documentation
    \item You have to use a specfic format in order for this tool to run
    \item More info - url{http://www.oracle.com/technetwork/articles/java/index-137868.html}
  \end{itemize}
\end{frame}

\begin{frame}
  \frametitle{Const Correctness}
  \begin{itemize}
    \item Const is documentation as much as a compiler directive
    \item If function arguments aren't modified by a function, ensure they are passed with const keyword
    \item Flag methods as const if they don't modify an object
    \item Use const interation over containers if the loop doesn't modify container
    \item Const should be preferred on by-value parameters and locals
  \end{itemize}
\end{frame}

\begin{frame}
  \frametitle{C++11 and Modern C++}
  \begin{itemize}
    \item \textbf{nullptr} - Used instead of NULL except on XBoxOne use \textbf{TYPE{\_}OF{\_}NULLPTR})
    \item \textbf{auto} - You should not use auto except in the following cases
    \begin{itemize}
      \item Binding to lambda types
      \item Iteration variables
      \item Template code (advance case)
    \end{itemize}
    \item \textbf{Range Based For} - Preferred to regular for loop
    \item \textbf{Lambdas} - Can be used, but be careful (see docs)
    \item \textbf{Enums} - Always use strongly type enums which inherit from \textbf{uint8}
    \item \textbf{Move Semantics} - All main container types have move constructors and move assignment
  \end{itemize}
\end{frame}

\begin{frame}
  \frametitle{Refrences}
  \url{https://docs.unrealengine.com/latest/INT/Programming/Development/CodingStandard/}
\end{frame}


\end{document}
