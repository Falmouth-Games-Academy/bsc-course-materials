\part{Polymorphism}
\frame{\partpage}

\begin{frame}{Introduction to Polymorphism}
	\begin{itemize}
		\pause \item Polymorphism is another key feature of OOP languages
		\pause \item The basic idea is that instances of a derived class can be treated as objects of the basic class
		\pause \item They can be used as parameters for functions and in collections
		\pause \item We then call the functions on these objects and our code will called the 'correct' version of the function
		\pause \item This is best illustrated by an example 
	\end{itemize}
\end{frame}

\begin{frame}[fragile]{Polymorphism example C\#}
		\begin{lstlisting}[language=C++,basicstyle=\tiny,]
		class Enemy{/*This has been define in previous slides*/}
		class Boss : Enemy{/*Again see previou slides*/}
		
		//This function will be in monobehavior
		void DoAttacks(Enemy enemy)
		{
			enemy.Attack();
		}
		
		//We probably have grabbed these from other game objects
		Enemy goblin=new Enemy();
		Eneny orc=new Enemy();
		Boss ogre=new Boss();
		
		//Call DoAttack on each one of these
		DoAttack(goblin);
		DoAttack(orc);
		DoAttack(ogre);
		
		//This even works if each instance is in a list
		List<Enemy> enemies=new List<Enemy>();
		enemies.Add(goblin);
		enemies.Add(orc);
		enemies.Add(ogre);
		
		foreach(Enemy e in enemies)
		{
			DoAttack(e);
		}
		\end{lstlisting}
\end{frame}

\begin{frame}[fragile]{Polymorphism example C++}
	\begin{lstlisting}[language=C++,basicstyle=\tiny,]
	class Enemy{/*This has been define in previous slides*/}
	class Boss : Enemy{/*Again see previou slides*/}
	
	//This function will be in monobehavior
	void DoAttacks(Enemy *enemy)
	{
		enemy->Attack();
	}
	
	//We probably have grabbed these from other game objects
	Enemy goblin=new Enemy();
	Eneny orc=new Enemy();
	Boss ogre=new Boss();
	
	//Call DoAttack on each one of these
	DoAttack(goblin);
	DoAttack(orc);
	DoAttack(ogre);
	
	//This even works if each instance is in a list
	std::vector<Enemy*> enemies;
	enemies.push_back(goblin);
	enemies.push_back(orc);
	enemies.push_back(ogre);
	
	for(Enemy * e : enemies)
	{
		DoAttack(e);
	}
	\end{lstlisting}
\end{frame}

