% Adjust these for the path of the theme and its graphics, relative to this file
%\usepackage{beamerthemeFalmouthGamesAcademy}
\usepackage{../../beamerthemeFalmouthGamesAcademy}
\usepackage{multimedia}
\graphicspath{ {../../} }

% Default language for code listings
\lstset{language=C++,
        morekeywords={each,in,nullptr,int32, TCHAR, uint8, int8, uint16, int16,
        uint32, int32, uint64, int64, PTRINT, UObject. AActor, SWidget, FName,
        FString, UClass, USoundCue, UTexture},
        breaklines=true,
        basicstyle=\tiny
}

% For strikethrough effect
\usepackage[normalem]{ulem}
\usepackage{wasysym}
\usepackage{listings}
\usepackage{pdfpages}

% http://www.texample.net/tikz/examples/state-machine/
\usetikzlibrary{arrows,automata}

\newcommand{\modulecode}{COMP260}\newcommand{\moduletitle}{Distributed Systems}\newcommand{\sessionnumber}{5}

\begin{document}
\title{\sessionnumber: Memory and Profiling}
\subtitle{\modulecode: \moduletitle}

\frame{\titlepage}

\begin{frame}
	\frametitle{Learning outcomes}
	\begin{itemize}
		\item \textbf{Understand} Memory in modern object orientated languages
		\item \textbf{Compare} memory models in managed and unmanaged languages
		\item \textbf{Understand} the role of the profiler in measuring performance in games
	\end{itemize}
\end{frame}

\begin{frame}
  \frametitle{Memory}
  \begin{itemize}
    \item Dynamic memory, allocated on the Heap and is growable
    \item Static memory, allocated on the Stack and is fixed size
  \end{itemize}
\end{frame}

\begin{frame}
  \frametitle{Stack Memory}
  \begin{itemize}
  	\item When you allocate value types (int, float, short, char etc), these are allocated on the stack
  	\item Values allocated on the stack are local, when they drop out of scope they are deallocated  
  	\item Values passed into functions are copied onto the stack
  	\item The stack is of fixed size
  	\begin{itemize}
  	\item C\# 1MB
  	\item C++ depends on compiler, Visual Studio is 1MB
  	\end{itemize}
  \end{itemize}
\end{frame}

\begin{frame}[fragile]
	\frametitle{Stack Memory Example 1}
	\begin{lstlisting}
		void Update()
		{
			int x=10;
			int y=10;
			
			Vector2 pos=Vector2(x,y);
		} //<-- x, y and pos drop out of scope here
	\end{lstlisting} 
\end{frame}

\begin{frame}[fragile]
	\frametitle{Stack Memory Example 2}
	\begin{lstlisting}
		class MonsterStats
		{
			private:
				int health;
				int strength;
			public:
				MonsterStats()
				{
					health=100;
					strength=10;
				}
			
				void ChangeHealth(int h)
				{
					health+=h;
				}//<- h drops out of scope here
			
				void ChangeStrength(int s)
				{
					strength+=s;
				}//<- s drops out of scope here
		}
		
		void main()
		{		
			//Create an instance of the class on the stack
			MonsterStats stats=MonsterStats();
			stats.ChangeHealth(10);
			stats.ChangeStrength(-2);
		}//<-- stats drops out of scope here
	\end{lstlisting}
\end{frame}

\begin{frame}
  \frametitle{Heap Memory}
  \begin{itemize}
  \item Otherwise known as dynamic memory
  \item Types allocated with the \textbf{new} keyword are allocated on the heap
  \item The new operator returns a reference to the type and can be allocated to a pointer (C++)
  \item This heap is managed by the programmer in C++ (see \textbf{delete} keyword) or the garbage collector in C\#
  \item In C++ is very important that you delete anything allocated on the heap
  \item \textbf{for every new, you need a matching delete}
  \item In the Unreal Engine objects can be Garbage Collected
  \end{itemize}
\end{frame}

\begin{frame}[fragile]
	\frametitle{Heap Memory Example 1 - C\#}
	\begin{lstlisting}
		public class MonsterStats
		{
			private int health;
			private int strength;
			public MonsterStats()
			{
				health=100;
				strength=10;
			}
			
			public void ChangeHealth(int h)
			{
				health+=h;
			}//<- h drops out of scope here
			
			void ChangeStrength(int s)
			{
				strength+=s;
			}//<- s drops out of scope here
		}
		
		void Start()
		{		
			//Create an instance of the class on the Heap
			MonsterStats new stats=MonsterStats();
			stats.ChangeHealth(10);
			stats.ChangeStrength(-2);
		}	
	\end{lstlisting}
\end{frame}

\begin{frame}[fragile]
	\frametitle{Heap Memory Example 2 - C++}
	\begin{lstlisting}
		class MonsterStats
		{
			private:
				int health;
				int strength;
			public:
				MonsterStats()
				{
					health=100;
					strength=10;
				}
				
				.....
		}
		
		void main()
		{		
			//Create an instance of the class on the Heap
			MonsterStats * stats=new MonsterStats();
			stats->ChangeHealth(10);
			stats->ChangeStrength(-2);
			
			if (stats)
			{
				delete stats;
				stats=nullptr;
			}
		}
	\end{lstlisting}
\end{frame}

\begin{frame}
  \frametitle{Types in C\#}
  \begin{itemize}
  	\item Value types are primitives such as int, bool, float etc
  	\item Structs are custom value types (see example)
  	\item Reference types are anything declare with the \textbf{class}, \textbf{interface} \& \textbf{delegate}
  	\item In addition to this strings are also reference types
  	\item Value types are allocated on stack
  	\item Reference type are allocated on the heap
  \end{itemize}
\end{frame}

\begin{frame}[fragile]
	\frametitle{Struct Example - C\#}
	\begin{lstlisting}
		public struct MonsterStats
		{
			private int health;
			private int strength;
			public MonsterStats()
			{
				health=100;
				strength=10;
			}
		}
		
		void Start()
		{		
			//Create an instance of the struct on the stack
			MonsterStats stats=new MonsterStats();
			stats.ChangeHealth(10);
			stats.ChangeStrength(-2);
		}	
	\end{lstlisting}
\end{frame}

\begin{frame}
	\frametitle{Passing Types}
	\begin{itemize}
	\item In C\#, when we call a method and pass some data as a parameter we either pass by value or we pass by reference. 
	\item We can mark a parameter with the \textbf{ref} or \textbf{out} keyword (see example)
	\item Reference types are always passed by reference
	\item In C++, we can also pass by value or reference. We can also pass in a pointer
	\item We can mark parameter with \textbf{\&} to pass by Reference 
	\item Larger data types should be passed by Pointer or Reference
	\end{itemize}
\end{frame}

\begin{frame}[fragile]
	\frametitle{Passing Example 1 - C\#}
	\begin{lstlisting}
		int x=10;
			
		void Adder(ref int value,int v)
		{
			value+=v;	
		}
		
		Adder(ref x,10);
		//x would now be 20 after this		
		
	\end{lstlisting}
\end{frame}

\begin{frame}[fragile]
	\frametitle{Passing Example 2 - C\#}
	\begin{lstlisting}
	
	void SetupMonster(ref MonsterStats stats, int health, int strength)
	{
		//if we use the ref keyword MonsterStats has to be initialised
		stats.health=health;
		stats.strength=strength;
	}
	
	void CreateMonster(out MonsterStats stats,int health, int strength)
	{
		//when we use out, it means we can initialise inside the function
		stats=new MonsterStats();
		stats.health=health;
		stats.strength=strength;
	}
	
	//Calling code
	MonsterStats goblinStats=new MonsterStats();
	SetupMonster(ref goblinStats,10,2);
	
	MonsterStats orcStats;
	CreateMonster(out orcStats,20,4);
	\end{lstlisting}
\end{frame}


\begin{frame}[fragile]
	\frametitle{Passing Example 1 - C++}
	\begin{lstlisting}
		int x=10;
			
		void Adder(int &value,int v)
		{
			value+=v;	
		}
		
		Adder(x,10);
		//x would now be 20 after this		
		
	\end{lstlisting}
\end{frame}

\begin{frame}[fragile]
	\frametitle{Passing Example 2 - C++}
	\begin{lstlisting}
	
	void SetupMonster(MonsterStats &stats, int health, int strength)
	{
		stats.health=health;
		stats.strength=strength;
	}
	
	//Calling code
	MonsterStats * goblinStats=new MonsterStats();
	SetupMonster(goblinStats,10,2);
	\end{lstlisting}
\end{frame}

\begin{frame}
  \frametitle{Strings}
  \begin{itemize}
  \item Strings act like value types but they are actually reference types (C\#)
  \item This means we need to be careful in allocating new strings
  \item \textbf{And} doing any operations using strings such as concatenation using +
  \item In C\# you should use the \textbf{StringBuilder} class 
  \item In C++ you should use the \textbf{stringstream} class
  \end{itemize}
\end{frame}
	
\begin{frame}
	\frametitle{Garbage Collection in C\#}
  	\begin{itemize}
  	\item In C\# there is an inbuilt Garbage Collector which will  walk through the Object Graph
	\item It will check to see if the object is still allocated, if not, the Garbage Collector will cleanup the object
	\item This process is automatic and is tuned for maximum performance
	\item However you should think of caching GetComponent calls using the \textbf{Start} or \textbf{Awake}
  	\end{itemize}
\end{frame}

\begin{frame}
	\frametitle{Memory Management in C++ \& Unreal}
	\begin{itemize}
	\item In C++ there is no Garbage Collection, you have to manually delete objects when no longer needed
  	\item Worth repeating  \textbf{for every new, you need a matching delete}
  	\item In the Unreal Engine there is Garbage Collection but you have to follow these rules
  	\begin{itemize}
  		\item Every Member of the class should be marked with \textbf{UPROPERTY}
  		\item Member pointers should only point to \textbf{UObject} or texttbf{UObject-derived objects}
  		\item The only container that is safe to have UObject or UObject-derived pointers in is a TArray
  	\end{itemize}
	\end{itemize}
\end{frame}

\begin{frame}
\frametitle{Exercises - BA students Only}
\begin{itemize}
	\item Debugging - Watch the following video - \url{https://unity3d.com/learn/tutorials/topics/scripting/debugging-unity-games-visual-studio}
	\item Debugging - Grab DebugExample.cs from learning space
	\item Put this in a Unity project and attach it to a game object in a new scene.
	\item Use the debugger to find answers to the questions Q1 and Q2 which are shown as comments within the Start() method.
	\item Profiling - Watch the following video - \url{https://unity3d.com/learn/tutorials/topics/interface-essentials/introduction-profiler}
\end{itemize}
\end{frame}

\end{document}
