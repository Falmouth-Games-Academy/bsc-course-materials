\part{Memory}
\frame{\partpage}

\begin{frame}{Memory Refresher}
	\begin{itemize}
		\pause \item Recall that:
		\begin{itemize}
			\pause \item Dynamic memory, allocated on the \textbf{Heap} and is \textbf{growable}
			\pause \item Static memory, allocated on the \textbf{Stack} and is \textbf{fixed size}
		\end{itemize}
	\end{itemize}
\end{frame}

\begin{frame}{Stack Memory}
	\begin{itemize}
		\pause \item When you allocate value types (int, float, short, char etc), these are allocated on the stack
		\pause \item Values allocated on the stack are local, when they drop out of scope they are deallocated  
		\pause \item Values passed into functions are copied onto the stack
		\pause \item The stack is of fixed size
		\begin{itemize}
			\item C\# - \textbf{1MB}
			\item C++ Visual Studio - \textbf{1MB}
		\end{itemize}
	\end{itemize}
\end{frame}

\begin{frame}[fragile]{Stack Memory Example 1}
	\begin{lstlisting}
		void Update()
		{
			int x=10;
			int y=10;
	
			Vector2 pos=Vector2(x,y);
		} //<-- x, y and pos drop out of scope here
	\end{lstlisting} 
\end{frame}

\begin{frame}[fragile]{Stack Memory Example 2}
	\begin{lstlisting}[language=C++,basicstyle=\tiny,]
		class MonsterStats
		{
		private:
			int health;
			int strength;
		public:
			MonsterStats()
			{
				health=100;
				strength=10;
			};
	
			void ChangeHealth(int h)
			{
				health+=h;
			};//<- h drops out of scope here
	
			void ChangeStrength(int s)
			{
				strength+=s;
			};//<- s drops out of scope here
		};
	
		void main()
		{		
			//Create an instance of the class on the stack
			MonsterStats stats=MonsterStats();
			stats.ChangeHealth(10);
			stats.ChangeStrength(-2);
		}//<-- stats drops out of scope here
	\end{lstlisting}
\end{frame}

\begin{frame}{Heap Memory}
	\begin{itemize}
	\pause \item Otherwise known as dynamic memory
	\pause \item Types allocated with the \textbf{new} keyword are allocated on the heap
	\pause \item The new operator returns a reference to the type and can be allocated to a pointer (C++)
	\pause \item This heap is managed by the programmer in C++ (see \textbf{delete} keyword) or the garbage collector in C\#
	\pause \item In C++ is very important that you delete anything allocated on the heap
	\pause \item \textbf{for every new, you need a matching delete}
	\pause \item In the Unreal Engine objects can be Garbage Collected
	\end{itemize}
\end{frame}

\begin{frame}[fragile]{Heap Memory Example 1 - C\#}
\begin{lstlisting}[language=C++,basicstyle=\tiny,]
	public class MonsterStats
	{
		private int health;
		private int strength;
		
		public MonsterStats()
		{
			health=100;
			strength=10;
		}
	
		public void ChangeHealth(int h)
		{
			health+=h;
		}//<- h drops out of scope here
	
		void ChangeStrength(int s)
		{
			strength+=s;
		}//<- s drops out of scope here
	}

	void Start()
	{		
		//Create an instance of the class on the Heap
		MonsterStats new stats=MonsterStats();
		stats.ChangeHealth(10);
		stats.ChangeStrength(-2);
	}	
\end{lstlisting}
\end{frame}

\begin{frame}[fragile]{Heap Memory Example 2 - C++}
\begin{lstlisting}[language=C++,basicstyle=\tiny,]
	class MonsterStats
	{
	private:
		int health;
		int strength;
	public:
		MonsterStats()
		{
			health=100;
			strength=10;
		}
		
	.....
	}
	
	void main()
	{		
		//Create an instance of the class on the Heap
		MonsterStats * stats=new MonsterStats();
		stats->ChangeHealth(10);
		stats->ChangeStrength(-2);
	
		if (stats)
		{
		delete stats;
		stats=nullptr;
		}
	}
\end{lstlisting}
\end{frame}

\begin{frame}{Data Types in C\#}
	\begin{itemize}
		\item Value types are primitives such as int, bool, float etc
		\item Structs are custom value types (see example)
		\item Reference types are anything declare with the \textbf{class}, \textbf{interface} \& \textbf{delegate}
		\item In addition to this strings are also reference types
		\item Value types are allocated on the stack
		\item Reference type are allocated on the heap
	\end{itemize}
\end{frame}

\begin{frame}[fragile]{Struct Example - C\#}
	\begin{lstlisting}[language=C++,basicstyle=\tiny,]
		public struct MonsterStats
		{
			private int health;
			private int strength;
			
			public MonsterStats()
			{
				health=100;
				strength=10;
			}
		}
	
		void Start()
		{		
			//Create an instance of the struct on the stack
			MonsterStats stats=new MonsterStats();
			stats.ChangeHealth(10);
			stats.ChangeStrength(-2);
		}	
	\end{lstlisting}
\end{frame}

\begin{frame}{Passing Types}
	\begin{itemize}
		\pause \item In C\#, when we call a method and pass some data as a parameter we either pass by value or we pass by reference. 
		\pause \item We can mark a parameter with the \textbf{ref} or \textbf{out} keyword (see example)
		\pause \item Reference types are always passed by reference
		\pause \item In C++, we can pass by value or reference. In addition to this, we can also pass in a pointer
		\pause \item We can mark parameter with \textbf{\&} to pass by Reference 
		\pause \item Larger data types should be passed by Pointer or Reference
	\end{itemize}
\end{frame}

\begin{frame}[fragile]{Passing Example 1 - C\#}
	\begin{lstlisting}
		int x=10;
		
		void Adder(ref int value,int v)
		{
			value+=v;	
		}
		
		Adder(ref x,10);
		//x would now be 20 after this		
	
	\end{lstlisting}
\end{frame}

\begin{frame}[fragile]{Passing Example 2 - C\#}
	\begin{lstlisting}[language=C++,basicstyle=\tiny,]
	
		void SetupMonster(ref MonsterStats stats, int health, int strength)
		{
			//if we use the ref keyword MonsterStats has to be initialised
			stats.health=health;
			stats.strength=strength;
		}
		
		void CreateMonster(out MonsterStats stats,int health, int strength)
		{
			//when we use out, it means we can initialise inside the function
			stats=new MonsterStats();
			stats.health=health;
			stats.strength=strength;
		}
		
		//Calling code
		MonsterStats goblinStats=new MonsterStats();
		SetupMonster(ref goblinStats,10,2);
		
		MonsterStats orcStats;
		CreateMonster(out orcStats,20,4);
	\end{lstlisting}
\end{frame}


\begin{frame}[fragile]{Passing Example 1 - C++}
	\begin{lstlisting}
		int x=10;
	
		void Adder(int &value,int v)
		{
			value+=v;	
		}
	
		Adder(x,10);
		//x would now be 20 after this		
	
	\end{lstlisting}
\end{frame}

\begin{frame}[fragile]{Passing Example 2 - C++}
	\begin{lstlisting}
	
		void SetupMonster(MonsterStats &stats, int health, int strength)
		{
			stats.health=health;
			stats.strength=strength;
		}
		
		//Calling code
		MonsterStats * goblinStats=new MonsterStats();
		SetupMonster(goblinStats,10,2);
	\end{lstlisting}
\end{frame}

