\part{Recursion}
\frame{\partpage}

\begin{frame}[fragile]{Recursion}
    \begin{itemize}
        \pause\item A \textbf{recursive} function is a function that \textbf{calls itself}
    \end{itemize}
    \pause
    \begin{lstlisting}
int factorial(int n)
{
    if (n <= 1)
        return 1;
    else
        return n * factorial(n-1);
}
    \end{lstlisting}
    \begin{itemize}
        \pause\item Recursive functions need a \textbf{base case} where they stop recursing,
			otherwise they will go \textbf{forever}
        \pause\item (Or rather, until a \textbf{stack overflow})
    \end{itemize}
\end{frame}

\begin{frame}{Thinking recursively}
	\begin{itemize}
		\pause\item I want to solve a problem
		\pause\item If I already had a function to solve smaller instances of the problem, I could use it
			to write my function
		\pause\item I can solve the smallest possible problem
		\pause\item Therefore I can write a recursive function
	\end{itemize}
\end{frame}

\begin{frame}{The call stack}
	\begin{itemize}
		\pause\item Recall: nested function calls are handled using a \textbf{stack}
		\pause\item Recursive functions are no different
		\pause\item This means if a recursive function contains \textbf{local variables},
			they are \textbf{independent} between instances of the function
	\end{itemize}
\end{frame}
