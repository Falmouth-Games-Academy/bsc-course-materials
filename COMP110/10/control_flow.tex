\part{Control flow in MIPS}
\frame{\partpage}

\begin{frame}[fragile]{Labels and jumping}
	\begin{itemize}
		\pause\item In assembly code, can set a \textbf{label} on any line:
	\end{itemize}
	\begin{lstlisting}
MyLabel: add $s0, $s1, 1
	\end{lstlisting}
	\begin{itemize}
		\pause\item Some instructions use labels to refer to a location in the code
		\pause\item E.g.\ the \lstinline{j} instruction
			simply jumps (backwards or forwards) to the specified line:
	\end{itemize}
	\begin{lstlisting}
j MyLabel
	\end{lstlisting}
\end{frame}

\begin{frame}[fragile]{Branching}
	\begin{itemize}
		\pause\item \textbf{Branching} is \textbf{conditional jumping}
	\end{itemize}
	\pause\begin{lstlisting}
beq $s, $t, Label
	\end{lstlisting}
	\begin{itemize}
		\pause\item This jumps to \lstinline{Label} \textbf{if and only if}
			the value of \lstinline{$s} equals the value of \lstinline{$t}
	\end{itemize}
	\pause\begin{lstlisting}
bne $s, $t, Label
	\end{lstlisting}
	\begin{itemize}
		\pause\item This jumps to \lstinline{Label} \textbf{if and only if}
			the value of \lstinline{$s} does not equal the value of \lstinline{$t}
	\end{itemize}
\end{frame}

\begin{frame}[fragile]{Conditionals}
	\begin{itemize}
		\pause\item Branching allows us to implement \textbf{if statements}
	\end{itemize}
	\pause
	\begin{columns}
		\begin{column}{0.43\textwidth}
			\begin{lstlisting}[language=Python,showlines=true]
if s0 != 0:
    s1 += 1
else:
    s2 += 1

			\end{lstlisting}
		\end{column}
		\pause
		\begin{column}{0.53\textwidth}
			\begin{lstlisting}
beq $s0, $zero, Else
addi $s1, $s1, 1
j End
Else: addi $s2, $s2, 1
End:
			\end{lstlisting}
		\end{column}
	\end{columns}
\end{frame}

\begin{frame}[fragile]{Loops}
	\begin{itemize}
		\pause\item Branching allows us to implement \textbf{while loops}
	\end{itemize}
	\pause
	\begin{columns}
		\begin{column}{0.43\textwidth}
			\begin{lstlisting}[language=Python,showlines=true]
i = 0
total = 0
limit = 10

while i != limit:
    total += i
    i += 1
#�end while

			\end{lstlisting}
		\end{column}
		\pause
		\begin{column}{0.53\textwidth}
			\begin{lstlisting}
addi $s0, $zero, 0
addi $s1, $zero, 0
addi $s2, $zero, 10

Loop: beq $s0, $s2, LoopEnd
add $s1, $s1, $s0
addi $s0, $s0, 1
j Loop
LoopEnd:
			\end{lstlisting}
		\end{column}
	\end{columns}
\end{frame}

\begin{frame}[fragile]{Exercise}
	Write a piece of Python code equivalent to the following:
	\begin{lstlisting}
addi $s0, $zero, 10
addi $s1, $zero, 0

Loop: beq $s0, $zero, LoopEnd
add $s1, $s1, $s0
addi $s0, $s0, -1
j Loop
LoopEnd:
	\end{lstlisting}
\end{frame}

\begin{frame}[fragile]{Not quite a while loop}
	\begin{itemize}
		\item Socrative \texttt{FALCOMPED}
		\item What is the difference between these two programs?
		\item (NB: assume \lstinline{$s0} $\geq 0$)
	\end{itemize}
	\begin{columns}
		\begin{column}{0.48\textwidth}
			\begin{lstlisting}
addi $s1, $zero, 0

Loop: beq $s0, $zero, LoopEnd
add $s1, $s1, $s0
addi $s0, $s0, -1
j Loop
LoopEnd:
			\end{lstlisting}
		\end{column}
		\begin{column}{0.48\textwidth}
			\begin{lstlisting}
addi $s1, $zero, 0

Loop: add $s1, $s1, $s0
addi $s0, $s0, -1
bne $s0, $zero, Loop
			\end{lstlisting}
		\end{column}
	\end{columns}
	\pause The code on the right implements a \textbf{do-while} loop (which Python doesn't have, but other languages do)
\end{frame}

\begin{frame}{Function calls}
	\begin{itemize}
		\pause\item Function calls can be implemented using the jump instruction:
			\begin{itemize}
				\pause\item To call the function: save the address of the instruction after the current one, then jump to the function
				\pause\item To return from the function: jump to the previously saved address
			\end{itemize}
		\pause\item MIPS has \lstinline{jal} and \lstinline{jr} instructions and \lstinline{$ra} register for this purpose
		\pause\item Socrative \texttt{FALCOMPED}: why save the return address? Why not just hard-code it into the program?
		\pause\item Nested function calls require a \textbf{stack} of return addresses
	\end{itemize}
\end{frame}
