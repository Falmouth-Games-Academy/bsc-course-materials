\part{Control structures}
\frame{\partpage}

\begin{frame}[fragile]{If statement}
	\begin{lstlisting}
if (x > 0)
{
    printf("x is positive\n");
}
else if (x < 0)
{
    printf("x is negative\n");
}
else
{
    printf("x is neither positive nor negative\n");
}
	\end{lstlisting}
\end{frame}

\begin{frame}[fragile]{If statement}
    \begin{itemize}
        \item Works just like the \lstinline{if} statement in Python \pause
        \item There can be zero, one or many \lstinline{else if} clauses \pause
        \item The \lstinline{else} clause is optional, but if present then there can only be one
    \end{itemize}
\end{frame}

\begin{frame}[fragile]{Conditions}
    \begin{itemize}
        \item Numerical comparison operators work just like Python:
        \lstinline{==  !=  <  >  <=  >=}
        \pause
        \item Boolean logic operators look a little different
    \end{itemize}
    \pause
    Python uses \lstinline[language=Python]{and, or, not}
    \begin{lstlisting}[language=Python]
if not (x < 0 or x > 100) and not (y < 0 or y > 100):
    print "Point is in rectangle"
    \end{lstlisting}
    \pause
    C++ uses \lstinline{&&, ||, !}
    \begin{lstlisting}[language=Python]
if (!(x < 0 || x > 100) && !(y < 0 || y > 100))
{
    printf("Point is in rectangle\n");
}
    \end{lstlisting}
\end{frame}

\begin{frame}[fragile]{Single-statement blocks}
    \begin{itemize}
        \item In many cases, if a block contains only a single statement then the curly braces can be omitted
    \end{itemize}
    \begin{lstlisting}
if (x > 0)
    printf("x is positive\n");
else if (x < 0)
    printf("x is negative\n");
else
    printf("x is neither positive nor negative\n");
    \end{lstlisting}
\end{frame}

\begin{frame}[fragile]{Single-statement blocks}
    \begin{itemize}
        \item Careful though! This can lead to obscure bugs
    \end{itemize}
    \begin{lstlisting}
if (z == 0)
    x = 0; y = 0;
    \end{lstlisting}
		Socrative \texttt{FALCOMPED}: what's wrong with this?
\end{frame}

%\begin{frame}[fragile]{Switch statement}
	%\begin{lstlisting}
%switch (x)
%{
%case 0:
    %printf("zero\n");
    %break;
%case 1:
    %printf("one\n");
    %break;
%case 2:
    %printf("two\n");
    %break;
%default:
    %printf("something else\n");
    %break;
%}
	%\end{lstlisting}
%\end{frame}

\begin{frame}[fragile]{While loop}
	\begin{lstlisting}
while (x > 0)
{
    printf("%d\n", x);
    x--;
}
	\end{lstlisting}
    \begin{itemize}
        \item Same as Python
    \end{itemize}
\end{frame}

\begin{frame}[fragile]{Do-while loop}
	\begin{lstlisting}
do
{
    printf("%d\n", x);
    x--;
} while (x > 0);
	\end{lstlisting}
	\pause
	\begin{itemize}
		\item \lstinline{while} loop checks the condition \textbf{before} executing the loop body \pause
		\item \lstinline{do-while} loop checks the condition \textbf{after} executing the loop body \pause
		\item e.g. if \lstinline{x == 0} to begin with, the \lstinline{while} body does not execute, the \lstinline{do-while} body executes once
	\end{itemize}
\end{frame}

\begin{frame}[fragile]{For-each loop}
	\begin{lstlisting}
std::vector<int> numbers { 1, 3, 5, 7, 9 };

for (int x : numbers)
{
    printf("%d\n", x);
}
	\end{lstlisting}
    \pause
	\begin{itemize}
		\item This works like the \lstinline{for} loop in Python \pause
		\item Used for iterating over data structures \pause
		\item For iterating over ranges of numbers, C++ has something different...
	\end{itemize}
\end{frame}

\begin{frame}[fragile]{For loop}
	\begin{lstlisting}
for (int i = 0; i < 10; i++)
{
    printf("%d\n", i);
}
	\end{lstlisting}
	\pause
    \begin{itemize}
        \item The \lstinline{for} loop has three parts: \pause
        \item The \textbf{initialiser} \lstinline{int i = 0}
        \begin{itemize}
            \item This is executed at the start of the loop
        \end{itemize} \pause
        \item The \textbf{condition} \lstinline{i < 10}
        \begin{itemize}
            \item The loop executes while this evaluates to \lstinline{true}
        \end{itemize} \pause
        \item The \textbf{loop statement} \lstinline{i++}
        \begin{itemize}
            \item This is executed at the end of each iteration of the loop
            \item \lstinline{i++} means ``increment \lstinline{i}'' --- this is shorthand for \lstinline{i = i + 1}
        \end{itemize}
    \end{itemize}
\end{frame}

\begin{frame}[fragile]{For loops and while loops}
	\begin{lstlisting}
for (int i = 0; i < 10; i++)
{
    printf("%d\n", i);
}
	\end{lstlisting}
    \begin{itemize}
        \item Any \lstinline{for} loop can easily be rewritten as a \lstinline{while} loop
    \end{itemize}
    \pause
	\begin{lstlisting}
int i = 0;
while (i < 10)
{
    printf("%d\n", i);
    i++;
}
	\end{lstlisting}
\end{frame}

\begin{frame}[fragile]{For loops in C++ and Python}
	\begin{lstlisting}
for (int i = 0; i < 10; i++)
{
    printf("%d\n", i);
}
	\end{lstlisting}
	\pause
    \begin{itemize}
        \item In Python, this would be written as a for-each loop, first using the \lstinline[language=Python]{xrange} function
        to construct the range of numbers $0, 1, 2, \dots, 9$:
    \end{itemize}
	\begin{lstlisting}[language=Python]
for i in xrange(10):
    print i
	\end{lstlisting}
	\pause
    \begin{itemize}
        \item The C++ way doesn't require construction of an iterable object, so is more efficient
    \end{itemize}
\end{frame}

\begin{frame}[fragile]{Socrative \texttt{FALCOMPED}}
    What would the first code fragment print?
	\begin{lstlisting}
for (int i = 0; i < 10; i++)
    printf("%d ", i);
	\end{lstlisting}
\end{frame}

\begin{frame}[fragile]{Socrative \texttt{FALCOMPED}}
    What would the second code fragment print?
	\begin{lstlisting}
for (int i = 0; i <= 10; i++)
    printf("%d ", i);
	\end{lstlisting}
\end{frame}

\begin{frame}[fragile]{Socrative \texttt{FALCOMPED}}
    What would the third code fragment print?
	\begin{lstlisting}
for (int i = 0; i < 10; i += 2)
    printf("%d ", i);
	\end{lstlisting}
\end{frame}

\begin{frame}[fragile]{Socrative \texttt{FALCOMPED}}
    What would the fourth code fragment print?
	\begin{lstlisting}
for (int i = 10; i < 0; i++)
    printf("%d ", i);
	\end{lstlisting}
\end{frame}

\begin{frame}[fragile]{Socrative \texttt{FALCOMPED}}
    What would the fifth code fragment print?
	\begin{lstlisting}
for (int i = 10; i > 0; i++)
    printf("%d ", i);
	\end{lstlisting}
\end{frame}

\begin{frame}[fragile]{Socrative \texttt{FALCOMPED}}
    What would the sixth code fragment print?
	\begin{lstlisting}
for (int i = 10; i > 0; i--)
    printf("%d ", i);
	\end{lstlisting}
\end{frame}
