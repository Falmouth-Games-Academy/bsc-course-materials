\part{Converting types}
\frame{\partpage}

\begin{frame}{Weak vs strong typing}
	\begin{itemize}
		\pause\item In \textbf{weakly typed} languages, a variable can hold a value of any type
        	\begin{itemize}
        	    \pause\item Examples: Python, JavaScript
        	\end{itemize}
		\pause\item In \textbf{strongly typed} languages, the type of a variable must be \textbf{declared}
        	\begin{itemize}
        	    \pause\item Examples: C\#, C++, Java
        	\end{itemize}
	\end{itemize}
\end{frame}

\begin{frame}[fragile]{Weak typing (example in Python)}
    \begin{lstlisting}
x = 7
# Now x has type int

x = "hello"
# Now x has type string
    \end{lstlisting}
\end{frame}

\begin{frame}[fragile]{Strong typing (example in C\#)}
    \begin{lstlisting}[language=C]
int x = 7;
// x is declared with type int

x = "hello";
// Compile error: cannot convert type "string" to "int"
    \end{lstlisting}
\end{frame}

\begin{frame}{Type casting}
	\begin{itemize}
		\pause\item It is often useful to \textbf{cast}, or \textbf{convert}, a value from one type to another
		\pause\item In Python, this is done by calling the type as if it were a function
			\begin{itemize}
				\pause\item \lstinline{float(17)} $\to$ \lstinline{17.0}
				\pause\item \lstinline{int(3.14)} $\to$ \lstinline{3}
				\pause\item \lstinline{str(3.14)} $\to$ \lstinline{"3.14"}
				\pause\item \lstinline{str(1 + 1 == 2)} $\to$ \lstinline{"True"}
				\pause\item \lstinline{int("123")} $\to$ \lstinline{123}
				\pause\item \lstinline{int("five")} gives an error
			\end{itemize}
	\end{itemize}
\end{frame}

\begin{frame}{Operations on types}
	\begin{itemize}
		\pause\item Certain operations can only be done on certain types of values
		\pause\item Can add two ints: \lstinline{2 + 3} $\to$ \lstinline{5}
		\pause\item Can add int and float: \lstinline{2 + 3.1} $\to$ \lstinline{5.1}
		\pause\item Can add two strings: \lstinline{"COMP" + "110"} $\to$ \lstinline{"COMP110"}
		\pause\item Can't add string and int: \lstinline{"COMP" + 110} $\to$ error
	\end{itemize}
\end{frame}

\begin{frame}{Implicit type conversion}
	\begin{itemize}
		\pause\item The type casts we saw a few slides ago are \textbf{explicit}
		\pause\item Some languages (not Python) can perform \textbf{implicit} type casts to make operations work
		\pause\item Sometimes called \textbf{type coercion}
		\pause\item E.g.\ in JavaScript, \lstinline{"COMP" + 110} $\to$ \lstinline{"COMP110"}
		\pause\item The integer \lstinline{110} is implicitly converted to a string \lstinline{"110"} to make the addition work
		\pause\item Equivalent in Python with explicit casts: \lstinline{"COMP" + str(110)}
	\end{itemize}
\end{frame}

\begin{frame}{Dangers of implicit type conversion}
	\begin{itemize}
		\pause\item Rules for implicit type conversion can sometimes be confusing
		\pause\item E.g.\ in JavaScript:
			\begin{itemize}
				\pause\item \lstinline{"5" + 3} $\to$ \lstinline{"53"}
				\pause\item \lstinline{"5" - 3} $\to$ \lstinline{2}
			\end{itemize}
	\end{itemize}
\end{frame}

