\part{Computability}
\frame{\partpage}

\begin{frame}{Computability theory}
	\begin{itemize}
		\pause\item Let $A$ and $B$ be \textbf{sets} of elements
			\begin{itemize}
				\pause\item NB: $A$ may be \textbf{infinite}
			\end{itemize}
		\pause\item A function $f : A \to B$ is \textbf{computable} if there exists a Turing machine
			which computes $f$
			\begin{itemize}
				\pause\item I.e.\ given an encoding of $a \in A$ as input, the Turing machine outputs an encoding of
					$f(a)$
			\end{itemize}
	\end{itemize}
\end{frame}

\begin{frame}{An uncomputable function}
	The \textbf{halting problem}
	\begin{itemize}
		\pause\item $A$ = the set of all Turing machines
		\pause\item $B = \{ \operatorname{true}, \operatorname{false} \}$
		\pause\item $f(a) = \begin{cases}
			\operatorname{true} & \text{ if $a$ halts in finite time on all inputs} \\
			\operatorname{false} & \text{ otherwise}
		\end{cases}$
		\pause\item There is \textbf{no} Turing machine that computes $f$
		\pause\item $f$ is \textbf{uncomputable}
	\end{itemize}
\end{frame}

\begin{frame}{Turing completeness}
	\begin{itemize}
		\pause\item A system (e.g.\ a computer or programming language) is \textbf{Turing complete}
			if it can implement any given Turing machine
	\end{itemize}
\end{frame}

\begin{frame}{Church-Turing Thesis}
	\begin{itemize}
		\pause\item If a function is \textbf{effectively calculable}, then it is \textbf{computable} by a Turing machine
		\pause\item Effectively calculable = there is a method or algorithm for computing it
		\pause\item So in terms of computability, Turing machines are as powerful as computers can be
	\end{itemize}
\end{frame}

\begin{frame}{Halting revisited}
	\begin{itemize}
		\pause\item Write a software tool that, given a Python program, predicts whether that program can go into an infinite loop
		\pause\item Your tool must work for \textbf{all} Python programs
		\pause\item Is this possible?
	\end{itemize}
\end{frame}
