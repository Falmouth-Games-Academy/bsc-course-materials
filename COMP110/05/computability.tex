\part{Computability}
\frame{\partpage}

\begin{frame}{Computability theory}
	\begin{itemize}
		\pause\item Let $A$ and $B$ be \textbf{sets} of elements
			\begin{itemize}
				\pause\item NB: $A$ may be \textbf{infinite}
			\end{itemize}
		\pause\item A function $f : A \to B$ is \textbf{computable} if there exists a Turing machine
			which computes $f$
			\begin{itemize}
				\pause\item I.e.\ given an encoding of $a \in A$ as input, the Turing machine outputs an encoding of
					$f(a)$
			\end{itemize}
	\end{itemize}
\end{frame}

\begin{frame}{An uncomputable function}
	The \textbf{halting problem}
	\begin{itemize}
		\pause\item $A$ = the set of all Turing machines (encoded as transition tables)
		\pause\item $B = \{ \operatorname{true}, \operatorname{false} \}$
		\pause\item $f(a) = \begin{cases}
			\operatorname{true} & \text{ if $a$ halts in finite time on all inputs} \\
			\operatorname{false} & \text{ otherwise}
		\end{cases}$
		\pause\item There is \textbf{no} Turing machine that computes $f$
		\pause\item $f$ is \textbf{uncomputable}
	\end{itemize}
\end{frame}

\begin{frame}{Computability and the Church-Turing Thesis}
	\begin{itemize}
		\pause\item Church-Turing tells us that Turing machines are as powerful as any other computer
		\pause\item Therefore if a function is uncomputable, there is \textbf{no conceivable machine} that can compute it
	\end{itemize}
\end{frame}

\begin{frame}{The halting problem}
	\begin{itemize}
		\pause\item Write a software tool that, given a C\# program, predicts whether that program can go into an infinite loop
		\pause\item Your tool must work for \textbf{all} C\# programs, considering \textbf{all} possible inputs to the program
		\pause\item This task is impossible!
	\end{itemize}
\end{frame}
