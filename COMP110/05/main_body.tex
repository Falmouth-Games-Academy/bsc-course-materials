% Adjust these for the path of the theme and its graphics, relative to this file
%\usepackage{beamerthemeFalmouthGamesAcademy}
\usepackage{../../beamerthemeFalmouthGamesAcademy}
\usepackage{multimedia}
\graphicspath{ {../../} }

% Default language for code listings
\lstset{language=Python
}

% For strikethrough effect
\usepackage[normalem]{ulem}
\usepackage{wasysym}

\usepackage{algpseudocode}

\usepackage{pdfpages}

% http://www.texample.net/tikz/examples/state-machine/
\usetikzlibrary{arrows,automata}

\newcommand{\modulecode}{COMP140 GAM160}\newcommand{\moduletitle}{Hacking Hardware/Advanced Programming}\newcommand{\sessionnumber}{Session 6}

\begin{document}
\title{\sessionnumber: Computational Complexity}
\subtitle{\modulecode: \moduletitle}

\frame{\titlepage} 

\begin{frame}
	\frametitle{Learning outcomes}
	\begin{itemize}
		\item \textbf{Explain} the notion of computability
		\item \textbf{Use} ``big $O$'' notation to express computational complexity
		\item \textbf{Apply} appropriate algorithms to achieve efficiency
	\end{itemize}
\end{frame}

\begin{frame}{Worksheet C}
	\begin{itemize}
		\item Computational complexity
		\item Due in class on \textbf{Monday 24th October} (next week)
	\end{itemize}
\end{frame}

\begin{frame}{Reading}
	E.\ G.\ Gilbert, D.\ W.\ Johnson, and S.\ S.\ Keerthi, 1988.
	A Fast Procedure for Computing the Distance Between Complex Objects in Three-Dimensional Space.
	\emph{IEEE Journal of Robotics and Automation},
	4(2):193--203.
\end{frame}

\part{Computation time}
\frame{\partpage}

\begin{frame}{Resources}
	\begin{itemize}
		\pause\item All programs use \textbf{resources}
			\begin{itemize}
				\pause\item Time
				\pause\item Memory
				\pause\item Network bandwidth
				\pause\item Power
				\pause\item ...
			\end{itemize}
		\pause\item Often \textbf{time} is the resource we care about the most
			\begin{itemize}
				\pause\item Particularly in games:
					want to maintain a good \textbf{frame rate}
					free of \textbf{lag} or \textbf{stuttering}
				\pause\item To run at 60 frames per second, we only have \textbf{16.666~milliseconds} to do everything that needs to be done on every frame
			\end{itemize}
	\end{itemize}
\end{frame}

\begin{frame}[fragile]{Basic time measurement in Python}
	\begin{lstlisting}
import time

start_time = time.perf_counter()

# ... do something here ...

end_time = time.perf_counter()
print("Time:", end_time - start_time, "seconds")
	\end{lstlisting}
	
	\begin{itemize}
		\pause\item \lstinline{time.perf_counter()} gives the ``current time'' in seconds
		\pause\item On Windows, this is the time since you first called \lstinline{time.perf_counter()}
		\pause\item Means little by itself, but \textbf{comparing} two values tells us how much time has \textbf{elapsed}
	\end{itemize}
\end{frame}

\begin{frame}[fragile]{Repeating for better accuracy}
	\begin{lstlisting}
import time

start_time = time.perf_counter()

repetition_count = 1000

for repetition in range(repetition_count):
    # ... do something here ...

end_time = time.perf_counter()
total_time = end_time - start_time
print("Time:", total_time, "seconds")
	\end{lstlisting}
	
	\begin{itemize}
		\pause\item There is some \textbf{overhead} from the \lstinline{for} loop, but in practice it is negligible
	\end{itemize}
\end{frame}

\begin{frame}
	\lstinputlisting[basicstyle=\scriptsize\ttfamily]{time_list_creation.py}
\end{frame}

\begin{frame}
	\lstinputlisting[basicstyle=\scriptsize\ttfamily]{time_list_append.py}
\end{frame}

\begin{frame}{Workshop exercise}
	\begin{itemize}
		\pause\item Investigate various operations on Python lists, and how their running time varies with the size of the list
		\pause\item For each of the operations listed on the next slide:
			\begin{itemize}
				\pause\item \textbf{Find out} how to do the operation
				\pause\item \textbf{Write} code similar to the previous slides, to generate a list of size $n$ (for various values of $n$) and then time the operation on that list
				\pause\item \textbf{Plot} graphs of the operations using Excel
				\pause\item (Advanced mode: instead of using Excel, plot graphs directly from Python using the Matplotlib library)
			\end{itemize}
	\end{itemize}
\end{frame}

\begin{frame}{Operations to time}
	\footnotesize
	\begin{columns}
		\begin{column}{0.45\textwidth}
			\begin{itemize}
				\item Append an element
				\item Insert an element at the beginning
				\item Insert an element at a random position
				\item Delete the first element
				\item Delete the last element
				\item Delete a random element
				\item Get the first element
				\item Get the last element
				\item Get a random element
				\item Find if the list contains a specific element
			\end{itemize}
		\end{column}
		\begin{column}{0.45\textwidth}
			\begin{itemize}
				\item Get the smallest element
				\item Get the largest element
				\item Get the sum of all elements
				\item Get the length of the list
				\item Copy the list
				\item Reverse the list
				\item Sort the list
				\item Randomly shuffle the list
				\item Convert the list to string
			\end{itemize}
		\end{column}
	\end{columns}
\end{frame}


\part{Search}
\frame{\partpage}

\newcommand{\namesunsorted}{
	\fbox{\parbox{0.9\textwidth}{\tiny
		Anderson, Martha \par
		Parker, Debra \par
		Russell, Mildred \par
		Stewart, Howard \par
		White, Amanda \par
		Perez, Diana \par
		Lewis, Rose \par
		Scott, Michelle \par
		Davis, Marilyn \par
		Cox, Shirley \par
		Young, Frank \par
		Collins, Jane \par
		Kelly, Philip \par
		Miller, Jeremy \par
		Clark, Stephanie \par
		Brown, Janet \par
		Diaz, Harold \par
		Hughes, Aaron \par
		Sanders, Phillip \par
		Williams, Billy \par
		Henderson, Lawrence \par
		Baker, Theresa \par
		Gonzalez, Adam \par
		Lopez, Jeffrey \par
		Ward, Jessica
	}}
}

\newcommand{\namessorted}{
	\fbox{\parbox{0.9\textwidth}{\tiny
		Anderson, Martha \par
		Baker, Theresa \par
		Brown, Janet \par
		Clark, Stephanie \par
		Collins, Jane \par
		Cox, Shirley \par
		Davis, Marilyn \par
		Diaz, Harold \par
		Gonzalez, Adam \par
		Henderson, Lawrence \par
		Hughes, Aaron \par
		Kelly, Philip \par
		Lewis, Rose \par
		Lopez, Jeffrey \par
		Miller, Jeremy \par
		Parker, Debra \par
		Perez, Diana \par
		Russell, Mildred \par
		Sanders, Phillip \par
		Scott, Michelle \par
		Stewart, Howard \par
		Ward, Jessica \par
		White, Amanda \par
		Williams, Billy \par
		Young, Frank
	}}
}

\begin{frame}{Search}
			\begin{itemize}
				\item We have a list of names, each with some data associated \pause
				\item We want to find one of them
			\end{itemize}
\end{frame}

\begin{frame}{Linear search}
			\begin{algorithmic}
				\Procedure{Find}{name, list} \pause
					\For{each item in list} \pause
						\If{item.name $=$ name} \pause
							\State \textbf{return} item \pause
						\EndIf
					\EndFor
					\State \textbf{raise error} ``Not found'' \pause
				\EndProcedure
			\end{algorithmic}
\end{frame}

\begin{frame}{How long does it take?}
	Socrative room code: \texttt{FALCOMPED}
	\begin{itemize}
		\item Suppose there are 25 items in the list \pause
		\item In the \textbf{best case}, how many items do we need to visit before finding the one we want? \pause
		\item How about in the \textbf{worst case}?
	\end{itemize}
\end{frame}

\begin{frame}{How long does it take?}
	Socrative room code: \texttt{FALCOMPED}
	\begin{itemize}
		\item If there are 25 items in the list, the \textbf{worst case} number of items visited is 25 \pause
		\item How about if there are 50 items? \pause
		\item How about 100 items? \pause
		\item If the number of items \textbf{doubles}, what happens to the amount of time the search takes?
	\end{itemize}
\end{frame}

\begin{frame}{Linear time}
	\begin{columns}
		\begin{column}{0.45\textwidth}
			\includegraphics[width=\textwidth]{plot2_linear}
		\end{column}
		\begin{column}{0.55\textwidth}
			\begin{itemize}
				\item The running time of linear search is \textbf{proportional} to the size $n$ of the list \pause
				\item Linear search is said to have \textbf{linear time complexity} \pause
				\item Also written as \textbf{$O(n)$ time complexity}
			\end{itemize}
		\end{column}
	\end{columns}
\end{frame}

\begin{frame}{Searching a sorted list}
			\begin{itemize}
				\item If the list is \textbf{sorted} in alphabetical order, we can do better than linear...
			\end{itemize}
\end{frame}

\begin{frame}{Binary search}
	\begin{algorithmic}
		\Procedure{Find}{name, list} \pause
			\If{list is empty}
				\State \textbf{raise error} ``Not found''
			\EndIf \pause
			\State mid $\gets$ the ``middle'' item of the list \pause
			\If{name $=$ mid.name}
				\State \textbf{return} mid \pause
			\ElsIf{name $<$ mid.name}
				\State \textbf{return} \Call{Find}{name, first half of list} \pause
			\ElsIf{name $>$ mid.name}
				\State \textbf{return} \Call{Find}{name, second half of list} \pause
			\EndIf
		\EndProcedure
	\end{algorithmic}
\end{frame}

\begin{frame}{Find ``Lopez, Jeffrey''}
	\begin{columns}
		\begin{column}{0.3\textwidth}
			\fbox{\parbox{0.9\textwidth}{\tiny
				$\phantom\longrightarrow$ Anderson, Martha \par
				$\phantom\longrightarrow$ Baker, Theresa \par
				$\phantom\longrightarrow$ Brown, Janet \par
				$\phantom\longrightarrow$ Clark, Stephanie \par
				$\phantom\longrightarrow$ Collins, Jane \par
				$\phantom\longrightarrow$ Cox, Shirley \par
				$\phantom\longrightarrow$ Davis, Marilyn \par
				$\phantom\longrightarrow$ Diaz, Harold \par
				$\phantom\longrightarrow$ Gonzalez, Adam \par
				$\phantom\longrightarrow$ Henderson, Lawrence \par
				$\phantom\longrightarrow$ Hughes, Aaron \par
				$\phantom\longrightarrow$ Kelly, Philip \par
				$\longrightarrow$ Lewis, Rose \par
				$\phantom\longrightarrow$ Lopez, Jeffrey \par
				$\phantom\longrightarrow$ Miller, Jeremy \par
				$\phantom\longrightarrow$ Parker, Debra \par
				$\phantom\longrightarrow$ Perez, Diana \par
				$\phantom\longrightarrow$ Russell, Mildred \par
				$\phantom\longrightarrow$ Sanders, Phillip \par
				$\phantom\longrightarrow$ Scott, Michelle \par
				$\phantom\longrightarrow$ Stewart, Howard \par
				$\phantom\longrightarrow$ Ward, Jessica \par
				$\phantom\longrightarrow$ White, Amanda \par
				$\phantom\longrightarrow$ Williams, Billy \par
				$\phantom\longrightarrow$ Young, Frank
			}}
		\end{column}
	\end{columns}
\end{frame}

\begin{frame}{Find ``Lopez, Jeffrey''}
	\begin{columns}
		\begin{column}{0.3\textwidth}
			\fbox{\parbox{0.9\textwidth}{\tiny
				{\color{gray}
				$\phantom\longrightarrow$ Anderson, Martha \par
				$\phantom\longrightarrow$ Baker, Theresa \par
				$\phantom\longrightarrow$ Brown, Janet \par
				$\phantom\longrightarrow$ Clark, Stephanie \par
				$\phantom\longrightarrow$ Collins, Jane \par
				$\phantom\longrightarrow$ Cox, Shirley \par
				$\phantom\longrightarrow$ Davis, Marilyn \par
				$\phantom\longrightarrow$ Diaz, Harold \par
				$\phantom\longrightarrow$ Gonzalez, Adam \par
				$\phantom\longrightarrow$ Henderson, Lawrence \par
				$\phantom\longrightarrow$ Hughes, Aaron \par
				$\phantom\longrightarrow$ Kelly, Philip \par
				$\phantom\longrightarrow$ Lewis, Rose \par
				}
				$\phantom\longrightarrow$ Lopez, Jeffrey \par
				$\phantom\longrightarrow$ Miller, Jeremy \par
				$\phantom\longrightarrow$ Parker, Debra \par
				$\phantom\longrightarrow$ Perez, Diana \par
				$\phantom\longrightarrow$ Russell, Mildred \par
				$\longrightarrow$ Sanders, Phillip \par
				$\phantom\longrightarrow$ Scott, Michelle \par
				$\phantom\longrightarrow$ Stewart, Howard \par
				$\phantom\longrightarrow$ Ward, Jessica \par
				$\phantom\longrightarrow$ White, Amanda \par
				$\phantom\longrightarrow$ Williams, Billy \par
				$\phantom\longrightarrow$ Young, Frank
			}}
		\end{column}
	\end{columns}
\end{frame}

\begin{frame}{Find ``Lopez, Jeffrey''}
	\begin{columns}
		\begin{column}{0.3\textwidth}
			\fbox{\parbox{0.9\textwidth}{\tiny
				{\color{gray}
				$\phantom\longrightarrow$ Anderson, Martha \par
				$\phantom\longrightarrow$ Baker, Theresa \par
				$\phantom\longrightarrow$ Brown, Janet \par
				$\phantom\longrightarrow$ Clark, Stephanie \par
				$\phantom\longrightarrow$ Collins, Jane \par
				$\phantom\longrightarrow$ Cox, Shirley \par
				$\phantom\longrightarrow$ Davis, Marilyn \par
				$\phantom\longrightarrow$ Diaz, Harold \par
				$\phantom\longrightarrow$ Gonzalez, Adam \par
				$\phantom\longrightarrow$ Henderson, Lawrence \par
				$\phantom\longrightarrow$ Hughes, Aaron \par
				$\phantom\longrightarrow$ Kelly, Philip \par
				$\phantom\longrightarrow$ Lewis, Rose \par
				}
				$\phantom\longrightarrow$ Lopez, Jeffrey \par
				$\phantom\longrightarrow$ Miller, Jeremy \par
				$\longrightarrow$ Parker, Debra \par
				$\phantom\longrightarrow$ Perez, Diana \par
				$\phantom\longrightarrow$ Russell, Mildred \par
				{\color{gray}$\phantom\longrightarrow$ Sanders, Phillip \par
				$\phantom\longrightarrow$ Scott, Michelle \par
				$\phantom\longrightarrow$ Stewart, Howard \par
				$\phantom\longrightarrow$ Ward, Jessica \par
				$\phantom\longrightarrow$ White, Amanda \par
				$\phantom\longrightarrow$ Williams, Billy \par
				$\phantom\longrightarrow$ Young, Frank
				}
			}}
		\end{column}
	\end{columns}
\end{frame}

\begin{frame}{Find ``Lopez, Jeffrey''}
	\begin{columns}
		\begin{column}{0.3\textwidth}
			\fbox{\parbox{0.9\textwidth}{\tiny
				{\color{gray}
				$\phantom\longrightarrow$ Anderson, Martha \par
				$\phantom\longrightarrow$ Baker, Theresa \par
				$\phantom\longrightarrow$ Brown, Janet \par
				$\phantom\longrightarrow$ Clark, Stephanie \par
				$\phantom\longrightarrow$ Collins, Jane \par
				$\phantom\longrightarrow$ Cox, Shirley \par
				$\phantom\longrightarrow$ Davis, Marilyn \par
				$\phantom\longrightarrow$ Diaz, Harold \par
				$\phantom\longrightarrow$ Gonzalez, Adam \par
				$\phantom\longrightarrow$ Henderson, Lawrence \par
				$\phantom\longrightarrow$ Hughes, Aaron \par
				$\phantom\longrightarrow$ Kelly, Philip \par
				$\phantom\longrightarrow$ Lewis, Rose \par
				}
				$\longrightarrow$ Lopez, Jeffrey \par
				$\phantom\longrightarrow$ Miller, Jeremy \par
				{\color{gray}$\phantom\longrightarrow$ Parker, Debra \par
				$\phantom\longrightarrow$ Perez, Diana \par
				$\phantom\longrightarrow$ Russell, Mildred \par
				$\phantom\longrightarrow$ Sanders, Phillip \par
				$\phantom\longrightarrow$ Scott, Michelle \par
				$\phantom\longrightarrow$ Stewart, Howard \par
				$\phantom\longrightarrow$ Ward, Jessica \par
				$\phantom\longrightarrow$ White, Amanda \par
				$\phantom\longrightarrow$ Williams, Billy \par
				$\phantom\longrightarrow$ Young, Frank
				}
			}}
		\end{column}
	\end{columns}
\end{frame}

\begin{frame}{How long does it take?}
	Socrative room code: \texttt{FALCOMPED}
	\begin{columns}
		\begin{column}{0.55\textwidth}
			\begin{itemize}
				\item Each iteration cuts the list in \textbf{half} \pause
				\item Worst case: we have to keep halving until we get down to a single element \pause
				\item If the size of the list is \textbf{doubled}, what happens to the worst-case
					\textbf{number of iterations} required? \pause
				\iftoggle{printable}{}{\item \textbf{Answer:} it increases by 1 \pause}
				\item The running time is \textbf{logarithmic} or $O(\log n)$ \pause
			\end{itemize}
		\end{column}
		\begin{column}{0.45\textwidth}
			\includegraphics[width=\textwidth]{plot2_log}
		\end{column}
	\end{columns}
\end{frame}

\begin{frame}{Hidden complexity}
	\begin{algorithmic}
			\If{name $<$ mid.name}
				\State \textbf{return} \Call{Find}{name, first half of list}
			\ElsIf{name $>$ mid.name}
				\State \textbf{return} \Call{Find}{name, second half of list}
			\EndIf
	\end{algorithmic}
	\pause
	\begin{columns}
		\begin{column}{0.45\textwidth}
			\only<4->{\includegraphics[width=\textwidth]{plot2_nlogn}}
		\end{column}
		\begin{column}{0.55\textwidth}
			\begin{itemize}
				\item Careful how you implement this! \pause
				\item \textbf{Copying} (half of) a list is \textbf{linear} $O(n)$ \pause
				\item The actual running time would be $O(n \log n)$ \pause
				\item Use \textbf{pointers} into the list instead of copying
			\end{itemize}
		\end{column}
	\end{columns}
\end{frame}

%\begin{frame}{Binary search done wrong}
%	\lstinputlisting[language=Python]{binary_search_bad.py}
%\end{frame}

%\begin{frame}{Binary search done right}
%	\lstinputlisting[language=Python]{binary_search_good.py}
%\end{frame}

\begin{frame}{Binary search vs linear search}
	\begin{columns}
		\begin{column}{0.45\textwidth}
			\includegraphics[width=\textwidth]{plot2_linear_log}
		\end{column}
		\begin{column}{0.55\textwidth}
			\begin{itemize}
				\item So binary search is better than linear search... right?
			\end{itemize}
		\end{column}
	\end{columns}
\end{frame}

\begin{frame}{Hashing}
	\begin{columns}
		\begin{column}{0.66\textwidth}
			\begin{itemize}
				\item Come up with a \textbf{hashing function} which maps elements to numbers \pause
				\item Example: assign $A=1, B=2, C=3$ etc, and add them together \pause
				\item Use these numbers to assign each element to a ``bin'' where it can be found \pause
			\end{itemize}
		\end{column}
		\begin{column}{0.3\textwidth}
			{\tiny
			\begin{tabular}{|c|l|}
$\vdots$ & $\vdots$ \\\hline
112 & Ward, Jessica \\\hline
113 & Baker, Theresa \\\hline
114 & Collins, Jane \\\hline
115 & --- \\\hline
116 & --- \\\hline
117 & Hughes, Aaron \\\hline
118 & --- \\\hline
119 & --- \\\hline
120 & --- \\\hline
121 & --- \\\hline
122 & Brown, Janet \\\hline
123 & --- \\\hline
124 & --- \\\hline
125 & Gonzalez, Adam \\ & Lewis, Rose \\\hline
126 & --- \\\hline
127 & --- \\\hline
128 & --- \\\hline
129 & --- \\\hline
130 & --- \\\hline
131 & --- \\\hline
132 & Young, Frank \\\hline
$\vdots$ & $\vdots$
			\end{tabular}
			}
		\end{column}
	\end{columns}
\end{frame}

\begin{frame}{Hash look-up}
	\begin{columns}
		\begin{column}{0.3\textwidth}
			{\tiny
			\begin{tabular}{|c|l|} \hline
				98 & Diaz, Harold \\\hline99 & Parker, Debra \\ & Perez, Diana \\ & White, Amanda \\\hline112 & Ward, Jessica \\\hline113 & Baker, Theresa \\\hline114 & Collins, Jane \\\hline117 & Hughes, Aaron \\\hline122 & Brown, Janet \\\hline125 & Gonzalez, Adam \\ & Lewis, Rose \\\hline132 & Young, Frank \\\hline135 & Kelly, Philip \\\hline138 & Cox, Shirley \\\hline142 & Clark, Stephanie \\\hline144 & Scott, Michelle \\\hline145 & Miller, Jeremy \\\hline147 & Davis, Marilyn \\\hline149 & Lopez, Jeffrey \\\hline151 & Anderson, Martha \\\hline158 & Williams, Billy \\\hline162 & Sanders, Phillip \\\hline171 & Russell, Mildred \\\hline175 & Stewart, Howard \\\hline183 & Henderson, Lawrence \\\hline
			\end{tabular}
			}
		\end{column}
		\begin{column}{0.66\textwidth}
			``Lopez, Jeffrey'' \pause
			
			$12 + 15 + 16 + 5 + 26 + 10 + 5 + 6 + 6 + 18 + 5 + 25 = 149$
		\end{column}
	\end{columns}
\end{frame}

\begin{frame}{How long does it take?}
	\begin{columns}
		\begin{column}{0.45\textwidth}
			\includegraphics[width=\textwidth]{plot2_constant}
		\end{column}
		\begin{column}{0.55\textwidth}
			\begin{itemize}
				\item If there are no ``collisions'', look-up time is \textbf{constant} or $O(1)$ \pause
					\begin{itemize}
						\item (NB: constant \textbf{with respect to} $n$) \pause
					\end{itemize}
				\item I.e. doubling the size of the list \textbf{does not change} the look-up time \pause
				\item When there are collisions, need to fall back on something like linear or binary search within each bin
			\end{itemize}
		\end{column}
	\end{columns}
\end{frame}

\begin{frame}[fragile]{Don't reinvent the wheel!}
	\begin{itemize}
		\pause\item We are using search as an \textbf{example}, to learn the \textbf{principles} --- in practice
			you should hardly ever implement your own search
		\pause\item Linear search in C\#:
			\begin{itemize}
				\pause\item \lstinline{List<T>.IndexOf()}
			\end{itemize}
		\pause\item Binary search in C\#:
			\begin{itemize}
				\pause\item \lstinline{List<T>.BinarySearch()}
			\end{itemize}
		\pause\item Hash tables in C\#:
			\begin{itemize}
				\pause\item \lstinline{Dictionary<TKey, TValue>}
				\pause\item \lstinline{GetHashCode()} is used to specify hash function
			\end{itemize}
	\end{itemize}
\end{frame}


\part{More on complexity}
\frame{\partpage}

\begin{frame}{Common complexity classes}
	\begin{center}
		\begin{tabular}{clc}
			\pause ``Faster'' & Constant & $O(1)$ \\
			\pause $\uparrow$ & Logarithmic & $O(\log n)$ \\
			\pause $|$ & Fractional power & $O(n^k)$, $k < 1$ \\
			\pause $|$ & Linear & $O(n)$ \\
			\pause $|$ & Quadratic & $O(n^2)$ \\
			\pause $|$ & Polynomial & $O(n^k)$, $k > 1$  \\
			\pause $\downarrow$ & Exponential & $O(e^n)$ \\
			\pause ``Slower'' & Factorial & $O(n!)$
		\end{tabular}
	\end{center}
\end{frame}

\begin{frame}{Common complexity classes}
	\begin{center}
		\includegraphics[width=\textwidth]{complexity_classes}
	\end{center}
\end{frame}

\begin{frame}{Working with big $O$ notation}
	\begin{itemize}
		\pause\item Can ignore \textbf{leading constants}
			\begin{itemize}
				\pause\item If one algorithm takes $n^2$ operations,
					another takes $500n^2$
					and a third takes $0.00000001n^2$,
					all three are $O(n^2)$
			\end{itemize}
		\pause\item Take only the \textbf{dominant term}
			\begin{itemize}
				\pause\item The term that is largest when $n$ is large
				\pause\item If an algorithm takes $0.1n^3 + 300n^2 + 7000$ operations,
					it is $O(n^3)$
			\end{itemize}
		\pause\item Multiply \textbf{compound} algorithms
			\begin{itemize}
				\pause\item If an algorithm does $n$ ``things'' and each ``thing'' is $O(n)$,
					then the overall algorithm is $O(n^2)$
			\end{itemize}
	\end{itemize}
\end{frame}

\begin{frame}{Quadratic complexity}
	\begin{columns}
		\begin{column}{0.4\textwidth}
			\only<3->{\includegraphics[width=\textwidth]{complete_graph}}
		\end{column}
		\begin{column}{0.6\textwidth}
			\begin{itemize}
				\item Collision detection between $n$ objects \pause
				\item The na\"ive way: check \textbf{each pair} of objects to see whether they have collided \pause
				\item This is \textbf{quadratic} or $O(n^2)$ \pause
				\item Doubling the number of objects would \textbf{quadruple} the time required! \pause
				\item Cleverer methods exist that are more scalable \pause
					\begin{itemize}
						\item Further reading: spatial hashing, quadtrees, octrees, Verlet lists
					\end{itemize}
			\end{itemize}
		\end{column}
	\end{columns}
\end{frame}

\begin{frame}{Exponential complexity}
	\begin{itemize}
		\pause\item A \text{prime number} is a number that is divisible only by $1$ and itself
		\pause\item Given an $n$-bit number $m = pq$ that is a product of two primes $p$ and $q$, find $p$ and $q$.
	\end{itemize}
	\pause
	\begin{algorithmic}
		\For{$p = 2, 3, \dots, m$}
			\State $q \gets m / p$
			\If{$q$ is an integer}
				\State \textbf{return} $p,q$
			\EndIf
		\EndFor
	\end{algorithmic}
	\begin{itemize}
		\pause\item Since $m \leq 2^n-1$, in the worst case this is $O(2^n)$
			\begin{itemize}
				\pause\item Actually even slower because division is not $O(1)$
			\end{itemize}
		\pause\item Adding 1 to $n$ potentially \textbf{doubles} the running time!
	\end{itemize}
\end{frame}

\begin{frame}{Aside: a famous unanswered question in computing}
	\begin{itemize}
		\pause\item A problem is ``in $P$'' if it can be solved with an
			algorithm running in $O(n^k)$ time
		\pause\item A problem is in $NP$ if a potential solution can be checked in $O(n^k)$ time
			\begin{itemize}
				\pause\item Equivalently, it can be solved with an algorithm running in $O(n^k)$ time on an infinitely parallel machine
			\end{itemize}
		\pause\item Are there any problems in $NP$ but not in $P$?
	\end{itemize}
\end{frame}

\begin{frame}{P versus NP}
	\begin{itemize}
		\pause\item If you can find a \textbf{mathematical proof} that either $P = NP$ or $P \neq NP$, there's a \$1 million prize...
		\pause\item It is believed that $P \neq NP$, so large instances of
			$NP$-hard problems are not solvable in a feasible amount of time
			\begin{itemize}
				\pause\item Many types of cryptography are based on this assumption
				\pause\item Quantum computers are ``infinitely parallel'' in a sense
					so \emph{can} solve some large $NP$-hard problems
			\end{itemize}
	\end{itemize}
\end{frame}

\begin{frame}{Caveats}
	\begin{itemize}
		\item Time complexity only tells us how an algorithm \textbf{scales} with the size of the input \pause
			\begin{itemize}
				\item If we know the input will always be \textbf{small}, time complexity is not so important \pause
				\item Linear search is quicker than binary search if you only ever have 3 elements \pause
				\item Na\"ive collision detection is fine if your game only ever has 4 objects on screen \pause
				\item Sometimes complexity in terms of other resources (e.g.\ space, bandwidth) are more important than time \pause
			\end{itemize}
		\item Software development is all about choosing \textbf{the right tool for the job} \pause
			\begin{itemize}
				\item If you need scalability, choose a scalable algorithm \pause
				\item Otherwise, choose simplicity
			\end{itemize}
	\end{itemize}
\end{frame}

\begin{frame}{Summary}
	\begin{itemize}
		\item Time complexity tells us how the running time of an algorithm \textbf{scales} with the size of the data
			it is given \pause
		\item Choice of data structures and algorithms can have a large impact on the efficiency of your software \pause
		\item ... but only if scalability is actually a factor
	\end{itemize}
\end{frame}


\end{document}
