\part{Computation time}
\frame{\partpage}

\begin{frame}{Resources}
	\begin{itemize}
		\pause\item All programs use \textbf{resources}
			\begin{itemize}
				\pause\item Time
				\pause\item Memory
				\pause\item Network bandwidth
				\pause\item Power
				\pause\item ...
			\end{itemize}
		\pause\item Often \textbf{time} is the resource we care about the most
			\begin{itemize}
				\pause\item Particularly in games:
					want to maintain a good \textbf{frame rate}
					free of \textbf{lag} or \textbf{stuttering}
			\end{itemize}
	\end{itemize}
\end{frame}

\begin{frame}[fragile]{Basic time measurement in Python}
	\begin{lstlisting}
import time

start_time = time.clock()

... do something here ...

end_time = time.clock()
print "Computation took", end_time - start_time, "seconds"
	\end{lstlisting}
\end{frame}

\begin{frame}[fragile]{Repeating for better accuracy}
	\begin{lstlisting}
import time

start_time = time.clock()

repetition_count = 1000

for repetition in xrange(repetition_count):
    ... do something here ...

end_time = time.clock()
time_per = (end_time - start_time) / repetition_count
print "Computation took", time_per, "seconds"
	\end{lstlisting}
\end{frame}

\begin{frame}{Scaling}
	\begin{itemize}
		\pause\item Timing is dependent on hardware and software issues
		\pause\item We are often less interested in how many milliseconds a particular computation takes on today's hardware, and more interested in how the execution time \textbf{scales} with the problem size
	\end{itemize}
\end{frame}
