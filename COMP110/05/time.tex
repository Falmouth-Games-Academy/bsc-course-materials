\part{Time benchmarking}
\frame{\partpage}

\begin{frame}{Resources}
    \begin{itemize}
        \pause\item All programs use \textbf{resources}
            \begin{itemize}
                \pause\item Time
                \pause\item Memory
                \pause\item Network bandwidth
                \pause\item Power
                \pause\item ...
            \end{itemize}
        \pause\item Often \textbf{time} is the resource we care about the most
            \begin{itemize}
                \pause\item Particularly in games:
                    want to maintain a good \textbf{frame rate}
                    free of \textbf{lag} or \textbf{stuttering}
                \pause\item To run at 60 frames per second, we only have \textbf{16.666~milliseconds} to do everything that needs to be done on every frame
                \pause\item E.g.\ in Unity, all of your \lstinline{Update()} methods as well as physics, audio, rendering etc.
            \end{itemize}
    \end{itemize}
\end{frame}

\begin{frame}[fragile]{Basic time measurement in C\#}
    \begin{lstlisting}
// using System.Diagnostics;
Stopwatch stopwatch = new Stopwatch();

stopwatch.Restart();
DoSomething();
stopwatch.Stop();

Console.WriteLine(stopwatch.Elapsed);

// Or for more accuracy
Console.WriteLine(stopwatch.ElapsedTicks);
    \end{lstlisting}
\end{frame}

\begin{frame}[fragile]{A benchmarking example}
    \begin{lstlisting}
Random random = new Random();
Stopwatch stopwatch = new Stopwatch();

using (StreamWriter writer 
            = new StreamWriter("results.csv"))
{
    for (int n = 1000; n < 100000; n += 1000)
    {
        stopwatch.Restart();

        List<int> list = new List<int>();
        for (int i = 0; i < n; i++)
            list.Append(random.Next(1000));

        stopwatch.Stop();

        writer.WriteLine("{0}, {1}", n, stopwatch.ElapsedTicks);
    }
}
    \end{lstlisting}
\end{frame}

\begin{frame}[fragile]{Repeating for better accuracy}
    \begin{lstlisting}
import time

start_time = time.perf_counter()

repetition_count = 1000

for repetition in range(repetition_count):
    # ... do something here ...

end_time = time.perf_counter()
total_time = end_time - start_time
print("Time:", total_time, "seconds")
    \end{lstlisting}
    
    \begin{itemize}
        \pause\item There is some \textbf{overhead} from the \lstinline{for} loop, but in practice it is negligible
    \end{itemize}
\end{frame}

\begin{frame}
    \lstinputlisting[basicstyle=\scriptsize\ttfamily]{timing/time_list_creation.py}
\end{frame}

\begin{frame}
    \lstinputlisting[basicstyle=\scriptsize\ttfamily]{timing/time_list_append.py}
\end{frame}

\begin{frame}{Workshop exercise}
    \begin{itemize}
        \pause\item Investigate various operations on Python lists, and how their running time varies with the size of the list
        \pause\item For each of the operations listed on the next slide:
            \begin{itemize}
                \pause\item \textbf{Find out} how to do the operation
                \pause\item \textbf{Write} code similar to the previous slides, to generate a list of size $n$ (for various values of $n$) and then time the operation on that list
                \pause\item \textbf{Plot} graphs of the operations using Excel
                \pause\item (Advanced mode: instead of using Excel, plot graphs directly from Python using the Matplotlib library)
            \end{itemize}
    \end{itemize}
\end{frame}

\begin{frame}{Operations to time}
    \footnotesize
    \begin{columns}
        \begin{column}{0.45\textwidth}
            \begin{itemize}
                \item Append an element
                \item Insert an element at the beginning
                \item Insert an element at a random position
                \item Delete the first element
                \item Delete the last element
                \item Delete a random element
                \item Get the first element
                \item Get the last element
                \item Get a random element
                \item Find if the list contains a specific element
            \end{itemize}
        \end{column}
        \begin{column}{0.45\textwidth}
            \begin{itemize}
                \item Get the smallest element
                \item Get the largest element
                \item Get the sum of all elements
                \item Get the length of the list
                \item Copy the list
                \item Reverse the list
                \item Sort the list
                \item Randomly shuffle the list
                \item Convert the list to string
            \end{itemize}
        \end{column}
    \end{columns}
\end{frame}
