\part{Generics in C\#}
\frame{\partpage}

\begin{frame}[fragile]{The problem}
    \begin{itemize}
        \pause\item Suppose we want to define a \lstinline{Pair} class to store two values
        \pause\item Something like this...
    \end{itemize}
    \pause
    \begin{lstlisting}
class PairOfInts
{
    public int first;
    public int second;
    
    public PairOfInts(int f, int s)
    {
        first = f;
        second = s;
    }
}
    \end{lstlisting}
\end{frame}

\begin{frame}[fragile]{The problem}
    \begin{itemize}
        \pause\item This is fine if we just want pairs of \lstinline{int}s
        \pause\item To store a pair of \lstinline{string}s we would need another class:
    \end{itemize}
    \pause
    \begin{lstlisting}
class PairOfStrings
{
    public string first;
    public string second;
    
    public PairOfStrings(string f, string s)
    {
        first = f;
        second = s;
    }
}
    \end{lstlisting}
\end{frame}

\begin{frame}[fragile]{The problem}
    \begin{itemize}
        \pause\item This quickly gets repetitive!
        \pause\item We could just store a pair of \lstinline{object}s --- in C\# \lstinline{object} can store values of any
            type
    \end{itemize}
    \pause
    \begin{lstlisting}
class PairOfObjects
{
    public object first;
    public object second;
    
    public PairOfObjects(object f, object s)
    {
        first = f;
        second = s;
    }
}
    \end{lstlisting}
    \begin{itemize}
        \pause\item However this doesn't let us impose type safety
    \end{itemize}
\end{frame}

\begin{frame}[fragile]{The solution}
    \begin{itemize}
        \pause\item \textbf{Generics} are a feature of C\# which let us pass types as ``parameters''
    \end{itemize}
    \pause
    \begin{lstlisting}
class Pair<ElementType>
{
    public ElementType first;
    public ElementType second;
    
    public PairOfObjects(ElementType f, ElementType s)
    {
        first = f;
        second = s;
    }
}
    \end{lstlisting}
    \begin{itemize}
        \pause\item \lstinline{ElementType} can be any type
    \end{itemize}
\end{frame}

\begin{frame}[fragile]{The solution}
    \begin{itemize}
        \pause\item When we instantiate the generic class, we pass in the type in angle brackets:
    \end{itemize}
    \pause
    \begin{lstlisting}
Pair<int> p1 = new Pair<int>(12, 34);
Pair<string> p2 = new Pair<string>("hello", "world");
    \end{lstlisting}
\end{frame}

\begin{frame}[fragile]{Multiple parameters}
    \begin{itemize}
        \pause\item Generics can take multiple parameters:
    \end{itemize}
    \pause
    \begin{lstlisting}
class Pair<Type1, Type2>
{
    public Type1 first;
    public Type2 second;
    
    public PairOfObjects(Type1 f, Type2 s)
    {
        first = f;
        second = s;
    }
}
    \end{lstlisting}
    \begin{lstlisting}
Pair<int, string> x = new Pair<int, string>(123, "hello");
    \end{lstlisting}
\end{frame}

\begin{frame}{Why generics?}
    \begin{itemize}
        \pause\item Generics let us write type safe code which can be adapted to data of different types
        \pause\item Standard libraries in .NET and Unity make use of generics for e.g.\ collection types
        \pause\item Similar to \textbf{templates} in C++
    \end{itemize}
\end{frame}

