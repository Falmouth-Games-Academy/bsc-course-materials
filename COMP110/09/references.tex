\newcommand{\socrative}{
	\begin{center}
		Socrative room code: \texttt{FALCOMPED}
	\end{center}
}

\part{Pass by reference}
\frame{\partpage}

\begin{frame}{References}
	\begin{itemize}
		\pause\item Our picture of a variable: a labelled box containing a value
		\pause\item For ``plain old data'' (e.g.\ numbers), this is accurate
		\pause\item For \textbf{objects} (i.e.\ instances of classes), variables actually hold
			\textbf{references} (a.k.a.\ \textbf{pointers})
		\pause\item It is possible (indeed common) to have \textbf{multiple references} to the same underlying object
	\end{itemize}
\end{frame}

\begin{frame}{The wrong picture}
	\begin{columns}
		\begin{column}{0.48\textwidth}
			\lstinputlisting{references_0.py}
		\end{column}
		\pause
		\begin{column}{0.48\textwidth}
			\begin{center}
				\colorbox{white}{
					\color{black}
					\begin{tabular}{|c|c|}
						\hline
						\textbf{Variable} & \textbf{Value} \\\hline
						\texttt{x} & \uncover<3->{\begin{tabular}{|c|c|}
							\hline
							\texttt{a} & 30 \\\hline
							\texttt{b} & 40 \\\hline
						\end{tabular}} \\\hline
						\texttt{y} & \uncover<4->{\begin{tabular}{|c|c|}
							\hline
							\texttt{a} & 50 \\\hline
							\texttt{b} & 60 \\\hline
						\end{tabular}} \\\hline
						\texttt{z} & \uncover<5->{\begin{tabular}{|c|c|}
							\hline
							\texttt{a} & 50 \\\hline
							\texttt{b} & 60 \\\hline
						\end{tabular}} \\\hline
					\end{tabular}
				}
			\end{center}
		\end{column}
	\end{columns}
\end{frame}

\begin{frame}{The right picture}
	\begin{columns}
		\begin{column}{0.48\textwidth}
			\lstinputlisting{references_0.py}
		\end{column}
		\pause
		\begin{column}{0.48\textwidth}
			\colorbox{white}{\parbox{0.9\textwidth}{
				\begin{center}
					\color{black}
					\begin{tabular}{|c|c|}
						\hline
						\textbf{Variable} & \textbf{Value} \\\hline
						\texttt{x} & \tikzmark{valuex} \\\hline
						\texttt{y} & \tikzmark{valuey} \\\hline
						\texttt{z} & \tikzmark{valuez} \\\hline
					\end{tabular}
					\par
					\vspace{3ex}
					\uncover<3->{\begin{tabular}{|c|c|}
						\hline
						\texttt{a} & \tikzmark{objectx}30 \\\hline
						\texttt{b} & 40 \\\hline
					\end{tabular}}
					\hspace{1ex}
					\uncover<4->{\begin{tabular}{|c|c|}
						\hline
						\texttt{a} & \tikzmark{objecty}50 \\\hline
						\texttt{b} & 60 \\\hline
					\end{tabular}}
				\end{center}
			}}
		\end{column}
	\end{columns}
	\begin{tikzpicture}
		[
		  remember picture,
		  overlay,
		  -latex,
		  color=red,
		  yshift=0.5ex,
		  shorten >=1pt,
		  shorten <=1pt,
		]
		\pause\draw ({pic cs:valuex}) to [bend right] ($ ({pic cs:objectx}) + (0, 1ex) $);
		\pause\draw ({pic cs:valuey}) to [bend right] ($ ({pic cs:objecty}) + (0, 1ex) $);
		\pause\draw ({pic cs:valuez}) to [bend left] ($ ({pic cs:objecty}) + (0, 1ex) $);
	\end{tikzpicture}
\end{frame}

\begin{frame}{Values and references}
	\socrative
	\lstinputlisting{references_1.py}
\end{frame}

\begin{frame}{Values and references}
	\socrative
	\lstinputlisting{references_2.py}
\end{frame}

\begin{frame}{Values and references}
	\socrative
	\lstinputlisting{references_3.py}
\end{frame}

\begin{frame}[fragile]{Pass by value}
	\pause
	In \textbf{function parameters},
	``plain old data'' is passed by \textbf{value}
	\pause
	\begin{lstlisting}
def double(x):
    x *= 2

a = 7
double(a)
print a
	\end{lstlisting}
	\pause
	\lstinline{double} does not actually do anything, as \lstinline{x} is just a local copy of
		whatever is passed in!
\end{frame}

\begin{frame}[fragile]{Pass by reference}
	\pause
	However, instances are passed by \textbf{reference}
	\begin{lstlisting}
class Box:
    def __init__(self, v):
        self.value = v

def double(x):
    x.value *= 2

a = Box(7)
double(a)
print a.value
	\end{lstlisting}
	\pause
	\lstinline{double} now has an effect, as \lstinline{x} gets a reference to the \lstinline{Box} instance
\end{frame}

\begin{frame}[fragile]{Lists are objects too}
	\pause
	\begin{lstlisting}
a = ["Hello"]
b = a
b.append("world")
print a  # ["Hello", "world"]
	\end{lstlisting}
	\pause
	... which means you should be careful when passing lists into functions,
	because the function might actually change the list!
\end{frame}

