\part{RISC vs CISC}
\frame{\partpage}

\begin{frame}{CISC}
    \begin{itemize}
        \pause\item x86 is what's known as a \textbf{CISC architecture}
        \pause\item CISC = Complex Instruction Set Computer
        \pause\item x86-64 has \textbf{thousands} of instructions % https://stefanheule.com/blog/how-many-x86-64-instructions-are-there-anyway/
    \end{itemize}
\end{frame}

\begin{frame}{RISC}
    \begin{itemize}
        \pause\item RISC = Reduced Instruction Set Computer
        \pause\item E.g. ARM (Advanced RISC Machine) architecture
        \pause\item Widely used in mobile (Apple and Android)
        \pause\item \textbf{Tens} or \textbf{hundreds} of instructions
    \end{itemize}
\end{frame}

\begin{frame}{RISC vs CISC}
    \begin{itemize}
        \pause\item CISC generally leads to \textbf{shorter programs} --- CISC does more per instruction
        \pause\item RISC generally leads to \textbf{simpler hardware} --- instructions are simpler
        \pause\item This also leads to better efficiency in terms of \textbf{power} and \textbf{heat}
        \pause\item Assembly programmers and compiler developers need to worry about the difference --- it is abstracted away from the rest of us
        \pause\item Nowadays the main reason to use CISC is \textbf{backwards compatibility}
        \pause\item Apple are moving to RISC for Mac computers --- will PCs ever follow suit?
    \end{itemize}
\end{frame}

