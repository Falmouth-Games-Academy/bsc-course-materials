\part{Turing machines}
\frame{\partpage}

\begin{frame}
	\frametitle{Activity - Groups of Five}
	\begin{figure}
		\includegraphics[scale=0.3]{assets/activity.png}
	\end{figure}
\end{frame}

\begin{frame}
	\begin{figure}
		\includegraphics[scale=0.2]{assets/turingvideo.png}
		\caption{\href{https://www.youtube.com/watch?v=gtRLmL70TH0}{https://www.youtube.com/watch?v=gtRLmL70TH0}}
	\end{figure}
\end{frame}

\begin{frame}

	\begin{figure}
		\includegraphics[scale=0.2]{assets/compute.png}
		\caption{\href{https://www.youtube.com/watch?v=dNRDvLACg5Q}{https://www.youtube.com/watch?v=dNRDvLACg5Q}}
	\end{figure}
\end{frame}

\begin{frame}
	\begin{figure}
		\includegraphics[scale=0.2]{assets/interactive.png}
		\caption{\href{http://turingmaschine.klickagent.ch/einband/}{http://turingmaschine.klickagent.ch/einband/}}
	\end{figure}
\end{frame}

\begin{frame}
	\frametitle{Turing Completeness}
	
	To show that something is Turing complete, it is enough to show that it can be used to simulate some Turing complete system. 	\\ ~ \\
	For an imperative language to be classed as Turing Complete it must have:

	\begin{itemize}
		\item Conditional branching (e.g., ``if'' and ``goto'' statements, or a ``branch if zero'' instruction)
		\item Ability to change an arbitrary amount of memory (e.g., the ability to maintain an arbitrary number of variables).
	\end{itemize}
\end{frame}

\begin{frame}
	!!! Since this is almost always the case, most if not all imperative languages are Turing complete if the limitations of finite memory are ignored. \\~\\
	\tiny{imperative programming is a programming paradigm that uses statements that change a program's state}
	
\end{frame}

