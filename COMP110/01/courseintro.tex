\part{Course introduction}
\frame{\partpage}

\begin{frame}{From the module guide}
This module is designed to introduce you to the basic principles of computing and programming in the context of digital games. It is designed to complement the other modules through providing a broad foundation on the different methods and techniques which will help you to be able to construct computer programs and able to use relevant scholarly sources. You will gain an understanding of software development and the various roles, pipelines, and terminology used within game development.
\end{frame}

\begin{frame}{Topic schedule}
	\begin{center}
		On LearningSpace...
	\end{center}
\end{frame}

\begin{frame}{Timetable}
	\url{http://mytimetable.falmouth.ac.uk}
	\begin{itemize}
		\item Odd numbered weeks: \\ Monday \textbf{13:30} \\ P/PL/Seminar 08 (here)
		\item Even numbered weeks: \\ Monday \textbf{09:00} \\ P/PL/Games Teaching Space
	\end{itemize}
\end{frame}

\begin{frame}{Worksheet A}
	\begin{itemize}
		\item SpaceChem!
		\item Due in class on \textbf{Monday 26th September} (next week)
	\end{itemize}
\end{frame}

\begin{frame}{Reading}
	C.\ Horsman, S.\ Stepney, R.C.\ Wagner and V.\ Kendon, 2014.
	When does a physical system compute?
	\emph{Proceedings of the Royal Society A},
	470:20140182.
	
	Available online: \url{http://rspa.royalsocietypublishing.org/content/royprsa/470/2169/20140182.full.pdf}
\end{frame}
