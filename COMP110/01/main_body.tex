% !TeX root = ./2020-21-COMP110-01-lecture-materials.tex
% Adjust these for the path of the theme and its graphics, relative to this file
%\usepackage{beamerthemeFalmouthGamesAcademy}
\usepackage{../../beamerthemeFalmouthGamesAcademy}
\usepackage{multimedia}
\graphicspath{ {../../} }

% Default language for code listings
\lstset{language=C++,
        morekeywords={each,in,nullptr}
}

% For strikethrough effect
\usepackage[normalem]{ulem}
\usepackage{wasysym}

\usepackage{pdfpages}

% http://www.texample.net/tikz/examples/state-machine/
\usetikzlibrary{arrows,automata}

\newcommand{\modulecode}{COMP260}\newcommand{\moduletitle}{Distributed Systems}\newcommand{\sessionnumber}{5}


\hypersetup{
pdftex,
pdftitle=\sessionnumber: Computing Foundations,
pdfauthor=Ed Powley,
pdfdisplaydoctitle,
pdflang=en-GB
}
 
\begin{document}
\title{\sessionnumber: Computing Foundations}
\subtitle{\modulecode: \moduletitle}

\frame{\titlepage} 

\begin{frame}{Learning outcomes}
	By the end of today's session, you will be able to:
	\begin{itemize}
		\item \textbf{Describe} the overall structure of the module and its assessments
		\item \textbf{Recall} the historical context of computing and gaming technology
		\item \textbf{Explain} the basic architecture of a computer
	\end{itemize}
\end{frame}

\begin{frame}{Today's agenda}
	\begin{itemize}
		\item COMP110 course outline
		\item History of computing
	\end{itemize}
\end{frame}

\part{Course introduction}
\frame{\partpage}

\begin{frame}{From the module guide}
This module will introduce you to the techniques of 3D graphics rendering and physics simulation used in modern computer games. Using the OpenGL library, you will develop an understanding of the 3D graphics pipeline, and how to program the GPU to produce advanced graphical effects.
\end{frame}

\begin{frame}{Topic schedule}
	\begin{center}
		On LearningSpace...
	\end{center}
\end{frame}

\begin{frame}{Assignment 1: Portfolio task}
	\begin{center}
		First worksheet is due in week 4.
	\end{center}
\end{frame}

\begin{frame}{Assignment 2: Research journal}
	\begin{center}
		First component due in week 3.
		\pause Don't forget to update the wiki!
	\end{center}
\end{frame}


\newcommand{\pictureslideb}[3]{
	\begin{frame}{#1}
		\begin{center}
			#3
			
			\vspace{6pt}
			
			\includegraphics[height=0.6\textheight]{#2}
		\end{center}
	\end{frame}
}

\newcommand{\pictureslide}[2]{
	\begin{frame}{#1}
		\begin{center}
			\includegraphics[height=0.6\textheight]{#2}
		\end{center}
	\end{frame}
}

\part{What was the first computer?}
\frame{\partpage}

\pictureslideb{Antikythera Mechanism ($\sim$150 BC)}{antikythera}{First mechanical computer?}
\pictureslideb{Babbage's Difference and Analytical Engines (1837)}{difference_engine}{First mechanical computer in modern age}
\pictureslideb{Colossus (1943)}{colossus}{First programmable electronic computer}
\pictureslideb{ENIAC (1946)}{eniac}{First general-purpose computer}
\pictureslideb{Manchester Small-Scale Experimental Machine (1948)}{manchester}{First stored program computer}
\pictureslideb{EDSAC (1949)}{edsac}{Many firsts in mathematics and science}
\pictureslideb{PDP-1 (1959)}{pdp1}{Influenced ``hacker culture''}
\pictureslideb{Datapoint 2200 (1970)}{datapoint2200}{First microcomputer}
\pictureslideb{Commodore VIC 20 (1980)}{vic20}{First computer to sell 1 million units}
\pictureslideb{IBM Personal Computer Model 5150 (1981)}{ibm_5150}{Precursor to the modern PC}

\part{What was the first computer game?}
\frame{\partpage}

\pictureslideb{Cathode Ray Tube Amusement Device (1948)}{crt}{First interactive electronic game}
\pictureslideb{Chess AI on the Ferranti Mark I (1951)}{ferranti}{First chess program}
\pictureslideb{Bertie the Brain (1950)}{bertie}{First computer game with a visual display}
\pictureslideb{OXO (1951)}{oxo}{First game with visuals on a general-purpose computer}
\pictureslideb{Tennis for Two (1959)}{tennis}{First to be created purely for entertainment}
\pictureslideb{SpaceWar! (1962)}{spacewar}{First widely available game, inspired first arcade games}
\pictureslideb{Pong (1972)}{pong}{First commercially successful game}

\part{What was the first games console?}
\frame{\partpage}

\pictureslideb{The Brown Box (1967)}{brownbox}{First prototype console}
\pictureslideb{Magnavox Odyssey (1972)}{magnavox}{First commercial console}

\begin{frame}{Game console timeline}
	\begin{center}
		\url{http://www.onlineeducation.net/videogame_timeline/video-game-timeline.jpg}
		
		(A little out of date!)
	\end{center}
\end{frame}


\begin{frame}{Debrief}
	\pause You should now be able to:
	\begin{itemize}
		\item \textbf{Describe} the overall structure of the module and its assessments
		\item \textbf{Recall} the historical context of computing and gaming technology
		\item \textbf{Explain} the basic architecture of a computer
	\end{itemize}
	\pause \textbf{Remember:} Worksheet A is due \textbf{next week}!
\end{frame}

\end{document}
