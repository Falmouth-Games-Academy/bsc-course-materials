% !TeX root = ./2020-21-COMP110-01-lecture-materials.tex
% Adjust these for the path of the theme and its graphics, relative to this file
%\usepackage{beamerthemeFalmouthGamesAcademy}
\usepackage{../../beamerthemeFalmouthGamesAcademy}
\usepackage{multimedia}
\graphicspath{ {../../} }

% Default language for code listings
\lstset{language=C++,
        morekeywords={each,in,nullptr}
}

% For strikethrough effect
\usepackage[normalem]{ulem}
\usepackage{wasysym}

\usepackage{pdfpages}

% http://www.texample.net/tikz/examples/state-machine/
\usetikzlibrary{arrows,automata}

\newcommand{\modulecode}{COMP140 GAM160}\newcommand{\moduletitle}{Hacking Hardware/Advanced Programming}\newcommand{\sessionnumber}{Session 6}


\hypersetup{
pdftex,
pdftitle=\sessionnumber: Computing Foundations,
pdfauthor=Ed Powley,
pdfdisplaydoctitle,
pdflang=en-GB
}
 
\begin{document}
\title{\sessionnumber: Computing Foundations}
\subtitle{\modulecode: \moduletitle}

\frame{\titlepage} 

\begin{frame}{Learning outcomes}
	By the end of today's session, you will be able to:
	\begin{itemize}
		\item \textbf{Describe} the overall structure of the module and its assessments
		\item \textbf{Recall} the historical context of computing and gaming technology
		\item \textbf{Explain} the basic architecture of a computer
	\end{itemize}
\end{frame}

\begin{frame}{Today's agenda}
	\begin{itemize}
		\item COMP110 course outline
		\item History of computing
	\end{itemize}
\end{frame}

\part{Module introduction}
\frame{\partpage}

\begin{frame}{Aim}
\begin{center}
To empower you to leverage mathematics and mathematical modelling in the design and implementation of real-time 3D worlds and simulations.
\end{center}
\end{frame}

\begin{frame}{Description}
On this module, you learn the fundamental mathematics involved in the design, development and maintenance of real-time 3D worlds and simulations. In doing so, you will leverage mathematics practically to generate and manipulate worlds and simulations relevant to a range of creative computing contexts. Indicatively, content spans topics such as linear algebra (vectors, matrices and quaternions), geometry, trigonometry, 3D transformation, collision detection, Newtonian mechanics, numerical control, calculus, and efficiency and optimisation of numerical methods.
\end{frame}

\begin{frame}{Learning Outcome}
	\begin{itemize}
		\item SOLVE
		\item Apply knowledge of algorithms, data structures, and mathematics to solve well-defined problems.
		\item Assessment criteria category: PROCESS
	\end{itemize}
\end{frame}

\begin{frame}{Topic schedule}
	\begin{center}
		On LearningSpace
	\end{center}
\end{frame}

\begin{frame}{Timetable}
	\begin{center}
		\url{http://mytimetable.falmouth.ac.uk}
	\end{center}
\end{frame}

\begin{frame}{Assignments}
	\begin{itemize}
		\pause\item Assignment 1: worksheet tasks
			\begin{itemize}
				\pause\item \textbf{Four} worksheets --- roughly 2 weeks each
			\end{itemize}
		\pause\item See LearningSpace for assignment briefs and worksheets
		\pause\item See MyFalmouth for deadlines
	\end{itemize}
\end{frame}

\begin{frame}{Worksheet A}
	\begin{itemize}
		\item B\'ezier curves
		\item Due \textbf{Monday week 4 (14th October)}
	\end{itemize}
\end{frame}


\part{A brief history of PCG}
\frame{\partpage}

\newcommand{\pictureslideb}[3]{
	\begin{frame}{#1}
		\begin{center}
			\includegraphics[height=0.6\textheight]{#2}
			
			\vspace{6pt}
			
			#3
		\end{center}
	\end{frame}
}

\newcommand{\pictureslide}[2]{
	\begin{frame}{#1}
		\begin{center}
			\includegraphics[height=0.6\textheight]{#2}
		\end{center}
	\end{frame}
}

\newcommand{\pictureslidew}[2]{
	\begin{frame}{#1}
		\begin{center}
			\includegraphics[width=\textwidth]{#2}
		\end{center}
	\end{frame}
}

\pictureslide{Dungeons \& Dragons (1974)}{dnd}
\pictureslide{Beneath Apple Manor (1978)}{beneathapplemanor}
\pictureslide{Rogue (1980)}{rogue}
\pictureslideb{Elite (1984)}{elite}{$8 \times 256=2048$ planets}
\pictureslide{Sid Meier's Civilization (1991)}{civilization}
\pictureslide{Frontier: Elite II (1993)}{frontier} 
\pictureslideb{The Elder Scrolls II: Daggerfall (1996)}{daggerfall}{Roughly half the size of Great Britain}
\pictureslide{SpeedTree (2002)}{speedtree}
\pictureslideb{.kkrieger (2004)}{kkrieger}{Full FPS game in 96kb}
\pictureslide{Dwarf Fortress (2006)}{dwarffortress}
\pictureslide{Spelunky (2008)}{spelunky}
\pictureslide{Spore (2008)}{spore}
\pictureslide{Left 4 Dead (2008)}{left4dead}
\pictureslide{Borderlands (2009)}{borderlands}
\pictureslideb{Minecraft (2011)}{minecraft}{Many times bigger than surface of Earth}
\pictureslide{The Binding of Isaac (2011)}{isaac}
\pictureslide{To That Sect (2013)}{tothatsect}
\pictureslide{Elite: Dangerous (2014)}{elitedangerous}
\pictureslide{Road Not Taken (2014)}{roadnottaken}
\pictureslidew{PROCJAM (2014--present)}{procjam}
\pictureslide{No Man's Sky (2016)}{nomanssky}


\begin{frame}{Debrief}
	\pause You should now be able to:
	\begin{itemize}
		\item \textbf{Describe} the overall structure of the module and its assessments
		\item \textbf{Recall} the historical context of computing and gaming technology
		\item \textbf{Explain} the basic architecture of a computer
	\end{itemize}
	\pause \textbf{Remember:} Worksheet A is due \textbf{next week}!
\end{frame}

\end{document}
