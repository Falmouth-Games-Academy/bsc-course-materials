% Adjust these for the path of the theme and its graphics, relative to this file
%\usepackage{beamerthemeFalmouthGamesAcademy}
\usepackage{../../beamerthemeFalmouthGamesAcademy}
\usepackage{multimedia}
\graphicspath{ {../../} }

% Default language for code listings
\lstset{language=C++,
        morekeywords={each,in,nullptr}
}

% For strikethrough effect
\usepackage[normalem]{ulem}
\usepackage{wasysym}

\usepackage{pdfpages}

\usepackage{circuitikz}

% http://www.texample.net/tikz/examples/state-machine/
\usetikzlibrary{arrows,automata}
\usetikzlibrary{calc}

\newcommand{\modulecode}{COMP140 GAM160}\newcommand{\moduletitle}{Hacking Hardware/Advanced Programming}\newcommand{\sessionnumber}{Session 6}

\hypersetup{
pdftex,
pdftitle=\sessionnumber: Research Skills,
pdfauthor=Ed Powley,
pdfdisplaydoctitle,
pdflang=en-GB
}
 
\begin{document}
\title{\sessionnumber: Research Skills}
\subtitle{\modulecode: \moduletitle}

\frame{\titlepage} 

\part{Research journal}
\frame{\partpage}

\begin{frame}{Research journal}
    \begin{itemize}
        \pause\item \textbf{Read} some seminal papers in computing (listed on the assignment brief)
        \pause\item \textbf{Choose} one of them
        \pause\item \textbf{Research} how this paper has influenced the field of computing
        \pause\item \textbf{Write up} your findings
        \begin{itemize}
            \pause\item Maximum 1500 words
            \pause\item With reference to appropriate academic sources
        \end{itemize}
    \end{itemize}
\end{frame}

\begin{frame}{Marking rubric}
    \begin{center}
        See assignment brief on LearningSpace/GitHub
    \end{center}
\end{frame}

\begin{frame}{Timeline}
    \begin{itemize}
        \item \textbf{Peer review} next week! (4th December)
        \item \textbf{Deadline} shortly after! (check MyFalmouth)
    \end{itemize}
\end{frame}


\part{Scholarly literature}
\frame{\partpage}

\begin{frame}{Scholarly work}
	\begin{itemize}
		\pause\item What is a ``scholarly'' work?
		\pause\item How do we know if something is scholarly?
	\end{itemize}
\end{frame}

\usetikzlibrary{shapes,arrows,intersections}
\usetikzlibrary{matrix,fit,calc,trees,positioning,arrows,chains,shapes.geometric,shapes}

\begin{frame}{Pyramid of sources}
	\centering
	\begin{tikzpicture}
	\coordinate (A) at (-4.5,0) {};
	\coordinate (B) at ( 4.5,0) {};
	\coordinate (C) at (0,5*1.2) {};
	\path[name path=AC,draw=none] (A) -- (C);
	\path[name path=BC,draw=none] (B) -- (C);
	\iftoggle{printable}{
		\filldraw[draw=Purple, ultra thick,fill=Purple!10!White] (A) -- (B) -- (C) -- cycle ;
	}{ % else
		\filldraw[draw=Purple, ultra thick,fill=Purple!10!Black] (A) -- (B) -- (C) -- cycle ;
	}	

	\foreach \y/\A in {4.5/{Scholarly journals and conference proceedings},
					   4.0/{Scholarly books and book chapters},
					   3.5/{Masters and PhD theses},
					   3.0/{Government documents, trade books and white papers},
					   2.5/{Specialised magazines},
					   2.0/{Pre-print papers (e.g.\ arXiv)},
					   1.5/{General interest books, magazines and newspapers},
					   1.0/{General encyclop\ae dias},
					   0.5/{Websites, blogs, Wikipedia, YouTube},
					   0.0/{Online discussion boards, personal communications}
					} {
		\pause
		\path[draw=none, very thick, dashed, name path=horiz] (A|-0,\y*1.2) -- (B|-0,\y*1.2);
		\draw[draw=Purple, very thick, dashed, 
			  name intersections={of=AC and horiz,by=P},
			  name intersections={of=BC and horiz,by=Q}] (P) -- (Q)
			  %node[midway,above,font=\bfseries\scshape,color=red!60!Brown] {\A};
			  node[midway,above] {\A};
	}
	\end{tikzpicture}
\end{frame}

\begin{frame}{Appropriateness of sources}
	\pause It is important to question the \textbf{appropriateness} of sources you use in academic work
	\begin{itemize}
		\pause\item \textbf{Validity}: Are claims based upon a correct interpretation of the evidence?
		\pause\item \textbf{Rigor}: Was the method of collecting evidence appropriate to ensure 
			comprehensive coverage while also avoiding bias?
	\end{itemize}
\end{frame}

\begin{frame}{Appropriateness of sources}
	\begin{itemize}
		\pause\item \textbf{Reliability}: has the claim been replicated, or at least reviewed, by other academics?
		\pause\item \textbf{Authoritativeness}: do we know who the author is?
			Does the author have enough experience in the field to present a fair and balanced argument?
		\pause\item \textbf{Venue}: Is the publisher reputable and free of undue editorial influences?
	\end{itemize}
\end{frame}

\begin{frame}{Appropriateness of sources}
	\pause There are of course exceptions where sources are presented as \textbf{artefacts} and/or \textbf{archives}:
	\begin{itemize}
		\pause\item Citing a newspaper as evidence for a claim based on the reception of a new technology
		\pause\item Citing a manufacturer's technical manual when describing a technical feature of a platform
		\pause\item Citing a Reddit post by a well-known industry figure as evidence for expert opinion
	\end{itemize}
	\pause The \textbf{way} in which sources are \textbf{used} is therefore important
\end{frame}

\part{Library resources}
\frame{\partpage}

\begin{frame}{The library}
	\begin{center}
		\includegraphics[height=0.7\textheight]{campus_map}
	\end{center}
\end{frame}

\begin{frame}{Library catalogue}
	\begin{center}
		\url{http://library.fxplus.ac.uk/}
	\end{center}
\end{frame}

\begin{frame}{Web proxy}
    \begin{itemize}
        \pause\item Some online resources are only available through the campus network
        \pause\item If not physically on campus, you can access them via \textbf{VPN}
        \pause\item \url{https://webvpn.falmouth.ac.uk/}
        \pause\item Some resources can also be accessed by \textbf{web proxy} through the library website
        \pause\item \url{https://library.fxplus.ac.uk/game-design-computing}
	\end{itemize}
\end{frame}

\begin{frame}{Important resources}
	\begin{itemize}
        \pause\item ACM Digital Library
		\pause\item IEEE Xplore
		\pause\item GDC Vault
	\end{itemize}
\end{frame}

\begin{frame}{How to find papers to read?}
	\begin{itemize}
		\pause\item Specialist databases: ACM Digital Library, IEEE Xplore
		\pause\item Google Scholar
		    \begin{itemize}
		        \pause\item Keyword searches
		        \pause\item Other work by the same author
		        \pause\item Work which has cited papers you have read
		    \end{itemize}
		\pause\item Wikipedia
		    \begin{itemize}
		        \pause\item Not a reliable source itself!
		        \pause\item However most articles have good bibiliographies
		    \end{itemize}
		\pause\item Bibliographies of papers you have read
	\end{itemize}
\end{frame}

\begin{frame}{Finding papers without paying}
	\begin{itemize}
		\pause\item Many papers are \textbf{paywalled}
		\pause\item Little known fact: \textbf{none} of the money from paywalls goes to the authors of the paper!
		\pause\item The university \textbf{subscribes} to ACM and IEEE to give free access to staff and students
		\pause\item Many journals offer free \textbf{open access}
		\pause\item Many authors put papers on their \textbf{personal websites}
		\pause\item Many universities (all UK universities) have \textbf{open access repositories}
			\begin{itemize}
				\pause\item Falmouth: \url{http://repository.falmouth.ac.uk}
			\end{itemize}
		\pause\item Sites like \textbf{sci-hub} have sprung up, providing illegal downloads of papers
	\end{itemize}
\end{frame}

\lstset{language=[LaTeX]{TeX}, morekeywords={part,pause,lstinputlisting}}

\part{Introducing LaTeX}
\frame{\partpage}

\begin{frame}{What is LaTeX?}
    \begin{itemize}
        \pause\item A \textbf{typesetting} system
        \pause\item A \textbf{markup language}
                (like HTML or Markdown)
        \pause\item \textbf{Not} a WYSIWYG system
    \end{itemize}
\end{frame}

\begin{frame}{These slides were written in LaTeX}
    % Display the first few lines of this file
    \lstinputlisting[firstline=3,lastline=18]{latex.tex}
\end{frame}

\begin{frame}{Why LaTeX?}
\begin{itemize}
	\pause\item Plain text format
	\begin{itemize}
		\pause\item Can use any text editor
		\pause\item Can use version control (e.g.\ Git)
		\pause\item Can use online editors (e.g.\ Overleaf)
	\end{itemize}
	\pause\item Separates content from formatting
	\begin{itemize}
		\pause\item Similar to HTML and CSS
		\pause\item Unlike most WYSIWYG systems
	\end{itemize}
	\pause\item Produces professional-looking papers, reports, theses, books, slideshows, ...
	\pause\item Excellent facilities for typesetting mathematical equations, pseudocode, source code listings etc.
	\pause\item Automatically handles cross-referencing of sections, figures etc.
	\pause\item Automatic tools for managing bibliographies (BibTeX)
\end{itemize}
\end{frame}

\begin{frame}{Getting LaTeX}
\begin{itemize}
	\pause\item LaTeX is \textbf{free open source software}
	\pause\item Consists of:
	\begin{itemize}
		\pause\item Several \textbf{executables} (pdflatex, bibtex, makeindex, ...)
		\pause\item A large library of \textbf{packages}
		\pause\item An \textbf{integrated development environment (IDE)} (optional)
	\end{itemize}
	\pause\item Distributions available for all major OSes
	\begin{itemize}
		\item Windows: MikTeX
		\item MacOS: MacTeX
		\item Linux: TeXLive
	\end{itemize}
	\pause\item Online services e.g.\ Overleaf (should also work on iPad / Android)
\end{itemize}
\end{frame}

\begin{frame}{Workshop Activity}
\begin{itemize}
	\item Go to \url{https://www.overleaf.com} and sign up for a free account
	\item Go to \url{https://www.latex-tutorial.com/tutorials/} and work through the tutorials
	\item Please prioritise the following tutorials (look at the others afterwards if you have time):
	    \begin{itemize}
            %\item 00 Installation
            \item 01 Your first document
            \item 02 Document structure (sections and paragraphs)
            \item 03 Packages
            %\item 04 Math
            \item 05 Adding pictures
            %\item 06 Table of contents
            \item 07 Bibliography
            %\item 08 Footnotes
            %\item 09 Tables
            %\item 10 Automatic table generation (from .csv)
            %\item 11 Automatic plot generation (from .csv)
            %\item 12 Drawing graphs (vector graphics with tikz)
            \item 13 Source code highlighting
            %\item 14 Circuit diagrams
            %\item 15 Advanced circuit diagrams
            \item 16 Hyperlinks
            \item 17 Lists
	    \end{itemize}
    \end{itemize}
\end{frame}


\part{Referencing}
\frame{\partpage}

\begin{frame}{Which referencing style?}
	\begin{itemize}
		\pause\item Many different referencing styles exist
		\pause\item Most Falmouth courses use \textbf{Harvard} style
		\pause\item In Computing we tend to prefer \textbf{IEEE} style
		\pause\item If assignments specify which one to use then use it
		\pause\item Otherwise choose whichever you prefer --- just be \textbf{consistent}
		\pause\item Tools like BibTeX make it easy to switch styles
	\end{itemize}
\end{frame}

\begin{frame}{IEEE referencing style}
	\begin{center}
		\small\url{https://ieeeauthorcenter.ieee.org/wp-content/uploads/IEEE-Reference-Guide.pdf}
	\end{center}
\end{frame}

\begin{frame}{BibTeX entry types}
	\begin{center}
		\small\url{https://en.wikibooks.org/wiki/LaTeX/Bibliography_Management\#BibTeX}
	\end{center}
\end{frame}

\begin{frame}{Writing BibTeX entries}
	\begin{itemize}
		\pause\item Some websites provide pre-written BibTeX entries for papers
		\pause\item Beware of copying and pasting these as they are often incomplete, incorrectly formatted or just wrong!
		\pause\item You must \textbf{always} check your bibliography in the compiled PDF and fix any errors
		\pause\item You \textbf{will} lose marks on your written assignments otherwise!
	\end{itemize}
\end{frame}



\begin{frame}{Workshop Activity}
	\begin{itemize}
		\pause\item See LearningSpace: work through the LaTeX tutorials
		\pause\item If you get stuck, post in chat here
		\pause\item As you're working through them, compare LaTeX to WYSIWYG systems such as Microsoft Word
		\pause\item Think of \textbf{one thing} which is better or easier in LaTeX...
		\pause\item ... and \textbf{one thing} which is better or easier in Word
		\pause\item Reconvene here at 5:30pm for a wrap-up
		% \item Go to \url{https://www.overleaf.com} and sign up for a free account
		% \item Go to \url{https://www.latex-tutorial.com/tutorials/} and work through the tutorials
		% \item Please prioritise the following tutorials (look at the others afterwards if you have time):
		% 	\begin{itemize}
		% 		%\item 00 Installation
		% 		\item 01 Your first document
		% 		\item 02 Document structure (sections and paragraphs)
		% 		\item 03 Packages
		% 		%\item 04 Math
		% 		\item 05 Adding pictures
		% 		%\item 06 Table of contents
		% 		\item 07 Bibliography
		% 		%\item 08 Footnotes
		% 		%\item 09 Tables
		% 		%\item 10 Automatic table generation (from .csv)
		% 		%\item 11 Automatic plot generation (from .csv)
		% 		%\item 12 Drawing graphs (vector graphics with tikz)
		% 		\item 13 Source code highlighting
		% 		%\item 14 Circuit diagrams
		% 		%\item 15 Advanced circuit diagrams
		% 		\item 16 Hyperlinks
		% 		\item 17 Lists
		% 	\end{itemize}
    \end{itemize}
\end{frame}

\end{document}
