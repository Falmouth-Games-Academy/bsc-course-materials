\lstset{language=[LaTeX]{TeX}, morekeywords={part,pause,lstinputlisting}}

\part{Introducing LaTeX}
\frame{\partpage}

\begin{frame}{What is LaTeX?}
    \begin{itemize}
        \pause\item A \textbf{typesetting} system
        \pause\item A \textbf{markup language}
                (like HTML or Markdown)
		\pause\item \textbf{Not} a WYSIWYG system
    \end{itemize}
\end{frame}

\begin{frame}{These slides were written in LaTeX}
    % Display the first few lines of this file
    \lstinputlisting[firstline=3,lastline=18]{latex.tex}
\end{frame}

\begin{frame}{Why LaTeX?}
\begin{itemize}
	\pause\item Plain text format
	\begin{itemize}
		\pause\item Can use any text editor
		\pause\item Can use version control (e.g.\ Git)
		\pause\item Can use online editors (e.g.\ Overleaf)
	\end{itemize}
	\pause\item Separates content from formatting
	\begin{itemize}
		\pause\item Similar to HTML and CSS
		\pause\item Unlike most WYSIWYG systems
	\end{itemize}
	\pause\item Produces professional-looking papers, reports, theses, books, slideshows, ...
	\pause\item Excellent facilities for typesetting mathematical equations, pseudocode, source code listings etc.
	\pause\item Automatically handles cross-referencing of sections, figures etc.
	\pause\item Automatic tools for managing bibliographies (BibTeX)
\end{itemize}
\end{frame}

\begin{frame}{Getting LaTeX}
\begin{itemize}
	\pause\item LaTeX is \textbf{free open source software}
	\pause\item Consists of:
	\begin{itemize}
		\pause\item Several \textbf{executables} (pdflatex, bibtex, makeindex, ...)
		\pause\item A large library of \textbf{packages}
		\pause\item An \textbf{integrated development environment (IDE)} (optional)
	\end{itemize}
	\pause\item Distributions available for all major OSes
	\begin{itemize}
		\item Windows: MikTeX
		\item MacOS: MacTeX
		\item Linux: TeXLive
	\end{itemize}
	\pause\item Online services e.g.\ Overleaf (should also work on iPad / Android)
\end{itemize}
\end{frame}

