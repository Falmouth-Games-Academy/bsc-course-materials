% Adjust these for the path of the theme and its graphics, relative to this file
%\usepackage{beamerthemeFalmouthGamesAcademy}
\usepackage{../../beamerthemeFalmouthGamesAcademy}
\usepackage{multimedia}
\graphicspath{ {../../} }

% Default language for code listings
\lstset{language=C++,
        morekeywords={each,in,nullptr}
}

% For strikethrough effect
\usepackage[normalem]{ulem}
\usepackage{wasysym}

\usepackage{algpseudocode}

\usepackage{pdfpages}

\usepackage{fancyvrb}

% http://www.texample.net/tikz/examples/state-machine/
\usetikzlibrary{arrows,automata}

\newcommand{\modulecode}{COMP140 GAM160}\newcommand{\moduletitle}{Hacking Hardware/Advanced Programming}\newcommand{\sessionnumber}{Session 6}

\begin{document}
\title{\sessionnumber: Conducting a literature review and forming the research question}
\subtitle{\modulecode: \moduletitle}

\frame{\titlepage} 

\begin{frame}
	\frametitle{Learning outcomes}
	\begin{itemize}
		\item \textbf{Explain} what makes a good research question
		\item \textbf{Formulate} research questions in the area of your chosen project
		\item \textbf{Conduct} a scholarly literature review
	\end{itemize}
\end{frame}

\begin{frame}{How is the dissertation marked}
	\begin{itemize}
		\pause\item The dissertation is marked \textbf{holistically}
		\pause\item There is a \textbf{rubric}, but \textbf{no weightings}
		\pause\item The markers will take account of \textbf{all} criteria,
		    and apply their \textbf{academic judgment} to reach a final mark
		\pause\item Don't neglect any part of the rubric --- all are important!
	\end{itemize}
\end{frame}

\part{Literature review}
\frame{\partpage}

\begin{frame}{A typical dissertation structure}
	\begin{itemize}
		\pause\item \textbf{Introduction}: introduce the broad context and motivation,
			culminating in your research question(s)
		\pause\item \textbf{Literature review}: survey existing work related to your project
		\pause\item \textbf{Method}: explain how you went about answering your research question
		\pause\item \textbf{Results}: present and analyse the data obtained, and discuss how it addresses
			your research question
		\pause\item \textbf{Conclusion}: summarise the dissertation, suggest directions for further work
		\pause\item \textbf{References}
	\end{itemize}
\end{frame}

\begin{frame}{The purpose of the literature review}
	\begin{itemize}
		\pause\item Understand the \textbf{context} of your work
		\pause\item Understand the \textbf{state of the art} in the field
			\begin{itemize}
				\pause\item What is currently known?
				\pause\item What are the important open questions?
				\pause\item What research methods are used in the field?
			\end{itemize}
		\pause\item Understand how \textbf{your work} fits in
			\begin{itemize}
				\pause\item Is your work \textbf{novel} (i.e.\ has it not been done before?)
				\pause\item Does it build sensibly on what has come before?
				\pause\item Is your research question one that others have asked, and possibly tried to answer?
			\end{itemize}
	\end{itemize}
\end{frame}

\begin{frame}{Which comes first: research question or literature review?}
	\pause\begin{center}
		\includegraphics[width=0.4\textwidth]{chickenegg1}
	\end{center}
	\begin{itemize}
		\pause\item Having an initial research question in mind will help focus your literature search
		\pause\item What you read will influence your research question
		\pause\item Be prepared to \textbf{refine} your research question
	\end{itemize}
\end{frame}

\begin{frame}{Tips}
	\begin{itemize}
		\pause\item Read widely!
		\pause\item Keep thorough notes
		\pause\item Annotate (either on paper or on screen)
		\pause\item Write up as you go along
	\end{itemize}
\end{frame}

\begin{frame}{Recommended reading}
	D.\ Boote and P.\ Beile. ``Scholars before researchers: on the centrality of the disseration
		literature review in research preparation,'' \textit{Educational Researcher} Vol.\ 34 No.\ 6, pp.\ 3--15,
		2005.
\end{frame}


\part{Formulating the research question}
\frame{\partpage}

\begin{frame}{What makes a good research question?}
	\begin{itemize}
		\pause\item \textbf{Motivates} and \textbf{focuses} your research
		\pause\item Is \textbf{relevant} to the field
		\pause\item Has \textbf{originality} (doesn't have to be completely original, but shouldn't be ``solved'')
		\pause\item Is \textbf{manageable} in the context of your project
		\pause\item Is neither \textbf{too broad} nor \textbf{too narrow}
		\pause\item Leads to \textbf{testable hypotheses}
		\pause\item Requires \textbf{argumentation} and \textbf{analysis}, not mere \textbf{statistics}
		\pause\item Is \textbf{interesting} and addresses a \textbf{need}
	\end{itemize}
\end{frame}

\begin{frame}{Scope of research questions}
	\begin{itemize}
		\pause\item Too broad:
			\begin{itemize}
				\item Are videogames bad for children?
			\end{itemize}
		\pause\item Too narrow, not interesting:
			\begin{itemize}
				\item How many children in Cornwall play Overwatch?
			\end{itemize}
		\pause\item Better:
			\begin{itemize}
				\item What effect does regular videogame playing have on the academic attainment of
					children ages 11--14?
			\end{itemize}
	\end{itemize}
\end{frame}

\begin{frame}{Research questions vs hypotheses}
	\begin{itemize}
		\pause\item A research question invites \textbf{exploration}
		\pause\item A hypothesis makes a \textbf{testable claim}
		\pause\item Research question:
			\begin{itemize}
				\item What effect does regular videogame playing have on the academic attainment of
					children ages 11--14?
			\end{itemize}
		\pause\item Hypothesis:
			\begin{itemize}
				\item There is a positive correlation in children ages 11--14
					between hours spent playing Minecraft and grades in computing
			\end{itemize}
		\pause\item A good research question leads to several hypotheses
	\end{itemize}
\end{frame}

\begin{frame}{Choosing a research problem}
	\includegraphics[width=\textwidth]{alon_fig1}
	
	{\tiny
	U.\ Alon, ``How to choose a good scientific problem,'' \textit{Molecular Cell} 35, pp.\ 726--728, 2009.
	}
\end{frame}

\begin{frame}{Exercise}
	\begin{itemize}
		\pause\item Look at some of the papers you have been reading
		\pause\item What are the \textbf{research questions} behind them?
	\end{itemize}
\end{frame}



\part{Research proposal}
\frame{\partpage}

\begin{frame}{Research proposal}
    \begin{itemize}
        \pause\item Last week you were asked to prepare a $\approx$ 500-word research proposal
        \pause\item Divide into pairs --- pair up with someone who \textbf{doesn't know} (much) about your proposed project
        \pause\item 5 minutes: \textbf{read} each other's proposals
        \pause\item I will ask \textbf{you} to explain:
            \begin{itemize}
                \pause\item In 1 sentence: what is \textbf{their} proposed research topic/question?
                \pause\item In 1 sentence: why is this interesting and/or important?
            \end{itemize}
    \end{itemize}
\end{frame}

\begin{frame}{Research proposal}
    Discuss in your pairs:
    \begin{itemize}
        \pause\item Did you understand each other's proposals?
        \pause\item Were there any misunderstandings or misrepresentations?
        \pause\item How can the proposal, and particularly the research question, be improved?
    \end{itemize}
    For the rest of this session:
    \begin{itemize}
        \pause\item Refine your proposal into a well-defined research question,
             to discuss with your supervisor after this
    \end{itemize}
\end{frame}



\end{document}
