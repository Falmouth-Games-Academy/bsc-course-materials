\part{The Role of Experimentation}
\frame{\partpage}

\begin{frame}{Definition of experiment}
An experiment can be characterized as a test (or a series of tests) wherein changes are introduced in the state of a system or process, enabling the observation and characterization of effects that can occur as a result of these changes.
\end{frame}

\begin{frame}{Experiments}
Usually performed with an objective in mind:

\bitems Uncovering \textbf{influential variables} in a given system or process;
	\spitem Determining \textbf{desired values} for certain \textbf{parameters}
	\spitem Characterize \textbf{behavior} of the system or process under study.
\eitem
\end{frame}

\begin{frame}{Retrospective study}
\begin{itemize}
	\item Characteristics
	\bitems Use of historical data;
	\item Investigating correlations;\eitem
	\item Problems
	\bitems Data representativeness;
	\item Availability of data;\eitem
\end{itemize}
\end{frame}

\begin{frame}{Observational study}
\begin{itemize}
	\item Characteristics
	\bitems Observation of the system with minimal disturbance;
	\item Investigation of usual behaviors;\eitem
	\item Problems
	\bitems Low representativeness of extreme cases;
	\item Low variability can affect observation of interesting effects;\eitem
\end{itemize}
\end{frame}

\begin{frame}{Designed experiment}
\begin{itemize}
	\item Characteristics
	\bitems Introduction of deliberate changes in the system;
		\item Inference on the \textit{causality} of the effects;
		\eitem
	\item Problems
	\bitems Requires rigorous experimental design and data analysis;
	\item Usually more expensive.
	\eitem
\end{itemize}
\end{frame}

\part{Experimentation strategies}
\frame{\partpage}

\begin{frame}{Educated guessing}
\bitems Select arbitrary combination of levels for the factors;
			\item Test and observe behavior; 
			\item Change one or two factors at a time, then re-test;
		\eitem
		\bitems Widely used in industry;
\item Can achieve good results, but has a lot of limitations;
\eitem
\end{frame}

\begin{frame}{COST: Change One Separate factor at a Time}
		\bitems Select a reference point;
			\item Change each factor individually, keeping all others constant;
		\eitem
	\bitems Also widely used;
		\item Can achieve good results as long as there are no interaction effects;
	\eitem
\begin{center}
	\colorbox{white}{\includegraphics[width=5.5cm]{../Campelo/figs/OFAT01c.png}}
\end{center}
% Image: (c) D.C. Montgomery
\end{frame}

\begin{frame}{Factorial designs}
		\bitems Select \textbf{levels} for each factor;
			\item Vary the factors simultaneously, in a systematic way;
		\eitem
	\bitems Estimation of main effects and interactions;
		\item Greater precision in the effect estimates;
		\item More efficient use of resources (information/observation);
	\eitem
\begin{center}
	\colorbox{white}{\includegraphics[width=5.5cm]{../Campelo/figs/FFD01a.png}}
\end{center}
% \lfr{Image: (c) D.C. Montgomery}
\end{frame}

\begin{frame}{Design of experiments}
Process of designing data gathering protocols to enable
accurate analyses by statistical tools, capable of supporting
sound and objective conclusions.
\end{frame}

\begin{frame}{Design of experiments}
\bitems Applicable to systems and processes subject to noise, experimental errors, uncertainties, etc.
	\item Necessary for the conclusions to have a quantifiable meaning;
	\item Helpful in avoiding errors due to personal biases or other artifacts of experimentation and analysis.
\eitem
\end{frame}

\begin{frame}{Design of the experiment}
	\bitems Scientific/technical question of interest;
		\item Selection of variables and values;
		\item Definition of the desired confidence level;
		\item Sample size calculations;
		\item Determination of protocols for data gathering;
	\eitem
\end{frame}

\begin{frame}{Statistical analyses of the data}
	\bitems Calculation of a test statistic;
		\item Validation of the assumptions of the statistical model;
		\item Calculation of the magnitude of effects;
		\item Drawing of conclusions and recommendations;
	\eitem
\end{frame}

\begin{frame}{Repetition and replication}
	\bitems Repeated measurements - estimation of within-group variability;
		\item Replication - estimative of the experimental error;
		\item Greater precision in estimating the model parameters;
	\eitem
\end{frame}

\begin{frame}{Randomization}
\bitems Avoids contamination of the data by order-dependent effects such as:

	\bitems Heating effects;
		\spitem Wear and tear effects;
		\spitem External interferences;
	\eitem
\eitem
\end{frame}

\begin{frame}{Blocking}
\bitems Isolation of nuisance variables (those that influence the response, but are not interesting for the analyses) that can be controlled;
	\spitem Improvement in the estimation of effects for the factors of interest;
	\spitem Reduction or eliminations of inconvenient factor effects;
\eitem
\end{frame}

\begin{frame}{The role of experimental design}
Experimental design is useful for avoiding the influence of spurious factors and personal biases on the results, by performing experiments in a impartial and objective way.

\vspace{2ex}

``\textit{The great tragedy of Science - the slaying of a
		beautiful hypothesis by an ugly fact.}''
		
		-- Thomas H. Huxley
%\lfr{Image: \url{http://www.iep.utm.edu/huxley/}}
\end{frame}

\begin{frame}{Jacques Benveniste and the memory of water}
	\bitems Nature (1988);
		\item Investigation committee: Maddox, Stewart, Randi;
		\item Retracted by Nature due to evidence of misconduct.
	\eitem
	Methodological problems
	\bitems Experimenter bias (absence of proper blinding);
		\item Cherrypicking (selective recording of results);
		\item Unaccounted sampling errors;
		\item Possible contamination;
		\item Complete lack of prior physical/ chemical plausibility;
		\item \textbf{Non-reproducibility}.
	\eitem
\end{frame}

\begin{frame}{Guidelines for a good design}
\bitems Pre-experimental design:

	\bitems Identification and definition of the problem;
		\spitem Selection of experimental and response variables of interest;
		\spitem Choice of experimental protocols;
	\eitem
	\spitem Choice of the experimental design;
	\spitem Collection of the data;
	\spitem Statistical data analyses;
	\spitem Conclusions and recommendations;
\eitem
\end{frame}

%=====

\begin{frame}{Before we start}
\bitems Is the investigation relevant?
	\spitem Would the results be interesting for the research community?
	\spitem Practical relevance?
	\bitems Employ exploratory experiments;\eitem
	\spitem Placement within the literature;
	\bitems Avoid repetition and irrelevance.\eitem
\eitem
\end{frame}

%=====

\begin{frame}{Definition of hypotheses}
The translation \textit{scientific question} $\rightarrow$ \textit{test hypothesis} requires special attention, and a solid knowledge of the technical area in which the experiment is being performed;
\end{frame}

%=====

\begin{frame}{Choice of Experimental Design}
\bitems (Relatively) simple, as long as the pre-experimental part is well done;
	\spitem Dependent on what is being tested (statistical question);
	\spitem A sound design tends to determine the analyses technique to be used, at least  qualitatively;
	\spitem Involves considerations about:
	\bitems Sample size;
		\item Ordering of observations;
		\item Determination of restrictions to the randomization and the use of blocks, etc.
	\eitem
	\spitem Available in several statistical/mathematical packages;
\eitem
\end{frame}

\begin{frame}{Problem-dependent}
\bitems Depending on the experimental question, different experimental designs are required
	\spitem A solid, statistically sound design tends to determine which statistical tests must be employed in the analysis step, at least qualitatively.
	\spitem Quantification of the proportion between intra-groups and inter-groups variability;
\eitem
\end{frame}

\begin{frame}{Data gathering}
\bitems Must be consistent with design, otherwise the validity of the results may be compromised - data collection must always follow the plan:
	\bitems No premature stops;	
		\spitem\textit{No-peeking rule} (except when planned, of course);
	\eitem
	\spitem Use of pilot experiments:
	\bitems Gathering of preliminary information;
		\spitem Practice with the experimental conditions;
	\eitem
\eitem
\end{frame}

\begin{frame}{A consequence of design}
\bitems Analysis techniques are generally relatively simple, but \textit{the devil is in the details};
	\spitem Use of existing statistical tools and frameworks, such as R
\eitem
\end{frame}

%=====

\begin{frame}{Statistical modeling}
\bitems General procedure for testing the experimental hypotheses:

	\bitems Definition of a \textit{null-model} (absence of effects) and of a desired level of significance;
		\spitem Determination of $P(data|\mbox{\textit{null-model}})$;
		\spitem Decision by rejection (or not) of the null hypothesis;
		\spitem Validation of model assumptions;
		\spitem Estimation of the  \textit{magnitude} of differences - \textbf{practical significance};
	\eitem
\eitem

	\centering\textit{Statistical methods do not \textbf{prove} anything, but they allow an objective definition of margins of plausibility for certain statements.}
\end{frame}

%=====

\begin{frame}{Reporting of results}
Combine textual, numeric and graphical elements to tell a story with your data. It simplifies the understanding and analysis of the results.
\begin{columns}[T]
\column{0.75\textwidth}
	\bitems Strive to achieve graphical excellence;
		\spitem Coherence of notation - special attention to figures and tables;
		\spitem Display simultaneous confidence intervals and other graphical indicators of effect size.
	\eitem
\column{0.25\textwidth}
	\centering\includegraphics[height=1.5cm]{../Campelo/figs/tufte.jpg}\\
	\centering\includegraphics[height=1.5cm]{../Campelo/figs/yau.png}
\end{columns}
\vspace{1ex}
{\footnotesize
Other great resources on graphical excellence:
\begin{itemize}
\item\textit{Flowing Data} (\url{http://flowingdata.com/})
\item\textit{Information is Beautiful} (\url{http://www.informationisbeautiful.net})
\end{itemize}
}
\end{frame}

\begin{frame}{Drawing and reporting conclusions}
\bitems Conclusions should be based on solid evidence from the data;
	\spitem Be conservative - it is common to exaggerate the generality of the results;
	\spitem Report significance levels and the assumptions under which the results are valid;
	\spitem \textit{Suggest explanations} to the observed results;
	\spitem Careful with \textit{anomaly hunting};
\eitem
\end{frame}

\begin{frame}{Some more relevant points}
\bitems Use of previous knowledge, theoretical or empirical;
	\spitem Iterative experimentation;
	\spitem Statistical $\times$ practical significance;
	\spitem Use of additional experiments to validate conclusions.
\eitem
\end{frame}
