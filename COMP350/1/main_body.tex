% Adjust these for the path of the theme and its graphics, relative to this file
%\usepackage{beamerthemeFalmouthGamesAcademy}
\usepackage{../../beamerthemeFalmouthGamesAcademy}
\usepackage{multimedia}
\graphicspath{ {../../} }

% Default language for code listings
\lstset{language=C++,
        morekeywords={each,in,nullptr}
}

% For strikethrough effect
\usepackage[normalem]{ulem}
\usepackage{wasysym}

\usepackage{pdfpages}

% http://www.texample.net/tikz/examples/state-machine/
\usetikzlibrary{arrows,automata}

\newcommand{\modulecode}{COMP140 GAM160}\newcommand{\moduletitle}{Hacking Hardware/Advanced Programming}\newcommand{\sessionnumber}{Session 6}

\begin{document}
\title{\sessionnumber: Module Intro \& The Optimisation Process}
\subtitle{\modulecode: \moduletitle}

\frame{\titlepage} 

\begin{frame}{Learning outcomes}
	By the end of today's session, you will be able to:
	\begin{itemize}
		\item \textbf{Recall} the key stages of the graphics pipeline
		\item \textbf{Explain} the differences between a CPU and a GPU
		\item \textbf{Write} basic programs using SDL and OpenGL
	\end{itemize}
\end{frame}

\part{Module Introduction}
\frame{\partpage}

\begin{frame}{Module Aims}
	\begin{itemize}
		\item Gain in understanding of techniques used professionally in the management of computing resources.
		\item Acquire knowledge and experience of concepts used to predict and model resource use.
		\item Acquire the knowledge and experience to enable critical evaluation of trade-offs to generate optimisation and efficiency.
	\end{itemize}
\end{frame}
\part{Assignment Details}
\frame{\partpage}

\begin{frame}{Assignment Overview}
	\begin{itemize}
		\item Optimisation Task - 50\%
		\item Porting Task - 30\%
		\item Research Journal  - 20\%
	\end{itemize}
\end{frame}

\begin{frame}{Assignment 1 - Optimisation Task}
	\begin{itemize}
		\item Take an existing project and optimise
		\item You have to identify the tools required for optimising
		\item I am more interested in your \textbf{process} during the task
		\item First Submission - \textbf{Friday 9th of February at 5pm}
		\item \url{https://github.com/Falmouth-Games-Academy/bsc-assignment-briefs/raw/2017-18/comp350/1/comp350_1.pdf}
	\end{itemize}
\end{frame}

\begin{frame}{Assignment 2 - Porting }
	\begin{itemize}
		\item Continue on with the project from Assignment 1
		\item Port your project to one of the following Platforms - PS4, Android, iOS
		\item You will have to fulfil some of the Technical Requirement for that platform
		\item \url{https://github.com/Falmouth-Games-Academy/bsc-assignment-briefs/raw/2017-18/comp350/2/comp350_2.pdf}
	\end{itemize}
\end{frame}

\begin{frame}{Assignment 3 - Research Journal }
	\begin{itemize}
		\item Write a 1200 word research journal on optimisation \& porting 
		\item Contribute to a community Wiki
		\item \url{https://github.com/Falmouth-Games-Academy/bsc-assignment-briefs/raw/2017-18/comp350/3/comp350_3.pdf}
	\end{itemize}
\end{frame}
\part{Optimisation}
\frame{\partpage}

\begin{frame}{Optimiser Mantra}
	\begin{enumerate}
		\pause \item Benchmark
		\pause \item Measure
		\pause \item Detect
		\pause \item Solve
		\pause \item Check
		\pause \item Repeat
	\end{enumerate}
\end{frame}

\begin{frame}{Benchmark}
	\begin{itemize}
		\pause \item This is a point of reference for your game, serves as a standard for comparison
		\pause \item A good benchmark should:
		\begin{enumerate}
			\pause \item Consistent between runs
			\pause \item Should be quick
			\pause \item Represent an actual game situation
			\pause \item Responsive to changes 
		\end{enumerate}
	\end{itemize}
\end{frame}

\begin{frame}{Measure}
	\begin{itemize}
		\pause \item You should be able to measures the performance of your code 
		\pause \item Tools like Profilers allow you to monitor the following:
		\begin{itemize}
			\pause \item CPU Usage - Across all cores, usually \% utilisation
			\pause \item Memory Usage - Ram, Stack, Heap etc
			\pause \item GPU - GPU core \& memory usage and shader performance
			\pause \item Code - Timings, function calls stats, call graphs
		\end{itemize}
	\end{itemize}
\end{frame}

\begin{frame}{Detect}
	\begin{itemize}
		\pause \item Usually the result of looking at the data from the profiler
		\pause \item With every change you are looking for the biggest possible performance increase
		\pause \item Always start with the big picture and work your way down
		\begin{itemize}
			\pause \item CPU or GPU slowing you down most?
			\pause \item If GPU is under utilised, perhaps shift some of work to the GPU (Compute Shaders) or perhaps GPU is waiting for CPU
			\pause \item If CPU is over utilised, perhaps look at profiling code in functions
		\end{itemize}
	\end{itemize}
\end{frame}

\begin{frame}{Solve}
	\begin{itemize}
		\pause \item Once you have detected the problem you need to solve it
		\pause \item This could involve rewriting an algorithm or changing data structures
		\pause \item In all cases the data captured should drive your work
	\end{itemize}
\end{frame}

\begin{frame}{Check}
	\begin{itemize}
		\pause \item After a change has been made you should always run the profiler again
		\pause \item Also check on different hardware!
	\end{itemize}
\end{frame}

\begin{frame}{Repeat}
	\begin{itemize}
		\pause \item A change in your code base can cause other issues to crop
		\pause \item Create a new benchmark and start the process again 
	\end{itemize}
\end{frame}

\begin{frame}{Levels of Optimisation}
	\begin{itemize}
		\pause \item System Level: Utilisation, Balancing and Efficiency
		\pause \item Algorithmic Level: Focus on removing work
		\pause \item Micro-Level: Line by line optimising (data structures is a good example here)
	\end{itemize}
\end{frame}

\begin{frame}{Optimisation Pitfalls}
	\begin{itemize}
		\pause \item Assumptions: Always measure!
		\pause \item Premature Optimisation: Don't optimise with data, or too early in the development process
		\pause \item Optimisation on Only One Machine: Test on the worst case system
		\pause \item Optimising Debug Builds
	\end{itemize}
\end{frame}


\part{Coffee Break}
\frame{\partpage}
\part{Housekeeping and Admin}
\frame{\partpage}
\part{Porting Hardware}
\frame{\partpage}

\begin{frame}{Exercise}
	\begin{enumerate}
		\item Fork the coursework repo - \url{https://github.com/Falmouth-Games-Academy/comp350-optimisation}
		\item Identify your main development tools (Unity or Unreal, Native Code)
		\item Investigate the various profiling options
		\item Record in a word doc (or similar) resources for these tools
		\item Answer the following questions
		\begin{itemize}
			\item What stats can be collected?
			\item Can you profile the GPU?
			\item What data can you record about your own code?
			\item Can you customise the Profiler, does it have an API?
		\end{itemize}
		\item Carry out a Pull Request for feedback 
	\end{enumerate}
\end{frame}

\begin{frame}{Debrief}
	\begin{itemize}
		\item \textbf{Recall} the key stages of the graphics pipeline
		\item \textbf{Explain} the differences between a CPU and a GPU
		\item \textbf{Write} basic programs using SDL and OpenGL
	\end{itemize}
\end{frame}

\end{document}
